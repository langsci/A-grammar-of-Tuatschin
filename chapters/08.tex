\chapter{The contact languages of Tuatschin}
The contact languages of Tuatschin are standard Sursilvan, Swiss German, and standard German. In this chapter only some few remarks will be given, since the subject is manifold and could easily fill a book-length publication. 

Standard Sursilvan is the language of instruction in school and is used among Tuatschin native speakers for written communication.

The contact with Swiss German starts at an early age through contact with Swiss German speakers who have a vacation house in the Tujetsch valley, or with tourists, and also with relatives which live in the German part of Switzerland and who have not learned Romansh.

Standard German starts being taught in school from the fourth form of primary school onwards and is also present through television. Broadcasting in Romansh is very scarce. Currently there is a ten minute news broadcasting from Monday to Friday, a broadcasting for children on Saturday that lasts 10 minutes, and a cultural broadcasting on Sunday which lasts 25 minutes. Therefore most broadcasting children (and adults) look are in German.

On a more general level, all Romansh varieties have been in contact with German, Swiss or standard, since a long time. \citet[176--181]{Liver2010} states that Romansh has been in contact with German since the times of Old High German (ca. 750-1050). Loans from OHG are for instance (I cite the Tuatschin forms) \textit{gljut} `people' (< OHG \textsc{liut}), \textit{uaut} `forest' (< OHG \textsc{wald}), or \textit{lubí} `permit' (< OHG \textsc{laubjan}).

More modern loans are e.g. \textit{clétg} `luck' (< German \textit{Glück}), \textit{stédi} `diligent' (< German \textit{stetig}), \textit{lura} `when' instead of \textit{scha} as a correlative element with conditional clauses and temporal clauses headed by \textit{cu} `when'.

Discourse particle of German origin are very often used. Examples are \textit{ábar} `but' (< German \textit{aber}), \textit{álṣò} `this is to say' (< German \textit{also}) \textit{sò} `well, OK' (< German \textit{so}), or \textit{zuar} `though' (< German \textit{zwar}).

Semantic broadening is also frequent. An example is \textit{unfrènda} `sacrifice, casualty', which is derived from Middle Latin \textsc{offerenda} \citet[1283]{Decurtins2012} whose Romance meaning is `sacrifice' and its German meaning `casualty'.

Calques are also very frequent, especially in the domain of the particle verbs (see § 4.1.3 above). Some more examples are \textit{mètar avaun} `imagine' (< German \textit{sich vorstellen}) (note that the Romansh synonym is not reflexive) \textit{curdá sé} (< German auffallen), or \textit{fá cun} `participate' (< German \textit{mitmachen}).

Sursilvan loans are less frequent than Germanisms, which is certainly related to the fact that a huge part of the lexicon Tuatschin shares with Sursilvan has the same form in both varieties.

Examples of sursilvanisms are \textit{pi} `more' instead of \textit{plé}, \textit{si} `up' instead of \textit{sé}, \textit{sa} `knows' instead of \textit{sò}, \textit{bugèn} `gladly' instead of \textit{ugèn}, \textit{uost} `August' instead of \textit{uést}.

A phonetic influence of Sursilvan is the use of [ʁ] instead of [r], which is not frequent among the native speakers I have consulted, but which one often hears when younger people or children speak Tuatschin.

