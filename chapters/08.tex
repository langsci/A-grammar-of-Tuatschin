\chapter{Varia}

\section{Reduplication}
In Tuatschin, reduplication only has an intensification function. Syntactic categories that may be reduplicated are attributive (\ref{ex:redadjattr1}) and predicative (\ref{ex:redadjpred1} and \ref{ex:redadjpred2}),  adjectives, adjectives used adverbially (\ref{ex:redadjpred3}) as well as adverbs modifying adjectives (\ref{ex:redadv1}) or used predicatively (\ref{ex:redadv2} and \ref{ex:redadv3}), or functioning as a discourse marker (\ref{ex:redadv4}).

\ea\label{ex:redadjattr1}
\langinfo{Tuatschín}{Sadrún}{m5, 1. 1199.}\\
\gll  Ála \textbf{véglja-véglja} tgèsa-parvènda, né?  \\
in.\textsc{def.art.f.sg} \textsc{red}\textasciitilde{old} presbytery right\\
\glt `At the very old presbytery, right?'
\z

\ea\label{ex:redadjpred1}
\langinfo{Tuatschin}{Surrain}{\citealt[128]{Büchli1966}}\\
\gll  […] i èra \textbf{sgtir-stgir} !\\
     […]  \textsc{expl} \textsc{cop.impf.3sg} \textsc{red}\textasciitilde{dark}\\
\glt `[…] it was pitch-dark.'
\z

\ea\label{ex:redadjpred2}
\langinfo{Tuatschín}{Ruèras}{m1, 1. 233ff.}\\
\gll    A quaj stèvnṣ èssar… \textbf{pulits-pulits} l’ jamna… tg’ al bap dètschi in frang a miaz.\\
and \textsc{dem.unm} must.\textsc{impf.1pl.1pl} \textsc{cop.inf} \textsc{red}\textasciitilde{well\_behaved}.\textsc{m.pl} \textsc{def.art.f.sg} week  \textsc{comp} \textsc{def.art.m.sg} father  give.\textsc{prs.sbjv.3sg} one.\textsc{m.sg} franc and half.\textsc{m.sg}\\
\glt `And we had to be … very well-behaved during the week … so that my father would give [us] one and a half francs.'
\z

\ea\label{ex:redadjpred3}
\langinfo{Tuatschín}{Sadrún}{m4, 1. 618ff.}\\
\gll  Al tat èr’ ajn a durméva lò grad sc’ in tajṣ, vèv’ udju \textbf{ṣchùbar-ṣchùbar} nuét.  \\
\textsc{def.art.m.sg} grandfather \textsc{cop.impf.3sg} up and sleep.\textsc{impf.3sg} there precisely like \textsc{indef.art.m.sg} badger have.\textsc{impf.3sg} hear.\textsc{ptcp.unm} \textsc{red}\textasciitilde{clean}.\textsc{adj.unm} nothing\\
\glt `My grandfather was up there and was sleeping like a log, he hadn’t heard anything at all.'
\z

\ea\label{ex:redadv1}
\langinfo{Tuatschín}{Sadrún}{m4, 1. 363f.}\\
\gll  Èl èr’ in tüp tga raṣdava bigja bjè, ju a gju \textbf{fétg-fétg} bian cun èl [...]. \\
\textsc{3sg} \textsc{cop.impf.3sg} \textsc{indef.art.m.sg} fellow \textsc{rel} speak.\textsc{impf.3sg} \textsc{neg} much \textsc{1sg} have.\textsc{prs.1sg} have.\textsc{ptcp.unm} \textsc{red}\textasciitilde{very} good.\textsc{unm} with \textsc{3sg.m}\\
\glt `He was a person who didn’t speak much, I went along very well with him [...].'
\z

\ea\label{ex:redadv2}
\langinfo{Tuatschin}{Tschamùt}{\citealt[18]{Büchli1966}}\\
\gll El ò mirau \textbf{antùrn-antùrn} […].\\
     \textsc{3sg} have.\textsc{prs.3sg} look.\textsc{ptcp.unm} \textsc{red}\textasciitilde{around} \\
\glt `He looked around and around.'
\z

\ea\label{ex:redadv3}
\langinfo{Tuatschín}{Ruèras}{m1, 1. 299}\\
\gll    A sjantar surprju acòrds \textbf{adin-adina}.\\
and after take\_over.\textsc{ptcp.unm} piecework.\textsc{m.pl} \textsc{red}\textasciitilde{always} \\
\glt `And afterwards I took over piecework, always.'
\z

\ea\label{ex:redadv4}
\langinfo{Tuatschín}{Sèlva}{f, 1. 937}\\
\gll Quaj èra in’ jèda... brutal tiar nus, \textbf{bèn-bèn}.   \\
\textsc{dem.unm} \textsc{cop.impf.3sg} one.\textsc{f.sg} time terrible.\textsc{adj.unm} among \textsc{1pl} \textsc{red}\textasciitilde{really}\\
\glt `Once it was terrible among us, really.'
\z

