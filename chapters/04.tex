\chapter{Verb phrase}

\section{The verb}
Tuatschin possesses intransitive (\ref{ex:intrans1}), mono-, and ditransitive verbs. Monotransitive verbs usually have a direct object (\ref{ex:trans1} and \ref{ex:trans2}), but in rare cases they also may have an indirect object (\ref{ex:trans:indir}). Ditransitive verbs (\ref{ex:ditrans1} and \ref{ex:ditrans2}) have a direct and an indirect object, the latter being marked by \textit{da/dad}\footnote{As shown in § 3.2.1.2 and 3.6.1 above, the dative articles \textit{li} (see example (\ref{ex:ditrans2})) as well as \textit{di} are obsolescent, and the dative marker \textit{a}, which is used in rare cases, is a loan from standard Sursilvan.}. An exception is the verb \textit{dumandá}, which has two direct objects (\ref{ex:trans:2DO}). Note that if the two objects are pronominal, the object of asking is usually not mentioned (\ref{ex:trans:2DO}). The indirect object usually precedes the direct object (\ref{ex:ditrans2}).

\ea
\label{ex:intrans1}
\langinfo{Tuatschín}{Sadrún}{m4, l. 535}\\
\gll  Quèl \textbf{durméva} sc’ in tajs.  \\
\textsc{dem.m.sg} sleep.\textsc{impf.3sg} like \textsc{indef.art.m.sg} badger\\
\glt `He used to sleep like a log.'
\z

\ea\label{ex:trans1}
\langinfo{Tuatschín}{Sedrun}{\citealt[106]{Büchli1966}}\\
\gll    In da Méjdal \textbf{vèv'} [\textbf{in}' \textbf{ura} \textbf{da} \textbf{Schwarzwald} \textbf{tg}' \textbf{èra} \textbf{ruta}].\\
     one.\textsc{m.sg} of \textsc{pln} have.\textsc{impf.3sg} \textsc{indef.art.f.sg} clock of \textsc{pln} \textsc{rel} \textsc{cop.impf.3sg} break.\textsc{ptcp.f.sg}\\
\glt `An inhabitant of Medel had a clock of the Black Forest that was broken.'
\z

\ea\label{ex:trans2}
\langinfo{Tuatschín}{Sadrún}{\citealt[103]{Büchli1966}}\\
\gll    In dé […] \textbf{partgirava} in buép [\textbf{las} \textbf{vacas}] sén Vòns.\\
     \textsc{indef.art.m.sg} day {} mind.\textsc{impf.3sg} \textsc{indef.art.m.sg} boy \textsc{def.art.f.pl} cow.\textsc{pl} up \textsc{pln}\\
\glt `One day [...] a boy was taking care of the cows at Vons.'
\z

\ea
\label{ex:trans:indir}
\langinfo{Tuatschín}{Sadrún}{m1, l. 254f.}\\
\gll    A quaj gang \textbf{udéva} [\textbf{dad} \textbf{òmasdús}].\\
and \textsc{dem.m.sg} corridor belong.\textsc{impf.3sg} \textsc{dat} both.\textsc{m.pl}\\
\glt `And there was only one corridor. And this corridor belonged to both [families].'
\z

\ea
\label{ex:ditrans1}
\langinfo{Tuatschín}{Camischùlas}{f6, l. 865f.}\\
\gll    [...] [\textbf{mia} \textbf{carèzja}] tgu stù \textbf{dá} [\textbf{da} \textbf{quèls}].\\
{} \textsc{poss.1sg.f.sg} love \textsc{rel.1sg} must.\textsc{prs.1sg} give.\textsc{inf} \textsc{dat} \textsc{dem.m.pl}\\
\glt `[...] my love that I have to give them.'
\z

\ea\label{ex:ditrans2}
\langinfo{Tuatschín}{Bugnaj}{\citealt[145]{Büchli1966}}\\
\gll    […] Nossadùna lèva \textbf{dá} [\textbf{li} \textbf{gjuven} \textbf{préir}] [\textbf{ina} \textbf{rancùnusciantscha} pal \textbf{survètsch}].\\
   {}  Our\_Lady.\textsc{f.sg} want.\textsc{impf.3sg} give \textsc{dat} young.\textsc{m.sg} priest \textsc{indef.art.f.sg} mark\_of\_gratitude.\textsc{f.sg} for.\textsc{def.art.m.sg} favour \\
\glt `[…] the holy Virgin wanted to give the young priest a mark of gratitude for the favour [he had done her].'
\z

\ea
\label{ex:trans:2DO}
\langinfo{Tuatschín}{Bugnaj} {\citealt[131]{Büchli1966}}\\
\gll  [...]  ina zagríndara […] \textbf{ò} \textbf{dumandau} [\textbf{la} \textbf{mùma} \textbf{da} \textbf{tgèsa}] [\textbf{in} \textbf{tgavégl} \textbf{da} \textbf{sia} \textbf{buéba}]. \\
{} \textsc{indef.art.f.sg} gipsy\_woman {} have.\textsc{prs.3sg}   ask.\textsc{ptcp.unm} \textsc{def.art.f.sg} mother of house one hair of \textsc{poss.3sg} girl \\
\glt `[…] a gipsy woman [...] asked the mother of the house for one hair of her daughter.'
\z


\subsection{Verbal morphology}
According to the ending of their infinitives, verbs can be divided into five classes:

\begin{itemize}

\item \textit{-á} (\textit{anflá} `find')
\item \textit{-è} (\textit{magljè} `eat')
\item  \textit{-aj} (\textit{tanaj} `hold')
\item \textit{-ˈar} (\textit{métar} `put')
\item \textit{-í} (\textit{fugí} `flee')
\end{itemize}

From a diachronic point of view, the \textit{-è}-class is a subclass of the \textit{-a}-class due to the general rule that \textit{a} becomes \textit{è} after a palatal consonant or glide.

The verbs ending in \textit{-aj} are all irregular and will be treated in Appendix II below, and the small amount of regular verbs ending in \textit{-í} all have the infix \textit{-èsch-} (see \tabref{tab:reg.verb-i} below).

Tuatschin has three nonfinite categories: infinitive, past participle, and gerund, whereby the gerund is not in use in current speech.

Within the finite categories, the language differentiates tense, aspect, and modal categories as well as simple,  compound, and doubly-compound categories.

The simple categories are present indicative, present subjunctive, imperfect indicative, imperfect subjunctive, conditional direct, conditional indirect, and imperative.

The compound categories are perfect indicative, perfect subjunctive, pluperfect indicative,  pluperfect subjunctive, and future. The compound tenses are formed with an auxiliary verb (either \textit{èssar} `be', \textit{vaj} `have', or \textit{vagní} `come') and the past participle or the infinitive. 

The doubly compound categories correspond to the perfect and the pluperfect, but with two past participles instead of one.

The personal ending for the first person singular present and imperfect indicative is \textit{-a} as in \textit{ju cònta} `I sing' and \textit{ju cantava} `I used to sing', but some irregular verbs lack this ending, as in  \textit{ju détsch} `I say', \textit{ju dùn} `I give', \textit{ju fétsch} `I do', \textit{ju végn} `I come', or \textit{ju vòn / ju mòn} `I go'. For further examples see Appendix II below.

Reflexive verbs are built with the prefix \textit{sa-} in all persons, tenses, moods, and nonfinite categories and use the auxiliary verb \textit{èssar} `be' for compound tenses: \textit{salavá} `wash (oneself)', \textit{ju salava} `I wash', \textit{té salavassas} `you (sg) would wash', \textit{nus èssan salavaj} `we have washed', \textit{vus vagnís á salavá} `you (pl) will wash', \textit{èlṣ èran salavaj} `they had washed'.

According to the DRG (1: 568), the choice of \textit{esser} as auxiliary verb for reflexives in Sursilvan is due to the demand of Sursilvan grammarians since the 18th century. Nowadays speakers seek to conform to this claim, but in spoken Sursilvan, one still can find \textit{haver} as auxiliary for reflexive verbs.

The reflexive verbs will not be treated this chapter, but their use will be presented below in § 5.6.1 on reflexive voice.

Verb forms that end in \textit{-n} in the first person singular present indicative take an euphonic \textit{-d} before \textit{ju} `I' or \textit{jus} `gone'.

\ea
\label{}
\langinfo{Tuatschín}{Sadrún}{m8, l. 1492}\\
\gll  Api sjantar \textbf{sùnd} \textbf{ju} sasjus gjù [...].\\
and after be.\textsc{prs.1sg} \textsc{1sg} sit.\textsc{ptcp.m.sg} down\\
\glt `And then I sat down [...].'
\z

\ea
\label{}
\langinfo{Tuatschín}{Sadrún}{m4, l. 414}\\
\gll   \textbf{Sùnd} \textbf{jus} ah gjù Surajn [...]. \\
be.\textsc{prs.1sg} go.\textsc{ptcp.m.sg} eh down \textsc{pln}\\
\glt `[I] went eh down to Surrein [...].'
\z


\subsubsection{Auxiliary verbs}
The auxiliary verbs \textit{èssar} (\tabref{tab:aux:èssar}) and \textit{vaj} (\tabref{tab:aux:vaj}) are used for compound tenses (and \textit{vaj} also for doubly-compound tenses), whereas \textit{vagní} (\tabref{tab:aux:vagní}) is used for future. In the following tables, only one compound tense will be listed, the perfect; as for the doubly-compound tenses, they are formed with the perfect or the imperfect of the auxiliary verb \textit{vaj}, the participle of \textit{vaj}, and the participle of the main verb and need not to be listed. Examples will be given in § 4.1.2.2.6 below.

\begin{table}
	\caption{Auxiliary verb \textit{èssar} `be'}
	\label{tab:aux:èssar}
	\begin{tabular}{llllll}
		\lsptoprule
		& \textsc{inf}  & \textsc{ptcp.m}  & \textsc{ptcp.f}  &  \textsc{ger}\\
		\midrule
		&\textit{èssar} &\textit{stauṣ}, \textit{staj}  & \textit{stada}, \textit{stadaṣ} & \textit{èss\underline{è}n}\\
		\lsptoprule
	\textsc{sbj.pron} 	&\textsc{prs.ind}  &\textsc{impf.ind} & \textsc{prf.ind} & \textsc{fut}\\
		\midrule
		\textsc{ju} &\textit{sùn} & \textit{èra} &\textit{sùn stauṣ/stada} &\textit{végn ád èssar}\\
		\textsc{té} &\textit{ajṣ} &\textit{èraṣ} &\textit{ajs stauṣ/stada} & \textit{végnaṣ ád èssar}\\
		\textsc{èl, èla, i, inṣ} &\textit{è} & \textit{èra} & \textit{è stauṣ/stada} &\textit{végn ád èssar}\\
		\textsc{nuṣ} &\textit{èssan} &\textit{èran} & \textit{èssan staj/stadaṣ} &\textit{vagnín ád èssar}\\
		\textsc{vuṣ} &\textit{èssaṣ} & \textit{èraṣ} & \textit{èssas staj/stadaṣ} &\textit{vagníṣ ád èssar}\\
		\textsc{èls, èlas, i}& \textit{èn} & \textit{èran} & \textit{èn staj/stadaṣ} & \textit{végnan ád èssar}\\
		\lsptoprule
		\textsc{sbj.pron}    &\textsc{prs.sbjv} & \textsc{impf.subj.}  &\textsc{cond.dir.} & \textsc{cond.ind.}\\
		\midrule
		\textsc{ju} & \textit{séjgi/ségi} & \textit{èri} & \textit{fùṣ} & \textit{fùssi}\\
		\textsc{té} & \textit{s\underline{é}jgiaṣ/s\underline{é}gias} & \textit{\underline{è}riaṣ} & \textit{fùssaṣ} & \textit{f\underline{ù}ssiaṣ}\\
		\textsc{èl, èla, i, inṣ} & \textit{séjgi/ségi} & \textit{èri} & \textit{fùṣ} & \textit{fùssi}\\
		\textsc{nuṣ} & \textit{s\underline{é}jgian, s\underline{é}gian} & \textit{\underline{è}rian} & \textit{fùssan} & \textit{f\underline{ù}ssian}\\
		\textsc{vuṣ} &  \textit{s\underline{é}jgiaṣ, s\underline{é}giaṣ} &  \textit{\underline{è}riaṣ} & \textit{fùssaṣ} & \textit{f\underline{ù}ssiaṣ}\\
		\textsc{èls, èlas, i} & \textit{s\underline{é}jgian, s\underline{é}gian} & \textit{\underline{è}rian} & \textit{fùssan} & \textit{f\underline{ù}ssian}\\
		\lspbottomrule
	\end{tabular}
\end{table}

In the third person singular and plural present and imperfect, the verb \textit{èssar} has a special form when there is subject inversion (which includes polar interrogatives): \textit{ṣè} or \textit{ásaj} and \textit{ṣèn}, as well as \textit{ṣèra} and \textit{ṣèran}. These forms go back to standard Sursilvan, where \textit{igl ei} `it is' becomes \textit{eiṣ ei} `is it' with subject inversion. In contrast to standard Sursilvan, the form of the copula does not include an expletive pronoun in the following examples.

\ea
\label{}
\langinfo{Tuatschín}{Ruèras}{m2, l. 215f.}\\
\gll    Òz \textbf{ṣè} quaj ah, òz \textbf{ṣèni} schòn autar, òz \textbf{ṣèn} \textbf{aj} ... la stradún. \\
today \textsc{cop.prs.3sg} \textsc{dem.unm} eh today \textsc{cop.prs.3pl.3pl} in\_fact different today \textsc{cop.prs.3pl} \textsc{3pl} {} \textsc{def.art.f.sg} street.\textsc{m.sg.augm} \\
\glt `Nowadays this is, eh, as a matter of fact they are different, nowadays they are [called] ... the «big street».'
\z

\ea
\label{}
\langinfo{Tuatschín}{Camischùlas}{f6, l. 734}\\
\gll    A Cazis \textbf{ṣèra} \textbf{quaj} al madèm.\\
in \textsc{pln} \textsc{cop.impf.3sg} \textsc{dem.unm} \textsc{def.art.m.sg} same\\
\glt `In Cazas this was the same thing.'
\z

\ea
\label{}
\langinfo{Tuatschín}{Sadrún}{m5}\\
\gll \textbf{Ṣè} quaj usché?\\
\textsc{cop.prs.3sg} \textsc{dem.unm} so\\
\glt `Is this so?'
\z

Children and very occasionally also older persons generalize this form and use it without subject inversion.

\ea
\label{}
\langinfo{Tuatschín}{Sadrún}{m8}\\
\gll \textbf{I} \textbf{ṣè} vit.\\
\textsc{expl} \textsc{cop.prs.3sg} empty.\textsc{adj.unm}\\
\glt `It is empty.'
\z

\ea
\label{}
\langinfo{Tuatschín}{Sadrún}{m5, l. 1275}\\
\gll [...] \textbf{i} \textbf{ṣèra} òns nùca tg’ èra awa [...].\\
{} \textsc{expl} \textsc{exist.impf.3sg} year.\textsc{m.pl} where \textsc{rel} \textsc{exist.impf.3sg} water \\
\glt `[...] there were years with rain [...].'
\z

\ea
\label{}
\langinfo{Tuatschín}{Sadrún}{m9, l. 1833}\\
\gll [...] anadas \textbf{tga} \textbf{ṣèn} uschéa [...].\\
{} age\_group.\textsc{f.pl} \textsc{rel} \textsc{cop.prs.3pl} so\\
\glt `[...] age groups which are like that [...].'
\z

In (\ref{ex:copexpl}) the form of the copula \textit{ṣè} does include an expletive pronoun as in the standard Sursilvan form noted above. 

\ea
\label{ex:copexpl}
\langinfo{Tuatschín}{Ruèras}{f7, l. 1712f.}\\
	\gll Basta, ju sùn id’ ál trèn, tòcan gjù Sògn Gagl \textbf{ṣè} bigja da fá bjè falju [...].   \\
enough \textsc{1sg} be.\textsc{prs.1sg} go.\textsc{ptcp.f.sg} to.\textsc{def.art.m.sg} train until down \textsc{pln} {} \textsc{cop.prs.3sg.expl} \textsc{neg} \textsc{comp} make.\textsc{inf} much wrong.\textsc{adj.unm}\\
\glt `Enough. I went to the train, to St. Gallen there is not much you could do wrong [...].'
\z

\begin{table}
\caption{Auxiliary verb \textit{vaj} `have'}
\label{tab:aux:vaj}
 \begin{tabular}{llllll}
 
  \lsptoprule
& \textsc{inf}  & \textsc{ptcp.m.unm} \\
  \midrule
&  \textit{vaj} &\textit{gju} \\
     
  \lsptoprule
\textsc{sbj.pron}  &\textsc{prs.ind}  &\textsc{impf.ind} & \textsc{prf.ind} & \textsc{fut}\\
   \midrule
\textsc{ju} &\textit{a} & \textit{vèva} & \textit{a gju} & \textit{végn á vaj}\\
\textsc{té} &\textit{aṣ} & \textit{vèvaṣ} & \textit{aṣ gju} & \textit{végnaṣ á vaj}\\
\textsc{èl, èla, i, ins} &\textit{ò} & \textit{vèva} & \textit{ò gju} &\textit{végn á vaj}\\
\textsc{nuṣ} &\textit{vajn} & \textit{vèvan} & \textit{vajn gju} &\textit{vagnín á vaj}\\
\textsc{vuṣ} &\textit{vajṣ} & \textit{vèvaṣ} & \textit{vajṣ gju} &\textit{vagníṣ á vaj}\\
\textsc{èls, èlas, i}& \textit{òn} & \textit{vèvan} & \textit{òn gju} &\textit{végnan á vaj}\\

 \lsptoprule
\textsc{sbj.pron}  &\textsc{prs.sbjv} & \textsc{impf.sbjv}  &\textsc{cond.dir} & \textsc{cond.ind} & \textsc{imp} \\
\midrule
\textsc{ju} & \textit{vagi}& \textit{vèvi} & \textit{vèṣ} & \textit{vèssi}\\
\textsc{té} & \textit{vájaṣ}& \textit{vèviaṣ} & \textit{vèssaṣ} & \textit{v\underline{è}ssiaṣ} &  \textit{vajaṣ}\\
\textsc{èl, èla, i, inṣ} & \textit{vagi} & \textit{vèvi} & \textit{vèṣ} & \textit{vèssi}\\
\textsc{nuṣ} & \textit{vájan}& \textit{vèvian} & \textit{vèssan} & \textit{v\underline{è}ssian}\\
\textsc{vuṣ} & \textit{vájaṣ}& \textit{vèviaṣ} & \textit{vèssaṣ}& \textit{vèssiaṣ} &  \textit{vajaṣ}\\
\textsc{èls, èlas, i} & \textit{vájan}& \textit{vèvian} & \textit{vèssan} & \textit{v\underline{è}ssian}\\
  \lspbottomrule
 \end{tabular}
\end{table}

\begin{table}
\caption{Auxiliary verb \textit{vagní} `come'}
\label{tab:aux:vagní}
 \begin{tabular}{llllll} 
  \lsptoprule
& \textsc{inf}  & \textsc{ptcp.m}  & \textsc{ptcp.f}\\
  \midrule
&\textit{vagní} &\textit{vagnúṣ}, \textit{vagní} & \textit{vagnida}, \textit{vagnidaṣ}\\
   
  \lsptoprule
\textsc{sbj.pron} & \textsc{prs.ind}  &\textsc{impf.ind} & \textsc{prf.ind} & \textsc{fut}\\
   \midrule
\textsc{ju} & \textit{végn} & \textit{vagnéva} &\textit{sùn vagnúṣ/vagnida} &\textit{végn á vagní}\\
\textsc{té} &\textit{végnaṣ} &\textit{vagnévaṣ} & \textit{ajṣ vagnúṣ/vagnida} & \textit{végnaṣ á vagní}\\
\textsc{èl, èla, i, inṣ} &\textit{végn} & \textit{vagnéva} &\textit{è vagnúṣ/vagnida} &\textit{végn á vagní}\\
\textsc{nuṣ} &\textit{vagnín} &\textit{vagnévan} &\textit{èssan vagní/vagnidaṣ} &\textit{vagnín á vagní}\\
\textsc{vuṣ} &\textit{vagníṣ} & \textit{vagnévaṣ} &\textit{èssaṣ vagní/vagnidaṣ} &\textit{vagníṣ á vagní}\\
\textsc{èls, èlas, i}& \textit{végnan} & \textit{vagnévan} &\textit{èn vagní/vagnidaṣ} &\textit{végnan á vagní}\\

 \lsptoprule
\textsc{sbj.pron} &\textsc{prs.sbjv} & \textsc{impf.sbjv}  &\textsc{cond.dir} & \textsc{cond.ind} & \textsc{imp}\\
\midrule
\textsc{ju} & \textit{végni}& \textit{vagnévi} & \textit{vagnéṣ}& \textit{vagnéssi}\\
\textsc{té} & \textit{v\underline{é}gniaṣ} & \textit{vagn\underline{é}viaṣ} & \textit{vagnéssaṣ} & \textit{vagn\underline{é}ssiaṣ} & \textit{nò}\\
\textsc{èl, èla, i, inṣ} & \textit{végni} & \textit{vagnévi} & \textit{vagnéṣ} & \textit{vagnéssi}\\
\textsc{nuṣ} & \textit{vagnian} & \textit{vagn\underline{é}vian} & \textit{vagnéssan} & \textit{vagn\underline{é}ssian}\\
\textsc{vuṣ} & \textit{vagnias}& \textit{vagn\underline{é}viaṣ} & \textit{vagnéssaṣ} & \textit{vagn\underline{é}ssiaṣ} & \textit{vagní}\\
\textsc{èls, èlas, i} & \textit{vagnian} & \textit{vagn\underline{é}vian}& \textit{vagnéssan} & \textit{vagn\underline{é}ssian}\\
  \lspbottomrule
 \end{tabular}
\end{table}


\subsubsection{Regular verbs}
Future tense is always built with the auxiliary verb \textit{vagní} `come', but it is not used in normal daily speech, where almost exclusively present tense is used for future reference. Therefore future tense will not be listed with the regular verbs. The same holds for the gerund, which was used by traditional story tellers until some decades ago, but which is not in use any more.\footnote{For examples, see § 4.1.2.1.2 below.}

As mentioned above, the \textit{è-}conjugation has split from the original \textit{á}-conjuga-tion (< Latin \textsc{-are})  because of the presence of a preceding palatal consonant. \tabref{tab:èconj} lists some examples of \textit{è}-verbs with their Sursilvan counterparts. Note that the final \textit{-r} of the infinitives in Standard Sursilvan orthography is not pronounced in any Sursilvan variety.

\begin{table}
\caption{Tuatschin verbs ending in \textit{è-} with their standard Sursilvan equivalents}
\label{tab:èconj}
 \begin{tabular}{llll}
 \lsptoprule
&\textsc{Tuatschín}  & \textsc{Sursilvan}  & \textsc{English} \\
  \midrule
 ʎ & \textit{magliè} &\textit{magliar}& 'eat' \\
ʥ&\textit{cargè}&\textit{cargar}&'carry'\\
ʨ&\textit{spatgè}&\textit{spitgar}&'wait'\\
ʧ&\textit{catschè}&\textit{catschar}&'hunt'\\
ʃ&\textit{schè}&\textit{schar}&'let'\\
j&\textit{sijè}&\textit{segar}&'mow'\\   
 \lspbottomrule
 \end{tabular}
\end{table}

\tabref{conja} and \tabref{conjè} illustrate the two conjugations deriving from the Latin first conjugation. The difference between the two conjgugations concerns above all the infinitive and the feminine past participle, which ends in  \textit{-èda} (in contrast to the masculine form which ends in \textit{-ada}). The imperfect morpheme of the \textit{è}-conjugation is normally \textit{-áva(-)}, but some verbs have \textit{-èva(-)}, as for example \textit{astgè} `be allowed', as in \textit{nuṣ \textbf{astgèvan} fá} `we were allowed to do' (§ 9.4, line 705) or \textit{schè} `let', as in \textit{a \textbf{schèvan} dá quèlas scúaṣ a da quaj} `and [we] would let these brooms and so on fall down' (§ 9.6, line 937).



\begin{table}
	\caption{Regular verbs ending in \textit{-á}}
	\label{conja}
	\begin{tabularx}{.7\textwidth}{llll}
		
		\lsptoprule
		\textsc{inf} & & \textsc{ptcp.m}  & \textsc{ptcp.f}\\
		\midrule
		\textit{gidá} & `help' & \textit{gidau}, \textit{gidauṣ} & \textit{gidada}, \textit{gidadaṣ}\\
		\lspbottomrule  
	\end{tabularx}
	
	\medskip
	
	\begin{tabularx}{\textwidth}{p{2cm}lllll}
		\lsptoprule
		\textsc{sbj.pron} &\textsc{prs.ind} &\textsc{prs.subj} &\textsc{impf.ind} & \textsc{impf.subj} &\textsc{prf.ind}\\
		\midrule
		\textsc{ju} & \textit{gida} & \textit{gidi} & \textit{gidava} & \textit{gidavi} & \textit{a gidau}  \\
		\textsc{té} & \textit{gidaṣ} & \textit{gídiaṣ} & \textit{gidavaṣ} & \textit{gidáviaṣ} & \textit{aṣ gidau}\\
		\textsc{èl, èla, i, inṣ} & \textit{gida} & \textit{gidi} & \textit{gidava} & \textit{gidavi}  & \textit{ò gidau} \\
		\textsc{nuṣ} & \textit{gidajn} & \textit{gídian} & \textit{gidavan} & \textit{gidávian} & \textit{vajn gidau} \\
		\textsc{vuṣ} & \textit{gidajṣ} & \textit{gídiaṣ} & \textit{gidavaṣ}  & \textit{gidáviaṣ} & \textit{vajṣ gidau} \\
		\textsc{èls, èlas, i} & \textit{gidan}  & \textit{gídian} & \textit{gidavan} & \textit{gidávian} & \textit{òn glidau}\\
		\lspbottomrule
	\end{tabularx}
	
	\medskip
	
	\begin{tabularx} {\textwidth}{p{2cm}XXXX}
		\lsptoprule
		\textsc{sbj.pron} &\textsc{dir.cond} &  \textsc{indir.cond} & \textsc{imp}\\
		\midrule
		\textsc{ju} & \textit{gidáṣ} & \textit{gidassi}\\
		\textsc{té} & \textit{gidassaṣ} & \textit{gidássiaṣ} & \textit{gida}\\
		\textsc{èl, èla, i, inṣ} & \textit{gidáṣ} &  \textit{gidassi}\\
		\textsc{nuṣ} & \textit{gidassan} & \textit{gidássian}\\
		\textsc{vuṣ} & \textit{gidassaṣ} & \textit{gidássiaṣ}  & \textit{gidaj}\\
		\textsc{èls, èlas, i} & \textit{gidassan}  & \textit{gidássian} \\
		\lspbottomrule
	\end{tabularx} 
\end{table}



\begin{table}
\caption{Regular verbs ending in \textit{-è}}
\label{conjè}
 \begin{tabularx}{.8\textwidth}{llll}
 
  \lsptoprule
  \textsc{inf}  && \textsc{ptcp.m}  & \textsc{ptcp.f}\\
  \midrule
  \textit{\textbf{magljè}} & `eat' & \textit{magljau}, \textit{magljauṣ} & \textit{\textbf{magljèda}, \textit{magljèdaṣ}}\\
  \lspbottomrule  
  \end{tabularx}
  
  \medskip
  
 \begin{tabularx}{\textwidth}{p{2cm}lllll}
  \lsptoprule
\textsc{sbj.pron} &\textsc{prs.ind} &\textsc{prs.sbjv} &\textsc{impf.ind} & \textsc{impf.sbjv} &\textsc{prf.ind}\\
 \midrule
\textsc{ju} &\textit{maglja} & \textit{magli}&\textit{magljava} &\textit{magljavi} & \textit{a magljau}\\
\textsc{té} &\textit{magljaṣ} & \textit{magljaṣ}&\textit{magljavaṣ} &\textit{magljáviaṣ} &\textit{as magljau}\\
\textsc{èl, èla, i, ins}  &\textit{maglja} &\textit{magli} & \textit{magljava} &\textit{magljavi} &\textit{ò magljau}\\
\textsc{nus} &\textit{magljáin} & \textit{magljan} &\textit{magljavan} &\textit{magljávian} &\textit{vajn magljau}\\
\textsc{vus} &\textit{magljáiṣ} &\textit{magljaṣ} & \textit{magljavaṣ} &\textit{magljáviaṣ} &\textit{vajs magljau}\\
\textsc{èls, èlas, i} & \textit{magljan} &\textit{magljan} &\textit{magljavan} &\textit{magljávian} &\textit{òn magljau}\\
  \lspbottomrule
\end{tabularx}

\medskip

\begin{tabularx} {\textwidth}{p{2cm}llll}
 \lsptoprule
  \textsc{sbj.pron} &\textsc{cond.dir} &  \textsc{cond.ind}  &\textsc{imp}\\
\midrule
\textsc{ju} & \textit{magljáṣ}&\textit{magljassi} &\\
\textsc{té}  & \textit{magljassaṣ}&\textit{magljássiaṣ} &\textit{maglja}\\
\textsc{èl, èla, i, ins} & \textit{magljáṣ} &\textit{magljassi} &\\
\textsc{nus} & \textit{magljassan} & \textit{magljássian} &\\
\textsc{vus} & \textit{magljassaṣ} & \textit{magljássiaṣ} &\textit{magljaj}\\
\textsc{èls, èlas, i} & \textit{magljassan} & \textit{magljássian}&\\
  \lspbottomrule
 \end{tabularx} 
\end{table}

The ending in -\textit{a} of the first person singular present and imperfect indicative is typical of Tuatschin Sursilvan. The standard ending in Sursilvan is -\textit{el} (\textit{jeu giavisch-el} `I wish'); in the DRG, however, I found one example of \textit{-a} in the variety of Riein, a village situated in the Lumnezia valley (\ref{rieina}).

\ea\label{rieina}
\langinfo{Sursilvan}{Riein}{\DRG{4}{327}}\\
\gll  Jeu giavisch-\textbf{a} bien cunfiert.  \\
     \textsc{1sg} wish-\textsc{prs.1sg} good consolation\\
\glt `My heartfelt sympathy.'
\z

This form was current in this local variety of Sursilvan for 1st person singular present and imperfect, as the following forms show: \textit{jeu astga} `I am allowed to', \textit{jeu gneva} `I used to come', \textit{jeu era} `I was', \textit{jeu suna} `I play (an instrument)', \textit{jeu sunava} `I used to play', and so on (examples taken out of the\textit{ Questiunari principal} of the DRG, recorded between 1900 and 1920; Ursin Lutz p.c., 2017/04/19).


\tabref{conjar1} lists the verbs ending in \textit{-ˈar}. Many verbs of this category are built as in \tabref{conjar1} but have one irregular form: the past participle. Some examples are \textit{árdar} `burn', \textit{árvar} `open', and \textit{bétar} `throw', whose participles are \textit{ars/arsa}, \textit{aviart/avjarta}, and \textit{béz/béza}. A list of irregular verbs ending in \textit{ˈar} is given in \tabref{stemaltvar} below.


\begin{table}
	\caption{Regular verbs ending in \textit{-ˈar}}
	\label{conjar1}
	\begin{tabularx}{.7\textwidth}{p{1,7cm}llll}
		
		\lsptoprule
		\textsc{inf} & & \textsc{ptcp.m}  & \textsc{ptcp.f}\\
		\midrule
		\textit{bátar} & `beat' & \textit{batju}, \textit{batjuṣ} & \textit{batida}, \textit{batidaṣ}\\
		\lspbottomrule  
	\end{tabularx}
	
	\medskip
	
	\begin{tabularx}{\textwidth}{p{1,7cm}llllll}
		\lsptoprule
		\textsc{sbj.pron} &\textsc{prs.ind} &\textsc{prs.subj} &\textsc{impf.ind} & \textsc{impf.subj} &\textsc{prf.ind}\\
		\midrule
		\textsc{ju} & \textit{bata} & \textit{bati} & \textit{batéva} & \textit{batévi} &  \textit{a batju}\\
		\textsc{té} & \textit{bataṣ} & \textit{bátiaṣ} & \textit{batévaṣ} & \textit{bat\underline{é}viaṣ} & \textit{aṣ batju}\\
		\textsc{èl, èla, i, inṣ} & \textit{bata} & \textit{bati} & \textit{batéva}  & \textit{batévi} & \textit{ò batju}\\
		\textsc{nuṣ} & \textit{batín} & \textit{bátian} & \textit{batévan} & \textit{bat\underline{é}vian} & \textit{vajn batju}\\
		\textsc{vuṣ} & \textit{batíṣ} & \textit{bátiaṣ} &  \textit{batévaṣ} & \textit{bat\underline{é}viaṣ} & \textit{vajṣ batju} \\
		\textsc{èls, èlas, i} & \textit{batan} & \textit{bátian} & \textit{batévan} & \textit{bat\underline{é}vian} & \textit{òn batju} \\
		\lspbottomrule
	\end{tabularx}
	
	\medskip
	
	\begin{tabularx} {\textwidth}{p{2cm}XXXX}
		\lsptoprule
		\textsc{sbj.pron} &\textsc{dir.cond} &  \textsc{indir.cond}    &\textsc{imp}\\
		\midrule
		\textsc{ju} & \textit{batéṣ} & \textit{batéssi} \\
		\textsc{té} & \textit{batéssaṣ} & \textit{bat\underline{é}ssiaṣ} & \textit{bata}\\
		\textsc{èl, èla, i, inṣ} & \textit{batéṣ} & \textit{batéssi}\\
		\textsc{nuṣ} & \textit{batéssan} & \textit{bat\underline{é}ssian} \\
		\textsc{vuṣ} & \textit{batéssaṣ} & \textit{bat\underline{é}ssiaṣ} & \textit{batí} \\
		\textsc{èls, èlas, i} & \textit{batéssan} & \textit{bat\underline{é}ssian}\\
		\lspbottomrule
	\end{tabularx} 
\end{table}

As mentioned above, verbs ending in \textit{-í} which are conjugated regularly always have the infix \textit{-èsch-}.

\begin{table}
	\caption{Regular verbs ending in \textit{-í} with the infix \textit{-èsch-}}
	\label{tab:reg.verb-i}
	\begin{tabularx}{.7\textwidth}{llll}
		
		\lsptoprule
		\textsc{inf} & & \textsc{ptcp.m}  & \textsc{ptcp.f}  \\
		\midrule
		\textit{finir} & `finish' & \textit{finju}, \textit{finjus} & \textit{finida}, \textit{finidas} \\
		\lspbottomrule  
	\end{tabularx}
	
	\medskip
	
	\begin{tabularx}{\textwidth}{p{1,7cm}lllll}
		\lsptoprule
		\textsc{sbj.pron} &\textsc{prs.ind} &\textsc{prs.subj} &\textsc{impf.ind} & \textsc{impf.subj} &\textsc{prf.ind}\\
		\midrule
		\textsc{ju} & \textit{finèscha} & \textit{finèschi} & \textit{finéva} & \textit{finévi} & \textit{a finju} \\
		\textsc{té} & \textit{finèschaṣ} & \textit{fin\underline{è}schiaṣ} & \textit{finévaṣ} & \textit{finévi} & \textit{as finju}\\
		\textsc{èl, èla, i, ins} & \textit{finèscha}  & \textit{finèschi} & \textit{finéva} & \textit{finévi} & \textit{ò finju}\\
		\textsc{nus} & \textit{finín} & \textit{finían} & \textit{finévan} & \textit{fin\underline{é}vian} &\textit{vajn finju}\\
		\textsc{vus} & \textit{finíṣ} & \textit{finíaṣ} & \textit{finévaṣ} & \textit{fin\underline{é}viaṣ} & \textit{vajs finju} \\
		\textsc{èls, èlas, i} & \textit{finèschan}  & \textit{fin\underline{è}schian} & \textit{finévan} & \textit{fin\underline{é}vian} & \textit{òn finju}\\
		\lspbottomrule
	\end{tabularx}
	
	\medskip
	
	\begin{tabularx} {\textwidth}{p{2cm}XXXX}
		\lsptoprule
		\textsc{sbj.pron} &\textsc{dir.cond} &  \textsc{indir.cond} &\textsc{imp}\\
		\midrule
		\textsc{ju} & \textit{finéṣ} & \textit{finéssi} \\
		\textsc{té} & \textit{finéssaṣ} &\textit{fin\underline{é}ssiaṣ}  &  \textit{finèscha}\\
		\textsc{èl, èla, i, ins} & \textit{finéṣ}  & \textit{finéssi}\\
		\textsc{nus} & \textit{finéssan} &  \textit{fin\underline{é}ssian}\\
		\textsc{vus} &  \textit{finéssaṣ} & \textit{fin\underline{é}ssiaṣ} & \textit{finí}\\
		\textsc{èls, èlas, i} & \textit{finéssan} & \textit{fin\underline{é}ssian}\\
		\lspbottomrule
	\end{tabularx} 
\end{table}



\subsubsection{Irregular verbs and verbs with regular stem alterations}
Since there are many irregular verbs and verbs with stem alterations, they will all be presented in Appendix I and II below. 

\subsection{Usage of nonfinite and finite verbal categories}
In this section the usage of the non-finite categories past participle, infinitive, and gerund as well as the finite categories will be analysed, with the exception of the imperative which will be treated in § 5.3 below.

\subsubsection{Nonfinite categories}

\paragraph{Past participle}


The past participle is used to form compound (\ref{ex:ptcp1}) and doubly-compound tenses (\ref{ex:dcomp:3}) as well as passive voice (\ref{ex:ptcp2}); it is furthermore used attributively and predicatively and may also be nominalized (\ref{ex:ptcp3} and \ref{ex:ptcp4}), usually in its feminine form. If the auxiliary verb is \textit{èssar} `be', the participle agrees with the subject (\ref{ex:ptcp1}).

\ea
\label{ex:ptcp1}
\langinfo{Tuatschín}{Sadrún}{m6, l. 1324f.}\\
\gll    [...] ju \textbf{sùn} \textbf{stauṣ} al davùs purtgè … da Sadrún [...].\\
{}  \textsc{1sg} be.\textsc{prs.1sg} \textsc{cop.ptcp.m.sg} \textsc{def.art.m.sg} last swineherd {} of \textsc{pln} \\
\glt `[...] I was the last swineherd … of Sedrun [...].'
\z

\ea\label{ex:ptcp2}
\langinfo{Tuatschín}{Sadrún}{f3, l. 98f.}\\
\gll  A … tgu \textbf{sùn} \textbf{vagnida} \textbf{panṣjunada} scha … ju fagèva zuar schòn avaun majnadistríct [...].\\
and {} when.\textsc{1sg} be.\textsc{prs.1sg} \textsc{pass.aux.ptcp.f.sg} pension\_off.\textsc{ptcp.f.sg} \textsc{corr} {} \textsc{1sg} do.\textsc{impf.1sg} although already before head\_of\_district.\textsc{m.sg}\\
\glt `And … when I got pensioned off … as a matter of fact, I had already worked as head of district before [...].'
\z

\ea\label{ex:ptcp3}
\langinfo{Tuatschín}{Sadrún}{m4, l. 512ff.}\\
\gll [...] gljèz èra magari è léjgar  tgé \textbf{cuṣchin-} \textbf{adaṣ} èl [fagèva] [...]s.\\
{} \textsc{dem.unm} \textsc{cop.impf.3sg} sometimes also funny.\textsc{adj.unm} what cook.\textsc{ptcp-} \textsc{f.pl} \textsc{3sg} [make.\textsc{impf.3sg}] \\
\glt `[...] it was sometimes also funny [to see] what he [cooked] [...].'
\z

\ea\label{ex:ptcp4}
\langinfo{Tuatschín}{Sadrún}{m6, l. 1464ff.}\\	
\gll [...] i dat ina fòtògrafia tgu sùn sé cun mju còl\underline{è}ga al dé da \textbf{la} \textbf{scarg-} \textbf{èda} [...].\\
{} \textsc{expl} \textsc{exist.prs.3sg}  \textsc{indef.art.f.sg} photograph \textsc{rel.1sg}  \textsc{cop.prs.1sg} on with  \textsc{poss.1sg.m.sg} mate \textsc{def.art.m.sg} day of  \textsc{def.art.f.sg}  drove.\textsc{ptcp-} \textsc{f.sg}\\
\glt `[...] there is a photograph in which I am with my mate the day of the pig driving [...].'
\z

\ea
\label{ex:dcomp:3}
\langinfo{Tuatschín}{Sadrún}{m9, l. 1808}\\
\gll Nuṣ \textbf{vajn} adina \textbf{gju} \textbf{fatg} parada.   \\
\textsc{1pl} have.\textsc{prs.1pl} always have.\textsc{ptcp.unm} make.\textsc{ptcp.unm} parade.\textsc{f.sg}\\
\glt `We always held a parade.'
\z

If the past participle is used predicatively, which is usually the case in passive voice, it is treated like an adjective, which means that (a) it agrees with the subject of the verb (\ref{ex:ptcp:agr1}), thus if the subject has no gender, the participle  takes its unmarked form (\ref{ex:ptcp:agr2}), and (b) if in a passive construction the subject follows the participle, it does not agree with it (\ref{ex:ptcp:agr3} and \ref{ex:ptcp:agr5}) (see also § 5.5.4 below).

í
\ea\label{ex:ptcp:agr1}
\langinfo{Tuatschin}{Sadrún}{f3; l. 107ff.}\\
\gll [...] al cantún ò circa trènta da quèls majnadistrícts, \textbf{quèls} èn \textbf{partí} \textbf{ajn} ajn ragjúns [...].  \\
{} \textsc{def.art.m.sg} canton  have.\textsc{prs.3sg} about thirty of \textsc{dem.m.pl} head\_of\_district.\textsc{pl} \textsc{dem.m.pl} \textsc{pass.aux.prs.3pl} divide.\textsc{ptcp.m.pl} in in region.\textsc{f.pl}\\
\glt `[...] the canton has about thirty of these heads of district, these are divided in regions [...].'
\z

\ea\label{ex:ptcp:agr2}
\langinfo{Tuatschin}{Zarcúns}{m2; l. 1637ff.}\\
\gll    Òh \textbf{gl’} \textbf{ampréndar} \textbf{tudèstg} è \textbf{stau}, l’ antschata ṣè \textbf{quaj} schòn \textbf{stau} in téc curj\underline{ù}s.\\
oh \textsc{def.art.m.sg} learn.\textsc{inf} German.\textsc{m.sg} be.\textsc{prs.3sg}  \textsc{cop.ptcp.unm} \textsc{def.art.f.sg} beginning be.\textsc{prs.3sg} \textsc{dem.unm} indeed \textsc{cop.ptcp.unm} \textsc{indef.art.m.sg} bit strange.\textsc{adj.unm}\\
\glt `Oh, to learn German was, at the beginning this was indeed a little bit strange.'
\z

\ea\label{ex:ptcp:agr3}
\langinfo{Tuatschin}{Sadrún}{f3; l. 56ff.}\\
\gll  Quaj è pròpi in ljuc ... nù tg’ \textbf{i} \textbf{vagnéva} \textbf{schau} tùt \textbf{la} \textbf{munizjun} tg’ i vèva, sigir.\\
\textsc{dem.unm} \textsc{cop.prs.3sg} exactly \textsc{indef.art.m.sg} place {} where \textsc{rel} \textsc{expl} \textsc{pass.aux.impf.3sg} leave.\textsc{ptcp.unm} all \textsc{def.art.f.sg} munition \textsc{rel} \textsc{expl} \textsc{exist.impf.3sg} sure.\textsc{adj.unm}\\
\glt `This is exactly a place ... where they stored all the munition, for sure.'
\z

\ea\label{ex:ptcp:agr5}
\langinfo{Tuatschín}{Sadrún}{m5, l. 1519f.}\\
\gll    A lò … sén quaj intènt ségi è \textbf{vagnú} \textbf{bagagjau} \textbf{quèla} \textbf{caplùta}.\\
and there {} upon \textsc{dem.m.sg} undertaking be.\textsc{prs.sbjv.3sg} also \textsc{pass.aux.ptcp.unm} build.\textsc{ptcp.unm} \textsc{dem.f.sg} chapel \\
\glt `And there … after this undertaking this chapel was built.'
\z


If the past participle is used attributively, the masculine singular form does not take the predicative \textit{-s} if it has no complements as in \textit{in ùm panṣjunau} `a retired man'; if it has complements, the participle is treated like a predicative adjective and takes \textit{-s} and can thus be considered an elliptic relative clause (\ref{ex:ptcp:pred2}).

\ea\label{ex:ptcp:pred2}
\langinfo{Tuatschín}{Camischùlas}{\DRG{3}{583}}\\
\gll  Al \textbf{tétg} da duaṣ alas \textbf{fatg-} \textbf{s} \textbf{cun} \textbf{ajssas} \textbf{bétga} \textbf{splanadas} […] aj sén latas.\\
\textsc{def.art.m.sg} roof of two.\textsc{f.pl} side.\textsc{pl} make.\textsc{ptcp-} \textsc{m.sg} with plank.\textsc{f.pl} \textsc{neg}  plane.\textsc{ptcp.f.pl} {} \textsc{cop.prs.3sg} on slat.\textsc{f.pl}\\
\glt `The two-sided roof made of planks that haven't been planed […] are on slats.'
\z

The negator \textit{bétga} and some temporal adverbs may intervene between the auxiliary verb and the past participle.

\ea\label{}
\langinfo{Tuatschín}{Sadrún}{m8, l. 1508}\\
\gll  Álṣò ju a \textbf{bigja} fatg aj agrèssíf [...].  \\
well \textsc{1sg} have.\textsc{prs.1sg} \textsc{neg} make.\textsc{ptcp.unm} \textsc{3sg} aggressive.\textsc{adj.unm}\\
\glt `Well, I didn’t do it in an aggressive way [...].'
\z

\ea\label{}
\langinfo{Tuatschín}{Ruèras}{m1, l. 290ff.}\\
\gll    A lu sjantar vau \textbf{adina} fatg al pur, ábar ju sùn \textbf{ùṣ} bigja staus … in dals fétg buns purs.\\
and then after have.\textsc{prs.1sg.1sg} always do.\textsc{ptcp.unm} \textsc{def.art.m.sg} farmer but  \textsc{1sg} be.\textsc{prs.1sg} now \textsc{neg} \textsc{cop.ptcp.m.sg} {} one.\textsc{m.sg} of.\textsc{def.art.m.sg} very good.\textsc{m.pl} farmer.\textsc{pl}\\
\glt `And after this I have always worked as a farmer, but I’ve never been … one of the very good farmers.'
\z

\ea
\label{}
\langinfo{Tuatschín}{Sadrún}{m5}\\
\gll Èl è \textbf{grad} arivaus sé da Cuéra.\\
\textsc{3sg.m} be.\textsc{prs.3sg} just arrive.\textsc{ptcp.m.sg} up from \textsc{pln}\\
\glt `He has just arrived from Chur.'
\z

\ea\label{}
\langinfo{Tuatschín}{}{\DRG{4}{304}}\\
\gll Ùssa, quèl da la quajda ṣaj \textbf{puspè} stauṣ ajn cul dét, miraj tschò!\\
   now \textsc{dem.m.sg} of \textsc{def.art.f.sg} desire be.\textsc{prs.3sg} again \textsc{cop.ptcp.m.sg} in with finger.\textsc{m.sg} look.\textsc{imp.2pl} here\\
\glt `Now the sweet-toothed has again stuck his finger into it, look here!'
\z

\ea
\label{}
\langinfo{Tuatschín}{Sadrún}{m6, 1. 1318f.}\\
\gll    Avaun nus èra sagir al tgavrè \textbf{era} \textbf{schòn} jus culas tgauras, lèz mava lu èra.\\
before \textsc{1pl} be.\textsc{impf.3sg} sure \textsc{def.art.m.sg} goatherd also already go.\textsc{ptcp.m.sg} with.\textsc{def.art.f.pl} goat.\textsc{pl} \textsc{dem.m.sg} go.\textsc{impf.3sg} then also\\
\glt `Before us the goatherd had certainly already gone with the goats, he also used to go.'
\z


If two clauses which both contain a verb modified by a compound tense are conjoined, either the subject (\ref{ex:diff:aux1}) or the subject and the auxiliary of the second or third verb may be omitted (\ref{ex:diff:aux3}). Note that the subject and the auxiliary verb may also be omitted if the second or third verb requires another auxiliary as in the first clause. An example is (\ref{ex:diff:aux2}) where \textit{lavá} `get up' requires \textit{èssar} `be' and \textit{mirá} `look' \textit{vaj} `have'.

\ea\label{ex:diff:aux1}
\langinfo{Tuatschín}{Ruèras} {\citealt[69]{Büchli1966}}\\
\gll Ju \textbf{sùn} \textbf{juṣ} avaun nuégl ad \textbf{a} \textbf{grju} a bargju […].\\
     \textsc{1sg}  be.\textsc{prs.1sg}  go.\textsc{ptcp.m.sg} before barn.\textsc{m.sg} and have.\textsc{prs.1sg} shout.\textsc{ptcp.unm} and cry.\textsc{ptcp.unm}\\
\glt `I went in front of the barn and shouted and cried.'
\z

\ea\label{ex:diff:aux3}
\langinfo{Tuatschín}{Sèlva}{\citealt[47]{Büchli1966}}\\
\gll Èl \textbf{ò} \textbf{prju} las duas sadjalas groma ad \textbf{è} \textbf{juṣ} òd tégja a \textbf{ṣvanjus}.\\
\textsc{3sg}  have.\textsc{prs.3sg} take.\textsc{ptcp.unm} \textsc{def.art.f.pl} two.\textsc{pl} bucket.\textsc{pl}  cream.\textsc{f.sg} and be.\textsc{prs.3sg} go.\textsc{ptcp.m.sg} out\_of hut.\textsc{f.sg} and disappear.\textsc{ptcp.m.sg}\\
\glt `He [the devil] took the two buckets full of cream and left the hut and disappeared.'
\z

\ea\label{ex:diff:aux2}
\langinfo{Tuatschín}{Rueras} {\citealt[68]{Büchli1966}}\\
\gll La damaun \textbf{èssan} aun \textbf{lavaj} baud a \textbf{mirau} da nòs tiars.\\
\textsc{def.art.f.sg}  morning be.\textsc{prs.1pl} still get\_up.\textsc{ptcp.m.pl} early and look.\textsc{ptcp.unm} of \textsc{poss.1pl.m.pl} animal.\textsc{pl}\\
\glt `In the morning we got up early and looked after our animals.'
\z

In narrative sequences where the perfect is used for story-line events, the auxiliary verbs may be omitted. In (\ref{ex:omit:1}) \textit{préndar} `take', \textit{magljè} `eat', and métar `put' would take \textit{vaj} `have', in contrast to \textit{turná} `go back' and \textit{ira} `go', which would take \textit{èssar} `be'.

\ea\label{ex:omit:1}
\langinfo{Tuatschín}{Zarcúns}{m2, l. 1648ff.}\\
\gll    A tschèls èran ajn stiva a dèvan tròcas né jass … a nus \textbf{prju} quèls … pinau tiar, quèls puschégns, quèlas … tablas cun sé tgarn a dal tùt … \textbf{prju} quaj ad \textbf{i} gjù ajn in clavau a \textbf{magljau} a sjantar \textbf{turnaj} sé cul cul … cun la vaschala vita a \textbf{méz} lò puspé api \textbf{i}.\\
and \textsc{dem.m.pl} \textsc{cop.impf.3pl} in living\_room.\textsc{f.sg} and give.\textsc{impf.3pl} k.o.card\_game.\textsc{f.pl} or k.o.card\_game.\textsc{m.sg} {} and \textsc{1pl} take.\textsc{ptcp.unm} \textsc{dem.m.pl} {} prepare.\textsc{ptcp.unm} by \textsc{dem.m.pl} snack.\textsc{m.pl}  \textsc{dem.f.pl} {} tray.\textsc{pl} with up meat.\textsc{f.sg} and of.\textsc{def.art.m.sg} all {} take.\textsc{ptcp.unm} \textsc{dem.unm} and go.\textsc{ptcp.m.pl} down in \textsc{indef.art.m.sg} hay\_barn and eat.\textsc{ptcp.unm} and after go\_back.\textsc{ptcp.m.pl} up with.\textsc{def.art.m.sg} with.\textsc{def.art.m.sg} {} with \textsc{def.art.f.sg} dishes.\textsc{f.sg} empty.\textsc{f.sg} and put.\textsc{ptcp.unm} there again and go.\textsc{ptcp.m.pl}\\
\glt `And the others were in the living room and were playing card games … and we took these … prepared, these snacks, these … trays with meat and all on it … we took this and went down into the hay barn and ate it and after we went up back with … with the empty dishes, put them there again and went away.'
\z

\paragraph{Gerund}

As mentioned above, the gerund is not used any more in spoken Tuatschin. There is no occurrence of this category in the oral corpus, but it was used by the traditional story tellers whose legends were published in \citet{Büchli1966}.

The gerund may be used as a complement of a verb of perception and is headed by the complementizer \textit{á/ád}. 

\ea\label{}
\langinfo{Tuatschín}{Surajn} {\citealt[53]{Büchli1966}}\\
\gll    Als pástars udévan adina \textbf{á} \textbf{vagnèn} tiars.\\
   \textsc{def.art.m.pl} herdsman.\textsc{pl} hear.\textsc{impf.3pl} always \textsc{comp} come.\textsc{ger}  animal.\textsc{m.pl}\\
\glt `The herdsmen were always hearing cattle coming […].'
\z

\ea\label{}
\langinfo{Tuatschín}{Sèlva} {\citealt[28]{Büchli1966}}\\
\gll   [...] ina sèra […] ò’ ‘ls pástars vju \textbf{ád} \textbf{èn} las vacas.\\
{} \textsc{indef.art.f.sg} evening {} have.\textsc{prs.3sg} \textsc{def.art.m.pl} herdsman.\textsc{pl} see.\textsc{ptcp.unm} \textsc{comp} go.\textsc{ger} \textsc{def.art.f.pl} cow.\textsc{pl} \\
\glt `[…] one evening [...] the herdsmen saw the cows going away.'
\z

The gerund also functions as head of a nonfinite causal or temporal subordinate clause.

\ea\label{}
\langinfo{Tuatschín}{Camischùlas} {\citealt[82]{Büchli1966}}\\
\gll   \textbf{Raturnòn} gl' ùm bétg anavùṣ da mjaṣdé, ò’ las zarclunzas tumju […].\\
come\_back.\textsc{ger} \textsc{def.art.m.sg} man \textsc{neg} back of noon have.\textsc{prs.3pl} \textsc{def.art.f.pl} weeder\_woman.\textsc{pl} be\_afraid.\textsc{ptcp.unm}\\
\glt `Since the man hadn’t come back by noon, the weeder women got afraid […].'
\z

\ea\label{}
\langinfo{Tuatschín}{Camischùlas} {\citealt[88]{Büchli1966}}\\
\gll    \textbf{Mònd} spèl’ aua da Ségnas sé òn èlṣ udju da tschèla vard anzatgi […].\\
     go.\textsc{ger} next\_to.\textsc{def.art.f.sg} water of  \textsc{pln} up have.\textsc{prs.3pl} \textsc{3pl.m} hear.\textsc{ptcp.unm} of \textsc{dem.f.sg} side somebody \\
\glt `When walking along the creek from Segnas up, they heard somebody on the other side […].'
\z


\paragraph{Infinitive}
The infinitive functions either as citation form of the verb or occurs as a non-finite verb phrase.

In the latter case, it may occur as the complement of a modal verb. Examples with modal verbs are \textit{savaj} `can, be able' (\ref{ex:inf:1}) or \textit{vaj da} `have to' (\ref{ex:inf:2}).

\ea\label{ex:inf:1}
\langinfo{Tuatschín}{Sadrún}{m5, l. 1195}\\
\gll  Qu’ è adina aviart a lu \textbf{saṣ} \textbf{í} \textbf{ajn} [...].\\
\textsc{dem.unm} \textsc{cop.prs.3sg} always open.\textsc{unm} and then can.\textsc{prs.2sg.gnr} go.\textsc{inf} in\\
\glt `This is always open, and then you can step in [...].'
\z

\ea\label{ex:inf:2}
\langinfo{Tuatschín}{Sadrún}{m6, l. 1454ff.}\\
\gll    Quèl \textbf{vès} lu aun \textbf{da} \textbf{pajè} da té al pustrètsch dal piartg tga té vèvas partgirau.\\
\textsc{dem.m.sg} have.\textsc{cond.3sg} then still \textsc{comp} pay.\textsc{inf} \textsc{dat} \textsc{2sg} \textsc{def.art.m.sg} money of.\textsc{def.art.m.sg} pig \textsc{rel} \textsc{2sg} have.\textsc{impf.2sg} look\_after.\textsc{ptcp.unm}\\
\glt `This one should still pay you the money of the pig you had looked after.'
\z

The infinitive is used in purposive clauses, be it after a verb of movement followed by the complementizer \textit{á} (\ref{ex:inf:3}) or after the complementizer \textit{para /pra} (\ref{ex:inf:4}).

\ea\label{ex:inf:3}
\langinfo{Tuatschín}{Sadrún}{m6, l. 928ff.}\\
\gll    A lu, agl aucségnar … da Sadrún … è saméz sén via par \textbf{í} ajnta Ruèras \textbf{á} \textbf{purtá} agit a \textbf{dá} sògn jéli [...].\\
and then \textsc{def.art.m.sg} priest {} of \textsc{pln} {} be.\textsc{prs.3sg} \textsc{refl.}put.\textsc{ptcp.m.sg} on way.\textsc{f.sg} \textsc{purp} go.\textsc{inf} into \textsc{pln}  \textsc{purp} bring.\textsc{inf} help.\textsc{m.sg} and give.\textsc{inf} holy.\textsc{m.sg} oil\\
\glt `And then, the priest … of Sedrun … set off in order to go to Rueras and bring help and administer the sacrament of anointing [...].'
\z

\ea\label{ex:inf:4}
\langinfo{Tuatschín}{Ruèras}{m10, l. 993ff.}\\
\gll [...] èl duvrava quaj mél pr trá, \textbf{pr} \textbf{trá} lèna sé da Cavòrgja.\\
{} \textsc{3sg.m} use.\textsc{impf.3sg} \textsc{dem.m.sg} mule \textsc{purp} pull.\textsc{inf} \textsc{purp} pull.\textsc{inf} wood.\textsc{coll} up from \textsc{pln}  \\
\glt `[...] he used that mule for transporting wood up from Cavorgia.'
\z

If a verb is fronted in order to topicalise it, it occurs nominalized, i.e. as an infinitive; the finite verb form remains in the background clause (\ref{ex:inf:7}).

\ea\label{ex:inf:7}
\langinfo{Tuatschín}{Surajn}{f5, l. 1319}\\
\gll Na na, a \textbf{durmí} durmévan nus cò. \\
no no and sleep.\textsc{inf} sleep.\textsc{impf.1pl} \textsc{1pl} here\\
\glt `No, no, and as for sleeping, we would sleep here.'
\z

In subject sentences the infinitive is either modified by the definitive masculine singular article (\ref{ex:inf:8}) or not (\ref{ex:inf:9}).
 
\ea
\label{ex:inf:8}
\langinfo{Tuatschín}{Camischùlas}{\DRG{3}{584}}\\
\gll \textbf{Al} \textbf{dèrgjar} \textbf{gjù} aj lu aun mal. Al Vagéli Mòn aj vagnús ṣut in caschnè.\\
\textsc{def.art.m.sg} demolish.\textsc{inf} down \textsc{cop.prs.3sg} then still bad \textsc{def.art.m.sg} \textsc{pn} \textsc{pn} be.\textsc{prs.3sg} come.\textsc{ptcp.m.sg} under \textsc{indef.art.m.sg} hayrack\\
\glt `Demolish [a hayrack] is indeed dangerous. Vigeli Monn came under a hayrack.'
\z

\ea
\label{ex:inf:9}
\langinfo{Tuatschín}{Sadrún}{m5}\\
\gll \textbf{Dèrgjar} \textbf{gjù} in caschnè è prigulús.\\
demolish.\textsc{inf} down \textsc{indef.art.m.sg} hayrack \textsc{cop.prs.3sg} bad.\textsc{adj.unm}\\
\glt `Demolish a hayrack is dangerous.'
\z

\subsubsection{Finite categories}


\paragraph{Present indicative}
Present tense is formed with the verb stem and the personal ending, which means that it is a zero-marked form, in contrast to for instance the imperfect which is characterized by the infix \textit{áv/èv/év}.

Present tense is used with all verbs that refer to an event that includes the moment of speech, independently of the aktionsart of the verb. In (\ref{ex:prs:1}) the present tense refers to an ongoing activity, in (\ref{ex:prs:2}) to a temporary state, and in (\ref{ex:prs:3}) to a permanent state.
                     
\ea\label{ex:prs:1}
\langinfo{Tuatschín}{Sèlva}{f2, l. 963f.}\\
\gll «Tatlaj! Las òndas, las òlmas \textbf{dian} ... rusari gjùn basèlgja.»  \\
listen.\textsc{imp.2pl} \textsc{def.art.f.pl} aunt.\textsc{pl} \textsc{def.art.f.pl} soul.\textsc{pl} say.\textsc{prs.3pl} {} rosary.\textsc{m.sg} down\_in church.\textsc{f.sg}\\
\glt `«Listen! The aunts, the spirits are saying ... a rosary down in the church.'
\z

\ea\label{ex:prs:2}
\langinfo{Tuatschín}{Sadrún}{m8, l. 1484}\\
\gll «Gè, \textbf{sùnd} ju ajn tju taritòri, \textbf{distùrb’} ju té?»   \\
yes \textsc{cop.prs.1sg} \textsc{1sg} in \textsc{poss.2sg.m.sg} territory disturb.\textsc{prs.1sg} \textsc{1sg} \textsc{2sg} \\
\glt `Yes, am I in your territory, do I disturb you?'
\z

\ea\label{ex:prs:3}
\langinfo{Tuatschín}{Sadrún}{m6, l. 875}\\
\gll    La Plata dl Barlòt \textbf{è} sé Caschlè.\\
\textsc{def.art.f.sg} slab of.\textsc{def.art.m.sg} sorcery \textsc{cop.prs.3sg} up \textsc{pln}\\
\glt `The sorcery slab is at Caschlè.'
\z

Present tense also fulfils the function of habitual (\ref{ex:prs:4}) or refers to other discontinuous activities (\ref{ex:prs:5}).
 
\ea\label{ex:prs:4}
\langinfo{Tuatschín}{Camischùlas}{f6, l.  695ff.}\\
\gll    Á Cazis èr’ ju ajn tgòmbra, álṣò qu’ èra tgòmbras da trajs, a lu qu’ \textbf{è} adina, ina \textbf{è} gè \textbf{adina} prsula, a nus trajs vèvan ábar … súpar!\\
in \textsc{pln} \textsc{cop.impf.1sg}	\textsc{1sg} in room.\textsc{f.sg} well \textsc{dem.unm} \textsc{cop.impf.3sg} room.\textsc{f.pl} of three and then \textsc{dem.unm} \textsc{cop.prs.3sg} always  one.\textsc{f.sg} \textsc{cop.prs.3sg} of\_course always alone.\textsc{f.sg} and \textsc{1pl} three have.\textsc{impf.3sg} but {} super\\
\glt `In Cazas I was in a room, well these were rooms for three, and then this was always, one [of the three] is always alone, of course, but the three of us, we had … a great time.'
\z

\ea\label{ex:prs:5}
\langinfo{Tuatschín}{Ruèras}{m1, l. 186}\\
\gll    Ina \textbf{studègja} … á á Winterthur [...].\\
one.\textsc{f.sg} study.\textsc{prs.3sg} {} in in \textsc{pln}\\
\glt `One studies … in in Winterthur [...].'
\z

Present tense also refers to an imminent future (\ref{ex:prs:6} and \ref{ex:prs:7}). 

\ea\label{ex:prs:6}
\langinfo{Tuatschín}{Sadrún}{m4, l. 404f.}\\
\gll  «Ju \textbf{cala} dad í á scùlèta, ju \textbf{pùs} bitg í plé.»\\
\textsc{1sg} stop.\textsc{prs.1sg} \textsc{comp} go.\textsc{inf} to nursery\_school.\textsc{f.sg} \textsc{1sg} can.\textsc{prs.1sg} \textsc{neg} go.\textsc{inf} any\_more  \\
\glt `I’ll stop going to nursery school, I can’t stand it any longer.'
\z

\ea\label{ex:prs:7}
\langinfo{Tuatschín}{Sadrún}{f3, l. 1f.}\\
\gll   Ju \textbf{raquénta} da mia lavur tga ju a fatg als dav\underline{ù}s òns. \\
\textsc{1sg} tell.\textsc{prs.1sg} of \textsc{poss.1sg.f.sg} job \textsc{rel} \textsc{1sg}  have.\textsc{prs.1sg} do.\textsc{ptcp.unm} \textsc{def.art.m.pl} last.\textsc{pl} year.\textsc{pl}\\
\glt `I’ll tell [you] about the job I have done during the last years.'
\z

Present tense is the usual way to refer to future situations of any type (\ref{ex:prs:8}).

\ea\label{ex:prs:8}
\langinfo{Tuatschín}{Sadrún}{m10}\\
\gll   \textbf{Damaun} / \textbf{Ajn} \textbf{duṣ} \textbf{òns} fagjajn nus quaj. \\
tomorrow {} in two.\textsc{m.pl} year.\textsc{pl} do.\textsc{prs.1pl} \textsc{1pl} \textsc{dem.unm}\\
\glt `Tomorrow / In two years we'll do that.'
\z

There are also instances of narrative present whose function is to render the story more vivid (\ref{ex:narr:prs}).

\ea\label{ex:narr:prs}
\langinfo{Tuatschín}{Sadrún}{m6, l. 928ff.}\\
\gll    A lu, agl aucségnar … da Sadrún … \textbf{è} \textbf{saméz} sén via par í ajnta Ruèras á purtá agit a dá sògn jéli, né?, al davùs sacramèn tga \textbf{dèvan} da quels … mòribúnds, basta, agl aucségnar \textbf{végn} atrás … Zarcúns a lu \textbf{auda} `l las stréjas sé cò séssum la val da Lòndadusa \textbf{òni} \textbf{clumau}:\\
and then \textsc{def.art.m.sg} priest {} of \textsc{pln} {} be.\textsc{prs.3sg}  \textsc{refl.}put.\textsc{ptcp.m.sg} on way.\textsc{f.sg} \textsc{purp} go.\textsc{inf} into \textsc{pln}  \textsc{purp} bring.\textsc{inf} help.\textsc{m.sg} and give.\textsc{inf} holy.\textsc{m.sg} oil right \textsc{def.art.m.sg} last sacrament \textsc{rel} give.\textsc{impf.3pl} \textsc{dat} \textsc{dem.m.pl} {} dying.\textsc{pl} enough \textsc{def.art.m.sg} priest come.\textsc{prs.3sg} through {} \textsc{pln} and then hear.\textsc{prs.3sg} \textsc{3sg.m} \textsc{def.art.f.pl} witch.\textsc{pl} up here uppermost \textsc{def.art.f.sg} valley of \textsc{pln} have.\textsc{prs.3pl.3pl} call.\textsc{ptcp.unm}\\
\glt `And then, the priest … of Sedrun … set off in order to go to Rueras and bring help and administer the sacrament of anointing, right? the Holy Sacrament they would give to those … dying people. Well, the priest comes through Zarcuns and then he hears the witches up there, they called from the uppermost part of the Londadusa valley:'
\z

In this example, the first verb referring to story line events is modified by the perfect tense (\textit{è saméz sén via}), the two verbs that follow are modified by the present tense (\textit{végn} and \textit{auda}); the last one (\textit{òni clumau}) is again modified by the perfect tense.

\paragraph{Imperfect indicative}
The imperfect indicative is formed by the infix \textit{áv/èv/év}, whereby the distribution of it is not straightforward. \textit{áv} is used with all verbs ending in \textit{-á} and with most verbs ending in \textit{-è}; \textit{èv} is used with some verbs ending in \textit{-è}, with most irregular verbs ending in \textit{-aj}, and with some other irregular verbs; and \textit{év} is used with all verbs ending in \textit{-ˈar} and \textit{-í} as well as with some irregular verbs ending in \textit{-aj}.

The basic functions of the imperfect indicative are to refer to imperfective aspect in the past with all types of lexical aspect (\ref{ex:impf:1}) and (\ref{ex:impf:2})\footnote{But see examples (\ref{ex:prf:inch:1}) to (\ref{ex:prf:inch:3}) below.}, to past habitual (\ref{ex:impf:3}) and (\ref{ex:impf:4}), or to an unspecified repetition of actions in the past (\ref{ex:impf:5}).
	
\ea\label{ex:impf:1}
\langinfo{Tuatschín}{Sadrún}{m8, l. 1480ff.}\\
\gll Api grad \textbf{ajn} \textbf{quèl} \textbf{mumèn} \textbf{vagnév’} in cégn … gròn ni– vi datiar ad èra lò usché in téc dòmin\underline{a}nt.\\
and exactly in \textsc{dem.m.sg} moment come.\textsc{impf.3sg} \textsc{indef.art.m.sg} swan {} big.\textsc{m.sg.unm} or over next\_to and \textsc{cop.impf.3sg} there so \textsc{indef.art.m.sg} bit dominant.\textsc{adj.unm}\\
\glt `And precisely at that moment a big swan … was coming to the place where I was, a bit a dominant one.'
\z

\ea\label{ex:impf:2}
\langinfo{Tuatschín}{Ruèras}{m1, l. 165}\\
\gll    Ju \textbf{lèv’} ampr\underline{è}ndar da majstar [...].\\
\textsc{1sg} want.\textsc{impf.1sg} learn.\textsc{inf} of joiner.\textsc{m.sg}\\
\glt `I wanted to become a joiner [...].'
\z

\ea\label{ex:impf:3}
\langinfo{Tuatschín}{Surajn}{f5, l. 1320}\\
\gll    Nus \textbf{mavan} la damaun api \textbf{vagnévan} la sèra. \\
\textsc{1pl} go.\textsc{impf.1pl}  \textsc{def.art.f.sg} morning and come.\textsc{impf.1pl} \textsc{def.art.f.sg} evening \\
\glt `We would go in the morning and come back in the evening.'
\z

\ea\label{ex:impf:4}
\langinfo{Tuatschín}{Sadrún}{m6, l. 1327f.}\\
\gll    A nus \textbf{mavan} culs pòrs sé Valtgèva, \textbf{mintga} \textbf{dé} sé a gjù, ju savès raquintá da té quaj.\\
and \textsc{1pl}  go.\textsc{impf.1pl} with.\textsc{def.art.m.pl} pig.\textsc{pl} up \textsc{pln} every day.\textsc{m.sg} up and down  \textsc{1sg}  can.\textsc{cond.1sg}  tell.\textsc{inf}  \textsc{dat}  \textsc{2sg} \textsc{dem.unm}\\
\glt `And we would go up to Valtgeva with the pigs, every day up and down, I could tell you about that.'
\z

\ea\label{ex:impf:5}
\langinfo{Tuatschín}{Sadrún}{f3, l. 11f.}\\
\gll  Api lura … ju \textbf{mava} è \textbf{mintgataun} cun èl á dá culur las sèndas [...].\\
and then {} \textsc{1sg} go.\textsc{impf.1sg} also sometimes with \textsc{3sg.m} \textsc{purp} give.\textsc{inf} colour.\textsc{f.sg} \textsc{def.art.f.pl} trail.\textsc{pl}\\
\glt `And then … from time to time I would go with him to give colour [to the stones indicating] the trails [...].'
\z


\paragraph{Perfect indicative}
The perfect is formed with the auxiliary verbs \textit{èssar} `be' or \textit{vaj} `have' and the past participle. If the verb is conjugated with \textit{èssar}, the participle agrees with the subject in gender and number.

The following verbs are conjugated with \textit{èssar}:

\begin{itemize}
	
	\item intransitive motion verbs: \textit{curdá} `fall', \textit{dá gjù} `fall down', \textit{í} `go', \textit{mitschá} `escape', \textit{ruclá} `fall down', \textit{saglí} `run', \textit{scapá} `escape', \textit{séjṣar gjù} `sit down', \textit{ṣgulá} `fly', \textit{ṣvaní} `disappear', \textit{vagní} `come'
	\item verbs of state: \textit{èssar} `be', \textit{rastá} `remain', \textit{vívar} `live'
	\item change-of-state verbs: \textit{capitá} `happen',  \textit{crèschar} `grow', \textit{maridá} `get married', \textit{murí} `die', \textit{néschar} `be born', \textit{schabagjá} `happen'
	\item reflexive verbs\footnote{According to the DRG (1: 568), the choice of \textit{esser} as auxiliary verb for reflexives in Sursilvan is due to the demand of Sursilvan grammarians since the 18th century. Nowadays speakers seek to conform to this claim, but in spoken Sursilvan, one still can find \textit{haver} as auxiliary for reflexive verbs. However, this does not seem to be the case in Tuatschin, where \textit{èssar} is exclusively used, at least in the corpus.}
	\item passive verbs
	\end{itemize}

The main function of the perfect is to express perfective aspect, i.e. to refer to the whole situation with beginning, middle, and end, without or with a relation to the present.

\ea\label{}
\langinfo{Tuatschín}{Surajn}{\citealt[128]{Büchli1966}}\\
\gll    Ju \textbf{sùnd} \textbf{jus} sé Culmatsch ina dumèngja. \\
\textsc{1sg} be.\textsc{prs.1sg} go.\textsc{ptpc.m.sg} up \textsc{pln} \textsc{indef.art.f.sg} Sunday\\
\glt `One Sunday I went up to Culmatsch.'
\z

\ea
\label{}
\langinfo{Tuatschín}{Sadrún}{m8, l. 1473}\\
\gll  [...] api \textbf{vòu} \textbf{anflau} in bi ljuc [...].\\
 {} and have.\textsc{prs.1sg.1sg} find.\textsc{ptcp.m.unm} \textsc{indef.art.m.sg} beautiful.\textsc{m.sg} place\\
\glt `[...] and then I found a nice place [...].'
\z

\ea\label{}
\langinfo{Tuatschín}{Sadrún}{m6, l. 928f.}\\
\gll    A lu, agl aucségnar … da Sadrún … \textbf{è} \textbf{saméz} sén via par í ajnta Ruèras [...].\\
and then \textsc{def.art.m.sg} priest {} of \textsc{pln} {} be.\textsc{prs.3sg} \textsc{refl.}put.\textsc{ptcp.m.sg} on way.\textsc{f.sg} \textsc{purp} go.\textsc{inf} into \textsc{pln}\\
\glt `And then, the priest … of Sedrun … set off in order to go to Rueras [...].'
\z

In Romance languages like French, when the perfect tenses modify a stative verb like \textit{connaître} `know' or \textit{savoir} `know', it usually gets an inchoative meaning: \textit{J'ai connu Michel à une fête.} `I met Michel at a party.', or \textit{J'ai su qu'elle était malade.} `I was told that she was ill.'. But in Tuatschin, the perfect is used with these stative verbs (which take the form \textit{ancanùschar} and \textit{savaj}) without an inchoative meaning. In other words, in these cases the verbs refer imperfectly to the situation, which is underlined by the use of the adverb \textit{schòn} `already' in (\ref{ex:prf:schon}).

\ea\label{ex:prf:inch:1}
\langinfo{Tuatschín}{Bugnaj} {\citealt[139]{Büchli1966}}\\
\gll    Èl \textbf{ò} \textbf{ancanùschju} la familja, mù maj détg òra tgi èri. \\
     \textsc{3sg.m} have.\textsc{prs.3sg} know.\textsc{ptcp.unm} \textsc{def.art.f.sg} family but never tell.\textsc{ptcp.unm} out who \textsc{cop.impf.sbjv.3sg}\\
\glt `He knew the family, but never said who they were.'
\z

\ea\label{ex:prf:schon}
\langinfo{Tuatschín}{Sadrún} {\citealt[103]{Büchli1966}}\\
\gll   Al buép \textbf{ò} schòn \textbf{ancanùschju} èlas. \\
     \textsc{def.art.m.sg} boy have.\textsc{prs.3sg} already know.\textsc{ptcp.unm} \textsc{3pl.f}\\
\glt `The boy already knew them [= the girls].'
\z

\ea\label{ex:prf:inch:2}
\langinfo{Tuatschín}{Cavòrgja} {\citealt[53]{Büchli1966}}\\
\gll    La fuméglja d’ alp \textbf{ò} \textbf{savju} nuét.\\
     \textsc{def.art.f.sg} farmhand.\textsc{coll} of alp have.\textsc{prs.3sg} know.\textsc{ptcp.unm} nothing\\
\glt `The alp shepherds didn’t know anything.'
\z

To get the inchoative meaning, Tuatschin uses \textit{amprèndar} \textit{d'} \textit{ancanùschar} (\ref{ex:prf:inch1}), literally `learn to know', and \textit{udí} (\ref{ex:prf:inch2}) `hear'.

\ea
\label{ex:prf:inch1}
\langinfo{Tuatschín}{Ruèras}{10, l. 1000ff.}\\
\gll  Api, ah, quaj ah fascinava pròpi mè, ju vèṣ ah gè ju vèṣ è ugèn \textbf{amprju} \textbf{d’} \textbf{ancanùschar} quaj mél, ábar ju ... sùn halt naschjus mèmja tart. \\
and eh \textsc{dem.unm} eh fascinate.\textsc{impf.3sg} really \textsc{1sg}  \textsc{1sg} have.\textsc{cond.1sg} ah yes \textsc{1sg} have.\textsc{cond.1sg} also with\_pleasure learn.\textsc{ptcp.m.unm} \textsc{comp} know.\textsc{inf} \textsc{dem.m.sg} mule but \textsc{1sg} {} be.\textsc{prs.1sg} just be\_born.\textsc{ptcp.m.sg} too late\\
\glt `And, eh, this really fascinated me, I would have eh yes I would have very much liked to get to know this mule, but I ... was just born too late.'
\z

\ea
\label{ex:prf:inch2}
\langinfo{Tuatschín}{Ruèras}{m10}\\
\gll Ju a \textbf{udju} tg' èl \textbf{ségi} mazauns.\\
\textsc{1sg} have.\textsc{prs.1sg} hear.\textsc{ptcp.unm} \textsc{comp} \textsc{3sg.m} \textsc{cop.prs.sbjv.3sg} ill.\textsc{m.sg}\\
\glt `I was told that he is ill.'
\z

As for \textit{vaj} `have', there is no difference between the use of the perfect or the imperfect, at least according to the native speakers I have consulted. Both the perfect in (\ref{ex:prf:inch:3}) and the imperfect in (\ref{ex:prf:inch:4}) could be interpreted as inchoative or as a permanent state.

\ea
\label{ex:prf:inch:3}
\langinfo{Tuatschín}{Ruèras}{m4, l. 336f.}\\
\gll Quaj è stau ina ... fétg grònda familja, èlṣ \textbf{òn} \textbf{gju} indiṣch ufauns [...].\\
\textsc{dem.unm}  be.\textsc{prs.3sg}  \textsc{cop.ptcp.m.unm}  \textsc{indef.art.f.sg} {} very big family \textsc{3pl.m} have.\textsc{prs.3pl} have.\textsc{ptcp.m.unm} eleven child.\textsc{m.pl}\\
\glt `This was a ... very big family, they had eleven children [...].'
\z

\ea
\label {ex:prf:inch:4}
\langinfo{Tuatschín}{Sadrún}{m5}\\
\gll Èls \textbf{vèvan} indiṣch ufauns.\\
\textsc{3pl.m} have.\textsc{impf.3pl} eleven child.\textsc{m.pl}\\
\glt `They had eleven children.'
\z

As seen in example (\ref{ex:omit:1}) in § 4.1.2.1.1  above, story-line events can also be referred to only with the past participle, without the auxiliary verbs \textit{èssar} or \textit{vaj}.


\paragraph{Pluperfect indicative}
The pluperfect fulfils the function of perfective aspect of a situation that is situated before another situation in the past.

\ea
\label{}
\langinfo{Tuatschín}{Sadrún}{m5, l. 1227f.}\\
\gll  Agl Andreòli \textbf{vèva} \textbf{finju} … las ... figuras … \textbf{avaun} \textbf{ca} la caplùta \textbf{èra} \textbf{stada} \textbf{finida}. \\
\textsc{def.art.m.sg} \textsc{pn} have.\textsc{impf.3sg} finish.\textsc{ptcp.unm} {} \textsc{def.art.f.pl} {} figure.\textsc{pl} {} before \textsc{rel} \textsc{def.art.f.sg} chapel be.\textsc{impf.3sg} \textsc{pass.aux.ptcp.f.sg} finish.\textsc{ptcp.f.sg} \\
\glt `Andreoli had finished ... the ... figures ... before the chapel was finished.'
\z

\ea
\label{}
\langinfo{Tuatschín}{Sadrún}{m4, l. 494ff.}\\
\gll  Qu' è lu ju tschèluisa tgé ca nuṣ \textbf{èssan} \textbf{vagní} vidòra, \textbf{turnaj} ò da Pardatsch, tg' èssan nus staj ajn lò fòrsa … quátar tschun jamnas, scha \textbf{vèva} `l \textbf{fatg} ṣchùber nuét. Quèla fascha \textbf{èra}  \textbf{satratg’} ansjaman [...]. \\
\textsc{dem.unm} be.\textsc{prs.3sg} then go.\textsc{ptcp.unm} such\_way \textsc{comp} when \textsc{1pl} be.\textsc{impf.1pl} come.\textsc{ptcp.m.pl} out return.\textsc{ptcp.m.pl} out of \textsc{pln} \textsc{comp} be.\textsc{prs.1pl} \textsc{1pl} \textsc{cop.ptcp.m.pl} in there maybe {} four five week.\textsc{f.pl} \textsc{corr} have.\textsc{impf.3sg} \textsc{3sg.m} do.\textsc{ptcp.unm} clean.\textsc{adj.unm} nothing \textsc{dem.f.sg} bandage be.\textsc{impf.3sg} \textsc{refl.}contract.\textsc{impf.3sg} together \\
\glt `This happened in such a way that when we returned down [to Surrein] from Pardatsch, then we had stayed there maybe … four or five weeks, he hadn’t done anything at all. That bandage had contracted [...].'
\z

\ea
\label{}
\langinfo{Tuatschín}{Camischùlas}{f6, l. 777ff.}\\
\gll   A nuṣ \textbf{vajn} \textbf{gju} schi súpar. Ju èra ùsa, ju \textbf{vèva} \textbf{gju} ajnsasèz al clétg dad èssar ajn tgòmbra cun ròm\underline{ò}ntschas.\\
and \textsc{1pl} have.\textsc{prs.1pl} have.\textsc{ptcp.unm} so super \textsc{1sf} \textsc{cop.impf.1sg} now \textsc{1sg} have.\textsc{impf.1sg} have.\textsc{ptcp.m.unm} in\_fact \textsc{def.art.m.sg} luck \textsc{comp} \textsc{cop.inf} in room.\textsc{f.sg} with Romansh.\textsc{f.pl}\\
\glt `And we had such a wonderful time. I was now, in fact I had been lucky to share the room with Romansh girls.'
\z

\paragraph{Future}
As already mentioned, the future is almost never used. The only example in the oral corpus is (\ref{ex:fut1}).

\ea\label{ex:fut1}
\langinfo{Tuatschín}{Sadrún}{f3, l. 31ff.}\\
\gll Ad anad' òtgòntasjat vajn nus gju ina vòtazjun fadarala ṣur da las sèndas, sch' i \textbf{végn} \textbf{á} \textbf{prèndar} \textbf{ajn} quaj né bétg.   \\
and year.\textsc{f.sg} eighty-seven have.\textsc{prs.1pl} \textsc{1pl} have.\textsc{ptcp.m.unm} \textsc{indef.art.f.sg} vote federal over of \textsc{def.art.f.pl} trail.\textsc{pl} whether \textsc{expl} \textsc{fut.aux.3sg} \textsc{comp} take.\textsc{inf} in \textsc{dem.unm} or \textsc{neg}  \\
\glt `And in 1987 we had a federal vote about the trails, whether it would be adopted or not.'
\z

\paragraph{Doubly-compound tenses}
There are two doubly-compound tenses: perfect (\ref{ex:dcomp:1} and \ref{ex:dcomp:2}) and pluperfect (\ref{ex:dcomp:4}). They fulfil the same functions as the simple compound tenses, but they express a longer temporal distance in the past.

\ea
\label{ex:dcomp:1}
\langinfo{Tuatschín}{Ruèras}{m1, l. 300ff.}\\
\gll    Ábar tschaj è bi, ju \textbf{a} lu sjantar \textbf{gju} … \textbf{calau} da fá `l pur tgu vèva tgéj? … tschuncònt’ òns.\\
but \textsc{dem.unm} \textsc{cop.prs.3sg} nice.\textsc{adj.unm} \textsc{1sg} have.\textsc{prs.1sg} then after have.\textsc{ptcp.unm} {} stop.\textsc{ptcp.unm} \textsc{comp} do.\textsc{inf} \textsc{def.art.m.sg} farmer when.\textsc{rel.1sg} have.\textsc{impf.1sg} what {} fifty year.\textsc{m.pl}\\
\glt `But that is nice, I then had … stopped working as a farmer when I was … fifty years old.'
\z

\ea
\label{ex:dcomp:2}
\langinfo{Tuatschín}{Sadrún}{m5, l. 1234ff.}\\
\gll A l’ òn ca tg’ \textbf{òn} \textbf{gju} \textbf{dépònju} quèlas ah figuras ò inṣ adina détg «la stiva dals gjadjus», ò quèla gju nùm sjantar.\\
and \textsc{def.art.m.sg} year \textsc{rel} \textsc{rel} have.\textsc{prs.3sg} have.\textsc{ptcp.m.unm} store.\textsc{ptcp.m.unm} \textsc{dem.f.pl} eh figure.\textsc{pl} have.\textsc{prs.3sg} \textsc{gnr} always say.\textsc{ptcp.m.unm} \textsc{def.art.f.sg} living\_room of.\textsc{def.art.m.pl} Jew.\textsc{pl} have.\textsc{prs.3sg} \textsc{dem.f.sg} have.\textsc{ptcp.m.unm} name after\\
\glt `And [since] the year they stored these eh figures one has always said «the living room of the Jews», has it been called since.'
\z

\ea
\label{ex:dcomp:4}
\langinfo{Tuatschín}{Zarcúns}{m2, l. 1633ff.}\\
\gll    A … ad in òn, sa ju aun bégn, lu \textbf{vèvan} nus lu \textbf{gju} \textbf{fatg} in tòc humòrístic  da la músic’ anòra.\\
and {} and  \textsc{indef.art.m.sg} year know.\textsc{prs.1sg} \textsc{1sg} still well then have.\textsc{impf.1pl} \textsc{1pl} then have.\textsc{ptcp.unm} do.\textsc{ptcp.unm} \textsc{indef.art.m.sg} prank funny from \textsc{def.art.f.sg} music out\\
\glt `And … and one year, I still know very well, we from the music had played a funny prank.'
\z




\paragraph{Progressive aspect}
The progressive aspect is formed with the copula \textit{èssar}, the preposition \textit{vid(a)} ‘at’, with (\ref{ex:prog:with}) or without (\ref{ex:prog:without}) the definite article masculine singular, and the infinitive.

\ea\label{ex:prog:with}
\langinfo{Tuatschín}{Bugnaj} {\citealt[132]{Büchli1966}}\\
\gll    Duas zarclunzas \textbf{èran} \textbf{vid} `\textbf{l} \textbf{zarclá}.\\
     two.\textsc{f.pl} weeder\_woman.\textsc{pl} \textsc{cop.impf.3pl} \textsc{prog} \textsc{def.art.m.sg} weed.\textsc{inf}\\
\glt `Two weeder women were weeding […].'
\z

\ea\label{ex:prog:without}
\langinfo{Tuatschín}{Sadrún}{f3, l. 75f.}\\
\gll  Api quèls da la vischnaunca \textbf{èran} grad vida ´l, \textbf{vida} \textbf{zaná} al bògn [...].  \\
and \textsc{dem.m.pl} of \textsc{def.art.f.sg} municipality \textsc{cop.impf.3pl} just \textsc{prog} \textsc{def.art.m.sg} \textsc{prog} renovate.\textsc{inf} \textsc{def.art.m.sg} bath\\
\glt `And the municipal employees were just renovating the swimming pool [...].'
\z

\paragraph{Present and perfect subjunctive}
Subjunctive mood, be it present, perfect, or imperfect, is characterised by the affix \textit{-i(-)}, which is located between stem and personal ending. Note that there is no personal ending in first and third person singular: \textit{tgu cònti / tg' èla cònti} `that I sing / that she sings'.

Subjunctive mood mostly occurs in some types of object clauses and in non-core argument clauses headed by \textit{avaun tga} `before', \textit{par tga} `in order to', \textit{tòca tga} `until', or \textit{sènza tga} `without that'.\footnote{The most thorough analysis of mood in standard Sursilvan is \citet{Grünert2003}, 578 pages.} In the corpus, subjunctive mood occurs in three tenses: present, perfect, and imperfect. Subjunctive imperfect will be treated in the next section.

The most important subjunctive triggers occurring in the corpus are

\begin{itemize}
\item (a) verbs of speaking: \textit{dí} `say', \textit{dumandá} `ask', \textit{raquintá} `tell', \textit{udí} `hear, be told';
\item (b) verbs of opinion: \textit{craj} `believe, think', \textit{paraj} `seem', \textit{tanaj} `think, hold', \textit{tartgá} `think';
\item (c) directive speech act verbs and optative: \textit{fá stém} and \textit{mirá}, both `make sure', \textit{rujè} `ask that', \textit{vulaj} `want';
\item (d) purposive conjunctions: \textit{par tga}, \textit{tga} `in order to';
\item (e) the conjunctions \textit{avaun (ca) tga} `before', \textit{sènza tga} `without', and \textit{tòcan} `until'.
\end{itemize}

Note that in object clauses the complementizer is often absent (\ref{ex:subj1} and \ref{ex:subj6}). Examples (\ref{ex:subj1} to \ref{ex:subj4}) illustrate the use of subjunctive mood, present and perfect, with verbs of speaking.

\ea
\label{ex:subj1}
\langinfo{Tuatschín}{Ruèras}{m10, l. 1086ff.}\\
\gll  A lu \textbf{ò} `l \textbf{détg} {\longrule} èl \textbf{sapi} bigja vagní da lò òra, ju, èl {\longrule} \textbf{stètgi} mal, èl {\longrule} \textbf{mòndi} da via òra [...].  \\
and then have.\textsc{prs.3sg} \textsc{3sg.m} say.\textsc{ptcp.unm} {} \textsc{3sg.m} can.\textsc{prs.sbjv.3sg} \textsc{neg} come.\textsc{inf} from there out \textsc{1sg} \textsc{3sg.m} {} stay.\textsc{prs.sbjv.3sg} bad \textsc{3sg.m} {} go.\textsc{prs.sbjv.3sg} from road.\textsc{f.sg} out\\
\glt `And then he said he couldn’t walk on that path, that I - that he was sorry, [but] that he would walk on the road [...].'
\z

\ea
\label{ex:subj2}
\langinfo{Tuatschín}{Sadrún}{m4, l. 686f.}\\
\gll Lu dumandavan nuṣ èl, \textbf{vèvan} \textbf{dumandau} núa èl \textbf{ségi} stauṣ ajn plaza [...].\\
then ask.\textsc{impf.1pl} \textsc{1pl} \textsc{3sg.m} have.\textsc{impf.3sg}  ask.\textsc{ptcp.unm} where \textsc{3sg.m} be.\textsc{prs.sbjv.3sg} \textsc{cop.ptcp.m.sg} in job.\textsc{f.sg}\\
\glt `Then we would ask him, we had asked [him] where he had been working [...].'
\z

\ea
\label{ex:subj3}
\langinfo{Tuatschín}{Sadrún}{m6, l. 880ff.}\\
\gll    A la détga \textbf{raquénta} … tga las stréjas dl Caschlè tg’ èran sé cò a fijèvan barlòt \textbf{vagjan} \textbf{trans-pòrtau} quèla plata sin\footnotemark{} in fil ... sajda [...].\\
and \textsc{def.art.f.sg} legend tell.\textsc{prs.3sg} {} \textsc{comp} \textsc{def.art.f.pl} witch.\textsc{pl} \textsc{def.art.m.sg} \textsc{pln} \textsc{rel} \textsc{cop.impf.3pl} up here and do.\textsc{impf.3pl} sorcery have.\textsc{prs.sbjv.3pl} carry.\textsc{ptcp.unm} \textsc{dem.f.sg} slab on \textsc{indef.art.m.sg} thread {} silk\\
\glt `And the legend says … that the witches of the Caschlè which were up there and used to do sorcery had carried this slab on a … silk thread [...].'\footnotetext{\textit{sin} instead of \textit{sén}}
\z

\ea
\label{ex:subj4}
\langinfo{Tuatschín}{Ruèras}{m10}\\
\gll Ju a \textbf{udju} tg' èl \textbf{ségi} mazauns.\\
\textsc{1sg} have.\textsc{prs.1sg} hear.\textsc{ptcp.unm} \textsc{comp} \textsc{3sg.m} \textsc{cop.prs.sbjv.3sg} ill.\textsc{m.sg}\\
\glt `I was told that he is ill.'
\z

Subjunctive mood is also used in free indirect speech, which is characterized by the lack of an introductory verb (\ref{ex:free.ind1}).

\ea
\label{ex:free.ind1}
\langinfo{Tuatschín}{Ruèras}{m10, l. 1171ff.}\\
\gll  Èl \textbf{èri} avaun caplùta a \textbf{vagi} \textbf{vju} tga quèls méls èn saspuantaj, api \textbf{vagi} èl \textbf{tartgau} ... dad í vi ajn via ... a tanaj sé èls.  \\
\textsc{3sg.m} \textsc{cop.impf.sbjv.3sg} in\_front chapel and have.\textsc{prs.sbjv.3sg} see.\textsc{ptcp.unm} \textsc{comp} \textsc{dem.m.pl} mule.\textsc{pl} be.\textsc{prs.3pl} \textsc{refl.}frighten.\textsc{ptcp.m.pl} and have.\textsc{prs.sbjv.3sg} \textsc{3sg.m} think.\textsc{ptcp.unm} {} \textsc{comp} go.\textsc{inf} over on road.\textsc{f.sg} {} and hold.\textsc{inf} up \textsc{3pl.m}\\
\glt `He was in front of the chapel and had seen that these mules ran away and he thought ... that he would go on the road ... and stop them.'
\z

In (\ref{ex:free.ind2}) and (\ref{ex:free.ind3}), the sentence starts with a verb in indicative mood, which represent the words of the narrator, and then goes on in subjunctive mood, which represents the words of the army.

\ea
\label{ex:free.ind2}
\langinfo{Tuatschín}{Sadrún}{f3, l. 68ff.}\\
\gll  L’ autar dé va ju gju la lubiantscha dad í vidajn, ábar \textbf{stòpi} prèndar malitèr cun mè, tga \textbf{vajan} … fùnc a \textbf{sapjan} prèndar ah, dí cu nus \textbf{vajan} da … ir davùṣ in cuélm.\\
\textsc{def.art.m.sg} other day have.\textsc{prs.1sg} \textsc{1sg} have.\textsc{ptcp.unm} \textsc{def.art.f.sg} permission \textsc{comp} go.\textsc{inf} in but must.\textsc{prs.sbjv.1sg}  take.\textsc{inf} military.\textsc{m.sg} with \textsc{1sg} \textsc{rel} have.\textsc{prs.sbjv.3pl} {} radio.\textsc{m.sg} and can.\textsc{prs.sbjv.3pl} take.\textsc{inf} eh say.\textsc{inf} when \textsc{1pl} have.\textsc{prs.sbjv.1pl} \textsc{comp} {} go.\textsc{inf} behind \textsc{indef.art.m.sg} mountain\\
\glt `The day after I got permission to go there, but I needed to take with me some soldiers that had a radio and would say when we should … go behind a mountain [to protect ourselves].'
\z

\ea
\label{ex:free.ind3}
\langinfo{Tuatschín}{Sadrún}{f3, l. 37ff.}\\
\gll [...] a mintga vaschnaunca \textbf{ò} lu \textbf{stavju} dá ajn tùt tgé ca la \textbf{vagi}, nùca la \textbf{vagi} lògans cun mussavias, a las sèndas, tùt.\\
{} and every municipality.\textsc{f.sg} have.\textsc{prs.3sg} then must.\textsc{ptcp.unm} give.\textsc{inf} in all what \textsc{rel} \textsc{3sg.f} have.\textsc{prs.sbjv.3sg} where \textsc{3sg.f} have.\textsc{prs.sbjv.3sg} place.\textsc{m.pl} with signpost.\textsc{f.pl} and \textsc{def.art.f.pl} trail.\textsc{pl} all \\
\glt `[...] and every municipality had then to inform about everything they had, where they had places with signposts and trails, everything.'
\z

Examples (\ref{ex:subj5}) to (\ref{ex:subj8}) illustrate the use of subjunctive mood with verbs of opinion.

\ea
\label{ex:subj5}
\langinfo{Tuatschín}{}{\DRG{3}{582}}\\
\gll  Da mé \textbf{par}' \textbf{aj} tg' al ajfar-piast da véjdar \textbf{èri} bétga schi lads.\\
\textsc{dat} \textsc{1sg} seem.\textsc{prs.3sg} \textsc{expl} \textsc{comp} \textsc{def.art.m.sg} hayrack\_post  of old.\textsc{adj.unm} \textsc{cop.impf.subj.3sg} \textsc{neg} so wide.\textsc{m.sg} \\
\glt `It seems to me that the hayrack posts of earlier times were not that wide.'
\z

\ea
\label{ex:subj6}
\langinfo{Tuatschín}{Ruèras}{\DRG{1}{393}}\\
\gll  Ju \textbf{tégn} tga quaj \textbf{végni} fatg pauc.  \\
\textsc{1sg} hold.\textsc{prs.1sg} \textsc{comp} \textsc{dem.unm} \textsc{pass.aux.prs.sbjv.3sg} make.\textsc{ptcp.unm} little\\
\glt `I think that this is not often done.'
\z

\ea
\label{ex:subj8}
\langinfo{Tuatschín}{Camischùlas}{f6, l. 769}\\
\gll    Api sjantar \textbf{vajn} nus \textbf{tartgau} {\longrule} nus \textbf{sapjan} durmí òra [...].\\
and after have.\textsc{prs.1pl} \textsc{1pl} think.\textsc{ptcp.unm} {} \textsc{1pl}  can.\textsc{prs.sbjv.1pl} sleep.\textsc{inf} out\\
\glt `And then we thought we would have a good sleep [...].'
\z

Examples (\ref{ex:subj9}) to (\ref{ex:subj11}) shows the use of subjunctive mood with directive speech act verbs.

\ea
\label{ex:subj9}
\langinfo{Tuatschín}{}{\DRG{5}{535}}\\
\gll  \textbf{Mira} tga quaj lò \textbf{davjanti} bétg. \\
look.\textsc{imp.2sg} \textsc{comp} \textsc{dem.unm} there become.\textsc{prs.sbjv.3sg} \textsc{neg}\\
\glt `Make sure that this does not happen.'
\z

\ea
\label{ex:subj10}
\langinfo{Tuatschín}{Sadrún}{m4, l. 478ff.}\\
\gll  Té \textbf{mira} lu tg’ al tat \textbf{fétschi} lu mintga dé, \textbf{préndi} gjù quaj a \textbf{ṣchubrègi} a \textbf{fétschi} sé da néjv.\\
\textsc{2sg} look.\textsc{imp.2sg} then \textsc{comp} \textsc{def.art.m.sg} grandfather do.\textsc{prs.sbjv.3sg} then every.\textsc{m.sg} day take.\textsc{prs.sbjv.3sg} down \textsc{dem.unm} and clean.\textsc{prs.sbjv.3sg} and do.\textsc{prs.sbjv.3sg} up of new.\textsc{adj.unm} \\
\glt `And you, make sure that your grandfather makes it every day, that he takes it off, that he cleans it and puts it on again.'
\z

\ea
\label{ex:subj11}
\langinfo{Tuatschín}{Camischùlas} {\citealt[94]{Büchli1966}}\\
\gll [...] lò végni [...] \textbf{rujau} tgé Nòssadùna \textbf{laschi} madirá bégn al graun ajn Tujétsch.\\
{} there \textsc{pass.aux.prs.3sg.expl} {} ask.\textsc{ptcp.unm} \textsc{comp} Our\_Lady.\textsc{f.sg} let.\textsc{prs.sbjv.3sg} ripen.\textsc{inf} well \textsc{def.art.m.sg} cereals in \textsc{pln}\\
\glt `[...] there they pray that the Virgin Mary let grow well the cereals in the Tujetsch Valley.'
\z

\ea
\label{}
\langinfo{Tuatschin}{Sadrún} {m5}\\
\gll Ju \textbf{vi} bétg tga la tgèsa \textbf{ardi}.\\
\textsc{1sg} want.\textsc{prs.1sg} \textsc{neg} \textsc{comp} \textsc{def.art.f.sg} house burn.\textsc{prs.sbjv.3sg}\\
\glt `I don't want that the house be on fire.'
\z

In purposive clauses, the conjunctions \textit{par tga} or \textit{tga} are used (examples \ref{ex:subj12} to \ref{ex:subj14}) .

\ea
\label{ex:subj12}
\langinfo{Tuatschín}{Sadrún}{m4, l. 595ff.}\\
\gll  A la sèra \textbf{par} \textbf{tga} \textbf{briṣchi} bétg … vagnéva quaj, quaj mava `l ajnagjù cul maun èra sènza … [vòns] a trèva vid\underline{ò} còtgla gjù sé sél plantschju.\\
and \textsc{def.art.f.sg} evening \textsc{purp} \textsc{comp} burn.\textsc{prs.sbjv.3sg} \textsc{neg} {} \textsc{pass.aux.impf.3sg} \textsc{dem.unm} \textsc{dem.unm} go.\textsc{impf.3sg} \textsc{3sg.m} in\_down with.\textsc{def.art.m.sg} hand also without {} [glove.\textsc{m.pl}] and pull.\textsc{impf.3sg} out charcoal.\textsc{coll} down up on.\textsc{def.art.m.sg} floor  \\
\glt `And in the evening, to avoid it burning … was that, there he went into [the fire] with one hand, also without [gloves], and pulled out charcoal from down there up to the floor.'
\z

\ea
\label{ex:subj13}
\langinfo{Tuatschín}{Ruèras}{m1, l. 235ff.}\\
	\gll    A quaj stèvnṣ èssar … pulits-pulits l’ jamna … {\longrule} \textbf{tg}’ al bap \textbf{dètschi} in frang a miaz.\\
	and \textsc{dem.unm} must.\textsc{impf.1pl.1pl} \textsc{cop.inf} {} \textsc{red}\textasciitilde{well\_behaved}.\textsc{m.pl} \textsc{def.art.f.sg} week {} {}  \textsc{comp} \textsc{def.art.m.sg} father  give.\textsc{prs.sbjv.3sg} one.\textsc{m.sg} franc and half.\textsc{m.sg}\\
\glt `And we had to be … very well-behaved during the week … so that my father would give [us] one and a half francs.'
\z

\ea
\label{ex:subj14}
\langinfo{Tuatschín}{}{\citealt[87]{Gadola1935}}\\
\gll  “[…] i ò dau las sjat.” “Lu cuschaj {\longrule} \textbf{tg}’ in \textbf{audi}.”\\
     {} \textsc{expl} have.\textsc{prs.3sg} give.\textsc{ptcp.unm} \textsc{def.art.f.pl} seven then be\_quiet.\textsc{imp.2pl} {}  \textsc{comp} \textsc{gnr} hear.\textsc{prs.sbjv.3sg}\\
\glt `”[…] It has struck seven o’clock.” “Then be quiet so we can hear.”'
\z


The subordinator \textit{avaun} `before' occurs as \textit{avaun tga} (\ref{ex:subj15}), \textit{avaun ca} (\ref{ex:subj16}), and \textit{avaun ca tga} (\ref{ex:subj17}). In (\ref{ex:subj17}) subjunctive mood is used, in contrast to (\ref{ex:subj15}) and (\ref{ex:subj16}) where indicative mood is used.

\ea
\label{ex:subj15}
\langinfo{Tuatschín}{}{\DRG{5}{777}}\\
\gll   \textbf{Avaun} \textbf{tgi} \textbf{végn} malaura isan las vacas ṣgarṣchajval.\\
before \textsc{comp.expl} come.\textsc{prs.ind.3sg} bad\_weather.\textsc{f.sg} run\_back\_and\_forth.\textsc{prs.3sg} \textsc{def.art.f.pl} cow.\textsc{pl} terrible.\textsc{adj.unm} \\
\glt `Before bad weather comes, the cows run back and forth like mad.'
\z

\ea
\label{ex:subj16}
\langinfo{Tuatschín}{Ruèras}{m5, l. 1227ff.}\\
\gll  Agl Andreòli vèva finju … las ... figuras … \textbf{avaun} \textbf{ca} la caplùta \textbf{èra} \textbf{stada} finida.\\
\textsc{def.art.m.sg} \textsc{pn} have.\textsc{impf.3sg} finish.\textsc{ptcp.unm} {} \textsc{def.art.f.pl} {} figure.\textsc{pl} {} before \textsc{rel} \textsc{def.art.f.sg} chapel be.\textsc{impf.3sg} \textsc{pass.aux.ptcp.f.sg} finish.\textsc{ptcp.f.sg} \\
\glt `Andreoli had finished ... the ... figures ... before the chapel was finished.'
\z

\ea
\label{ex:subj17}
\langinfo{Tuatschín}{Sadrún}{m6, l. 1329f.}\\
\gll    Qu’ è stau … mataj … gl' òn \textbf{avaun} \textbf{ca} \textbf{tgu} \textbf{mondi} … ál’ ampréma classa.\\
\textsc{dem.unm} be.\textsc{prs.3sg} \textsc{cop.ptcp.unm} {} probably {} \textsc{def.art.m.sg} year before \textsc{rel} \textsc{rel.1sg} go.\textsc{prs.sbjv.1sg} {} to.\textsc{def.art.f.sg} first form\\
\glt `This was … probably… the year before I attended … the first form [of primary school].'
\z

A similar hesitation between indicative and subjunctive can be observed with \textit{tòca} or \textit{tòca tga} `until'. In (\ref{ex:toca:subj}) \textit{tòca} triggers subjunctive and in (\ref{ex:toca:ind}) \textit{tòca tga} triggers indicative.

\ea
\label{ex:toca:subj}
\langinfo{Tuatschín}{Sadrún}{f6, l. 740ff.}\\
\gll    Api èra la sòra òra uschéja … avaun niaṣ ésch ad \textbf{ò} \textbf{spatgau} a spatgau \textbf{tòca} la \textbf{audi} anzatgéj [...].\\
and \textsc{cop.impf.3sg} \textsc{def.art.f.sg} nun out so {} in\_front\_of \textsc{poss.1pl.m.sg} door and have.\textsc{prs.3sg} wait.\textsc{ptcp.unm} and wait.\textsc{ptcp.unm} until \textsc{3sg.f} hear.\textsc{prs.sbjv.3sg} something\\
\glt `And then the nun was out [on the corridor] like this ... in front of our door, waiting and waiting until she would hear something [...].'
\z

\ea
\label{ex:toca:ind}
\langinfo{Tuatschín}{Ruèras}{m3, l. 2142ff.}\\
	\gll  Anqual jèda vagnéva lu al pás[tar] … né usché cu aj vasévan a gidavan \textbf{tòca} \textbf{tg}’ ins \textbf{èr}’ ajn … ajn «ṣchwung» [...].\\
some time.\textsc{f.sg} \textsc{come.impf.3sg} then \textsc{def.art.m.sg} herdsman {} or so when \textsc{3pl} see.\textsc{impf.3pl} and help.\textsc{impf.3pl} until \textsc{comp} \textsc{gnr} \textsc{cop.impf.3sg} in {} in momentum.\textsc{m.sg}\\
\glt `Sometimes the herdsman would come ... or so, when they saw and they would help until one was again in momentum [...].'
\z

In the corpus, \textit{sènza tga} `without' only occurs with subjunctive.

\ea
\label{}
\langinfo{Tuatschín}{Ruèras}{f7, l. 1750f.}\\
\gll  [...] èla savèv’ è í vidò gljunsch á paj \textbf{sènza} \textbf{tgu} \textbf{stòpi} tumaj tga la mòndi á funs. \\
{} \textsc{3sg.f} can.\textsc{impf.3sg} also go.\textsc{inf} over\_out far on foot.\textsc{m.sg} without \textsc{comp.1sg} must.\textsc{prs.sbjv.1sg} fear.\textsc{inf} \textsc{comp} \textsc{3sg.f} go.\textsc{prs.sbjv.3sg} to ground.\textsc{m.sg}\\
\glt `[...] she could go far on foot without me having to be afraid that she could drown.'
\z

\ea
\label{}
\langinfo{Tuatschín}{Bugnaj} {\citealt[132]{Büchli1966}}\\
\gll Èla végn sjantar \textbf{sènza} tga nus \textbf{lajan}.\\
\textsc{3sg.f} come.\textsc{prs.3sg} after without \textsc{comp} \textsc{1pl} want.\textsc{prs.sbjv.1pl}\\
\glt `She follows us without us wanting [it].'
\z

If a subordinate clause depends on a clause whose verb occurs in subjunctive mood, the clause which normally dos not take subjunctive takes it by syntactic attraction. An example is (\ref{ex:subj:synt.attr}), where the subjunctive occurs in the relative clause which normally requires indicative.

\ea
\label{ex:subj:synt.attr}
\langinfo{Tuatschín}{Zarcúns}{m2, l. 1580ff.}\\
	\gll    [...] quaj \textbf{fagèva} las gjufnas lu schòn \textbf{stém} \textbf{sch}’ i \textbf{vajan} sé la nègla tg’ èla \textbf{vaj} dau né bétg.\\
{} \textsc{dem.unm} do.\textsc{impf.3sg} \textsc{def.art.f.pl} young\_woman.\textsc{pl} then in\_fact attention.\textsc{m.sg} if \textsc{3pl}  have.\textsc{sbjv.prs.3pl} up \textsc{indef.art.f.sg} carnation \textsc{rel} \textsc{3sg.f} have.\textsc{sbjv.prs.3sg}  give.\textsc{ptcp.unm} or \textsc{neg} \\
\glt `[...] the young women would pay close attention to whether they had put on the hat the carnation they had given them or not.'
\z

There are some cases where conditional is used instead of subjunctive (\ref{ex:condforsubj}).


\ea
\label{ex:condforsubj}
\langinfo{Tuatschín}{Tschamùt} {\citealt[132]{Büchli1966}}\\
\gll Quèl lèva bétga \textbf{craj} tga `ls tiars \textbf{rasdassan} da Nadal-nòtg durònt mèssa [...].\\
\textsc{dem.m.sg} want.\textsc{impf.3sg} \textsc{neg} believe.\textsc{inf} \textsc{comp} \textsc{def.art.m.pl} animal.\textsc{pl} speak.\textsc{cond.3pl} of Christmas-night.\textsc{f.sg} during mass.\textsc{f.sg}\\
\glt `He didn't want to believe that the animals speak during mass on Christmas Eve [...].'
\z

Following examples: sent a mail to Matthias Grünert 2020/05/20

Matthias' answer:

Il conjunctiv en construcziuns cumpletivas suenter saver (senza negaziun) ei bein documentaus en sursilvan, surtut sche saver vegn duvraus egl imperfect. Il fatg presentaus ella construcziun subordinada vegn mess en ina dependenza pli ferma dalla perspectiva dil protagonist. Mira ils suandonts exempels ch’jeu hai citau en mia dissertaziun (Modussyntax im Surselvischen, Basel/Tübingen: Francke 2003):

p. 439
El tschentava adina puspei quella damonda desperada e savevach’el survegnimai la risposta. (TUOR 1988:120)

p. 440
... adina savev’jeu ch’jeu vegnianavos en Svizra, mo che quei New York era aschi fascinont che en in onn ves’ins buc tut, aschia che quei ei vegniu quater onns e miez ordlunder. (Profil 90/57)

Ils onns avon eran nus disai da gudignar savens. Uonn savevannus che quei seigibuca pli pusseivel. (La Quotidiana 210 [1997]:7)

Jeu savevelgia il december che jeu possediin calzer ch’ei optimals per mei. (La Quotidiana 4 [1997]:9)

L’autra construcziun che ti citeschas para a mi ualti particulara. Tenor il senn dalla passascha savess “lèza savèva ròmòntsch” esser la construcziun d’object che savess depender da “api vèvan nuṣ anflau òra”, pia:

api vèvan nuṣ anflau òra tga [=che] lèza savèva ròmòntsch.

En ina tala construcziun subordinada fuss il conjunctiv pusseivels:

Gia il pievel egipzian haveva anflau ora che mèl d’aviuls seigi in remiedi. (04.05.2018 / La Quotidiana)

… il recent studi d’Economiesuisse che ha anflau ora che 20 procent dalla populaziun paghien 35 procent dallas prestaziuns dil stadi. (18.09.2007 / La Quotidiana)

Che propi la construcziun cundiziunala (u temporala?) “scha nus mòndian a séjṣian spèr la sòr’ Andréa” ei el conjunctiv, ei denton surprendent.

Zaco dat ei cheu ina cruschada: anflar ora provochescha il conjunctiv, denton buc en ina construcziun d’object, nua ch’ins spitgass el, mobein en ina subordinada adverbiala ch’ei pli datier dil verb anflar ora. 

---


\ea\label{}
\langinfo{Tuatschín}{Sadrún}{f6, l. 797ff.}\\
\gll   [...] api vèvan nuṣ anflau òra scha nus \textbf{mòndjan} a \textbf{séjṣian} spèr la sòr’ Andréa, lèza savèva ròmòntsch.\\
{} and have.\textsc{impf.1pl} \textsc{1pl}  find.\textsc{ptcp.unm} out if \textsc{1pl} go.\textsc{prs.sbjv.1pl} and sit.\textsc{prs.sbjv.1pl} next  \textsc{def.art.f.sg} nun \textsc{pn} \textsc{dem.f.sg} know.\textsc{impf.3sg} Romansh.\textsc{m.sg}\\
\glt `[...] and then we had found out that if we went to sit next to Sister Andrea, she knew Romansh.'
\z

\ea
\label{}
\langinfo{Tuatschín}{Sadrún}{f6, l. 717ff.}\\
\gll    A las sòras \textbf{savèvan} tga nus trajs nus \textbf{vagjan} adina u-léjgar, a nus \textbf{mondian} bugèn cò gjù à scùla, a nus \textbf{fètschjan} filistùcas [...].\\
and \textsc{def.art.f.pl} nun.\textsc{pl} know.\textsc{impf.3pl} \textsc{comp} \textsc{1pl} three \textsc{1pl} have.\textsc{prs.sbjv.1pl} always \textsc{elat}-funny.\textsc{adj.unm} and \textsc{1pl} go.\textsc{prs.sbjv.1pl} with\_pleasure here down to school.\textsc{f.sg} and \textsc{1pl} do.\textsc{prs.sbjv.1pl} prank.\textsc{pl}\\
\glt `And the nuns knew that the three of us, we always had fun, and that we liked to come to school down here, and that we used to play pranks [...].'
\z

\paragraph{Imperfect subjunctive}
Imperfect subjunctive is very rare in the corpus, where it only occurs with speech act verbs (examples \ref{ex:subjimpf1} and \ref{ex:subjimpf2}) and verbs of opinion (\ref{ex:subjimpf3}).

\ea
\label{ex:subjimpf1}
\langinfo{Tuatschín}{Sadrún}{m4, l. 457ff.}\\
\gll Préndar ajn, pù schòn èssar tga samidav’ al grép, tga …  a dí di la …, la détga di tg’ \textbf{èrian} schindanajn tg' i \textbf{udévian} ch’ i tu- \textbf{tucavi} da mjazdé ajnt Ruèras.\\
take.\textsc{inf} in can.\textsc{prs.3sg} well be.\textsc{inf}  \textsc{comp} \textsc{refl}.change.\textsc{impf.3sg} \textsc{def.art.m.sg} rock \textsc{comp} {} and say.\textsc{inf} say.\textsc{prs.3sg} \textsc{def.art.f.sg} {} \textsc{def.art.f.sg} legend say.\textsc{prs.3sg}  \textsc{comp} \textsc{cop.impf.sbjv.3pl} so\_in \textsc{comp} \textsc{3pl} hear.\textsc{impf.sbjv.3pl} \textsc{comp} \textsc{expl} beat- beat.\textsc{impf.sbjv.3sg} of noon.\textsc{m.sg} in \textsc{pln}\\
\glt `As for mining, it could well be that the rock changed, that …, the legend says that they were so deeply in the cave that they heard that the clock pealed at noon in Rueras.'
\z

\ea\label{ex:subjimpf2}
\langinfo{Tuatschín}{Sadrún}{m10, l. 1139ff.}\\
\gll  Èl \textbf{èri} avaun caplùta a vagi vju tga quèls méls èn saspuantaj, api vagi èl tartgau ... dad í vi ajn via ... a tanaj sé èls.  \\
\textsc{3sg.m} \textsc{cop.impf.sbjv.3sg} in\_front chapel and have.\textsc{prs.sbjv.3sg} see.\textsc{ptcp.unm} \textsc{comp} \textsc{dem.m.pl} mule.\textsc{pl} be.\textsc{prs.3pl} \textsc{refl.}frighten.\textsc{ptcp.m.pl} and have.\textsc{prs.sbjv.3sg} \textsc{3sg.m} think.\textsc{ptcp.unm} {} \textsc{comp} go.\textsc{inf} over on road.\textsc{f.sg} {} and hold.\textsc{inf} up \textsc{3pl.m}\\
\glt `[The priest said that] He was in front of the chapel and had seen that these mules ran away and he thought ... that he would go on the road ... and stop them.'
\z

\ea
\label{ex:subjimpf3}
\langinfo{Tuatschín}{Ruèras}{m3, l. 2114f.}\\
\gll Ju \textbf{craj} tgu \textbf{vèvi} òtg vacas [...].\\
\textsc{1sg} believe.\textsc{prs.1sg} \textsc{comp.1sg} have.\textsc{sbjv.impf.1sg} eight cow.\textsc{f.pl}\\
\glt `I think I had eight cows [...].'
\z

\paragraph{Direct and indirect conditional}
The direct conditional mostly occurs in conditional sentences, in the protasis as well as in the apodosis. The apodosis is often not expressed overtly. The direct conditional has a simple and a compound form. The simple form expresses present counterfactuality.

\ea\label{ex:conddir1}
\langinfo{Tuatschín}{Sadrún}{m4, l. 437ff.}\\
\gll A lu Prdatsch … plénansé cò ancúntar Tgòm … ṣaj ina rùsna, quaj \textbf{fùṣ} è aun intarassant \textbf{sch’} ins \textbf{savés}, quaj datèscha da gl' òn ju … a ussa bigja grat prèsèn, méli a sistschian a zatgéj.\\
and then \textsc{pln} {} more\_uphill here in\_direction \textsc{pln} {} \textsc{cop.prs.3sg} \textsc{indef.art.f.sg} hole \textsc{dem.unm} \textsc{cop.cond.3sg} also indeed interesting.\textsc{m.unm} if \textsc{gnr} know.\textsc{cond.3sg} \textsc{dem.unm} date.\textsc{prs.3sg} from \textsc{def.art.m.sg} year \textsc{1sg} {} have.\textsc{prs.1sg} now \textsc{neg} just present thousand and six\_hundred and something\\
\glt `And then Pardatsch … a bit more uphill here in direction of Tgom … there is a cave, it would indeed be interesting if one knew that this is dated, I … dont' have it exactly in mind, thousand six hundred something.'
\z

\ea
\label{ex:conddir2}
\langinfo{Tuatschín}{Sadrún}{m4, l. 402f.}\\
\gll  «Té \textbf{savèssaṣ} í cul tat ajn Pardatsch.»  \\
\textsc{2sg} can.\textsc{cond.2sg} go.\textsc{inf} with.\textsc{def.art.m.sg} grandfather up \textsc{pln}  \\
\glt `«You could go up to Pardatsch with your grandfather.»'
\z

The compound form expresses counterfactuality in the past.

\ea
\label{ex:conddir3}
\langinfo{Tuatschín}{Camischùlas}{f6, l. 787ff.}\\
\gll    A … api parví dal ròmòntsch èri á Cazis scha …, a … nus astgèvan bégja raṣdá ròmòntsch, inṣ \textbf{vèṣ} gè \textbf{savju} dá la bùca ṣur dlas sòras.\\
and {} and because of.\textsc{def.art.m.sg} Romansh \textsc{cop.impf.3sg.expl} in \textsc{pln} if {} and {} \textsc{1pl} be\_allowed.\textsc{impf.1pl} \textsc{neg} speak.\textsc{inf} Romansh.\textsc{m.sg} \textsc{gnr}  have.\textsc{cond.3sg} after\_all can.\textsc{ptcp.unm} give.\textsc{inf} \textsc{def.art.f.sg} mouth over of.\textsc{def.art.f.pl} nun.\textsc{pl}\\
\glt `And … and as for Romansh, in Cazas it was if …, and … we were not allowed to speak Romansh, as a matter of fact one could have made derisive remarks about the nuns.'
\z

\ea
\label{ex:conddir4}
\langinfo{Tuatschín}{Sadrún}{f2, l. 950f.}\\
\gll  [...] ju \textbf{fùṣ} ina sèra maj \textbf{id’} ò da tgèṣa la sèra da stgir. \\
{} \textsc{1sg} be.\textsc{cond.1sg} \textsc{indef.art.f.sg} evening never go.\textsc{ptcp.f.sg} out of home.\textsc{f.sg} \textsc{def.art.f.sg} evening of dark.\textsc{m.unm}\\
\glt `[...] I would never have left home in the evening when it was dark.'
\z

As examples (\ref{ex:conddir1}) to (\ref{ex:conddir4}) show, the final \textit{-s} of the singular persons and of the second person plural of the direct conditional is realised [ṣ] if it is followed by a vowel without a pause as is the case with all forms of the verbal paradigms that end in -s. Note, however, that if the conditional is followed by the expletive pronoun or the pronoun of the third person plural which is not marked for gender, both \textit{i}, the ending of the conditional is pronounced [i] as in \textit{duèss-i} `should.\textsc{cond.3sg}-\textsc{expl}', line 1218 in § 9.8 below.


Examples (\ref{ex:cond.indir1}) and (\ref{ex:cond.indir2}) illustrate the indirect conditional, which occurs in object clauses that are governed by a speech act verb like \textit{dumandá} `ask' or \textit{dí} `say'.

\ea
\label{ex:cond.indir1}
\langinfo{Tuatschín}{Ruèras}{m10, l. 1138f.}\\
\gll  [...]  a … lu vajn nuṣ, va ju dumandau sch’ èl \textbf{prèndèssi} mè tòcan … á Ruèras. \\
{} and {} then have.\textsc{prs.1pl} \textsc{1pl} have.\textsc{prs.1sg}  \textsc{1sg} ask.\textsc{ptcp.m.unm} if \textsc{3sg.m} take.\textsc{cond.indir.3sg} \textsc{1sg} until {} to \textsc{pln}\\
\glt `[...] and … then we, I asked whether he could take me down to Rueras.'
\z

\ea
\label{ex:cond.indir2}
\langinfo{Tuatschín}{Sadrún}{f3, l. 23ff.}\\
\gll  [...] api lu va ju … tlafònau dad èl a détg, éba, mi' ùm ségi èba mòrts scù i sápjan, ábar … ju \textbf{fagèssi} ugèn vinavaun quèla lavur, api ò `l détg:  \\
{} and then have.\textsc{prs.1sg} \textsc{1sg} {} call.\textsc{ptcp.m.unm} \textsc{dat} \textsc{3sg.m} and say.\textsc{ptcp.m.unm} exactly \textsc{poss.1sg.m.sg} man be.\textsc{prs.sbjv.3sg} precisely die.\textsc{ptcp.m.sg} as \textsc{3pl} know.\textsc{prs.sbjv.3pl} but {} \textsc{1sg} do.\textsc{cond.indir.1sg} with\_pleasure still \textsc{dem.f.sg} job and have.\textsc{prs.3sg} \textsc{3sg.m} say.\textsc{ptcp.m.unm}\\ 
\glt `[...] and then I … phoned him and said that my husband had died as he knew, but … that I would like to keep doing this job [...].'
\z

\paragraph{Tense agreement}
In contrast to other Romance varieties, Tuatschin has no tense agreement. In object sentences, it is always the tense that would occur in direct speech which is used. This is probably connected to the fact that Tuatschin (and Sursilvan in general) uses subjunctive after direct speech act verbs or verbs of opinion, be they affirmative or negated, also in contrast to other Romance varieties. An example is (\ref{ex:notenseagr}).

\ea
\label{ex:notenseagr}
\langinfo{Tuatschín}{Sadrún}{f3, l. 24f.}\\
\gll  [...] api lu va ju … tlafònau dad èl a détg, éba, mi' ùm \textbf{ségi} èba \textbf{mòrts} [...]. \\
{} and then have.\textsc{prs.1sg} \textsc{1sg} {} call.\textsc{ptcp.unm} \textsc{dat} \textsc{3sg.m} and say.\textsc{ptcp.unm} exactly \textsc{poss.1sg.m.sg} man be.\textsc{prs.sbjv.3sg} precisely die.\textsc{ptcp.m.sg} \\ 
\glt `[...] and then I … phoned him and said that my husband had died [...].'
\z

In this example, perfect subjunctive is used (\textit{ségi mòrts}) and not pluperfect subjunctive (\textit{*èri mòrts}), which does not occur in the corpus. In direct speech, perfect indicative would be used: «Mi' ùm \textit{è mòrts}.» `My husband has died.'

\paragraph{The construction \textit{vaj} \textit{tga} `have that'}
It has not been possible to determine the exact function of \textit{vaj tga} (\ref{ex:vajtga1} - \ref{ex:vajtga2}) or \textit{végn tga} (\ref{ex:vegntga1}, but the examples suggest that the construction focalises on the present relevance of the event the verbs refers to (\ref{ex:vajtga1} and \ref{ex:vegntga1}) or on past habitual (\ref{ex:vajtga2} and \ref{ex:vajtga3}).

\ea
\label{ex:vajtga1}
\langinfo{Tuataschin}{Ruèras}{\DRG{2}{397}}\\
\gll  Ju \textbf{a} \textbf{tga} fò bléd.\\
\textsc{1sg} have.\textsc{prs.3sg} \textsc{rel} do.\textsc{prs.3sg} sick\\
\glt `I feel sick.'
\z

\ea
\label{ex:vajtga2}
\langinfo{Tuatschín}{Camischùlas}{f6, l. 822f.}\\
\gll    A lu \textbf{vèvan} nus lò \textbf{tga} nus astgèvan raṣdá ròmòntsch.\\
and then have.\textsc{impf.1pl} \textsc{1pl} there \textsc{comp} \textsc{1pl} be\_allowed.\textsc{impf.1pl} speak.\textsc{inf} Romansh.\textsc{m.sg} \\
\glt `And then we had the opportunity to be allowed to speak Romansh there.'
\z

\ea
\label{ex:vajtga3}
\langinfo{Tuatschín}{Sadrún}{m9, l. 1894}\\
\gll  [...] api sjantar mav’ ins lu ... ségi quaj ajn quèla bar né ajn tschèla … nùca tg’ i \textbf{vèva} lu \textbf{tga} trèva ... dad ira.  \\
{} and after go.\textsc{impf.3sg} \textsc{gnr} then {} \textsc{cop.prs.sbjv.3sg} \textsc{dem.unm} in \textsc{dem.f.sg} bar or in \textsc{dem.f.sg} {} where \textsc{rel} \textsc{expl} have.\textsc{impf.3sg} then \textsc{comp} pull.\textsc{impf.3sg} {} \textsc{comp} go.\textsc{inf}\\
\glt `[...] and then we would go into this bar or into that one ... wherever it drew us to go.'
\z

\ea
\label{ex:vegntga1}
\langinfo{Tuatschín}{}{\DRG{2}{215}}\\
\gll I briṣcha la cazèta, i \textbf{végn} \textbf{tga} sufla. \\
\textsc{expl} burn.\textsc{prs.3sg} \textsc{def.art.f.sg} pot \textsc{expl} come.\textsc{prs.3sg} \textsc{rel} blow.\textsc{prs.3sg} \\
\glt `[The soot] on the pot is burning, it is getting stormy.'
\z


\subsection{Particle verbs}
A particle verb is a verb that combines with an adverb to form a semantic unit. An example is \textit{fá gjù}, literally `make down', which means `make an appointment'. The origin of such structures is controversial: it is considered either a genuine Romansh structure, a loan from German or Swiss German, or both. \textit{fá gjù}, however, is a clear case of calque from Swiss German. In Swiss German, `make an appointment' is [ˈˈabˈmaxə]. In this lexeme, the prefix \textit{ab-} has been interpreted as [ˈabə] `down', hence \textit{gjù}, and [ˈmaxə] means 'do, make', which leads to the particle verb \textit{fá gjù}.

There is an important difference between the German and the Romansh construction: In German, standard or Swiss, the particle is a verbal prefix which in simple tenses is located at the end of the sentence, as in (\ref{ex:pcl:chg1}).

\ea\label{ex:pcl:chg1}
\langinfo{Swiss German}{}{own knowledge}\\
\gll  [ix \textbf{max} jedə ta:g mit im \textbf{ab}]\\
     \textsc{1sg}  make every day with him  \textsc{ptcl} \\
\glt `I make an appointment with him every day.'
\z

In such cases, the particle is adjacent to the verb in Tuatschin (and other Romansh varieties).

\ea\label{}
\langinfo{Tuatschín}{Sadrún}{m4}\\
\gll  Ju \textbf{fétsch} \textbf{gjù} cun èl mintga dé.\\
     \textsc{1sg}  make.\textsc{prs.1sg}  \textsc{ptcl} with \textsc{3sg.m}  every day\\
\glt `I make an appointment with him every day.'
\z

However, in Tuatschin and other Romansh varieties, the particle is not immediately adjacent to the verb, since some elements may intervene between the verb and the particle. These elements are inverted subjects (pronouns (\ref{ex:pv:1}) or full noun phrases (\ref{ex:pv:2})), the negator \textit{bétga} and its variants (\ref{ex:pv:3}), as well as other adverbs like \textit{aun} `still', \textit{è/èra} (\ref{ex:pv:4}) `also', \textit{lu} `then' \ref{ex:pv:4}, \textit{magari} `sometimes' (\ref{ex:pv:5}), \textit{maj} `never' (\ref{ex:pv:6}), \textit{pròpi} `exactly' (\ref{ex:pv:7}), \textit{puspè} `again' (\ref{ex:pv:8}), or \textit{schòn} `certainly' (\ref{ex:pv:9}).

\ea\label{ex:pv:1}
\langinfo{Tuatschín}{Sadrún}{m6}\\
\gll   Damaun prèn \textbf{ju} sé èl.\\
     tomorrow take.\textsc{prs.1sg} \textsc{1sg} up \textsc{3sg}\\
\glt `Tomorrow I will lift him up.'
\z

\ea\label{ex:pv:2}
\langinfo{Tuatschín}{}{\citealt[91]{Gadola1935}}\\
\gll  Té mù trafica usché vinavaun, api sièta \textbf{al} \textbf{malitèr} gjù té in dé.\\
     \textsc{2sg} just be\_up\_to.\textsc{prs.3sg} so further and shoot.\textsc{prs.3sg} \textsc{def.art.m.sg} army down \textsc{2sg} \textsc{indef.art.m.sg} day\\
\glt `You just go on behaving this way, and the army will shoot you down one day.' 
\z

\ea\label{ex:pv:3}
\langinfo{Tuatschín}{Sadrún}{m6}\\
\gll   Damaun prèn ju \textbf{bégja} sé èl.\\
     tomorrow take.\textsc{prs.1sg} \textsc{neg} \textsc{1sg} up \textsc{3sg}\\
\glt `Tomorrow I won’t lift him up.'
\z

\ea\label{ex:pv:4}
\langinfo{Tuatschín}{Zarcúns}{m2, l. 1650ff.}\\
	\gll    [...] a lu dèvani \textbf{lu} \textbf{è} sé da, da scrívar tòcs, tg’ ins stuèva scrívar … navèn dal ròm\underline{ò}ntsch … vi sél tudèstg … tga vagnéva curagju.\\
{} and then give.\textsc{3pl.3pl} then also up \textsc{comp} \textsc{comp} write.\textsc{inf} play.\textsc{m.pl} \textsc{rel} \textsc{gnr} must.\textsc{impf.3sg} write.\textsc{inf} {} from  of.\textsc{def.art.m.sg} Romansh {} over on.\textsc{def.art.m.sg} German {} \textsc{rel} \textsc{pass.aux.impf.3sg} correct.\textsc{ptcp.unm}\\

\glt `[...] and then they also gave [us as homework] to write plays which one had to translate … from Romansh to German … [and which] were corrected.'
\z

\ea
\label{ex:pv:5}
\langinfo{Tuatschín}{Sadrún}{m6}\\
\gll  Ju prèn \textbf{magari} sé èl.  \\
\textsc{1sg} take.\textsc{prs.1sg} sometimes up \textsc{3sg.m}\\
\glt `Sometimes I lift him up.'
\z

\ea
\label{ex:pv:6}
\langinfo{Tuatschín}{Sadrún}{m6}\\
\gll  Cun quèl fétschu \textbf{maj} gjù.  \\
with \textsc{dem.m.sg} make.\textsc{prs.1sg}.\textsc{1sg} never down\\
\glt `With this person I never make an appointment.'
\z

\ea
\label{ex:pv:7}
\langinfo{Tuatschín}{Sadrún}{m4, l. 453}\\
\gll   Aah, tgé .. prandévan \textbf{pròpi} òra sa ins bégj' éxáct [...]. \\
ah what {} take.\textsc{impf.3pl} exactly out know.\textsc{prs.3sg} \textsc{gnr} \textsc{neg} exactly\\
\glt `Ah, what … they really mined one does not know exactly [...].'
\z

\ea
\label{ex:pv:8}
\langinfo{Tuatschín}{Sadrún}{m9}\\
\gll  Prèn \textbf{puspè} sé quaj! \\
take.\textsc{imp.2sg} again up \textsc{dem.unm}  \\
\glt `Lift this again! '
\z

\ea
\label{ex:pv:9}
\langinfo{Tuatschín}{} {\citealt[51]{Berther1998}}\\
\gll  Al plé mal stùn ju pal bian cazè. Al pòlisch crèscha \textbf{schòn} ansjaman.\\
     \textsc{def.art.m.sg}  most bad stay.\textsc{prs.1sg}  \textsc{1sg} for.\textsc{def.art.m.sg} good shoe \textsc{def.art.m.sg} thumb grow.\textsc{prs.3sg}  certainly together\\
\glt `I am very sorry for the shoe of good quality. My big toe will certainly knit together again.'
\z

With the adverb \textit{savèns} `often' as well as with \textit{dabòt} and \textit{spèrt}, both `rapidly', the adverb may occur between the verb and its particle or not.

\ea
\label{ex:pv:10}
\langinfo{Tuatschín}{Sadrún}{m6, f1}\\
\gll    Ju prèn \textbf{savèns} sé èl.\\
     \textsc{1sg} take.\textsc{prs.1sg} often up \textsc{3sg.m}\\
\glt `I often lift him up.'
\z

\ea
\label{ex:pv:11}
\langinfo{Tuatschín}{Sadrún}{f1}\\
\gll    Quèl prènd ju sé \textbf{savèns}.\\
     \textsc{dem.m.sg} take.\textsc{prs.1sg} \textsc{1sg} up often\\
\glt `I often lift him up.'
\z

\ea
\label{ex:pv:12}
\langinfo{Tuatschín}{Sadrún}{m9}\\
\gll Èls prendan \textbf{spèrt} sé als ufauns.   \\
\textsc{3pl.m} take.\textsc{prs.3pl} rapidly up \textsc{def.art.m.pl} child.\textsc{pl}  \\
\glt `They lift the children rapidly.'
\z

\ea
\label{ex:pv:13}
\langinfo{Tuatschín}{Sadrún}{m9}\\
\gll Els prendan sé \textbf{spèrt} als ufauns.   \\
   \textsc{3pl} take.\textsc{prs.3pl} up rapidly \textsc{def.art.m.pl} child.\textsc{pl}  \\
\glt `They lift the children rapidly.'
\z

\ea
\label{ex:pv:14}
\langinfo{Tuatschín}{Ruèras}{m3}\\
\gll Ju prèn \textbf{aun} sé \textbf{spèrt} agl ufaun.    \\
    \textsc{1sg} take.\textsc{prs.1sg} still up rapidly \textsc{def.art.m.sg} child \\
\glt `Right now, I’ll lift the child rapidly.'
\z

\ea
\label{ex:pv:15}
\langinfo{Tuatschín}{Ruèras}{m3}\\
\gll Ju prèn \textbf{spèrt} \textbf{aun} sé agl ufaun.    \\
    \textsc{1sg} take.\textsc{prs.1sg} rapidly still up \textsc{def.art.m.sg} child \\
\glt `Right now, I’ll lift the child rapidly.'
\z

\ea
\label{ex:pv:16}
\langinfo{Tuatschín}{Ruèras}{m3}\\
\gll Ju prèn \textbf{aun} \textbf{dabòt} sé agl ufaun.    \\
    \textsc{1sg} take.\textsc{prs.1sg} rapidly still up \textsc{def.art.m.sg} child \\
\glt `Right now, I’ll lift the child rapidly.'
\z

The adverb \textit{mintgataun} `sometimes', which is a synonym of \textit{magari}, may not intervene between the verb and the particle.% Or do some consultants hesitate?

\ea
\label{17ex:pv:}
\langinfo{Tuatschín}{Sadrún}{m6}\\
\gll   *Ju prèn \textbf{mintgataun} sé èl. \\
     \textsc{1sg} take.\textsc{prs.1sg} sometimes up \textsc{3sg.m}\\
\glt `Sometimes I lift him up.'
\z

In contrast to German, pronouns may not stand between the verb and the particle: \textit{bétar navèn quaj} `throw away this' vs. \textit{*bétar quaj navèn} `throw this away', or \textit{prèndar sé èl} `lift up him' vs. \textit{*prèndar èl sé} 'lift him up'. A further example is (\ref{ex:schao}).

\ea
\label{ex:schao}
\langinfo{Tuatschín}{Sadrún}{m6, l. 1311f.}\\
	\gll    A nuṣ duṣ vèvan dad í á rimná quèls pòrs, prèndar òr, \textbf{schá} \textbf{ò} \textbf{èls}, ò da nuégl.\\
	and \textsc{1pl} two.\textsc{m.pl} have.\textsc{impf.1pl} \textsc{comp} go.\textsc{inf}  \textsc{purp} collect.\textsc{inf} \textsc{dem.m.pl} pig.\textsc{pl} take.\textsc{inf} out  let.\textsc{inf} out \textsc{3pl.m} out of barn.\textsc{m.sg}\\
\glt `And the two of us had to go and collect these pigs, take out, let them out, out of the barn.'
\z


\subsection{Copulative verbs}
Copulative verbs are \textit{èssar} `be', \textit{pará dad èssar} `seem', and the change of state verb \textit{vagní} `become'.

The copula \textit{èssar} `be' is a general copula which allows non-verbal elements to fulfil predicative functions, as e.g. nouns (\ref{ex:cop:1}), adjectives (\ref{ex:cop:2}), prepositional phrases in locative (\ref{ex:cop:3}), temporal (\ref{ex:cop:4}), or comitative (\ref{ex:cop:6}) function, or adverbs (\ref{ex:cop:5}).

\ea
\label{ex:cop:1}
\langinfo{Tuatschín}{Sadrún}{m6, l. 868}\\
	\gll    A qu’ \textbf{è} pròpi \textbf{ina} … pulit grònda \textbf{plata} [...].\\
	and \textsc{dem.unm} \textsc{cop.prs.3sg} precisely \textsc{indef.art.f.sg} {} very big slab\\
\glt `And this really is a … very big slab [...].'
\z

\ea
\label{ex:cop:2}
\langinfo{Tuatschín}{Sèlva}{f2, l. 938}\\
\gll Quaj \textbf{èra} in’ jèda ... \textbf{brutal} tiar nus, bèn-bèn.\\
\textsc{dem.unm} \textsc{cop.impf.3sg} one.\textsc{f.sg} time {} terrible.\textsc{adj.unm} among \textsc{1pl} \textsc{red}\textasciitilde{really}\\
\glt `Once it was terrible among us, really.'
\z

\ea
\label{ex:cop:3}
\langinfo{Tuatschín}{Ruèras}{m10, l. 1139f.}\\
\gll  Èl \textbf{èri} \textbf{avaun} \textbf{caplùta} a vagi vju tga quèls méls èn saspuantaj [...].\\
\textsc{3sg.m} \textsc{cop.impf.sbjv.3sg} in\_front chapel and have.\textsc{prs.sbjv.3sg} see.\textsc{ptcp.unm} \textsc{comp} \textsc{dem.m.pl} mule.\textsc{pl} be.\textsc{prs.3pl} \textsc{refl.}frighten.\textsc{ptcp.m.pl}\\
\glt `He was in front of the chapel and had seen that these mules ran away [...].'
\z

\ea
\label{ex:cop:4}
\langinfo{Tuatschín}{Sadrún}{m5, l. 1206}\\
\gll Quaj hanlétg da tiars, ál sén scù quaj tg' i \textbf{èra} \textbf{ál} \textbf{ṣchèniv\underline{á}val} \textbf{tschantanè} [...].\\
	\textsc{dem.m.sg} business of animal.\textsc{m.pl} in.\textsc{def.art.m.sg} sense like \textsc{dem.unm} \textsc{rel} \textsc{expl} \textsc{cop.impf.3sg} in.\textsc{def.art.m.sg} nineteenth century.\textsc{m.sg}\\
\glt`This cattle business, in the sense of how it was in the nineteenth century [...].'
\z

\ea
\label{ex:cop:6}
\langinfo{Tuatschín}{Sadrún}{m6, l. 1334ff.}\\
\gll    Ajn tuta cas mia, mia mùma a la mùma da mju còl\underline{è}ga tg’ \textbf{èra} è \textbf{cun} \textbf{mè}… vèvan stju gidá nus in pèr dis [...].\\
in every case.\textsc{m.sg} \textsc{poss.1sg.f.sg}  \textsc{poss.1sg.f.sg} mother and \textsc{def.art.f.sg} mother of  \textsc{poss.1sg.m.sg} mate \textsc{rel}  \textsc{cop.impf.3sg} also with \textsc{1sg}  have.\textsc{impf.3pl} must.\textsc{ptcp.unm} help.\textsc{inf} \textsc{1pl} \textsc{indef.art.m.sg} couple day.\textsc{m.pl}\\
\glt `Anyhow my, my mother and the mother of my mate who was with me … had had to help us a couple of days [...].'
\z

\ea
\label{ex:cop:5}
\langinfo{Tuatschín}{Surajn}{f5, l. 1290f.}\\
\gll    Gè scù bjè autar è tg’ è samidau. Quaj \textbf{è} \textbf{usché}.\\
yes as a\_lot other.\textsc{adj.unm} also \textsc{rel} be.\textsc{prs.3sg} \textsc{refl.}change.\textsc{ptcp.unm}  \textsc{dem.unm} \textsc{cop.prs.3sg} so\\
\glt `Yes, as many other things that also have changed. That’s how things are.'
\z

The following examples illustrate the functions of \textit{pará dad èssar} `seem' and \textit{vagní} `become'.

\ea
\label{ex:cop:7}
\langinfo{Tuatschín}{Sadrún}{m5}\\
\gll Èla \textbf{para} \textbf{dad} \textbf{èssar} stauncla.   \\
\textsc{3sg.f} seem.\textsc{prs.3sg} \textsc{comp} \textsc{cop.inf} tired.\textsc{f.sg} \\
\glt `She seems to be tired.'
\z

\ea
\label{ex:cop:8}
\langinfo{Tuatschín}{Bugnaj} {\citealt[145]{Büchli1966}}\\
\gll    Té stùs stá cò tùt parsula ad i \textbf{végn} \textbf{unviarn} a \textbf{végn} \textbf{frajd} […].\\
    \textsc{2sg} must.\textsc{prs.2sg} stay here all alone.\textsc{f.sg} and \textsc{expl} come.\textsc{prs.3sg} winter and come.\textsc{prs.3sg} cold.\textsc{adj.unm}  \\
\glt `You must stay here alone, and winter is coming and it is getting cold.'
\z

\ea
\label{ex:cop:9}
\langinfo{Tuatschín}{Sadrún}{m6, l. 1385f.}\\
\gll  [...] èl’ èra \textbf{vagnida} tùt \textbf{còtschna} [...].\\
{} \textsc{3sg.f} be.\textsc{impf.3sg} become.\textsc{ptcp.f.sg} completely red.\textsc{f.sg}\\
\glt `[...] she had turned completely red [...].'
\z

\subsection{Existential verbs}
Existential constructions are formed with the expletive pronoun \textit{i}, less frequently with \textit{quaj} `this', and the verbs \textit{èssar} `be' (\ref{ex:exist.essar1} to \ref{ex:exist.essar6}) or \textit{dá} `give' (\ref{{ex:exist.da1}} to \ref{{ex:exist.da6}}).

As is the case with \textit{quaj} (see § 3.2.2.1, examples \ref{ex:quaj:agrwithsubj}ff. above), the existential verb agrees with the expletive subject pronoun, but not with the predicative noun, which means that if the predicative noun is plural, the verb form is singular(\ref{ex:exist.essar1}).

\ea
\label{ex:exist.essar1}
\langinfo{Tuatschín}{Ruèras}{m3, l. 2219f.}\\
\gll [...] ancunt’ agl atún \textbf{èri} plé \textbf{paucs} \textbf{tiarṣ} [...].\\
 {} towards \textsc{def.art.m.sg} autumn \textsc{exist.impf.3sg.expl} more little.\textsc{m.pl} animal.\textsc{pl} \\
\glt `[...] towards autumn there were fewer animals [...].'
\z

\ea
\label{ex:exist.essar2}
\langinfo{Tuatschín}{Sadrún}{m4, l. 419ff.}\\
\gll [...] lu \textbf{ṣè} aun dus trajs intarassants lògans ajn cò [...].\\
{} then \textsc{exist.prs.3sg.expl} still two three interesting place.\textsc{m.pl} in here\\
\glt `[...] there are furthermore two or three interesting places up there [...].'
\z

\ea
\label{ex:exist.essar3}
\langinfo{Tuatschín}{Ruèras}{m10, l. 1048f.}\\
\gll   [...] a mintgataun \textbf{èri} aun grépa tga stèv’ ò in téc … . \\
{} and sometimes  \textsc{exist.impf.3sg.expl} moreover rock.\textsc{coll} \textsc{rel}  stand.\textsc{impf.3sg}  out \textsc{indef.art.m.sg} bit\\
\glt `[...] and from time to time there were rocks protruding a bit ... .'
\z

\ea
\label{ex:exist.essar4}
\langinfo{Tuatschín}{Sadrún}{m4, l. 419ff.}\\
\gll Ér \textbf{ṣaj} \textbf{stau} bjè turists sé lò.\\
yesterday be.\textsc{prs.3sg.expl} \textsc{cop.ptcp.unm} many tourist.\textsc{pl} up there\\
\glt `Yesterday there were many tourists up there.'
\z

\ea
\label{ex:exist.essar5}
\langinfo{Tuatschín}{Sèlva}{f2, l. 926f.}\\
\gll    Api plinèngjù, èri gl' unviarn, \textbf{qu’} \textbf{èra} baghétgs aun, ad \textbf{èra} pravasèdars [...].\\
and more\_down \textsc{cop.impf.3sg.expl} \textsc{def.art.m.sg} winter \textsc{dem.unm} \textsc{exist.impf.3sg} building.\textsc{m.pl} still and \textsc{exist.impf.3sg} herdsman.\textsc{m.pl} \\
\glt `And down there, it was winter, there were still buildings [there], and there were men who would feed the animals [...].'
\z

\ea
\label{ex:exist.essar6}
\langinfo{Tuatschín}{Ruèras}{m1, l. 252f.}\\
\gll    A \textbf{quaj} \textbf{èra} mù in ganc tras. A quaj gang udéva dad òmasdús.\\
and \textsc{dem.unm} \textsc{exist.impf.3sg} only one.\textsc{m.sg} corridor through and \textsc{dem.m.sg} corridor belong.\textsc{impf.3sg} to both.\textsc{m.pl}\\
\glt `And there was only one corridor. And this corridor belonged to both [families].'
\z

\ea
\label{ex:exist.da1}
\langinfo{Tuatschín}{Sadrún}{m4, l. 461f.}\\
\gll  Quaj è da dubitá, quaj è mataj scù quaj tga … \textbf{i} \textbf{dat} aun bjè da quèlas détgas.  \\
\textsc{dem.unm}  \textsc{cop.prs.3sg} \textsc{comp} doubt.\textsc{inf} \textsc{dem.unm} \textsc{cop.prs.3sg} probably like \textsc{dem.unm} \textsc{rel} {} \textsc{expl} \textsc{exist.prs.3sg} still many of  \textsc{dem.f.pl} legend.\textsc{pl}\\
\glt `This one has to doubt, this is probably like what … there still are many such legends.'
\z

\ea
\label{ex:exist.da2}
\langinfo{Tuatschín}{Sadrún}{m6, l. 1428ff.}\\
\gll    [...] \textbf{i} \textbf{dat} ina fòtògrafia tgu sùn sé cun miu còlèga al dé da la scargèda [...].\\
{} \textsc{expl} \textsc{exist.prs.3sg}  \textsc{indef.art.f.sg} photograph \textsc{rel.1sg}  \textsc{cop.prs.1sg} on with \textsc{poss.1sg.m.sg} mate \textsc{def.art.m.sg} day of  \textsc{def.art.f.sg}  drove.\textsc{ptcp.f.sg}\\
\glt `But this was mi first job, and all are very proud and there is a photograph on which I am with my mate the day of the pig driving [...].'
\z

\ea
\label{ex:exist.da3}
\langinfo{Tuatschín}{Ruèras}{m1, l. 274ff.}\\
\gll    Las nòt\underline{í}zjas sa ju bétg danùndar als gjaniturs, als dus baps prandèvan aj, \textbf{i} \textbf{dèva} ajnta Ruèras, dèv’ aj in ca vèva r\underline{á}djò. \\
\textsc{def.art.f.pl} news.\textsc{pl} know.\textsc{prs.1sg} \textsc{1sg} \textsc{neg} from\_where \textsc{def.art.m.pl} parents.\textsc{pl} \textsc{def.art.m.pl} two.\textsc{m.pl} father.\textsc{pl} take.\textsc{impf.3pl} \textsc{3sg} \textsc{expl} \textsc{exist.impf.3sg} in \textsc{pln} \textsc{exist.impf.3sg} \textsc{expl}  one.\textsc{m.sg} \textsc{rel} have.\textsc{impf.3sg} radio.\textsc{m.sg}\\
\glt `I don’t know where my parents had the news from, the two fathers took them, there was in Rueras, there was [only] one who had a radio.'
\z

\ea
\label{ex:exist.da4}
\langinfo{Tuatschín}{Sadrún}{m4, l. 511ff.}\\
\gll  Quaj sch’ \textbf{i} \textbf{dèva} rèsts scha vagnévi lu magari rimnau quaj dus trajs dis api méz tùt ajn ina… tùt ansjaman.  \\
well if \textsc{expl} \textsc{exist.impf.3sg} leftovers.\textsc{m.pl} then \textsc{pass.aux.impf.3sg.expl} then sometimes collect.\textsc{ptcp.unm} \textsc{dem.unm} two.\textsc{m.pl} three day.\textsc{pl} and put.\textsc{ptcp.m.pl} all in  \textsc{indef.art.f.sg} all together\\
\glt `Well, when there were leftovers, they would sometimes be collected during two or three days and then all put together in a … all together.'
\z

\ea
\label{ex:exist.da5}
\langinfo{Tuatschín}{Sadrún}{m6, l. 882f.}\\
\gll    A lu al pròxim, in dls pròximṣ unvjarns ... \textbf{òi} \textbf{dau} ina grònda navada [...].\\
and then \textsc{def.art.m.sg} next one of.\textsc{def.art.m.pl} next.\textsc{pl} winter.\textsc{pl} {} have.\textsc{prs.3sg}.\textsc{expl} \textsc{exist.ptcp.unm} \textsc{indef.art.f.sg} big snowfall \\
\glt `And then the next, one of the next winters … there was a big snowfall [...].'
\z

\ea
\label{ex:exist.da6}
\langinfo{Tuatschín}{Sadrún}{m6, l. 891f.}\\
\gll    Quaj èra ju gjù ina grònda lavina … a vèva … déstruí ina grònda part dl vitg ajntadém Ruèras … a \textbf{vèv’} è \textbf{dau} mòrts [...].\\
\textsc{dem.unm} be.\textsc{impf.3sg} go.\textsc{ptcp.unm} down \textsc{indef.art.f.sg} big avalanche {} and have.\textsc{impf.3sg} {} destroy.\textsc{ptcp.unm} \textsc{indef.art.f.sg} huge part of.\textsc{def.art.m.sg} village uppermost \textsc{pln} {} and have.\textsc{impf.3sg} also \textsc{exist.ptcp.unm} dead.\textsc{m.pl}\\
\glt `Then a huge avalanche went down … and had … destroyed a big part of the village in the upper part of Rueras … and there had also been dead persons [...].'
\z

Examples (\ref{ex:exist:daessar1}) and (\ref{ex:exist:daessar2}) show the occurrence of \textit{dá} `give' and \textit{èssar} `be' in the same context. Furthermore, (\ref{ex:exist:daessar1}) also contains two examples with the expletive pronoun \textit{i} and two examples without it.

\ea
\label{ex:exist:daessar1}
\langinfo{Tuatschín}{Sadrún}{m5, l. 1241ff.}\\
\gll Álṣò \textbf{i} \textbf{dèva} òns nùca tga gudignavan ... nùndétg, ad \textbf{i} \textbf{ṣèra} òns nùca tg’ \textbf{èra} awa, ad \textbf{èra} òns nùca tga spardévan.\\
well \textsc{expl} \textsc{exist.impf.3sg} year.\textsc{m.pl} where \textsc{rel} earn.\textsc{impf.3pl} {} incredibly and \textsc{expl} \textsc{exist.impf.3sg} year.\textsc{m.pl} where \textsc{rel} \textsc{exist.impf.3sg} water and \textsc{exist.impf.3sg} year.\textsc{m.pl} where \textsc{rel} lose.\textsc{impf.3pl} \\
\glt `Well, there were years when they earned ... a lot of money, and there were years with rain, and precisely years when they would lose money.'
\z

\ea
\label{ex:exist:daessar2}
\langinfo{Tuatschín}{Sadrún}{m5, l. 1248ff.}\\
\gll [...] a lu \textbf{èri} è quèls prígals tga \textbf{dèva} [...] sén ira, naturálmajn cu i vagnévan anavùs [...].\\
{} and then \textsc{exist.impf.3sg.expl} also \textsc{dem.m.pl} danger \textsc{rel} \textsc{exist.impf.3sg} {} on go.\textsc{inf} natural.\textsc{f.sg.adv} when \textsc{3pl} come.\textsc{impf.3pl} back\\
\glt `[...] and then there were these dangers which [one encountered] when travelling, of course when they would come back [...].'
\z



\subsection{Modal verbs}
The following modal verbs occur in the corpus: \textit{astgè} `be allowed', \textit{duaj} `must, should', \textit{èssar da} `must, have to' \textit{munglá} `must', \textit{pudaj} `can, be able', \textit{savaj} `can', \textit{schè/schá} `let, allow', \textit{stuaj} `must, have to', \textit{vaj da} `have to', and \textit{vulaj} `want'.

Obligation is expressed by \textit{èssar da} `must, have to' (\ref{ex:essarda1}), \textit{duaj} `must, should' (\ref{ex:duaj1}), \textit{munglá} `must' (\ref{ex:mung1})\footnote{Nowadays \textit{munglá} is only used with conditional mood.}, \textit{vaj da} `have to' (\ref{ex:vajda1} and \ref{ex:vajda2}), and \textit{stuaj} `must, have to' (\ref{ex:stuaj1} and \ref{ex:stuaj2}). Note that \textit{èssar da} is impersonal.

\ea
\label{ex:essarda1}
\langinfo{Tuatschín}{Ruèras}{f4, l. 1968}\\
	\gll [...] api \textbf{èri} \textbf{da} \textbf{fá} \textbf{fajn}.\\
{} and be.\textsc{impf.3sg.expl} \textsc{comp} do.\textsc{inf} hay.\textsc{m.sg}\\
\glt `[...] and then one had to do hay.'
\z

\ea
\label{ex:duaj1}
\langinfo{Tuatschín}{Sadrún}{f3, l. 158ff.}\\
\gll Gè grat uschéja, ábar ju, ina da la natira ad a adina … vulju fá mia òbligazjun a finju, a tschèls \textbf{dajan} è fá.\\
yes exactly so but \textsc{1sg} one.\textsc{f.sg} of \textsc{def.art.f.sg} nature and have.\textsc{prs.3sg} always {} want.\textsc{ptcp.unm} do.\textsc{inf} \textsc{poss.1sg.f.sg} obligation and finish.\textsc{ptcp.unm} and \textsc{dem.m.pl} must.\textsc{prs.3pl} also do.\textsc{inf}\\
\glt `Yes, exactly like that. But I, a person who likes nature, and I have always … wanted to meet my obligations, and the other persons should also do [the same].'
\z

\ea\label{ex:mung1}
\langinfo{Tuatschín}{Cavòrgja}{f1}\\
\gll  Ju \textbf{munglaṣ} ir' á tgèsa.\\
\textsc{1sgl}  must.\textsc{cond.1sg} go.\textsc{inf} to house.\textsc{f.sg} \\
\glt `I should go home.'
\z

\ea\label{ex:vajda1}
\langinfo{Tuatschín}{Sadrún}{m6, l. 1413f.}\\
\gll    A tiar in pur… èn nuṣ í, èr’ ju, \textbf{vèv}’ ju \textbf{gju} \textbf{dad} incassá quèls raps [...].\\ 
and by \textsc{indef.art.m.sg} farmer  be.\textsc{prs.1sg} \textsc{1pl} go.\textsc{ptcp.m.pl}  be.\textsc{impf.1sg} \textsc{1sg}  have.\textsc{impf.1sg} \textsc{1sg} have.\textsc{ptcp.unm} \textsc{comp} collect.\textsc{inf}  \textsc{dem.m.pl} cent.\textsc{pl}\\
\glt `And we went … to a farmer, I was, I had to collect this money [...]'
\z


The infinitive does not have to be adjacent to to the modal verb \textit{vaj} (examples \ref{ex:vajda2} and \ref{ex:vajda3}). 

\ea
\label{ex:vajda2}
\langinfo{Tuatschín}{Sadrún}{m4, l. 486f.}\\
\gll  Vus vèssas \textbf{lu} \textbf{aun} da fá quèls bogns né mirá da la plaja.\\
\textsc{2sg.pol}  have.\textsc{cond.2pl} then still \textsc{comp} do.\textsc{inf} \textsc{dem.m.pl} bath.\textsc{pl} or look\_after.\textsc{inf} of \textsc{def.art.f.sg} wound\\
\glt `You  still should take these baths or look after the wound.'
\z

\ea
\label{ex:vajda3}
\langinfo{Tuatschín}{Ruèras}{f4, l. 2047f.}\\
\gll A sch’ i èra malaura, scha vèv’ inṣ \textbf{á} \textbf{tgèsa} da fá [...].\\
and if \textsc{expl} \textsc{cop.impf.3sg} bad\_weather.\textsc{f.sg} \textsc{corr} have.\textsc{impf.3sg} \textsc{gnr} at house.\textsc{f.sg} \textsc{comp} do.\textsc{inf} \\
\glt `And if the weather was bad, we, the women, had to work in the house [...].'
\z

\ea
\label{ex:stuaj1}
\langinfo{Tuatschín}{Sadrún}{m4, l. 518ff.}\\
\gll  [...] l’ antschata da la parmavèra \textbf{stuèv}’ ins schè ajn als tiars ajn nuégl [...]. \\
{} \textsc{def.art.f.sg} beginning of \textsc{def.art.f.sg} spring must.\textsc{impf.3sg} \textsc{gnr} let.\textsc{inf} in \textsc{def.art.m.pl} animal.\textsc{pl} in barn.\textsc{m.sg}\\
\glt `[...] at the beginning of spring one had to let the animals into the barn [...]'
\z

\ea
\label{ex:stuaj2}
\langinfo{Tuatschín}{Sadrún}{f3, l. 34ff.}\\
\gll Api … ṣè quaj vagnú prju ajn, a lu ò la cònfadarazjun dau vi quaj da mintga cantún, a lèzs òn sèzs \textbf{stavju} lura … métar sén pajs quaj [...].\\
and {} be.\textsc{prs.3sg} \textsc{dem.unm} \textsc{pass.aux.ptcp.unm} take.\textsc{ptcp.unm} in and then have.\textsc{prs.3sg} \textsc{def.art.f.sg} confederation give.\textsc{ptcp.unm} over \textsc{dem.unm} \textsc{dat} every canton.\textsc{m.sg} and \textsc{dem.m.pl} have.\textsc{prs.3pl} self.\textsc{m.pl} must.\textsc{ptcp.unm} then {} put.\textsc{inf} on foot.\textsc{m.pl} \textsc{dem.unm}\\
\glt `Then … this has been accepted, and then the confederation handed it over to every canton, and these had ... to get it off the ground themselves [...].'
\z

\textit{Astgè} `be allowed, can' (\ref{ex:astg1}) and \textit{schè/schá} (\ref{ex:astg2}) `let, allow' refer to permission.

\ea
\label{ex:astg1}
\langinfo{Tuatschín}{Sadrún}{f6, l. 748f.}\\
\gll[...] api vajn nuṣ gju quèla gròndjuṣ’ idéa scha nuṣ \textbf{astgan} cuṣchiná.\\
{} and have.\textsc{prs.1pl} \textsc{1sg} have.\textsc{ptcp.unm} \textsc{dem.f.sg} great.\textsc{f.sg} idea if \textsc{1pl} be\_allowed.\textsc{prs.1pl} cook.\textsc{inf}\\
\glt `[...] and then we had that great idea [to ask] whether we were allowed to cook.'
\z

\ea
\label{ex:astg2}
\langinfo{Tuatschín}{}{\citealt[85]{Gadola1935}}\\
\gll  In autar òn \textbf{schajṣ} vuṣ èra ir ad alp mè. \\
\textsc{indef.art.m.sg} other year let.\textsc{prs.2pl} \textsc{2pl} also go.\textsc{inf} to alp \textsc{1sg}\\
\glt `Another year you will also let me go to alp.'
\z

\textit{Astgè} is also used in polite requests as in  (\ref{ex:astg3}).

\ea
\label{ex:astg3}
\langinfo{Tuatschín}{Ruèras}{f4, l. 2078}\\
	\gll  «\textbf{Astg}’ \textbf{ju} aun dá in glas aua [...] ?»\\
be\_allowed.\textsc{prs.1sg} \textsc{1sg} in\_addition give.\textsc{inf} \textsc{indef.art.m.sg} cup water\\
\glt `«May I give [you] another glass of water [...] ?»'
\z

Ability is expressed by \textit{pudaj} `can, be able' and \textit{savaj} `can'. \textit{Pudaj} refers to non-learned participant internal ability (\ref{ex:pudaj1} - \ref{ex:pudaj3}) or to permission (\ref{ex:pudaj4}).

\ea
\label{ex:pudaj1}
\langinfo{Tuatschín}{Camischùlas}{f6, l. 717ff.}\\
\gll    A las sòras savèvan tga [...] nus fètschian filistùcas, ad èlas \textbf{pudévan} maj tiar nus.\\
and \textsc{def.art.f.pl} nun.\textsc{pl} know.\textsc{impf.3pl} \textsc{comp} [...] \textsc{1pl} do.\textsc{prs.sbjv.1pl} prank.\textsc{pl} and \textsc{3pl.f} can.\textsc{impf.3pl} never to \textsc{1pl}\\
\glt `And the nuns knew that [...] we used to play pranks, and that they would never be able to prove anything to us.'
\z

\ea
\label{ex:pudaj2}
\langinfo{Tuatschín}{Sadrún}{m6, l. 1397ff.}\\
\gll    Ad ad in piartg … ah … vèva \textbf{pudju} scapá sé Valtgèva tras la sajf, qu’ er’ ina, la saiv èri, vèva rùt in palé, quaj vèva \textbf{pudju} atrás [...].\\
and and \textsc{indef.art.m.sg} pig {} eh {}  have.\textsc{impf.3sg} can.\textsc{ptcp.unm} escape.\textsc{inf} up \textsc{pln} through \textsc{def.art.f.sg} fence  \textsc{dem.unm} \textsc{cop.impf.3sg}  \textsc{indef.art.f.sg} \textsc{def.art.f.sg} fence \textsc{cop.impf.3sg.expl} have.\textsc{impf.3sg} break.\textsc{ptcp.unm}  \textsc{indef.art.m.sg} post \textsc{dem.unm}  have.\textsc{impf.3sg} can.\textsc{ptcp.unm} through \\
\glt `And and a pig … eh … had been able to escape in Valtgeva through the fence, that was a, the fence was, had a broken post, it had been able to go through [...].'
\z

\ea
\label{ex:pudaj3}
\langinfo{Tuatschín}{Sadrún}{m4, l. 399f.}\\
\gll  «Ju cala dad í á scùlèta, ju \textbf{pùs} bitg í plé.»\\
\textsc{1sg} stop.\textsc{prs.1sg} \textsc{comp} go.\textsc{inf} to nursery\_school.\textsc{f.sg} \textsc{1sg} can.\textsc{prs.1sg} \textsc{neg} go.\textsc{inf} any\_more  \\
\glt `I’ll stop going to nursery school, I can’t stand it any longer.'
\z

\ea
\label{ex:pudaj4}
\langinfo{Tuatschín}{Sadrún}{f3, l. 67ff.}\\
\gll  Api, quaj èra … quajnasé in pòst da survigilònza, bétga schanza da \textbf{pudaj} atrás, quaj dé òni bigja schau í nuṣ atrás.\\
and \textsc{dem.unm} \textsc{cop.impf.3sg} {} \textsc{dem.unm}.in\_up \textsc{indef.art.m.sg} guard of vigilance.\textsc{f.sg} \textsc{neg} chance.\textsc{f.sg} \textsc{comp} can.\textsc{inf} through \textsc{dem.m.sg} day have.\textsc{prs.3pl.3pl} \textsc{neg} let.\textsc{ptcp.unm} go.\textsc{inf} \textsc{1pl} through\\
\glt `And there was … up there a vigilance guard, no way to go through, that day they didn’t let us go through [that place].'
\z


\textit{Savaj} refers to participant external (\ref{ex:savaj1} - \ref{ex:savaj4}) or learned participant internal ability (\ref{ex:savaj5} and \ref{ex:savaj6}). 

\ea
\label{ex:savaj1}
\langinfo{Tuatschín}{Sadrún}{m4, l. 401f.}\\
\gll  Pi ò èla dét[g]: «Té \textbf{savèssaṣ} í cul tat ajn Pardatsch.»  \\
and have.\textsc{prs.3sg} \textsc{3sg} say.\textsc{ptcp.unm} \textsc{2sg} can.\textsc{cond.2sg} go.\textsc{inf} with.\textsc{def.art.m.sg} grandfather up \textsc{pln}  \\
\glt `Then she said: «You could go up to Pardatsch with your grandfather.'
\z

\ea
\label{ex:savaj2}
\langinfo{Tuatschín}{Camischùlas}{f6, l. 762}\\
\gll    Api sjantar vajn nus tartgau nus \textbf{sapjan} durmí òra [...].\\
and after have.\textsc{prs.1pl} \textsc{1pl} think.\textsc{ptcp.unm} \textsc{1pl}  can.\textsc{prs.sbjv.1pl} sleep.\textsc{inf} out\\
\glt `And then we thought we would have a good sleep [...].'
\z

\ea
\label{ex:savaj3}
\langinfo{Tuatschín}{Ruèras}{m10, l. 1042ff.}\\
\gll  A lu vèvan, òni fatg ò dal mir, álṣò òr dal grép òni fatg ina pintga …  sènda tg’ ins \textbf{sò} ira ah á paj flòt.\\
and then have.\textsc{impf.3pl} have.\textsc{prs.3pl.3pl} make.\textsc{ptcp.unm} out of.\textsc{def.art.m.sg} rock\_face this\_is\_to\_say out of.\textsc{def.art.m.sg} rock have.\textsc{prs.3pl.3pl} make.\textsc{ptcp.unm} \textsc{indef.art.f.sg} small {} path \textsc{rel} \textsc{gnr} can.\textsc{prs.3sg} go.\textsc{inf} ah on foot.\textsc{m.sg} easy.\textsc{adj.unm} \\
\glt `And then they made, out of the rock face, this is to say out of the rock they made a small … path through which one could easily go ah on foot.'
\z


\ea
\label{ex:savaj4}
\langinfo{Tuatschín}{Sadrún}{m8, l. 1492ff.}\\
\gll  Api sjantar sùnd ju sasjus gjù, api vau tartgau gè ábar ah, ju stu gè, ju \textbf{sa} us bigj’ ajfach í ál’ awa.\\
and after be.\textsc{prs.1sg} \textsc{1sg} sit.\textsc{ptcp.m.sg} down and have.\textsc{prs.1sg.1sg} think.\textsc{ptcp.unm} yes but ah \textsc{1sg} must.\textsc{prs.1sg} after\_all \textsc{1sg} can.\textsc{prs.1sg} now \textsc{neg} simply go.\textsc{inf} into.\textsc{def.art.f.sg} water\\
\glt `And then I sat down and thought yes, but, ah, I should after all, I cannot simply jump into the water.'
\z                                       

\ea
\label{ex:savaj5}
\langinfo{Tuatschín}{Sadrún}{f3, l. 154ff.}\\
\gll  Quaj crajs bé, l’ antschata cu ju a surprju quaj, èri, èri da quèls tgé… mataj\footnotemark{} tg’ ina fèmna \textbf{sapi} fá da quaj.\\
\textsc{dem.unm} believe.\textsc{prs.2sg.gnr} \textsc{neg} \textsc{def.art.f.sg} beginning when \textsc{1sg} have.\textsc{prs.1sg} take\_on.\textsc{ptcp.unm} \textsc{dem.unm} \textsc{exist.impf.3sg.expl} \textsc{exist.impf.3sg.expl} of \textsc{dem.m.pl} \textsc{comp} probably \textsc{comp} \textsc{indef.art.f.sg} woman can.\textsc{prs.sbjv.3sg} do.\textsc{inf} of \textsc{dem.unm}\\
\glt `This you don’t believe, at the beginning when I took on this job, there were, there were some men who … [would say] that a woman is not able to do that.'\footnotetext{\textit{Mataj} means 'probably'; in this context, it is used ironically in the sense of 'impossibly'.}
\z

\ea
\label{ex:savaj6}
\langinfo{Tuatschín}{Sadrún}{m4, l. 679ff.}\\
\gll Lu dumandavan nuṣ èl, vevan dumandau nùa èl ségi stauṣ ajn plaza, èra ’l staus zatgé vid Andermatt– a tudèstg \textbf{savèv’} ju è bigja– vèvan nuṣ dumandau in' jèda sch’ èl \textbf{sapi}, \textbf{savèva} ’l lu schòn in téc tudèstg, \textbf{savèva} ’l lu aun, quaj tg’ èra lu bigja ’l cas tiar quèls végls aun.   \\
then ask.\textsc{impf.1pl} \textsc{1pl} \textsc{3sg.m} have.\textsc{impf.3sg}  ask.\textsc{ptcp.unm} where \textsc{3sg.m} be.\textsc{prs.sbjv.3sg} \textsc{cop.ptcp.m.sg} in job.\textsc{f.sg} be.\textsc{impf.3sg} \textsc{3sg.m}  \textsc{cop.ptcp.m.sg} something over \textsc{pln} and German know.\textsc{impf.1sg} \textsc{1sg} also \textsc{neg} have.\textsc{impf.1pl} \textsc{1pl} ask.\textsc{ptcp.unm} one.\textsc{f.sg} time whether \textsc{3sg.m} can.\textsc{prs.sbjv.3sg} know.\textsc{impf.3sg} \textsc{3sg.m} then indeed \textsc{indef.art.m.sg} piece German know.\textsc{impf.3sg} \textsc{3sg.m} then really \textsc{dem.unm} \textsc{rel} \textsc{cop.impf.3sg} then \textsc{neg} \textsc{def.art.m.sg} case at \textsc{dem.m.pl} old.\textsc{pl} really \\
\glt `Then we would ask him, we had asked him where he had been working, he had been working for a certain time in Andermatt – and [whether he knew] German I didn't know either – we had asked him whether he knew, he knew some German indeed, he really knew, which then was not the case with these old people.'
\z

If \textit{savaj} modifies a verb with complements, this verb is sometimes omitted, probably under influence of Swiss German influence. In example (\ref{ex:savaj7}) it is the verb \textit{í} `go' which is omitted.

\ea
\label{ex:savaj7}
\langinfo{Tuatschín}{Sadrún}{m9, l. 1796ff.}\\
	\gll [...] da nòs tjams salagravan nus da vagní ò da scùla par è \textbf{savaj} \textbf{á} \textbf{la} \textbf{gjuvantétgna}.\\
 {} of \textsc{ poss.1pl.m.sg} time \textsc{refl}.appreciate.\textsc{impf.1pl} \textsc{1pl} \textsc{comp} come.\textsc{inf} out of school.\textsc{f.sg} \textsc{purp} also can.\textsc{inf} to \textsc{def.art.f.sg} youth\\
\glt `[...] when we were young we were happy to come out of school in order to also be able to go to the association of young men.'
\z

The opposition between \textit{pudaj} and \textit{savaj} is not always clear-cut. In (\ref{ex:pudaj6}), the modal verb refers to participant external possibility and one would expect \textit{savaj} instead of \textit{pudaj}.

\ea
\label{ex:pudaj6}
\langinfo{Tuatschín}{}{\DRG{3}{376}}\\
\gll   Cò \textbf{pùn} ins cargè tschuncònta vacas.\\
here can.\textsc{prs.3sg} \textsc{gnr} charge fifty cow.\textsc{f.pl}\\
\glt `Here one can put to graze fifty cows.'
\z
                               
Volition is expressed by \textit{vulaj} `want' (\ref{ex:vul1} and \ref{ex:vul2}). 

\ea
\label{ex:vul1}
\langinfo{Tuatschín}{Sedrun}{\citealt[106]{Büchli1966}}\\
\gll    «Gjòn, \textbf{vul} té bétga gidá mè da cargè quèla bùra?» «Ben ben, scù ju \textbf{pùs}, vi ju schòn gidá.»\\
\textsc{pn} want.\textsc{prs.1sg} \textsc{2sg} \textsc{neg} help.\textsc{inf} \textsc{1sg} \textsc{comp} carry.\textsc{inf} \textsc{dem.f.sg} block yes yes as \textsc{1sg} can.\textsc{prs.1sg} want.\textsc{prs.1sg} \textsc{1sg} certanily help.\textsc{inf}\\ 
\glt `«Gion, don’t you want to help me charge this block?» «Yes, sure, I will certainly help [you] as well as I can.»'
\z

\ea
\label{ex:vul2}
\langinfo{Tuatschín}{Sadrún}{m8, l. 1473f.}\\
\gll  [...] api vòu anflau in bi ljuc, api lu… \textbf{lèv’} ju fá bògn lò [...].\\
{} and have.\textsc{prs.1sg.1sg} find.\textsc{ptcp.unm} \textsc{indef.art.m.sg} beautiful.\textsc{m.sg} place and then want.\textsc{impf.1sg} \textsc{1sg} do.\textsc{inf} bath there\\
\glt `[...] and then I found a nice place, and then I wanted to take a bath there [...].'
\z

\textit{Vut dí} or \textit{vuta dí}, both `mean'  (literally `wants say'), is best considered a lexicalised expression.

\ea
\label{ex:vutdi1}
\langinfo{Tuatschín}{Ruèras}{f4, l. 2010}\\
\gll Sas tgé quaj \textbf{vut} \textbf{dí}?   \\
know.\textsc{prs.2sg} what \textsc{dem.unm} want.\textsc{prs.3sg} say.\textsc{inf}\\
\glt `Do you know what this means?'
\z

Epistemic modality is expressed by \textit{duaj}  (\ref{ex:duaj2}), \textit{pudaj} (\ref{ex:pudaj5}), and \textit{pudaj èssar} (\ref{ex:pudsav1}) as well as \textit{savaj èssar} (\ref{ex:pudsav2}).

\ea
\label{ex:duaj2}
\langinfo{Tuatschín}{Sadrún}{m5, l. 1186ff.}\\
\gll   [...] sén quaj pas \textbf{duèssi} èssar ina samagljònta caplùta [...] \\
{} on \textsc{dem.m.sg} pass should.\textsc{cond.3sg.expl} \textsc{cop.inf} \textsc{indef.art.f.sg} similar chapel\\
\glt `[...] on this pass there should be a similar chapel [...]'
\z

\ea
\label{ex:pudaj5}
\langinfo{Tuatschín}{Sadrún}{m4, l. 377ff.}\\
\gll   [...] quèlṣ vèvan in purèsser plétòst … pign, tgé \textbf{pudévan} èls vaj, déjsch quindisch armaulṣ gronṣ api lu aun tgauras [...]. \\
{} \textsc{dem.m.pl} have.\textsc{impf.3pl} \textsc{indef.art.m.sg} farm rather {} small what can.\textsc{impf.3pl} \textsc{3pl.m} have.\textsc{inf} ten fifteen animal.\textsc{m.pl} big.\textsc{pl} and then besides goat.\textsc{f.pl}\\
\glt `But this was ... real poverty, they would save as much as they  could, they had to save, and … I know from my uncles, they had a rather ... small farm, what could they ha- have, maybe ten, fifteen big animals and then also goats [...].'
\z

\ea
\label{ex:pudsav1}
\langinfo{Tuatschín}{Sadrún}{m4, l. 457f.}\\
\gll Préndar ajn, \textbf{pù} schòn \textbf{èssar} tga samidav’ al grép [...].\\
take.\textsc{inf} in can.\textsc{prs.3sg} well be.\textsc{inf}  \textsc{comp} \textsc{refl}.change.\textsc{impf.3sg} \textsc{def.art.m.sg} rock\\
\glt `As for mining, it could well be that the rock changed [...].'
\z

\ea
\label{ex:pudsav2}
\langinfo{Tuatschín}{Sadrún}{m5}\\
\gll Préndar ajn, \textbf{sò} schòn \textbf{èssar} tga samidav’ al grép [...].\\
take.\textsc{inf} in can.\textsc{prs.3sg} well be.\textsc{inf}  \textsc{comp} \textsc{refl}.change.\textsc{impf.3sg} \textsc{def.art.m.sg} rock\\
\glt `As for mining, it could well be that the rock changed [...].'
\z

Epistemic modality is also expressed by adverbs like \textit{fòrsa} (line 1273) / \textit{fòrza} (line 1506) `maybe', \textit{mataj} (line 1329) `probably', or \textit{sagir} (line 1692) `certainly'.


\section{Arguments of the verb} 

\subsection{Core arguments}

\subsubsection{Subject}
Subjects are not a part of the verb phrase, but since they are arguments of the verb, they will be treated in this section.

The subject is not marked morphologically but is defined by its position either before or after the verb according to the verb-second syntax of Tuatschin (and more generally of Sursilvan). Subject inversion in general will be treated in § 5.1.1 below about argument order.

Singular subject nouns which have a plural reference such as \textit{gljut} `people' trigger the third person plural in the verb. Since this phenomenon is attested in the DRG (about 100 years ago) and in Büchli (1966, at least 50 years ago), it can be assumed that it has been in the language already for a long time.

\ea
\label{}
\langinfo{Tuatschín}{}{\DRG{9}{575}}\\
\gll Cuélm a vals statan a la \textbf{gljut} \textbf{s'antaupan}.\\
mountanin.\textsc{m.pl} and valley.\textsc{f.pl} stay.p\textsc{rs.3pl} and \textsc{def.art.f.sg} people \textsc{refl}.meet.\textsc{prs.3pl}\\
\glt `Mountains and valleys stay, and people meet.'
\z

\ea
\label{}
\langinfo{Tuatschín}{Bugnaj} {\citealt[142f.]{Büchli1966}}\\
\gll \textbf{La} \textbf{gljut} tga \textbf{mavan} da quèla via ancùntar Bugnaj [...] \textbf{udévan} [...] ina vusch [...].\\
\textsc{def.art.f.sg} people \textsc{rel} go.\textsc{impf.3pl} from \textsc{dem.f.sg} way towards \textsc{pln} {} hear.\textsc{impf.3pl} {} \textsc{indef.art.f.sg} voice\\
\glt `The people who took that way towards Bugnei [...] would hear [...] a voice [...].'
\z

\ea
\label{}
\langinfo{Tuatschín}{Sadrún}{m4, 1. 600}\\
\gll  A bjè \textbf{gljut} \textbf{tumévan} è mju tat [...].  \\
and many people.\textsc{f.sg} be.afraid.\textsc{impf.3pl} also \textsc{poss.1sg.m.sg} grandfather\\
\glt `And many people were afraid of my grandfather [...].'
\z

\ea
\label{}
\langinfo{Tuatschín}{Zarcúns}{m2, l. 1658}\\
\gll    [...] la \textbf{gjuvantétgna} … \textbf{fòn} parada.\\
{} \textsc{def.art.f.sg} youth {} do.\textsc{prs.3pl} parade.\textsc{f.s}g\\
\glt `[...] the association of young men … holds a parade.'
\z

The following phenomenon is also attested in the DRG materials and in Büchli (1966). If there is subject inversion and the subject corresponds to a third person plural, the verb form is in the singular.


\ea
\label{}
\langinfo{Tuatschín}{Camischùlas}{\DRG{3}{592}}\\
\gll La salín \textbf{schava} `\textbf{ls} \textbf{utschals} bétga stá ugèn.\\
\textsc{def.art.f.sg} wheat let.\textsc{impf.3sg} \textsc{def.art.m.pl} bird.\textsc{pl} \textsc{neg} stay.\textsc{inf} with\_pleasure\\
\glt `The birds didn't like to let the wheat be.'
\z

\ea
\label{}
\langinfo{Tuatschín}{Sèlva} {\citealt[28]{Büchli1966}}\\
\gll [...] ina sèra [...] \textbf{ò} `\textbf{ls} \textbf{pástars} vju ad èn las vacas.\\
{} \textsc{indef.art.f.sg} afternoon {} have.\textsc{prs.3sg} \textsc{def.art.m.pl} herdsman.\textsc{pl} see.\textsc{ptcp.unm} \textsc{comp} go.\textsc{ger} \textsc{def.art.f.pl} cow.\textsc{pl}\\
\glt `[...] one afternoon [...] the herdsmen saw the cows going.'
\z

\ea
\label{}
\langinfo{Tuatschín}{Ruèras}{f4, 1. 1953f.}\\
\gll [...] lu \textbf{vèva} \textbf{las} \textbf{fèmnas} da lavá ò la tgèsa [...].   \\
{} then have.\textsc{impf.3sg} \textsc{def.art.f.pl} woman.\textsc{pl} \textsc{comp} wash.\textsc{inf} out \textsc{def.art.f.sg} house\\
\glt [...] then the women had to clean the house [...].'
\z

\ea
\label{}
\langinfo{Tuatschín}{Zarcúns}{m2, l. 1625ff.}\\
\gll    [...] quaj \textbf{fagèva} \textbf{las} \textbf{gjufnas} lu schòn stém sch’ i vajan sé la nègla tg’ èla vai dau né bétg.\\
{} \textsc{dem.unm} do.\textsc{impf.3sg} \textsc{def.art.f.pl} young\_woman.\textsc{pl} then in\_fact attention.\textsc{m.sg} if \textsc{3pl}  have.\textsc{sbjv.prs.3pl} up \textsc{indef.art.f.sg} carnation \textsc{rel} \textsc{3sg.f} have.\textsc{sbjv.prs.3sg}  give.\textsc{ptcp.unm} or \textsc{neg} \\
\glt `[...] the young women would pay close attention to whether they had put on the hat the carnation they had given them or not.'
\z

\ea
\label{}
\langinfo{Tuatschín}{Zarcúns}{m2, l. 1661f.}\\
\gll    Ad òz \textbf{fò} è \textbf{las} \textbf{gjufnas} … par tga … ségi avùnda.\\
and today do.\textsc{prs.3sg} also  \textsc{def.art.f.pl} young\_woman.\textsc{pl} {} \textsc{purp} \textsc{comp} {} \textsc{exist.prs.sbjv.3sg} enough\\
\glt `And today the young women also take part … so that … there are enough people.'
\z

\ea
\label{}
\langinfo{Tuatschín}{Sadrún}{m5, 1. 1202ff.}\\
\gll  [...] fòrsa scha ju ṣbaglja bitg \textbf{ṣè} \textbf{quèlas} \textbf{figuras} lu vagnidas trans-pòrtadaṣ a mézaṣ ajn quèla, ajn quaj sòntg\underline{è}t.\\
{} maybe if \textsc{1sg} be\_wrong.\textsc{prs.1sg} \textsc{neg} \textsc{cop.prs.3sg} \textsc{dem.f.pl} figure.\textsc{pl} then \textsc{pass.aux.ptcp.f.pl} transport.\textsc{ptcp.f.pl} and put.\textsc{ptcp.f.pl} in \textsc{dem.f.sg} in \textsc{dem.m.sg} little\_chapel\\
\glt `[...] maybe, if I am not wrong, yes, when these figures were transported and put into this little chapel.'
\z

\ea
\label{}
\langinfo{Tuatschín}{Sadrún}{f3, 1. 82f.}\\
\gll  [...] quaj è vagnú da bètòn’ ajn, a tanju \textbf{ò} \textbf{laṣ} aun adina. \\
{} \textsc{dem.unm} be.\textsc{prs.3sg} come.\textsc{ptcp.unm} \textsc{comp} concrete.\textsc{inf} in and hold.\textsc{ptcp.unm} have.\textsc{prs.3sg} \textsc{3pl.f} still always \\
\glt `[...] this has been concreted, and they still hold.'
\z


\ea
\label{}
\langinfo{Tuatschín}{Sadrún}{m4, 1. 382f.}\\
\gll   A zatgéj \textbf{mav'} \textbf{alṣ} \textbf{aucs} mavan lu aun anzatgé … ád uáut.\\
and something go.\textsc{impf.3sg}  \textsc{def.art.m.pl} uncle.\textsc{pl}  go.\textsc{impf.3pl} then also something {} to forest.\textsc{m.sg} \\
\glt `And sometimes my uncles would also go sometimes ... to the forest.'
\z

\ea
\label{}
\langinfo{Tuatschín}{Cavòrgja}{m7, l. 2235}\\
\gll  A ... cò \textbf{mav}’ \textbf{als} \textbf{buéts} la stad ad alp [...].  \\
and {} here go.\textsc{impf.3sg} \textsc{def.art.m.pl} boy.\textsc{pl} \textsc{def.art.f.sg} summer to alp.\textsc{m.sg}\\
\glt `And ... here, during summer, the boys would go to the summer pastures [...].'
\z


\subsubsection{Direct object}
The direct object is not marked morphologically, but is defined through its syntactic position, be it a pronoun or a full noun phrase. With simple tenses, it is located after the verb (\ref{ex:do1}) or after the subject in case of subject inversion (\ref{ex:do2}), as well as after the negator \textit{bétga}, particles, and adverbs that have been treated above in § 4.1.3 about particle verbs.

\ea
\label{ex:do1}
\langinfo{Tuatschín}{Sadrún}{m4, l. 647ff.}\\
\gll  A… vagnéva mèndar a mèndar a dumagnavan \textbf{bigj}' [\textbf{èl}] ál, ál spital lèva  `l bitg í né tiar miadis.  \\
and become.\textsc{impf.3sg} worse and worse and induce.\textsc{impf.3pl} \textsc{neg} \textsc{3sg.m} to.\textsc{def.art.m.sg} to.\textsc{def.art.m.sg} hospital want.\textsc{impf.3sg} \textsc{3sg.m} \textsc{neg} go.\textsc{inf} or to doctor.\textsc{m.pl} \\
\glt `And … it became worse and worse and they couldn’t induce [him] to the, to the hospital he didn’t want to go, nor to the doctors.'
\z

\ea\label{ex:do2}
\langinfo{Tuatschín}{Sadrún}{m5, l. 1163f.}\\
\gll  Qu’ è adina aviart a lu saṣ í ajn api vèzas [\textbf{té}] [\textbf{quèlas} ah... \textbf{figuras}] ajn grondèzja da carstgaun.\\
\textsc{dem.unm} \textsc{cop.prs.3sg} always open.\textsc{adj.unm} and then can.\textsc{prs.2sg} go.\textsc{inf} in and see.\textsc{prs.2sg} \textsc{2sg} \textsc{dem.f.pl} ah figure.\textsc{pl} in size.\textsc{f.sg} of human\_being.\textsc{m.sg} \\
\glt `This is always open, and then you can step in and then you see these ah ... figures of the size of human beings.'
\z

With compound tenses, the direct object is located after the participle (\ref{ex:do3}) or after the verbal particle if there is one (\ref{ex:do4}), but not after the inverted subject or the negator, since these elements follow the finite verb.

\ea\label{ex:do3}
\langinfo{Tuatschín}{Ruèras}{m1, l. 179}\\
\gll    Quaj ò \textbf{bégja} dau [\textbf{discusjun}].\\
\textsc{dem.unm} have.\textsc{prs.3sg} \textsc{neg} \textsc{exist.ptcp.unm} discussion.\textsc{m.sg}\\
\glt `There was no discussion.'
\z

\ea
\label{ex:do4}
\langinfo{Tuatschín}{Sadrún}{m6}\\
\gll  Ju prèn magari sé [èl].  \\
\textsc{1sg} take.\textsc{prs.1sg} sometimes up \textsc{3sg.m}\\
\glt `Sometimes I lift him up.'
\z

In rare cases, ditransitive verbs have two direct objects. This is especially the case with \textit{dumandá} `ask, ask for' (\ref{ex:doubledo1}).

\ea
\label{ex:doubledo1}
\langinfo{Tuatschín}{Bugnaj} {\citealt[131]{Büchli1966}}\\
\gll  [...]  ina zagríndara […] ò dumandau [\textbf{la} \textbf{mùma} \textbf{da} \textbf{tgèsa}] [\textbf{in} \textbf{tgavégl} \textbf{da} \textbf{sia} \textbf{buéba}]. \\
{} \textsc{indef.art.f.sg} gipsy\_woman {} have.\textsc{prs.3sg}   ask.\textsc{ptcp.unm} \textsc{def.art.f.sg} mother of house one hair of \textsc{poss.3sg} girl \\
\glt `[…] a gipsy woman [...] asked the mother of the house for one hair of her daughter.'
\z

Another case could be \textit{dá fjuc la lèna} (m10) `light firewood', literally `give fire the firewood', but in this case the syntax of the complements is inverted: the theme precedes the beneficiary, so this construction cannot be viewed as a case of verbs having two direct objects. In my opinion, the verb and the first complement have to be considered as a single semantic and syntactic unit (e.g. \textit{dá fjuc} `light') having a single complement (\textit{la lèna} `the firewood') which functions as direct object. A similar case is \textit{dá culur las sèndas} (f3, § 9.1, l. 11f.), literally `give colour the trails', where \textit{dá culur} can be interpreted as a semantic unit like \textit{dá fjuc}.

With ditransitive verbs, the direct object may be omitted if it has been mentioned before.

\ea\label{}
\langinfo{Tuatschín}{Camischùlas}{f6, l. 740ff.}\\
\gll    Api ò èla cò détg: «Cool, ju mòn grad a raquénta [\textbf{dad} \textbf{èlas}] [\textbf{{\longrule}}].»\\
and have.\textsc{prs.3sg} \textsc{3sg.f} here say.\textsc{ptcp.unm} cool \textsc{1sg}  go.\textsc{prs.1sg} right\_away and tell.\textsc{prs.1sg} \textsc{dat} \textsc{3pl.f} {}\\
\glt `And then she said there: «Cool, I’ll just go and tell them.»'
\z

\ea\label{}
\langinfo{Tuatschín}{Sadrún}{f3, l. 141ff.}\\
\gll  Gè, gè. A lura… nus, nuṣ ṣchajn [\textbf{dis} \textbf{vischnauncas}] [\textbf{{\longrule}}], tarmèttajn ábar tutina «eine Mängelmeldung»\footnotemark{} scù quaj ò nùm.  \\
yes yes and then \textsc{1pl} \textsc{1pl} tell.\textsc{prs.1pl} \textsc{dat.pl} municipality.\textsc{pl} {} send.\textsc{prs.1pl} but nevertheless a report\_of\_damage as \textsc{dem.unm} have.\textsc{prs.3sg} name.\textsc{m.sg}\\
\glt `Yes, yes. And then we tell it to the municipalities, but we nevertheless send «a report of damages» as this is called.'\footnotetext{Said in standard German.}
\z

The direct object does not have to be immediately adjacent to the verb (\ref{ex:do5}).

\ea
\label{ex:do5}
\langinfo{Tuatschín}{Ruèras}{f4, l. 1920}\\
	\gll Ju \textbf{vèv}’ als véntgatschún d’ avrél \textbf{natalézi} [...].   \\
\textsc{1sg} have.\textsc{impf.1sg} \textsc{def.art.m.pl} twenty-five of April.\textsc{m.sg} birthday.\textsc{m.sg} \\
\glt `I had my birthday on April 25 [...].'
\z


\subsubsection{Indirect object}
As mentioned above in §§ 3.2.1.2 and 3.6.1, the nominal and pronominal definite indirect object was headed by \textit{di/dis} or \textit{li/lis}. Nowadays, the indirect object, whether definite or not, is almost exclusively headed by \textit{da} (but see (\ref{datart7}) to (\ref{datart11}) in § 3.2.1.3 above about the speech of some older people).

In most cases, the indirect object precedes the direct object (\ref{ex:io:2} and \ref{ex:io:3}), but (\ref{ex:io:1}) shows that the inverse also occurs.
 
\ea\label{ex:io:1}
\langinfo{Tuatschín}{Sadrún}{f3, l.34ff.}\\
\gll Api … ṣè quaj vagnú prju ajn, a lu ò la cònfadarazjun dau vi [\textbf{quaj}] [\textbf{da} \textbf{mintga} \textbf{cantún}] [...].   \\
and {} be.\textsc{prs.3sg} \textsc{dem.unm} \textsc{pa ss.aux.ptcp.unm} take.\textsc{ptcp.unm} in and then have.\textsc{prs.3sg} \textsc{def.art.f.sg} confederation give.\textsc{ptcp.unm} over \textsc{dem.unm} \textsc{dat} every canton.\textsc{m.sg}\\
\glt `Then … this has been accepted, and then the confederation handed it over to every canton [...].'
\z

\ea\label{ex:io:2}
\langinfo{Tuatschín}{Sadrún}{m6, l.1419f.}\\
\gll  «Quèl vès lu aun da pajè [\textbf{da} \textbf{té}] [\textbf{al} \textbf{pustrètsch} \textbf{dal} \textbf{piertg} \textbf{tga} \textbf{té} \textbf{vèvas} \textbf{partgirau}].» \\
\textsc{dem.m.sg} have.\textsc{cond.3sg} then still \textsc{comp} pay.\textsc{inf} \textsc{dat} \textsc{2sg} \textsc{def.art.m.sg} money of.\textsc{def.art.m.sg} pig \textsc{rel} \textsc{2sg} have.\textsc{impf.2sg} look\_after.\textsc{ptcp.unm}\\
\glt `This one should still pay you the money of the pig you had looked after.'
\z

\ea
\label{ex:io:3}
\langinfo{Tuatschín}{Sadrún}{m6, l.1295f.}\\
\gll    A nus mavan culs pòrs sé Valtgèva, mintga dé sé a gjù, ju savès raquintá [\textbf{da} \textbf{té}] [\textbf{quaj}].\\
and \textsc{1pl}  go.\textsc{impf.1pl} with.\textsc{def.art.m.pl} pig.\textsc{pl} up \textsc{pln} every day.\textsc{m.sg} up and down  \textsc{1sg}  can.\textsc{cond.1sg}  tell.\textsc{inf}  \textsc{dat}  \textsc{2sg} \textsc{dem.unm}\\
\glt `And we would go up to Valtgeva with the pigs, every day up and down, I could tell you about that.'
\z

The usual semantic role of an indirect object is \textsc{recipient} as in the examples above, but with verbs like \textit{plaṣchaj} `please' or \textit{fá plaṣchaj} `make pleasure', the semantic role is  \textsc{experiencer} as in (\ref{ex:dat:exp:1}) and (\ref{ex:dat:exp:2}).

\ea\label{ex:dat:exp:1}
\langinfo{Tuatschín}{Sadrún}{m4, l.386}\\
\gll A quaj \textbf{plaṣchéva} nuéta pròpi \textbf{da} \textbf{mé}. \\
and \textsc{dem.unm} please.\textsc{impf.3sg} nothing really \textsc{dat} \textsc{1sg}   \\
\glt `And I didn’t really like that.'
\z

\ea
\label{ex:dat:exp:2}
\langinfo{Tuatschín}{Sadrún}{f3, l.113}\\
\gll  A quaj fò adina \textbf{da} \textbf{mé} … plaṣchaj.  \\
and \textsc{dem.unm} make.\textsc{prs.3sg} always \textsc{dat} \textsc{1sg} {} pleasure.\textsc{m.sg} \\
\glt `And this is always a … pleasure for me. '
\z

\textit{Da} as a dative marker is also reported for the Surmiran dialect of Marmorera:

\ea\label{}
\langinfo{Surmiran}{Marmorera}{\DRG{5}{19}}\\
\gll  Ja da detg \textbf{da} \textbf{mia} \textbf{sora} tgi la vegna no.\\
\textsc{1sg} have.\textsc{prs.1sg} say.\textsc{ptcp.unm} \textsc{dat} \textsc{poss.1sg} sister \textsc{comp} \textsc{3sg.f} come.\textsc{prs.sbjv.3sg} here\\
\glt `I told my sister to come here.'
\z

\ea\label{}
\langinfo{Surmiran}{Marmorera}{\DRG{5}{19}}\\
\gll  Ja do(u)m in mail \textbf{da} \textbf{chel} \textbf{umfant}.\\
\textsc{1sg} give.\textsc{prs.1sg} \textsc{infef.art.m.sg} apple \textsc{dat} \textsc{dem.m.sg} child\\
\glt `I give an apple to this child.'
\z

\subsection{Non-core arguments}

\subsubsection{Locative arguments}
Locative arguments are realised as adverbs or combinations of adverbs, as noun phrases, or as adpositional phrases. The latter are formed with simple or complex prepositions as well as with circumpositions.

Adverbs are either simple or combinations of adpositions as well as combinations of adpositions with adverbs; very frequent are also combinations of adverbs which form locative arguments.

Simple adverbs are \textit{cò} `here', \textit{daspèras} `next to it', \textit{drétg} `right', \textit{gljunsch} `far away', \textit{lò} `there', \textit{nagljú} `nowhere',  \textit{saniastar} `left', \textit{tschò} `here'\footnote{\textit{Tschò} `here' is usually used together with \textit{lò} `there'; see example (\ref{ex:tscholo}).}, and \textit{zanú/zanúa} `somewhere'.

\ea
\label{}
\langinfo{Tuatschín}{Ruèras}{m10, l. 1174f.}\\
\gll  [...] al bùrdi stèva ò ualti \textbf{gljunsch}, quaj balantschava in téc.\\
{} \textsc{def.art.m.sg} load  stand.\textsc{impf.3sg} out quite far \textsc{dem.unm} roll.\textsc{impf.3sg}  \textsc{indef.art.m.sg} bit\\
\glt `[...] the load was sticking out quite a lot, it was rolling a bit.'
\z

\ea
\label{}
\langinfo{Tuatschín}{Ruèras}{m10, l. 1226}\\
\gll A lura … quaj è grat stau in téc, quèls mavan tùt \textbf{saniastar} sén via.   \\
and then {} \textsc{dem.unm} be.\textsc{prs.3sg} just \textsc{cop.ptcp.unm} \textsc{indef.art.m.sg} bit \textsc{dem.m.pl} go.\textsc{impf.3pl} completely left.\textsc{adj.unm} on road.\textsc{f.sg}  \\
\glt `And then … this wa just for a bit, they were walking on the very left side of the road.'\footnote{The left side is the side of the precipice if one comes up from e.g. Rueras.}
\z

\ea
\label{ex:tscholo}
\langinfo{Tuatschín}{Cavòrgja}{m7, l. 2160f.}\\
\gll [...] ábar savènṣ è aun da quèls tg’ èran fumégl \textbf{tschò} né \textbf{lò}.\\
{} but often also in\_addition of \textsc{dem.m.pl} \textsc{rel} \textsc{cop.impf.3pl} farmhand.\textsc{m.sg} here or there\\
\glt `[...] but often also one of those that were farmhands here and there.'
\z

\ea
\label{}
\langinfo{Tuatschín}{Sadrún}{m4, l. 689ff.}\\
\gll  A … api ṣè lu capitau … mù gè quaj cu `l vèva sjatòntanòv òns circa, ṣè `l \textbf{zanúa} para i è ruclaus.  \\
and {} and be.\textsc{prs.3sg} then happen.\textsc{ptcp.unm} {} but yes \textsc{dem.unm} when \textsc{3sg.m} have.\textsc{impf.3sg} seventy-nine year.\textsc{m.pl} around be.\textsc{prs.3sg} \textsc{3sg.m} somewhere  seem.\textsc{prs.3sg} \textsc{expl} also fall.\textsc{ptcp.m.sg}\\
\glt `And … and then it happened, ... well when he was about seventy-nine years old, it seems that he also fell down somewhere.'
\z

Combinations of prepositions or of prepositions with adverbs are e.g. \textit{ajndadájns} `inside (< in + inside)', \textit{angjù} `down' (< in + down), \textit{ansé} `up' (< in + up), \textit{daváuntiar} `in front' (< in front + towards), òrdavaun `in front' (out + before), \textit{sédangjù} (< up + down), \textit{ṣurangjù} (over + down), \textit{vinavaun} `farther' (< over + in + in front).

\ea
\label{}
\langinfo{Tuatschín}{Cavòrgja}{f1, l. 2160f.}\\
\gll Ju spétga té \textbf{ajndadajns}, ajn stiva.\\
\textsc{1sg} wait\textsc{.prs.1sg} \textsc{2sg} inside in living-room.\textsc{f.sg}\\
\glt `I wait for you inside, in the living-room.'
\z

\ea
\label{}
\langinfo{Tuatschín}{Sadrún}{m6, l. 1500ff.}\\
	\gll    [...] a lu mava … in \textbf{òrdavaun} a lu mavan quèls pòrs tùt ajn còrda [...].\\
{} and then go.\textsc{impf.3sg} {} one.\textsc{m.sg} in\_front and then go.\textsc{impf.3pl} \textsc{dem.m.pl} pig.\textsc{pl} all in single\_file.\textsc{f.sg} \\
\glt `[...] and then … one would move in front and then the other pigs would follow in single file[...].'
\z

\ea
\label{}
\langinfo{Tuatschín}{Ruèras}{m10, l. 1125f.}\\
\gll A lu èssan nuṣ i sé api vagní \textbf{ṣurangjù} ad i gjù Tiefenbach.\\
and then be.\textsc{prs.1pl} \textsc{1pl} go.\textsc{ptcp.m.pl} up and come.\textsc{ptcp.m.pl} over\_down and go.\textsc{ptcp.m.pl} down \textsc{pln} \\
\glt `And then we went up and came [from] over [the avalanche barriers] down and went down to Tiefenbach.'
\z




A special case are \textit{gjùdém} `at the very bottom' and \textit{séssum} `upmost', which are a combination of the preposition \textit{gjù} and \textit{sé} with the derivational morpheme -\textit{dém} `(down)most', respectively -\textit{sum} `(up)most'.


\ea
\label{}
\langinfo{Tuatschín}{Camischùlas}{f6, l. 810ff.}\\
\gll [...] ad amplanjú sé agl ésch cul … ròlas da pupí da tualèta tòcan \textbf{séssum} [...].\\
{} and fill.\textsc{ptcp.unm} up \textsc{def.art.m.sg} door with.\textsc{def.art.m.sg} {} roll.\textsc{f.pl} of paper.\textsc{m.sg} of toilet.\textsc{f.sg} until very\_top\\
\glt `[...] and filled up the doorway with the … rolls of toilet paper until the very top [...].'
\z

Locative adverbs are also formed by combinations of the demonstrative \textit{quaj} `this' (\ref{ex:quajnase}), the comparative \textit{plé} `more (than)', and the consecutive \textit{schi} `so (that)' (\ref{ex:schindanajn}). With \textit{plé} the compared element may be explicit (\ref{ex:plendanoragju}) or implicit (\ref{ex:plendanora}).

\ea
\label{ex:quajnase}
\langinfo{Tuatschín}{Sadrún}{f3, l. 67f.}\\
\gll  Api, quaj èra … \textbf{quajnasé} in pòst da survigilònza [...].\\
and \textsc{dem.unm} \textsc{cop.impf.3sg} {} \textsc{dem.unm}.in\_up \textsc{indef.art.m.sg} guard of vigilance.\textsc{f.sg}\\
\glt `And there was … up there a vigilance guard [...].'
\z

\ea
\label{ex:plendanoragju}
\langinfo{Tuatschín}{Ruèras}{m10, l. 1238ff.}\\
\gll  [...] quèls tg’ èran staj á mèssa èran schòn [...] \textbf{pléndanòragj\underline{ù}} \textbf{tga} quaj tga nuṣ èssan, tg’ als, tg’ als méls èn galòpaj.\\
{} \textsc{dem.m.pl} \textsc{rel} be.\textsc{impf.3pl} \textsc{cop.ptcp.m.pl} at mass.\textsc{f.sg} \textsc{cop.impf.3pl} already [...] more\_out\_down than \textsc{dem.unm} \textsc{rel} \textsc{1pl} be.\textsc{prs.1pl} \textsc{rel} \textsc{def.art.m.pl} \textsc{rel} \textsc{def.art.m.pl} mule.\textsc{pl} \textsc{cop.prs.3pl} gallop.\textsc{ptcp.m.pl}\\
\glt `[...] those who had assisted the mass were already [...] farther down than we were, that the, that the mules galloped.'
\z

\ea
\label{ex:plendanora}
\langinfo{Tuatschín}{Sadrún}{m4, l. 425ff.}\\
\gll [...] lu ṣè aun dus trajs intarassants lògans ajn cò, ajn Burganèz, qu' è in tòc \textbf{pléndanòra} [...].\\
{}  then \textsc{exist.prs.3sg} still two three interesting place.\textsc{m.pl} in here in \textsc{pln} \textsc{dem.unm} \textsc{cop.prs.3sg} \textsc{indef.art.m.sg} bit more\_out \\
\glt `[...] there were furthermore two or three interesting places up there, in Burganèz, this is a little bit more down the valley [...].'
\z

\ea
\label{ex:schindanajn}
\langinfo{Tuatschín}{Sadrún}{m4, l. 489ff.}\\
\gll [...] la détga di tg’ èrian \textbf{schindanajn} tg' i udévian c’ i tucavi da mjadṣdé ajnt Ruèras.\\
{} \textsc{def.art.f.sg} legend say.\textsc{prs.3sg}  \textsc{comp} \textsc{cop.impf.sbjv.3pl} so\_in \textsc{comp} \textsc{3pl} hear.\textsc{impf.sbjv.3pl} \textsc{comp} \textsc{expl} beat.\textsc{impf.sbjv.3sg} of noon.\textsc{m.sg} in \textsc{pln}\\
\glt `[...] the legend says that they were so deep in the cave that they heard that they heard the clock strike noon in Rueras.'
\z

Some combinations of adverbs are \textit{cò angjù} `down here', \textit{gjù cò} `down here', and \textit{ò lò} `out there'.

\ea
\label{}
\langinfo{Tuatschín}{Sadrún}{f3, l. 98}\\
\gll  \textbf{Cò} \textbf{angj\underline{ù}} va ju la finala nagíns.\\
here in\_down have.\textsc{prs.1sg} \textsc{1sg} \textsc{def.art.f.sg} end no.\textsc{m.pl}  \\
\glt `In the end I don’t have any down here.'
\z

\ea
\label{}
\langinfo{Tuatschín}{Sadrún}{m4, l. 598f.}\\
\gll  \textbf{Ò} \textbf{lò} vòu fòrza schòn è survagnú in téc quajda d' í par crapa [...].\\
out there  have.\textsc{prs.1sg.1sg} maybe really also get.\textsc{ptcp.unm} \textsc{indef.art.m.sg} bit desire.\textsc{f.sg} \textsc{comp} go.\textsc{inf} for stone.\textsc{coll}\\
\glt `Out there I might have started enjoying looking for stones a bit [...].'
\z



The adverbs \textit{cò} `here' and \textit{lò} `there' combine with the prepositions \textit{sé} `up' and \textit{gjù} `down'. In the case of \textit{cò} and \textit{sé}, they combine either as \textit{sé cò} or as \textit{cò sé}.\footnote{The syntactic status of \textit{sé} in the combination \textit{cò sé} is not clear to me, because it looks like a postposition. However, there are no postpositions in Tuatschín except in such cases.} If one says

\ea
\label{}
\langinfo{Tuatschín}{Ruèras}{m10}\\
\gll Ju sùn \textbf{sé} \textbf{cò}.\\
\textsc{1sg} \textsc{cop.prs.1sg} up here\\
\glt `I am up here.'
\z

it implies that speaker and hearer are at the same place. If one says

\ea
\label{}
\langinfo{Tuatschín}{Ruèras}{m10}\\
\gll Ju sùn \textbf{cò} \textbf{sé}.\\
\textsc{1sg} \textsc{cop.prs.1sg} here up\\
\glt `I am up here.'
\z

it implies that the speaker is at a higher place than the hearer.

Mail a Tarcisi per survegnir ina declaraziun.

\ea
\label{}
\langinfo{Tuatschín}{Cavòrgja}{f6, l. 2410ff.}\\
\gll Nuṣ vèvan da partgirá als tiars, als buéts, a stèvan \textbf{lò} \textbf{sé}, sé majṣès.\\
\textsc{1pl} have.\textsc{impf.1pl} \textsc{comp} \textsc{mind.inf} \textsc{def.art.m.pl} animal.\textsc{pl} \textsc{def.art.m.pl} boy.\textsc{pl} and stay.\textsc{impf.1pl} there up up assembly\_of\_houses.\textsc{m.sg}\\
\glt `We had to mind the animals, we the boys, and stayed up there, at the \textit{majṣès}.'
\z

In Tuatschin - as well as in other Romansh varieties - adpositions heading place names or names of important buildings like churches or schools get a special meaning and will be treated below in this section (see \tabref{loc1} - \tabref{loc6}).

The following simple prepositions occur in the corpus: \textit{ajn} `in, into', \textit{ajnt/ajnta} `in, into', \textit{amiaz} `in the middle of', \textit{ancùntar} `towards', ant\underline{ù}rn `around', \textit{avaun} `before', \textit{dadajns/dadajnt} `inside', \textit{dadò} `outside', \textit{\textit{davùs}} `behind', \textit{gjù} `down', \textit{nùca} `by, next to', \textit{ò/òra} `out', \textit{sé} `up', \textit{sén} `on', \textit{séssum} `on top of', \textit{spèr} `next to', \textit{ṣur/ṣu} `above', \textit{ṣut} `under', \textit{tòca/tòcan} `until', \textit{viars} `towards'.

\ea
\label{}
\langinfo{Tuatschín}{Cavòrgja}{\citealt[121]{Büchli1966}}\\
\gll a la nòtg ṣè ‘l gjat vagnús \textbf{ajnta} \textbf{létg} […].\\
and \textsc{def.art.f.sg} night be.\textsc{prs.3sg} \textsc{def.art.m.sg} cat come.\textsc{ptcp.m.sg} into bed\\
\glt `And at night the cat came into [his] bed […].'
\z

\ea
\label{}
\langinfo{Tuatschín}{Sadrún}{m6, l. 1530f.}\\
\gll    [...] fatg ant\underline{ù}rn in sujèt mataj a méz sé quaj \textbf{ant\underline{ù}rn} \textbf{al} \textbf{vjantar} dals pòrs.\\
{} do.\textsc{ptcp.unm} around \textsc{indef.m.sg} rope probably and put.\textsc{ptcp.unm} up  \textsc{dem.unm} around \textsc{def.art.m.sg} belly of.\textsc{def.art.m.pl} pig.\textsc{pl} \\
\glt `[...] tied a rope around, and put them around the belly of the pigs.'
\z


\ea
\label{}
\langinfo{Tuatschín}{Cavòrgja}{f1}\\
\gll Ju spétga \textbf{avaun} \textbf{tgèsa}.\\
\textsc{1sg} wait\textsc{.prs.3sg} before house.\textsc{f.sg}\\
\glt `I'm waiting in front of the house.'
\z

\ea
\label{}
\langinfo{Tuatschín}{Camischùlas}{f6, l. 801ff.}\\
\gll [...] prandévan nòssa … sarvjèta, matévan sé \textbf{nùca} \textbf{la} \textbf{sòr}’ \textbf{Andréa}, a lu astgèvan nus raṣdá ròmòntsch.\\
{} take.\textsc{impf.1pl} \textsc{poss.1pl.f.sg} {} napkin put.\textsc{impf.1pl} up  where \textsc{def.art.f.sg} nun \textsc{pn} and then be\_allowed.\textsc{impf.1pl} \textsc{1pl} speak.\textsc{inf} Romansh.\textsc{m.sg}\\
\glt `[...] [we] would take our … napkin, would put it next to Sister Andrea, and then we were allowed to speak Romansh.'
\z

\ea
\label{}
\langinfo{Tuatschín}{Sadrún}{m4, l. 501ff.}\\
\gll [...] api vèvanṣ da partgirá als tiars, mav’ ins lò sédòr ál sit, grat cò \textbf{nùca} \textbf{quèla} \textbf{rùsna} tgu a raquintau.\\
{} and have.\textsc{impf.1pl.1pl} \textsc{comp} mind.\textsc{inf} \textsc{def.art.m.pl} animal.\textsc{pl} go.\textsc{impf.3sg} \textsc{gnr} there up\_out in.\textsc{def.art.m.sg} south just there by \textsc{dem.f.sg} hole \textsc{rel.1sg} have.\textsc{prs.1sg} tell.\textsc{ptcp.unm}\\  
\glt `[...] and we also had to mind the animals; we would then go up to the south, just by that cave I have told about.'
\z

If the referent is known to or can by inferred by the hearer, noun phrases headed by a preposition preclude the use of determiners: \textit{ajn cuṣchina, nuégl, tgòmbra} `in the kitchen, living-room, barn, room', \textit{ajnta létg} `in bed', or \textit{avaun tgèsa} `in front of the house'.

But if a noun headed by a preposition refers to an entity that is not known to the hearer, the noun must be modified by the indefinite article (\ref{ex:ajnina1}).

\ea
\label{ex:ajnina1}
\langinfo{Tuatschín}{Sadrún}{m4, l. 386f.}\\
\gll  [...] èl è curdauṣ gjùdajn \textbf{ajn} \textbf{ina} \textbf{rùsna} nùndétga [...].\\
{} \textsc{3sg.m} be.\textsc{prs.3sg} fall.\textsc{ptcp.m.sg} down\_into in \textsc{indef.art.f.sg} hole awful\\
\glt `[...] he fell in an awful hole [...].'
\z

The noun may be modified by a demonstrative determiner in anaphoric function (\ref{ex:ajnquela1}).

\ea
\label{ex:ajnquela1}
\langinfo{Tuatschín}{Sadrún}{m5, l. 1270f.}\\
\gll \textbf{Ajn} \textbf{quèla} \textbf{caplùta} ṣè ajn ina, la quarta stazjun da la via da la crusch.\\
in \textsc{dem.f.sg} chapel \textsc{cop.prs.3sg} in \textsc{indef.art.f.sg} \textsc{def.art.f.sg} fourth station of \textsc{def.art.f.sg} way of \textsc{def.art.f.sg} cross \\
\glt `In this chapel there is a, the fourth station of Christ’s way of the Cross.'
\z

If the noun phrase is modified by an adjective, the definite article must occur: \textit{ajn la tgèsa véglja} `in the old house' (Büchli 1966: 30). Note that usually the preposition \textit{ajn} and the definite article are fused: \textit{ajn al} → \textit{ajl/ál}, \textit{ajn la} → \textit{ajla/ála}.

Some more examples of bare nouns in prepositional phrases are \textit{ajn tégja} (Büchli 1966: 122), \textit{ajn stizjun} (Büchli 1966: 123), \textit{ajn nuégl} (B 66: 125), \textit{ajn stiva} (§9.13, l. 1768), \textit{ajn caplùta} (Büchli 1966: 45), \textit{ajn caplùta da Sòntgaclau} (Büchli 1966: 45), \textit{avaun tégja} (§ 9.17, l. 2493), \textit{ṣur Lucmagn} `over the Lucmagn pass' (§ 9.9, l. 1337).


The simple prepositions cannot stand alone, i.e. they cannot function as adverbs. In order to do so, they need a derivational morpheme, which is \textit{an}- in case of \textit{gjù} `down' and \textit{sé} `up', and \textit{vid}- with \textit{ajn} `in(to)' and \textit{ò/òr/òra} `out'. The adverbial equivalent of \textit{spèr} `next to' is \textit{daspèras}.

\ea
\label{}
\langinfo{Tuatschín}{Cavòrgja}{m7; l. 2459f.}\\
\gll A … la duméngja … alṣ ùmanṣ èn ì \textbf{angj\underline{ù}} [...].\\
{} and {} \textsc{def.art.f.sg} Sunday {} \textsc{def.art.m.sg} man.\textsc{pl} be.\textsc{prs.3pl} go.\textsc{ptcp.m.pl} down {} \textsc{def.art.m.pl} boy.\textsc{pl} \textsc{cop.impf.3pl} over alpine\_hut.\textsc{f.sg} {} and\\
\glt `And ... on Sunday ... the men went down [...]. '
\z

\ea
\label{}
\langinfo{Tuatschín}{Sadrún}{m4, l. 480ff.}\\
\gll A quèla tauna vò \textbf{vidajn} – quaj tgu sùn stauṣ ajn – vò lò \textbf{vidajn} circa véntgatschún métars, san ins í \textbf{vidajn} da quèla, api sasparti, vòi ajn duas.\\
and \textsc{dem.f.sg} cave go.\textsc{prs.3sg} into {}  \textsc{dem.unm} \textsc{rel.1sg} be.\textsc{prs.1sg} \textsc{cop.ptcp.m.sg} in {} go.\textsc{prs.3sg} there into about twenty-five meter.\textsc{m.pl} can.\textsc{prs.3sg} \textsc{gnr} go.\textsc{inf} into of \textsc{dem.f.sg} and \textsc{refl}.divide.\textsc{prs.3sg.expl} go.\textsc{prs.3sg.expl} in two.\textsc{f.pl}  \\
\glt `And this cave – [judging from] where I have been into it – one can go into it about 25 meters, and then it splits into two.'
\z

\ea
\label{}
\langinfo{Tuatschín}{Sadrún}{m4, l. 630ff.}\\
\gll  A la sèra par tga briṣchi bétg … vagnéva quaj, quaj mava `l ajnagjù cul maun èra sènza … [vòns] a trèva \textbf{vid\underline{ò}} còtgla giù sé sé  `l plantschju.  \\
and \textsc{def.art.f.sg} evening \textsc{purp} \textsc{comp} burn.\textsc{prs.sbjv.3sg} \textsc{neg} {} \textsc{pass.aux.impf.3sg} \textsc{dem.unm} \textsc{dem.unm} go.\textsc{impf.3sg} \textsc{3sg.m} in\_down with.\textsc{def.art.m.sg} hand also without {} [glove.\textsc{m.pl}] and pull.\textsc{impf.3sg} out charcoal.\textsc{coll} down up on \textsc{def.art.m.sg} floor  \\
\glt `And in the evening, to avoid it burning … was that, there he went into [the fire] with one hand, also without [gloves], and pulled out charcoal from down there up to the floor.'
\z

\ea
\label{}
\langinfo{Tuatschín}{Sadrún}{m4, l. 568f.}\\
\gll Ṣùtajn èri la tégja nùca `l caṣchav' èra, \textbf{daspèras} quèls dus nuégls [...].\\
under\_in \textsc{cop.impf.3sg.expl} \textsc{def.art.f.sg} alpine\_hut \textsc{rel} \textsc{3sg.m} make\_cheese.\textsc{impf.3sg} also next \textsc{dem.m.pl} two.\textsc{m.pl} cow\_barn.\textsc{pl} \\
\glt `Below was the alpine hut where he would also make cheese, next to it those two cow barns [...].'
\z

However, \textit{sé} and \textit{gjù} may stand alone in the combination \textit{sé a gjù} `up and down'.

\ea
\label{}
\langinfo{Tuatschín}{Sadrún}{m6, l. 1427}\\
\gll    A nus mavan culs pòrs \textbf{sé} \textbf{Valtgèva}, mintga dé \textbf{sé} a \textbf{gjù} [...].\\
and\textbf{} \textsc{1pl}  go.\textsc{impf.1pl} with.\textsc{def.art.m.pl} pig.\textsc{pl} up \textsc{pln} every day.\textsc{m.sg} up and down  \textsc{1sg}  can.\textsc{cond.1sg}  tell.\textsc{inf}  \textsc{dat}  \textsc{2sg} \textsc{dem.unm}\\
\glt `And we would go up to Valtgeva with the pigs, every day up and down [...].'
\z

The preposition \textit{ancùntar} constitutes a special case in the sense that it triggers dative with human nouns but not with non-human nouns.

\ea
\label{}
\langinfo{Tuatschín}{Ruèras}{\citealt[64]{Büchli1966}}\\
\gll Cò ṣaj	vagnú	ina fèmna \textbf{ancùntar} \textbf{li}	\textbf{quaj} \textbf{pur} […].\\
here \textsc{cop.prs.3sg} come.\textsc{ptcp.unm} \textsc{indef.art.f.sg} woman towards \textsc{def.dat.art.sg} \textsc{dem.m.sg} peasant \\
\glt `At this moment a woman came towards this peasant […].'
\z

\ea
\label{}
\langinfo{Tuatschín}{Sadrún}{m5}\\
\gll Èl è juṣ \textbf{ancùntar} \textbf{da} \textbf{la} \textbf{mùma}.\\
\textsc{3sg.m} be.\textsc{prs.3sg} go.\textsc{ptcp.m.sg} towards \textsc{dat} \textsc{def.art.f.sg} mother\\
\glt He went towards his mother.'
\z

\ea
\label{}
\langinfo{Tuatschín}{Sèlva}{\citealt[26]{Büchli1966}}\\
\gll A lu ṣgulavan las còcas ò da la cazèta \textbf{ancunt}' \textbf{al} \textbf{tgamin}.\\
and then fly.\textsc{impf.3pl} \textsc{def.art.f.pl} small\_cake.\textsc{pl} out of \textsc{def.art.f.sg} pan towards \textsc{def.art.m.sg} chimney\\
\glt `And then the small cakes flew out of the pan towards the chimney.'
\z

\ea
\label{}
\langinfo{Tuatschín}{Sadrún}{m6, l. 1000ff.}\\
\gll    Ad ajnaquèla … ṣaj sadèrs … ina grònda lavina gjù da la val Lòndadusa gjù \textbf{ancùntar} \textbf{al} \textbf{vitg} [..].\\
and at\_that\_moment {} be.\textsc{prs.3sg.expl} \textsc{refl.}fall.\textsc{ptcp.unm} {} \textsc{indef.art.f.sg} huge avalanche down of \textsc{def.art.f.sg} valley \textsc{pln} down towards \textsc{def.art.m.sg} village\\
\glt `And precisely at that moment … a huge avalanche … came down from the Londadusa valley, down towards the village [...].'
\z

\textit{Via} instead of \textit{vi} `over' is rejected by some informants; it occurs, however, in the oral corpus as well as in Büchli (1966).

\ea
\label{ex:via1}
\langinfo{Tuatschín}{Bugnaj} {\citealt[132]{Büchli1966}}\\
\gll A lu ségi la cúa ida \textbf{via} da Sòntg Antòni gjù a vagi suatíu als zagríndars ò ṣùt Bugnaj.\\
and then be.\textsc{prs.sbjv.3sg} \textsc{def.art.f.sg} tail go.\textsc{ptcp.f.sg} over from holy.\textsc{m.sg} \textsc{pn} down and have.\textsc{prs.sbjv.3sg} catch\_up.\textsc{ptcp.unm} \textsc{def.art.m.pl} Yenish.\textsc{pl} out under \textsc{pln}\\
\glt `And then the tail went down by [the chapel of] Saint Anthony and caught up the Yenish beneath Bugnei.'
\z

\ea
\label{ex:via2}
\langinfo{Tuatschín}{Tschamùt} {\citealt[15]{Büchli1966}}\\
\gll I èr' in artg \textbf{ṣul} Rajn \textbf{via}.\\
\textsc{expl} \textsc{cop.impf.3sg} \textsc{indef.art.m.sg} rainbow on.\textsc{def.art.m.sg} \textsc{rn} over\\
\glt `There was a rainbow over the Rhine.'
\z

\ea
\label{ex:via3}
\langinfo{Tuatschín}{Cavòrgja}{m7, l. 2498f.}\\
\gll  [...] api mir’ al bab \textbf{via} sén mè:\\
{} and look.\textsc{prs.3sg} \textsc{def.art.m.sg} father over on \textsc{1sg} \\
\glt `[...] and then my father looks over to me:'
\z

Complex prepositions are \textit{damanajval da} `near', \textit{navèn da} `from', \textit{ò da/òrd} `out of', \textit{òn} (< \textit{òra ajn} `out (in)to'), \textit{vi da} `(over) to'.

\ea
\label{}
\langinfo{Tuatschín}{Ruèras}{m10, l. 1675f.}\\
\gll  Immis, quaj è gjù, gjù tschò \textbf{damanajval} \textbf{da} … \textbf{Interlaken}.\\
\textsc{pln} \textsc{dem.unm} \textsc{cop.prs.3sg} down down there near of {} \textsc{pln} \\
\glt `Immis, that is down, down there, near … Interlaken.'
\z

\ea
\label{}
\langinfo{Tuatschín}{Ruèras}{m3, l. 2206ff.}\\
\gll  Tùts … ò da scùla á fumegl … a sjantar ád alp a \textbf{navèn} \textbf{dad} \textbf{alp} vagnévas pér al davùs mumèn a mavas á scùla. \\
all.\textsc{m.pl} {} out of school.\textsc{f.sg} to farmhand.\textsc{m.sg} {} and after to alp.\textsc{f.sg} and away from alp come.\textsc{impf.2sg.gnr} only \textsc{def.art.m.sg} last moment and go.\textsc{impf.2sg.gnr} to school.\textsc{f.sg}\\
\glt `All ... out of school to farmhand ... and after this to the alpine pasture and you would only come away from the pasture at the last moment and then you would go to school.'
\z

\ea
\label{}
\langinfo{Tuatschín}{Camischùlas}{f6, l. 785f.}\\
\gll    [...] api ṣè …  ina da nòssa tgòmbra id’ \textbf{òn} \textbf{tualèta} [...].\\
{} and \textsc{cop.prs.3sg} {} one.\textsc{f.sg} of \textsc{poss.1pl.f.sg} room go.\textsc{ptcp.f.sg} out\_in toilet\\
\glt `[...] and then … one of our room went out to the toilet [...]'
\z

\ea
\label{}
\langinfo{Tuatschín}{Ruèras}{m3, l. 2333ff.}\\
\gll Las quátar \textbf{òrd} \textbf{létg} ad í á rimná las vacas ajn stával [...].\\
\textsc{def.art.f.pl} four out\_of bed.\textsc{m.sg} and go.\textsc{inf} \textsc{comp} collect.\textsc{inf} \textsc{def.art.f.pl} cow.\textsc{pl} in cowshed.\textsc{m.sg}\\
\glt `At four o'clock out of bed and go and gather the cows in the cowshed [...].'
\z

\ea
\label{}
\langinfo{Tuatschín}{Camischùlas}{f6}\\
\gll Surajn è \textbf{vi} \textbf{da} \textbf{tschèla} vart dil Rajn.\\
\textsc{pln} \textsc{cop.prs.3sg} over of \textsc{dem.f.sg} side of.\textsc{def.art.m.sg} \textsc{rn} \\
\glt `Surrein is on the other side of the Rhine.'
\z

The derived adverbs with -\textit{dém} or -\textit{sum} are also used as prepositions.

\ea
\label{}
\langinfo{Tuatschín}{Camischùlas}{m6, l. 983ff.}\\
\gll    Quaj èra ju gjù ina grònda lavina … a vèva … déstruí ina grònda part dl vitg \textbf{ajntadém} \textbf{Ruèras} [..].\\
\textsc{dem.unm} be.\textsc{impf.3sg} go.\textsc{ptcp.unm} down \textsc{indef.art.f.sg} big avalanche {} and have.\textsc{impf.3sg} {} destroy.\textsc{ptcp.unm} \textsc{indef.art.f.sg} huge part of.\textsc{def.art.m.sg} village uppermost \textsc{pln}\\
\glt `Then a huge avalanche went down … and had … destroyed a big part of the village in the upper part of Rueras [...].'
\z

\ea
\label{}
\langinfo{Tuatschín}{Sadrún}{m5, l. 992ff.}\\
\gll A lu ṣchajnṣ adina al ... sòntgèt \textbf{òdém} \textbf{al} \textbf{vitg}.\\
and then say.\textsc{prs.1pl.1pl} always \textsc{def.art.m.sg} {} little\_chapel out\_most \textsc{def.art.m.sg} village\\
\glt `And then we always say the ... little chapel at the lowest part of the village.'
\z

\ea
\label{}
\langinfo{Tuatschín}{Sadrún}{m6, l. 1273f.}\\
\gll    [...] a lu auda `l las stréjas sé cò, \textbf{séssum} \textbf{la} \textbf{val} \textbf{da} \textbf{Lòndadusa} òni clumau:\\
{} and then hear.\textsc{prs.3sg} \textsc{3sg.m} \textsc{def.art.f.pl} witch.\textsc{pl} up here uppermost \textsc{def.art.f.sg} valley of \textsc{pln} have.\textsc{prs.3pl.3pl} call.\textsc{ptcp.unm}\\
\glt `[...] and then he hears the witches up there, they called from the uppermost part of the Londadusa valley:'
\z

\ea
\label{}
\langinfo{Tuatschín}{Ruèras}{m10, l. 1127f.}\\
\gll Quaj è ina, asch’ ina stazjun amiaz al pas circa né … strusch \textbf{séssum} \textbf{al} \textbf{pʰas}.   \\
\textsc{dem.unm} \textsc{cop.prs.3sg} \textsc{indef.art.f.sg} such \textsc{indef.art.f.sg} station amid \textsc{def.art.m.sg} pass around or {} almost on\_top \textsc{def.art.m.sg} pass\\
\glt `This is a, such a station in the middle [of the road to] the pass, approximately, or … almost on top of the pass.'
\z


Circumpositions are e.g. \textit{da \textsc{n} ajn} `through N into', \textit{da \textsc{n} sé} `up', `from N up', \textit{da \textsc{n} sédòra} `up', `from N up', \textit{da \textsc{n} òra} `out of', `through N out', \textit{par \textsc{n} antùrn} `around', \textit{spèr \textsc{n} vi} `next to N over', \textit{ṣùt \textsc{n} ajn} `under N in(to)', \textit{ṣùt \textsc{n} gjù} `under N down'.

In circumpositions, the preposed element \textit{da} often refers to source (`from') (\ref{ex:da.source1}) or to path (`through') (\ref{ex:da.path1}).

\ea
\label{ex:da.source1}
\langinfo{Tuatschín}{Cavòrgja}{\citealt[123]{Büchli1966}}\\
\gll Cu i òn purtau la bara ò da \textbf{tgèsa}, mirav’ èl \textbf{da} \textbf{fanèstr}’ \textbf{òra}.\\
when \textsc{3pl} have.\textsc{prs.3pl} carry.\textsc{ptcp.unm} \textsc{def.art.f.sg} corpse out of house
look.\textsc{impf.3sg} \textsc{3sg.m} from window out\\
\glt `When they carried the corpse out of the house, he was looking out of the window.'
\z


\ea
\label{ex:da.path1}
\langinfo{Tuatschín}{Cavòrgja}{\citealt[121]{Büchli1966}}\\
\gll […] a mava \textbf{da} la pòrta a \textbf{da} las rèmas \textbf{ajn} ajn clavau.\\
[…] and go.\textsc{impf.3sg} from \textsc{def.art.f.sg} door and from \textsc{def.art.f.pl} crack.\textsc{pl} in in barn \\
\glt `[…] and [the hay] came into the barn through the door and the cracks.'
\z

\ea
\label{}
\langinfo{Tuatschín}{Sadrún}{f3, l. 91ff.}\\
\gll  [...] sau bétg c’ ju sùn ida \textbf{da} la val Strém \textbf{ajnasé} [..].\\
{} know.\textsc{prs.1sg.1sg} \textsc{neg} when \textsc{1sg} be.\textsc{prs.1sg} go.\textsc{ptcp.f.sg} from \textsc{def.art.f.sg} valley \textsc{pln} in\_up\\
\glt `[...] I don’t know when I went up the Strem valley {}.'
\z

\ea
\label{}
\langinfo{Tuatschín}{Ruèras}{m10, l. 1117ff.}\\
\gll  Avaun, navèn da Realp essan nuṣ i dad ina … \textbf{dad} in … trùtg \textbf{sédòra} tòcan sésúr las lavinèras [...]. \\
before away from \textsc{pln} be.\textsc{prs.1pl} \textsc{1pl} go.\textsc{ptcp.m.pl} from \textsc{indef.art.f.sg} {} from \textsc{indef.art.m.sg} {} footpath up until up\_over \textsc{def.art.f.pl} avalanche\_barrier.\textsc{pl} \\
\glt `Before, from Realp we went on a footpath [which lead us] above the avalanche barriers [...].'
\z

\ea
\label{}
\langinfo{Tuatschín}{Bugnaj}{\citealt[135]{Büchli1966}}\\
\gll Cu ‘l è vagnús á tgèsa la sèra \textbf{spèr} ina gronda prajt-crap \textbf{vi}, ò ‘l schau dá la sagir \textbf{ṣu} la prajt-crap \textbf{gjù} […].\\ 
when \textsc{3sg.m} \textsc{cop.prs.3sg} come.\textsc{ptcp.m.sg} to house.\textsc{f} \textsc{def.art.f.sg} afternoon next\_to \textsc{indef.art.f.sg} big rock\_face over have.\textsc{prs.3sg} \textsc{3sg.m} let.\textsc{ptcp.unm} give.\textsc{inf} \textsc{def.art.f.sg} saw under \textsc{def.art.f.sg} rock\-face down\\
\glt `When in the evening he came back home, passing a huge rock face, he let the saw fall down under the rock face […].'
\z

One consultant uses the verb \textit{í} `go' without preposition in the case of \textit{majṣès}. The others use either \textit{á} `to' or \textit{sé} `up' in such cases.

\ea
\label{}
\langinfo{Tuatschín}{Cavòrgja}{m7, l.2146f.}\\
	\gll  [...] ábar avaun c’ al tiams dad alp èri lu aun dad \textbf{í} \textbf{majṣès} … culs tiars.  \\
{} but before \textsc{comp} \textsc{def.art.m.sg} time of alp be.\textsc{impf.3sg.expl} then in\_addition \textsc{comp} go.\textsc{inf} assembly\_of\_houses {} with.\textsc{def.art.m.pl} animal.\textsc{pl}\\
\glt `[...] but before going to the summer pastures one had to go to the \textit{majṣès} with the animals.'
\z

\ea
\label{}
\langinfo{Tuatschín}{Cavòrgja}{m7, l.2150f.}\\
\gll Scadín cas, quaj èra schòn … ah … in désidéri, savaj \textbf{í} \textbf{majṣès} a durmí sé lò.\\
each.\textsc{m.sg} case dem.\textsc{unm} \textsc{cop.impf.3sg} indeed {} eh {} indef.\textsc{art.m.sg} longing can.\textsc{inf} go.\textsc{inf} assembly\_of\_houses and sleep.\textsc{inf} up there\\
\glt `In any case, this was indeed ... eh ... a longing, be able to go to the \textit{majṣès} and sleep up there.'
\z

\ea
\label{}
\langinfo{Tuatschín}{Ruèras}{f4, l.f.}\\
	\gll A las fèmnas stèv’ ins aun gidá á tgèsa a culs ùmans mavan lu a … mavan aj lu \textbf{á} \textbf{majṣès} [...].\\
and def.\textsc{art.f.pl} woman.\textsc{pl} must.\textsc{impf.3sg} \textsc{gnr} moreover help.\textsc{inf} at home.\textsc{f.sg} and when.\textsc{def.art.m.pl} man.\textsc{pl} go.i\textsc{mpf.3pl} then and {} go.\textsc{impf.3pl} \textsc{3pl} then to assembly\_of\_house.\textsc{m.sg}\\
\glt `And the women one had to help them at home and when the men would then go and ... would then go up to the \textit{majṣès} [...].'
\z


In Tuatschin - as well as in other Romansh varieties - locative arguments are very important.\footnote{See \citet[4-126]{Ebneter1994} for standard Sursilvan and other Romansh varieties.} When the speaker is located in the Tujetsch valley, he or she must indicate whether they go down the valley (\textit{ò} `out'), up the valley (\textit{ajnta} `into'), outside the valley (\textit{ò} `out' or \textit{gjù} `down'), or over to a place (\textit{vi}), usually seen from the speech act place. If he or she is outside the valley and goes into the valley, they must decide whether they use \textit{sé} (`up') or \textit{ajn} (`into').

\tabref{loc1} shows the prepositions used when the speaker is located or moves within the Lower Valley (which starts in Bugnei and ends in Dieni), and \tabref{loc2} the prepositions used when going to the other side of the Rhine or to the Medel Valley.

\begin{table}
	\caption{Locatives I}
	\label{loc1}
	\begin{tabular}{lllllll}
		\lsptoprule
		& & Bugnaj & Sadrún & Camischùlas & Ruèras & Diani \\
		\midrule
		Bugnaj  & →&   & \textit{ajnta} & \textit{ajnta} & \textit{ajnt} & \textit{ajnta}\\
		Sadrún & → & \textit{ò}  &  & \textit{ajnta} & \textit{ajnt} & \textit{ajnta}\\
		Camischùlas & →& \textit{ò} & \textit{ò} & & \textit{ajnt} & \textit{ajnta}\\
		Ruèras & →& \textit{ò} & \textit{ò} & \textit{ò} & & \textit{vi}\\
		Diani & →& \textit{ò} & \textit{ò} & \textit{ò} & \textit{ò} &\\
		\lspbottomrule
	\end{tabular}
\end{table}

\begin{table}
	\caption{Locatives II}
	\label{loc2}
	\begin{tabular}{lllllll}
		\lsptoprule
		& & Sadrún & Surajn & Cavòrgja & Méjdal & Curaglja\\ 
		\midrule
		Sadrún  &    →& & \textit{gjù/vi}  & \textit{gjùn}    & \textit{vin} & \textit{vi}\\
		Surajn  &   → &  \textit{sé/vi} & & \textit{gjùn}    & \textit{vin} & \textit{vi}   \\
		\lspbottomrule
	\end{tabular}
\end{table}

The prepositions \textit{gjùn} and \textit{vin} are a combination of \textit{gjù} `down' and \textit{vi} `over' with \textit{ajn} `into'. The use of \textit{gjùn} and \textit{vin} is due to the the fact that \textit{cavòrgja} means `canyon' and that \textit{Méjdal} is a valley, hence `into'.

\tabref{loc3} shows the prepositions used between the Upper and the Lower part of the Tujetsch valley, as well as to the bordering canton of Uri with two villages, \textit{Ursera} (\textit{Andermatt} in German) and \textit{Caschinutta} (\textit{Göschenen} in German).

\begin{table}
	\caption{Locatives III}
	\label{loc3}
	\begin{tabular}{lllllll}
		\lsptoprule
		& & Sèlva & Tschamùt & Ursèra & Caṣchinùta & Uri\\ 
		\midrule
		Sadrún  & → & \textit{ajnta} & \textit{sé/sén} & \textit{vid} & \textit{vi} & \textit{vi gl}\\
		Surajn  & → & \textit{sé} & \textit{sén}  & \textit{vid} & \textit{gjù} \\
		\lspbottomrule
	\end{tabular}
\end{table}

\begin{table}
	\caption{Locatives IV}
	\label{loc4}
	\begin{tabular}{lllllll}
		\lsptoprule
		& & Tschamùt & Sèlva & Ruèras & Sadrún & Bugnaj\\
		\midrule
		Tschamùt  & →   & \textit{}    & \textit{}     & \textit{òragjù}  & \textit{} & \textit{}\\
		Sèlva  & → & \textit{} &  \textit{}  & \textit{}   & \textit{} \\
		\lspbottomrule
	\end{tabular}
\end{table}

\tabref{loc4} shows the prepositions used outside the Tujetsch valley, going down from Mompé Tujetsch, the first village outside the valley, until Trun, which is still located in the Surselva.

\begin{table}
	\caption{locatives V}
	\label{loc5}
	\begin{tabular}{lllllll}
		\lsptoprule
		&       & Mumpé Tujètsch & Ségnas & Mustajr & Surajn/Sumvitg & Trùn\\ 
		\midrule
		Sadrún  & →  & \textit{ò} & \textit{ò}  &    \textit{ò} & \textit{gjù} & \textit{gjù}\\
		Sadrún & ← & \textit{ajnta} & \textit{ajnta} & \textit{ajnta} & \textit{sé}  & \textit{sé} \\
		\lspbottomrule
	\end{tabular}
\end{table}

The villages until Mustér  are treated as if they still belonged to the Tujetsch valley.

\ea
\label{}
\langinfo{Tuatschín}{Bugnaj}{\citealt[132]{Büchli1966}}\\
\gll Lu ṣèn quèls da Sadrún i \textbf{da} \textbf{las} \textbf{Cavòrgjas} \textbf{òra} a staj \textbf{ò} \textbf{Mustajr} avaun ca `ls zagrinders.\\
then \textsc{cop.prs.3pl} \textsc{dem.m.pl} of \textsc{pln} go.\textsc{ptcp.m.pl} of \textsc{def.art.f.pl} \textsc{pln} out and \textsc{cop.ptcp.m.pl} out \textsc{pln} before \textsc{comp} \textsc{def.art.m.pl} yenish.\textsc{pl}\\
\glt `Then the people of Sedrun passed  Cavorgia and were in Mustér before the Yenish.'
\z

The combinations of \textit{cò} `here' with \textit{òra} and \textit{ajn} have two meanings: either `here (down or up the valley)', or `outside (in direction down or up the valley)'.

\ea
\label{}
\langinfo{Tuatschín}{Sadrún}{m5}\\
\gll \textbf{Quòra} ò Mustajr ṣaj bi.\\
here\_out out \textsc{pln} \textsc{cop.prs.3sg.expl} beautiful.\textsc{adj.unm}\\
\z

\ea
\label{}
\langinfo{Tuatschín}{Sadrún}{m5}\\
\gll Quaj è \textbf{quòra}, òn Cavòrgja.\\
\textsc{dem} \textsc{cop.prs.3sg} here\_out, out\_in \textsc{pln}\\
\glt `This is here (in direction down the valley), in Cavorgia.'
\z

\ea
\label{}
\langinfo{Tuatschín}{Cavòrgja}{f1}\\
\gll Ju spétga té \textbf{quòra}.\\
\textsc{1sg} wait.\textsc{prs.1sg} \textsc{2sg} here\_out\\
\glt `I'm waiting for you outside (the house) (in direction down the valley).'
\z

The villages and cities from Trun downwards are all modified by the preposition \textit{gjù} if they are relevant to the speakers, as is Chur or Zürich. If they are not relevant to them, the preposition \textit{á} `to' is used.

\ea
\label{}
\langinfo{Tuatschín}{Cavòrgja}{f1}\\
\gll  Ju vòn \textbf{gjù} \textbf{Turitg}.  \\
\textsc{1sg} go.\textsc{prs.1sg} down \textsc{pln}\\
\glt `I am going (down) to Zurich.'
\z


Regarding the lateral valleys of the Tujetsch valley, most speakers use \textit{ajn} `in(to)', but some use \textit{sé}.

\ea
\label{}
\langinfo{Tuatschín}{Sadrún}{m5}\\
\gll Ju mòn \textbf{ajn} Val Gjuf, \textbf{ajn} Val Val.\\
\textsc{1sg} go.\textsc{prs.1sg} into valley \textsc{pln} into valley \textsc{pln}\\
\glt `I go into the Gjuv valley, into the Val valley.'
\z

\ea
\label{}
\langinfo{Tuatschín}{Sadrún}{m6}\\
\gll Ju mòn \textbf{sé} Val Gjuv, \textbf{sé} Val Val.\\
\textsc{1sg} go.\textsc{prs.1sg} valley \textsc{pln} up valley \textsc{pln}\\
\glt `I go into the Gjuv valley, into the Val valley.'
\z

When coming from the lateral valleys, \textit{ajnagù} or \textit{òragjù} is used.

\ea
\label{}
\langinfo{Tuatschín}{Sadrún}{m2, l. 1627ff.}\\
\gll [...] in … tga stèva ajnasé Gjuf ábar vaj gju da vagní \textbf{navèn} \textbf{da} \textbf{Gjuf} \textbf{tòcan} \textbf{òragjù} … \textbf{Zarcúns}, í da dí páternias tòcan òragjù Zarcúns.\\
{} one.\textsc{m.sg} {} \textsc{rel} live.\textsc{impf.3sg} in\_and\_up \textsc{pln} but  have.\textsc{sbjv.prs.3sg} have.\textsc{ptcp.unm} \textsc{comp} come.\textsc{inf} from of \textsc{pln} until out\_down {} \textsc{pln} \\
\glt `[...] one … that lived up in Gjuf, but who had to come down from Gjuf to … Zarcuns [...].'
\z

\textit{Dadajns/dadajnt} and \textit{dad\underline{ò}} in combination with a place name or another reference point (e.g. a church or a school) means that the subject of the sentence is located outside the place or away form the reference point, in direction up or down the valley.

\ea
\label{}
\langinfo{Tuatschín}{Sadrún}{m5}\\
\gll Ju sùn \textbf{dadajns} \textbf{Ségnas}.\\
\textsc{1sg} \textsc{cop.prs.1sg} from\_in \textsc{pln}\\
\glt `I am outside Ségnas, in direction of the Tujetsch valley.'
\z

\ea
\label{}
\langinfo{Tuatschín}{Sadrún}{m5}\\
\gll Ju sùn \textbf{dad\underline{ò}} \textbf{Ségnas}.\\
\textsc{1sg} \textsc{cop.prs.1sg} from\_out \textsc{pln}\\
\glt `I am outside Ségnas, in direction of Mustér.'
\z

\ea
\label{}
\langinfo{Tuatschín}{Sèlva}{f2, l. 1015f.}\\
\gll Nus vèvan nòssa scùla ajn Sùtcrèstas, qu’ è \textbf{dadajnt} \textbf{Sèlva} [...]. \\
\textsc{1pl} have.\textsc{impf.1pl} \textsc{poss.1pl.f.sg} school in \textsc{pln} \textsc{dem.unm} \textsc{cop.prs.3sg} from\_in \textsc{pln}\\
\glt `We had our school in Sutcrestas, this is outside Selva [in direction up the valley] [...].'
\z

\ea
\label{}
\langinfo{Tuatschín}{Sadrún}{m5, l. 645f.}\\
\gll [...] Nacla, quaj é ... \textbf{dadajns} ... \textbf{Surajn}, fòrza végn minutas vidajn.\\
{} \textsc{pln} \textsc{dem.unm} \textsc{cop.prs.3sg} {} more\_back {} \textsc{pln} maybe twenty minute.\textsc{f.pl} into\\
\glt `[...] Nacla, this is … farther behind … Surrein, maybe twenty minutes farther behind.'
\z

\ea
\label{}
\langinfo{Tuatschín}{Ruèras}{m3 and f4, l. 2153ff.}\\
	\gll \texttt{[m3]} Sas tgé quaj vut dí? Quaj è ljung, quaj è sé Miléz, Ṣcharinas. \texttt{[f4]} Sé Miléz, \textbf{dadajns} \textbf{Miléz} ... \\ 
{} know.\textsc{prs.2sg} what \textsc{dem.unm} want.\textsc{prs.3sg} say.\textsc{inf} \textsc{dem.unm} \textsc{cop.prs.3sg} long.\textsc{adj.unm} \textsc{dem.unm} \textsc{cop.prs.3sg} up \textsc{pln} \textsc{pln} {} up \textsc{pln} in \textsc{pln}\\
\glt  `\texttt{[m3]} Do you know what this means? This is a long way, this is up at Milez, Scharinas. \texttt{[f4]} Up at Milez, to the west of Milez ... '
\z

\ea
\label{}
\langinfo{Tuatschín}{Bugnaj}{\citealt[139]{Büchli1966}}\\
\gll Lu ò ‘l vju \textbf{òragjù} \textbf{dadòr} \textbf{Camischùlas} sén in prau ina familja tga sùlvèva.\\
then have.\textsc{prs.3sg} \textsc{3sg} see.\textsc{ptcp.unm} out.down outside \textsc{pln} on \textsc{indef.art.m.sg} field \textsc{indef.art.f.sg} family \textsc{rel} have.breakfast.\textsc{impf.3sg}\\
\glt `Then he saw down there, outside Camischolas [in direction down the valley], a family which was having breakfast on a field.'
\z

As mentioned, the reference point does not have to be a village or a town; in (\ref{ex:dadajnspun}) it is the bridge over the Drun river. In order to explain to me the meaning of \textit{dadajns}, my consultant told me when we were in the Krüzli hotel:

\ea
\label{ex:dadajnspun}
\langinfo{Tuatschín}{Sadrún}{m5}\\
\gll Nuṣ duṣ èssan \textbf{dadajns} \textbf{la} \textbf{pùn}.\\
\textsc{1pl} two.\textsc{m.pl} \textsc{cop.prs.1pl} from\_in \textsc{def.art.f.sg} bridge\\
\glt `The two of us are away from the bridge [in direction up the valley].'
\z


\tabref{loc5} shows the prepositions that are used with the neighbouring countries (Italy, Germany, Austria, France) or regions (Bavaria).

\begin{table}
	\caption{Locatives VI}
	\label{loc6}
	\begin{tabular}{lllllll}
		\lsptoprule
		& & Italja & Tjaratudèstga  & Baviara & Austrja & Fròntscha\\ 
		\midrule
		Tujétsch  &    →& \textit{gjù l'} &  \textit{ajn}   &  \textit{òn} & \textit{ò l'} & \textit{ajn}\\
		\lspbottomrule
	\end{tabular}
\end{table}

The adverbs follow the same rules as the prepositional phrases. In (\ref{ex:vidòr1}) the speech act participants are in Sedrun and \textit{vidòr} `down the valley' is used because the hearer will go to Naclas (referred to as \textit{ajn lò} `up there') and the speaker wants the hearer to come back to Sedrun.

\ea
\label{ex:vidòr1}
\langinfo{Tuatschín}{Ruèras}{m4; l. 651f.}\\
\gll «Té nò lu \textbf{vidòr} ùssa. Lò, quèsta sèra dòrma lu bigja \textbf{ajn} \textbf{lò}.»\\
\textsc{2sg} come.\textsc{imp.2sg} then down now there  \textsc{dem.f.sg} evening sleep.\textsc{imp.2sg} then \textsc{neg} in there\\
\glt `Come down here now. Don’t sleep up there this evening.»'
\z

Sometimes the combination of locative adverbs do not refer to the direction up or down the valley. This is the case e.g. with \textit{sédòra} in (\ref{ex:sedora1}).

\ea
\label{ex:sedora1}
\langinfo{Tuatschín}{Ruèras}{m1; l. 230ff.}\\
\gll    A sjantar c’ ins mava, sch’ mav’ ins sél Albṣu cul trén, a quaj custav’ in franc dad í \textbf{sédòra}.\\
and after when \textsc{gnr}  go.\textsc{impf.3sg} then go.\textsc{impf.3sg} \textsc{gnr} on.\textsc{def.art.m.sg} \textsc{pln} with.\textsc{def.art.m.sg} train and \textsc{dem.unm} cost.\textsc{impf.3sg} one.\textsc{m.sg}  franc \textsc{comp} go.\textsc{inf} up\_out \\
\glt `And after this, if one went, one would go up to the Alpsu [pass] by train and this costed one franc to go up there.'
\z

In this example \textit{sé} means `up', however \textit{òra} does not mean `down the valley' but `outside'. As a matter of fact, \textit{òra} refers to the fact that the Alpsu pass is not a village where one could be inside, but an open space. The opposite of \textit{sédòra} in such a context is \textit{sédajn} `up and into', as in (\ref{ex:sedajn1}).

\ea
\label{ex:sedajn1}
\langinfo{Tuatschín}{Sadrún}{m5}\\
\gll Èls ajn i \textbf{sédajn} ajn tgèsa.\\
\textsc{3pl.m} be.\textsc{prs.3pl} go.\textsc{ptcp.unm} up\_into into house.\textsc{f.sg}\\
\glt `They went up into the house.'
\z

\subsubsection{Temporal arguments}
Temporal arguments may consist of adverbs, noun phrases, and prepositional phrases. Nouns involved in the forming of temporal arguments are

avaun ca `ls zagrinders Büchli 132

òz an damaun `this morning'

als davùs òns (2)

mintgataun (11)

anad' òtgòntasját (31)

la sèra (93, 707)

adina (113)

zacú (132), zacuras (704)

la duméngja sjantar vjaspras (197)

gl antiar sjantarmjazdé (201f.)

durònt l'ujara (224)

la duméngja sjantarmjazdé (225)

djantarájn (507)

sjantar tschajna (521)

vònzaj `later' (531, 1498)

sònda-duméngja `during week-end' (553)

antiar dé (563)

al dé òra (575f.)

quèsta sèra (614)

ana sissòntasjat (693)

sé sissum (708)

mjasa las déjsch (713)

da las déjsch (714)

al vèndardís sèra (718f.)

la sònda andamaun (721)

da quèlas uras, da quaj tjams (727)

igl avrél (736)

òrdavaun `in advance' (1039)

bjè jèdas (1119)

avaun mèssa (1153)

Nus mavan la damaun api vagnévan la sèra. (1320)

mintga dé (1327)

méasa/mjasa las nùv (1355)

grad ajn quaj mumèn (1480)

ju vèva grat cumprau (1499)

òzaldé (1775)

da nòs tjams (1796)

ditg (1841)

djantarájn (1841)

durònt l'jamna né la sònda-duméngja (1879f.)

las òtg (1898)

tòca gl avrél (1918f.)

la sònda (1922)

gljèndisdís (1922)

às davùs mumèn (2059)

lavá las quátar (2174)

dal dé avaun né da la sèr' avaun (2193f.)

dantaun `meanwhile' (2196)

ancùnt' agl atún (2219)

ancùntar sèra, las quátar, las tschun (2264)

da las quátar (2320)

gljéndisdís andamaun (2353)

la sèra da las quátar (2354)


\ea
\label{}
\langinfo{Tuatschín}{Sadrún}{f3; l.111f.}\\
\gll Ùsa \textbf{quèst}’ \textbf{jamna} vau fatg gròndas turas… da… da sis sjat uras… gè bunamajn \textbf{mintga} \textbf{dé}.  \\
now \textsc{dem.f.sg} week have.\textsc{prs.1sg.1sg} do.\textsc{ptcp.unm} big.\textsc{f.pl} tour.\textsc{pl} of of six seven hour.\textsc{f.pl} yes almost every day.\textsc{m.sg}  \\
\glt `Now this week I made long tours … of … of six seven hours … yes, almost every day.'
\z

\ea
\label{}
\langinfo{Tuatschín}{Surajn}{f5, l.1270ff.}\\
\gll Mintgataun mavan nuṣ èra… plas pitgògnas á cavá cristalas anstagl mirá dlas tgauras, pi vignévan nus halt \textbf{in} \textbf{téc} \textbf{tart}. \\
sometimes go.\textsc{impf.1pl} \textsc{1pl} also around.\textsc{def.art.f.pl}  steep\_slope.\textsc{pl} \textsc{purp} dig.\textsc{inf} crystal.\textsc{f.pl} instead look\_for.\textsc{inf} of.\textsc{def.art.f.pl} goat.\textsc{pl} and come.\textsc{impf.1pl} \textsc{1pl} simply \textsc{indef.art.m.sg} bit late \\
\glt `From time to time we would also … go farther up to extract crystals instead of looking for the goats, and then we would come back a bit late.'
\z


\ea\label{}
\langinfo{Tuatschín}{Camischùlas}{f6; l.710ff.}\\
\gll    [...] tgi ca vagnéva traplaus stuèva \textbf{al} \textbf{vèndardis} \textbf{sèra}… stá lò, stgèvan bigj’ í á tgèsa, api stèvan lu á ṣchùbargè in’ ura zatgéj, durmí lò, api stèvan lu í pèr \textbf{la} \textbf{sònda} \textbf{andamaun} á tgèsa.\\
{} who \textsc{rel} \textsc{pass.aux.impf.3sg} catch.\textsc{ptcp.m.sg} must.\textsc{impf.3sg} \textsc{def.art.m.sg} Friday evening stay.\textsc{inf} there be\_allowed.\textsc{impf.3pl} \textsc{neg} go.\textsc{inf} to home.\textsc{f.sg} and stay.\textsc{impf.3pl} then \textsc{comp} clean.\textsc{inf} one.\textsc{f.sg} hour something sleep.\textsc{inf} there and must.\textsc{impf.3pl} then go.\textsc{inf} only \textsc{def.art.f.sg} Saturday in\_morning to home.\textsc{f.sg}\\
\glt `And the person who … didn’t do that and who got caught, the nuns would just walk around on guard duty, the person who got caught had to … remain there on Friday evening, they were not allowed to go home, and then they had to clean for more or less one hour, sleep there, and then could only go home on Saturday morning.'
\z

\ea\label{}
\langinfo{Tuatschín}{Sadrún}{m4; l.540ff.}\\
	\gll  Ajn quèla végljadétgna, api al pròblèm èra, èra, al pròblèm èra \textbf{las} \textbf{sòndas} a \textbf{dumèngjas}.   \\
	in  \textsc{dem.f.sg} age and \textsc{def.art.m.sg} problem \textsc{cop.impf.3sg} \textsc{cop.impf.3sg} \textsc{def.art.m.sg} problem \textsc{cop.impf.3sg} \textsc{def.art.f.pl} Saturday.\textsc{pl} and Sunday.\textsc{pl} \\
\glt `At that age, and the problem was, was, the problem was on Saturdays and Sundays.'
\z

\ea\label{}
\langinfo{Tuatschín}{Camischùlas}{f6; l.721ff.}\\
\gll A lu èri… da quèlaṣ uras, ah da quaj tjams aun tga… ahm, als amprandissadis antschavévan par part \textbf{ajgl} \textbf{avrél}.\\
and then \textsc{cop.impf.3sg.expl} of \textsc{dem.f.pl} hour.\textsc{pl} ah of \textsc{dem.m.sg} time still  \textsc{comp} ahm \textsc{def.art.m.pl} apprenticeship.\textsc{pl} begin.\textsc{impf.3pl} for part.\textsc{f.sg} in.\textsc{def.art.m.sg} April\\
\glt `And then there was … at that time, ah at that time still that … ahm, the apprenticeships would partly begin in April.'
\z

 Nòssadùna… d’Úast (l.1547, m2, Zarcúns)
 


\ea
\label{}
\langinfo{Tuatschín}{Bugnaj} {\citealt[132]{Büchli1966}}\\
\gll [...] lu ṣèn quèls da Sadrún [...] staj ò Mustajr \textbf{avaun} \textbf{ca} `\textbf{ls} \textbf{zagríndars}.\\
{} then be.\textsc{prs.3pl} \textsc{dem.m.pl} of \textsc{pln} {} \textsc{cop.ptcp.m.pl} out \textsc{pln} before \textsc{comp} \textsc{def.art.m.pl} Yenish\\
\glt `[...] then those from Sedrun [...] were in Musté before the Yenish.'
\z




\subsubsection{Manner arguments}

The comparative of \textit{bégn} `well' can also be formed analytically by \textit{plé} `more'.

\ea\label{}
\langinfo{Tuatschín}{}{\DRG{1}{296}}\\
\gll  I vò ònz \textbf{plé} \textbf{bégn}. \\
     \textsc{expl} go.\textsc{prs.3sg} rather more well\\
\glt `I feel rather better.'
\z


in pèr cazès nùfs, bégn fatgs (Büchli 18, Tschamùt)


\ea
\label{}
\langinfo{Tuatschín}{Cavòrgja}{m7, l.2198f.}\\
	\gll A … la sèrvala dèv’ ju schòn \textbf{nuídis}.    \\
and {} \textsc{def.art.f.sg} cervelat give.\textsc{cond.1sg} \textsc{1sg} indeed reluctantly\\
\glt `And ... the cervelat I would only give away reluctantly.'
\z

faruct ugèn (96)

usché, uschéja

quèluisa (484, 1524, 2053 )

tschèluisa (494)

plaunsjú (660)

ugèn (680)

zacù (1860)

bégn (1874)

a paj (2268)

nuídis (2311)

{\color{red}-majn}



\subsubsection{Quantifiers}

\ea\label{}
\langinfo{Tuatschín}{Ruèras}{m1, l. 292ff.}\\
\gll    A gju quèl al plé gròn plaṣchaj… da… surprèndar lavurs da maridur a da májstar, a mava plé \textbf{bjè} sén gljèz.\\
and have.\textsc{ptcp.unm} \textsc{dem.m.sg} \textsc{def.art.m.sg} more big.\textsc{m.sg.unm} pleasure of  take\_over.\textsc{inf} job.\textsc{f.pl} of bricklayer.\textsc{m.sg} and of joiner.\textsc{m.sg} and  go.\textsc{impf.1sg} more often on \textsc{dem.unm}\\
\glt `And had the greatest pleasure … to take over bricklayers’s or joiners’ jobs, and I did more often that [kind of work.]'
\z

\ea
\label{}
\langinfo{Tuatschín}{Camischùlas}{f6, l. 851f.}\\
\gll    Quaj fùs stau \textbf{pi} \textbf{pir} par mé da stuaj raṣdá ajn tgòmbra cun tschèlas ròmòntschas … tudèstg.\\
\textsc{dem.unm} \textsc{cop.cond.3sg} \textsc{cop.ptcp.unm} more worse for \textsc{1sg} \textsc{comp} must.\textsc{inf} speak.\textsc{inf} in room.\textsc{f.sg} with \textsc{dem.f.pl} Romansh.\textsc{pl} {} German.\textsc{m.sg}\\
\glt `It would have been worse for me if I'd had to speak German in the room with those Romansh room-mates.'
\z

ins vèza bjè (1848f.)

purschju daplé (1860)

\subsubsection{Beneficiaries}
The preposition heading nouns and pronouns is \textit{par/pr}.

\ea\label{}
\langinfo{Tuatschín}{Sadrún}{m4, l. 364ff.}\\
\gll  Èl èr’ in tüp tga raṣdava bigja bjè, ju a gju fétg-fétg bian cun èl, ábar eh, gè qu’ è lu stau in tjams ualti dir cunzún \textbf{par} \textbf{la} \textbf{tata} [...].\\
\textsc{3sg} \textsc{cop.impf.3sg} \textsc{indef.art.m.sg} fellow \textsc{rel} speak.\textsc{impf.3sg} \textsc{neg} much \textsc{1sg} have.\textsc{prs.1sg} have.\textsc{ptcp.unm} \textsc{red}\textasciitilde{very} good.\textsc{m.unm} with \textsc{3sg.m} but eh yes \textsc{dem.unm} be.\textsc{prs.3sg} then \textsc{cop.ptcp.unm} \textsc{indef.art.m.sg} time quite hard especially for \textsc{def.art.f.sg} grandmother\\
\glt `He was a person who didn’t speak much, I went along very well with him, but, eh, yes, this has then been a very hard time especially for my grandmother [...].'
\z

\ea
\label{}
\langinfo{Tuatschín}{Sadrún}{f3, l. 14}\\
\gll «Ah, quaj fùṣ è ina lavur \textbf{pr} \textbf{mè}.»   \\
ah \textsc{dem.unm} \textsc{cop.cond.3sg} also \textsc{indef.art.f.sg} job for \textsc{1sg} \\
\glt `Ah, this could also be a job for me.'
\z





\subsubsection{Further arguments}

Causative arguments are introduced by parví da: \textit{parvi dal ròm\underline{ò}ntsch} (line 792).

Comitative is expressed by \textit{ansjaman} or \textit{ansjaman cun} `together with' if the second element is mentioned.

\ea
\label{}
\langinfo{Tuatschín}{Sadrún}{f6, l. }\\
\gll    Tgé! Quèlas taljánaras èran ampʰau \textbf{ansjaman} a las ròmòntschas né las tudèstgas, né è dal vitg matévani schòn in téc \textbf{ansjaman}.\\
what \textsc{dem.f.pl} Italian.\textsc{f.pl} \textsc{cop.impf.3pl} a\_bit together and \textsc{def\textbf{}.art.f.pl} Romansh.\textsc{.pl} or  \textsc{def.art.f.pl} German.\textsc{.pl} or also of.\textsc{def.art.m.sg} village put.\textsc{impf.3pl.3pl} in\_fact \textsc{indef.m.sg} bit together\\
\glt `Look! these Italians were a bit together, and the Romansh or the Germans, or they put them together even from the [same] village.'
\z

\ea
\label{}
\langinfo{Tuatschín}{Ruèras}{\DRG{1}{602}}\\
\gll   La rég\underline{i}na végn cupanada; èla sgùla a vò á spaz \textbf{ansjaman} \textbf{cun} in grias […]. \\
    \textsc{def.art.f.sg} queen  \textsc{pass.aux.prs.3sg} fertilize.\textsc{ptcp.f.sg} \textsc{3sg} fly.\textsc{prs.3sg} and go.\textsc{prs.3sg} for walk together with \textsc{indef.art.m.sg} drone \\
\glt `The queen is fertilized; she flies away and goes for a trip with a drone […].'
\z


òr dal grép `out of the rock face' (instrumental?) (1075)

antrás ls lavurs `dank der Arbeiten' 

\ea
\label{}
\langinfo{Tuatschín}{Zarcúns}{m2, l. 1579f.}\\
	\gll    Api \textbf{anstagl} \textbf{bájbar} \textbf{al} \textbf{vin} … èran nus lu i sé, vèvani fatg ina bòla.\\
and instead drink.\textsc{inf} \textsc{def.art.m.sg} wine {} be.\textsc{impf.1pl} \textsc{1sg} then go.\textsc{ptcp.m.pl} up have.\textsc{impf.3pl.3pl} do.\textsc{ptcp.unm} \textsc{indef.art.m.sg} punch \\
\glt `And instead of drinking the wine … we went up, they had prepared a punch.'
\z

\section{Negation}
The verb phrase negator has several allomorphs:  \textit{bétga/bétg'/bé/bigja/bgja/bigj'}. The form \textit{bé} is only used by the young generation, with some exceptions.

The negator is located after the non-finite verb (\ref{}), which means that with compound tenses, it is located after the auxiliary verb (\ref{}) and with modal verbs heading an infinitive clause, the negator is situated after the modal verb. There is, however, one exception: in case of an inverted subject, the negator follows the subject (\ref{}).

\ea
\label{}
\langinfo{Tuatschín}{Sadrún}{m4, l. 343ff.}\\
\gll  [...] avaun c’ ju sùn staus tial tat savévu da quaj nuét a \textbf{vèṣ} \textbf{è} \textbf{bitga} safatg ajn zatgé spacjal.  \\
{} before \textsc{comp} \textsc{1sg} be.\textsc{prs.1sg} \textsc{cop.ptcp.m.sg} at.\textsc{def.art.m.sg} grandfather know.\textsc{impf.1sg.1sg} of \textsc{dem.unm} nothing and have.\textsc{cond.1sg} also \textsc{neg} \textsc{refl.}do.\textsc{ptcp.unm} in something special.\textsc{m.sg}\\
\glt `[...] before I stayed with my grandfather I didn’t know anything and I wouldn’t have noticed anything either.'
\z

\ea\label{}
\langinfo{Tuatschín}{Camischùlas}{f6, l. 843ff.}\\
\gll    Ál’ antschatta cu té \textbf{capèschaṣ} \textbf{aun} \textbf{bigja} quèls… curjòs plaids tg’ èls òn, stòs halt dumandá [...].\\
at.\textsc{def.art.f.sg} beginning when \textsc{2sg.gnr} understand.\textsc{prs.2sg.gnr} yet \textsc{neg} \textsc{dem.m.pl} strange.\textsc{pl} word.\textsc{pl} \textsc{rel} \textsc{3pl.m} have.\textsc{prs.3pl} must.\textsc{prs.2sg.gnr} just ask.\textsc{inf}\\
\glt `At the beginning when you don’t understand yet those … strange words they use, you must just ask [...].'
\z

\ea\label{}
\langinfo{Tuatschín}{Sadrún}{f6, l. 754f.}\\
\gll    A la sèra èssan nus, \textbf{stuèvan} \textbf{nuṣ} \textbf{èba} \textbf{bigj’} í ad uraṣ ajnta létg [...].\\
and \textsc{def.art.f.sg} evening be.\textsc{prs.1pl} \textsc{1pl} must.\textsc{impf.1pl} \textsc{1pl} just \textsc{neg} go.\textsc{inf} at hour.\textsc{f.pl} in  bed.\textsc{m.sg} [...].\\
\glt `And in the evening, we went, we just didn’t have to go early to bed [...].'
\z

\ea\label{}
\langinfo{Tuatschín}{Sadrún}{m8, l. 1585f.}\\
\gll  [...] ju \textbf{sa} \textbf{us} \textbf{bigj’} ajfach í ál’ awa.\\
{} \textsc{1sg} can.\textsc{prs.1sg} now \textsc{neg} simply go.\textsc{inf} into.\textsc{def.art.f.sg} water\\
\medskip
\glt `[...] I cannot simply jump into the water.'
\z

\ea
\label{}
\langinfo{Tuatschín}{Sadrún}{m8, l. 1600}\\
\gll  Álṣò ju \textbf{a} \textbf{bigja} \textbf{fatg} aj agrèssíf [...]. \\
well \textsc{1sg} have.\textsc{prs.1sg} \textsc{neg} make.\textsc{ptcm.unm} \textsc{3sg} aggressive.\textsc{adj.unm}\\
\glt `Well, I didn’t do it in an aggressive way [...].'
\z

\ea
\label{}
\langinfo{Tuatschín}{Sadrún}{m8, l. 1505}\\
\gll Ju sa \textbf{bé} dac\underline{ù}.   \\
\textsc{1sg} know.\textsc{prs.1sg} \textsc{neg} why \\
\glt `I don’t know why.'
\z

\ea\label{}
\langinfo{Tuatschín}{Sadrún}{m8, l. 1577}\\
\gll  Ju \textbf{sa} \textbf{bigj’} \textbf{í} ál’ aua.\\
\textsc{1sg} can.\textsc{prs.1sg} \textsc{neg} go.\textsc{inf} into.\textsc{def.art.f.sg} water \\
\glt `Well, I couldn’t jump into the water.'
\z


\ea\label{}
\langinfo{Tuatschín}{Zarcúns}{m2, l. 1485f.}\\
\gll    Quaj è hald ina détga tgu sa, ábar plé gròn savès \textbf{ju} \textbf{lu} \textbf{è} \textbf{bétga} [...]\\
\textsc{dem.unm} \textsc{cop.prs.3sg} simply \textsc{indef.art.f.sg} legend \textsc{rel.1sg} know.\textsc{prs.1sg} but more big.\textsc{m.sg} can.\textsc{cond.1sg} \textsc{1sg} then also \textsc{neg}\\
\glt `This is a legend I know, but a longer one I would not be able [...].'
\z

\ea\label{}
\langinfo{Tuatschín}{Sadrún}{m5, l. 1156}\\
\gll  Álṣò ùṣ è `l \textbf{bigja} \textbf{plé} òdém al vitg [...]. \\
well now  \textsc{cop.prs.3sg} \textsc{3sg.m} \textsc{neg} any\_more low\_most \textsc{def.art.m.sg} village\\
\glt `Well, now it is not at the lowest part of the village [...].'
\z

\ea\label{}
\langinfo{Tuatschín}{Ruèras}{m10, l. 1107f.}\\
\gll  A lu vès ju \textbf{bitga} ugagjau \textbf{plé}, da, dad ira cun èls vinanavaun.\\
and then have.\textsc{cond.1sg} \textsc{1sg} \textsc{neg} dare.\textsc{ptcp.unm} more \textsc{comp} \textsc{comp} go.\textsc{inf} with \textsc{3pl.m} farther\\
\glt `And then I wouldn’t have dared to, to, to go farther with them any more.'
\z

\ea\label{}
\langinfo{Tuatschín}{Sadrún}{m4, l. 660f.}\\
\gll Api òni tractau quaj, a sjantar mava quaj \textbf{bigja} \textbf{plé} gjù Surajn [...].\\
and have.\textsc{prs.3pl.3pl} treat.\textsc{ptcp.unm} \textsc{dem.unm} and after go.\textsc{impf.3sg} \textsc{dem.unm} \textsc{neg} any\_more down \textsc{pln}\\
\glt `And they treated that, and after this it was not possible any more for him to live in Surrein [...].'
\z

\ea
\label{}
\langinfo{Tuatschín}{Sadrún}{m4, l.399f.}\\
\gll  «Ju cala dad í á scùlèta, ju pùs \textbf{bitg} í \textbf{plé}.»\\
\textsc{1sg} stop.\textsc{prs.1sg} \textsc{comp} go.\textsc{inf} to nursery\_school.\textsc{f.sg} \textsc{1sg} can.\textsc{prs.1sg} \textsc{neg} go.\textsc{inf} any\_more  \\
\glt `I’ll stop going to nursery school, I can’t stand it any longer.'
\z

\ea
\label{}
\langinfo{Tuatschín}{Sadrún}{m4}\\
\gll Ábar ju sa \textbf{bigja} tgé quaj è \textbf{plé}.\\
but \textsc{1sg} know.\textsc{prs.1sg} \textsc{neg} what \textsc{dem.unm} \textsc{cop.prs.3sg} more\\
\glt `But I don't know what this is any more.'
\z

\ea
\label{}
\langinfo{Tuatschín}{Tschamùt} {\citealt[12]{Büchli1966}}\\
\gll   Èl è \textbf{bétga} jus \textbf{plé} cun quaj catschadur. \\
      \textsc{3sg} be.\textsc{prs.3sg} \textsc{neg} go.\textsc{ptcp.m.sg} any.more with \textsc{dem.m.sg} hunter\\
\glt `[…] he didn’t go with this hunter any more.'
\z

\ea
\label{}
\langinfo{Tuatschín}{Sadrún}{m6, l. 1435ff.}\\
	\gll    [..] quaj piartg èra juṣ atráṣ a vèva rùt gjù al matg tga vèva \textbf{mù} la còrda \textbf{plé} antùrn.\\
{} \textsc{dem.m.sg} pig be.\textsc{impf.3sg} go.\textsc{ptcp.m.sg} through and have.\textsc{impf.3sg} break.\textsc{ptcp.unm} down \textsc{def.art.m.sg} bunch  \textsc{comp} have.\textsc{impf.3sg} only \textsc{def.art.f.sg} rope more around\\
\glt `[...] this pig had gone through and had broken the bunch of flowers so that he only had the rope around [his belly].'
\z


\ea\label{}
\langinfo{Tuatschín}{Tschamùt} {\citealt[15]{Büchli1966}}\\
\gll  [...] scha vagi èl \textbf{mù} in cazè \textbf{plé}.\\
     then have.\textsc{prs.sbjv.3sg} \textsc{3sg} only one.\textsc{m} shoe any.more \\
\glt `[…] then he would have only one shoe left.'
\z

\ea
\label{}
\langinfo{Tuatschín}{Bugnaj} {\citealt[147]{Büchli1966}}\\
\gll    Als pádars savèvan \textbf{bétga} spatgè plé ditg \textbf{plé}.\\
      \textsc{def.art.m.pl} Father.\textsc{pl} know.\textsc{impf.3pl} \textsc{neg} wait.\textsc{inf} more long any\_more\\
\glt `The fathers couldn’t wait any longer […].'
\z

\ea
\label{}
\langinfo{Tuatschín}{Cavorgia} {\citealt[119]{Büchli1966}}\\
\gll    Lura lèvani \textbf{bétga} schè fá pástar gròn èla \textbf{plè} […].\\
     then want.\textsc{impf.3pl.3pl} \textsc{neg} let.\textsc{inf} make.\textsc{inf} shepherd big \textsc{3sg.f} any\_more\\
\glt `Then they didn’t want to let her be the main shepherdess […] any more.'
\z

\ea\label{}
\langinfo{Tuatschín}{}{\citealt[69]{Berther2007}}\\
\gll   I tégnan \textbf{bétga} sé schi bégn als praus cò \textbf{plé}. \\
     \textsc{3pl} hold.\textsc{prs.3pl} \textsc{neg} up so well \textsc{def.art.m.pl} field.\textsc{pl} here any.more \\
\glt `Here they don’t see to the fields well any more.'
\z

\ea
\label{}
\langinfo{Tuatschín}{Sadrún}{m9, l. 1775ff.}\\
	\gll [...] òzaldé … sa ju \textbf{bigja} métar avaun \textbf{plé} tg’ i fòn da gljèz, né?    \\
{} nowadays {} can.\textsc{prs.1sg} \textsc{1sg} \textsc{neg} put before more \textsc{comp} \textsc{3pl} do.\textsc{prs.3pl} of \textsc{dem.unm} right \\
\glt `And I can imagine nowadays ... I cannot imagine any more that they play that, right?'
\z



The following example illustrates \textit{mù … plé} `only ... more', which displays the same syntax as \textit{bétga plé}.



in infinitive clause

\ea\label{}
\langinfo{Tuatschín}{Ruèras}{m10, l. 1017ff.}\\
\gll  [...] quaj èssan nuṣ i sé pr… \textbf{pr} \textbf{bitga} \textbf{stuaj} \textbf{ira} sé la… sé la via dad autos, sé la via dal pas. \\
{} \textsc{dem.unm} be.\textsc{prs.1pl} \textsc{1pl} go.\textsc{ptcp.m.pl} up \textsc{purp} \textsc{purp} \textsc{neg} must.\textsc{inf} go.\textsc{inf} up \textsc{def.art.f.sg} up \textsc{def.art.f.sg} way of car.\textsc{m.pl} up \textsc{def.art.f.sg} way of.\textsc{def.art.m.sg} pass  \\
\glt `[...] there we went up in order to avoid the car way, the way of the pass.'
\z



`not yet' apparently bétg' aun

\ea\label{}
\langinfo{Tuatschín}{Sadrún}{m6, l. 1377f.}\\
\gll  Ju vèva \textbf{bigja} \textbf{aun} amprju, ajn scùlèta amprandèvan nus pauc.\\
\textsc{1sg} have.\textsc{impf.1sg} \textsc{neg} yet  learn.\textsc{ptcp.unm} in  nursery\_school.\textsc{f.sg} learn.\textsc{impf.1pl}  \textsc{1pl} little \\
\glt `I hadn’t learned yet, we didn’t learn much in nursery school.'
\z

\ea\label{}
\langinfo{Tuatschín}{Surajn}{f5, l. 1261f.}\\
\gll    [...] lu èri \textbf{aun} \textbf{bigja} turists [...]. \\
{} then \textsc{exist.impf.3sg.expl} yet \textsc{neg} tourist.\textsc{m.pl}\\
\glt [...]  then there weren’t yet tourists [...].'
\z

\ea
\label{}
\langinfo{Tuatschín}{Sadrún}{f3, l. 10}\\
\gll  Ábar lu èra quaj \textbf{bigja} \textbf{aun} schi bjè.  \\
but then \textsc{cop.impf.3sg} \textsc{dem.unm} \textsc{neg} yet so much  \\
\glt `But at that time this was not that much yet.'
\z

nuéta

\ea
\label{}
\langinfo{Tuatschín}{Sadrún}{m4, l. 385}\\
\gll A quaj plaṣchéva \textbf{nuéta} pròpi da mé. \\
and \textsc{dem.unm} please.\textsc{impf.3sg} \textsc{neg} really \textsc{dat} \textsc{1sg}\\
\glt `And I really didn’t  like that.'
\z

\ea
\label{}
\langinfo{Tuatschín}{Ruèras}{f4, l. 2033f.}\\
	\gll «Ò quaj pùs té schòn, ùṣa quaj è \textbf{nuéta} schi nausch.»\\
oh \textsc{dem.unm} can.\textsc{prs.2sg} \textsc{2sg} indeed now \textsc{dem.unm} \textsc{cop.prs.3sg} \textsc{neg} so bad.\textsc{adj.unm}\\
\glt «Oh, you are certainly able to do that, now this is not so bad».'
\z

nuét `nothing'

\ea
\label{}
\langinfo{Tuatschín}{Cavòrgja}{f7, l. 2335f.}\\
	\gll Al bap ò détg \textbf{nuét}.\\
\textsc{def.art.m.sg} father have.\textsc{prs.3sg} say.\textsc{ptcp.unm} nothing\\
\glt `My father didn't say anything.'
\z


There is no double negation after \textit{senza} ‘without’.

\ea\label{}
\langinfo{Tuatschín}{Bugnaj} {\citealt[147]{Büchli1966}}\\
\gll    Al gjával ò stiu í \textbf{sènza} savaj fá \textbf{zatgéj}.\\
      \textsc{def.art.m.sg} devil have.\textsc{prs.3sg} must.\textsc{ptcp.unm} go.\textsc{inf} without know do.\textsc{inf} something \\
\glt `The devil had to leave without being able to do anything.'
\z






