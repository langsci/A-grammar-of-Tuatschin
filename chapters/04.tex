\chapter{Verb phrase}

\section{The verb}
Tuatschin possesses intransitive, mono-, and ditransitive verbs. Usually ditransitive verbs have a direct and an indirect object (which is marked by  \textit{də}, \textit{a}) \textit{di} or \textit{li}.

\ea\label{}
\langinfo{Tuatschin}{Sedrun}{\citealt[106]{Büchli1966}}\\
\gll    In də Méidel vèev' in' ura də Schwarzwald tg' èera rutta.\\
     one of \textsc{pln} have.\textsc{impf.3sg} \textsc{indef.art.f.sg} clock of \textsc{pln} \textsc{rel} \textsc{cop.impf.3sg} break.\textsc{ptcp}\\
\glt `An inhabitant of Meidal had a clock of the Black Forest that was broken.'
\z

\ea\label{}
\langinfo{Tuatschin}{Sedrun}{\citealt[103]{Büchli1966}}\\
\gll    In dé […] pərtgirava in buep las vaccas sen Vons.\\
     \textsc{indef.art.m.sg} day […] tend.\textsc{impf.3sg} \textsc{indef.art.m.sg} boy \textsc{def.art.f.pl} cow.\textsc{pl} up \textsc{pln}\\
\glt `One day a boy was tending the cows at Vons.'
\z

\ea\label{}
\langinfo{Tuatschin}{Bugnai}{\citealt[145]{Büchli1966}}\\
\gll    […] Nossadunna lèeva dâ li giuven préir ina rǝncůnuscientscha pǝl survetsch.\\
   [...]  Virgin want.\textsc{impf.3sg} give \textsc{dat} young.\textsc{m.sg} priest \textsc{indef.art.f.sg} mark.of.gratitude for.\textsc{def.art.m.sg} favour \\
\glt `[…] the holy Virgin wanted to give the young priest a mark of gratitude for the favour [he had done her].'
\z

In rare cases, ditransitive verbs have two direct objects. This is especially the case with \textit{dumandá} `ask'.

\ea\label{}
\langinfo{Tuatschin}{Bugnei} {\citealt[131]{Büchli1966}}\\
\gll  [...]  ina zagrindəra […] ò dumandau \textbf{la} \textbf{mumma} də tgèesa \textbf{in} \textbf{tgavegl} də sia bueba. \\
    [...] \textsc{indef.art.f.sg}   gipsy.woman [...] have.\textsc{prs.3sg}   ask.\textsc{ptcp} \textsc{def.art.f.sg} mother of house one hair of \textsc{poss.3sg} girl \\
\glt `[…] a gipsy woman asked the mother of the house one hair of her daughter.'
\z

Another case is \textit{dá fjuc la lèna} (m11) `to light firewood', literally `give fire the firewood'.

\subsection{Verbal morphology}
Tuatschin displays five conjugation classes whose infinitives end in \textit{-á} (\textit{anflá}) 'find', \textit{-è} (\textit{magliè}) `eat', \textit{-ai} (\textit{tanai} `hold'), \textit{-í} (\textit{fugí} `flee'), and \textit{-ar} (\textit{métar} `put').  From a diachronic point of view, the \textit{-è}-class is a subclass of the \textit{-a}-class, due to the general rule that \textit{a} becomes \textit{è} after a palatal consonant or glide.

Tuatschin has three non-finite categories: infinitive, past participle, and gerund.  Within the finite categories, the language differentiates tense, aspect, and modal categories as well as simple,  compound, and doubly-compound categories.

The simple categories are present indicative, present subjunctive, imperfect indicative, imperfect subjunctive, conditional direct, conditional indirect, and imperative.

The compound categories are perfect indicative, perfect subjunctive, pluperfect indicative,  pluperfect subjunctive, and future. The compound tenses are formed with an auxiliary verb (either \textit{èssar} 'be', \textit{vaj} 'have', or \textit{vagní} 'come') and the past participle or the infinitive. 

The doubly compound categories correspond to the perfect and the pluperfect,
but with two auxiliary verbs instead of one.


\subsubsection{Auxiliary verbs}
The auxiliary verbs \textit{èssar} (\tabref{tab:aux:èssar}) and \textit{vai} (\tabref{tab:aux:vai}) are used for compound and doubly-compound tenses, whereas \textit{vagní} (\tabref{tab:aux:vagní}) is used for future. In the following tables, only one compound tense will be listed, the perfect; as for the doubly-compound tenses, they are formed with the perfect or the imperfect of the auxiliary verb vai, the participle of vai and the participle of the main verb and need not to be listed. Examples will be given below in section xxx. 

According to the DRG (1: 568), the choice of \textit{èssar} as auxiliary verb for reflexives in Sursilvan is due to the demand of Sursilvan grammarians since the 18th century. Nowadays speakers seek to conform to this claim, but in spoken Sursilvan, \textit{haver} (or \textit{vai} in Tuatschin) is still widespread.

\begin{table}
\caption{Auxiliary verb \textit{èssar}}
\label{tab:aux:èssar}
 \begin{tabular}{lllll}
 
  \lsptoprule
 & \textsc{inf}  & \textsc{ptcp.m}  & \textsc{ptcp.f}  &  \textsc{ger}\\
  \midrule
  &èssar &staus & stada & èssèn\\
  &&stai &stadas\\
   
  \lsptoprule
    &\textsc{prs.ind}  &\textsc{impf.ind} & \textsc{prf.ind} & \textsc{fut}\\
   \midrule
ju &sùn & èra &sun staus/stada &végn ad èssar\\
té &ais &èras &ais staus/stada & végnas ad èssar\\
èl, èla &è & èra &è staus/stada &végn ad èssar\\
nus &èssan &èran &èssan stai/stadas &vagnin ad èssar\\
vus &èssas & èras &èssas stai/stadas &vagnis ad èssar\\
èls, èlas& èn & èran &èn stai/stadas &végnan ad èssar\\

 \lsptoprule
   &\textsc{prs.sbjv} & \textsc{impf.subj.}  &\textsc{cond.dir.} & \textsc{cond.ind.}\\
\midrule
ju& &&& \\
té\\
èl/èla\\
nus\\
vus\\
èls, èlas\\
  \lspbottomrule
 \end{tabular}
\end{table}


\begin{table}
\caption{Auxiliary verb \textit{vai}}
\label{tab:aux:vai}
 \begin{tabular}{lllll} % add l for every additional column or remove as necessary
 
  \lsptoprule
 & \textsc{inf}  & \textsc{ptcp.m}  & \textsc{ptcp.f}  &  \textsc{ger}\\
  \midrule
  &vai &gju &  & \\
     
  \lsptoprule
    &\textsc{prs.ind}  &\textsc{impf.ind} & \textsc{prf.ind} & \textsc{fut}\\
   \midrule
ju &a & vèva &a gju &végn a vai\\
té &as &vèvas &as gju & végnas a vai\\
èl, èla &ò & vèva &ò gju &végn a vai\\
nus &vain &vèvan &vain gju &vagnín a vai\\
vus &vais & vèvas &vais gju &vagnís a vai\\
èls, èlas& òn & vèvan &òn gju &végnan a vai\\

 \lsptoprule
   &\textsc{prs.sbjv} & \textsc{impf.subj.}  &\textsc{cond.dir.} & \textsc{cond.ind.}\\
\midrule
ju& &&& \\
té\\
èl/èla\\
nus\\
vus\\
èls, èlas\\
  \lspbottomrule
 \end{tabular}
\end{table}

\begin{table}
\caption{Auxiliary verb \textit{vagní}}
\label{tab:aux:vagní}
 \begin{tabular}{lllll} % add l for every additional column or remove as necessary
 
  \lsptoprule
 & \textsc{inf}  & \textsc{ptcp.m}  & \textsc{ptcp.f}  &  \textsc{ger}\\
  \midrule
  &èssar &staus & stada & èssèn\\
  &&stai &stadas\\
   
  \lsptoprule
    &\textsc{prs.ind}  &\textsc{impf.ind} & \textsc{prf.ind} & \textsc{fut}\\
   \midrule
ju &sùn & èra &sun staus/stada &végn ad èssar\\
té &ais &èras &ais staus/stada & végnas ad èssar\\
èl, èla &è & èra &è staus/stada &végn ad èssar\\
nus &èssan &èran &èssan stai/stadas &vagnin ad èssar\\
vus &èssas & èran &èssas stai/stadas &vagnis ad èssar\\
èls, èlas& èn & èran &èn stai/stadas &végnan ad èssar\\

 \lsptoprule
   &\textsc{prs.sbjv} & \textsc{impf.subj.}  &\textsc{cond.dir.} & \textsc{cond.ind.}\\
\midrule
ju& &&& \\
té\\
èl/èla\\
nus\\
vus\\
èls, èlas\\
  \lspbottomrule
 \end{tabular}
\end{table}




\subsubsection{Regular verbs}
As mentioned above, the \textit{è-}conjugation has split from the original \textit{á}-conjugation (< Latin \textsc{-are})  because of the presence of a preceding palatal consonant. \tabref{tab:èconj} lists some examples of \textit{è}-verbs with their Sursilvan counterpart. Note that the final \textit{-r} of the infinitives is not pronounced in any Sursilvan variety.

\begin{table}
\caption{è-conjugation in Tuatschin and Sursilvan}
\label{tab:èconj}
 \begin{tabular}{llll} % add l for every additional column or remove as necessary
 \lsptoprule
&\textsc{Tuatschin}  & \textsc{Sursilvan}  & \textsc{English} \\
  \midrule
 ʎ & magliè &magliar& 'eat' \\
ʥ&cargè&cargar&'carry'\\
ʨ&spatgè&spitgar&'wait'\\
ʧ&catschè&catschar&'hunt'\\
ʃ&schè&schar&'let'\\
j&sijê&segar&'mowle'\\   
 \lspbottomrule
 \end{tabular}
\end{table}


\begin{table}
\caption{Regular verbs in -è}
\label{tab:reg:verb:-è}
 \begin{tabularx}{.6\textwidth}{XXXX} % add l for every additional column or remove as necessary
 
  \lsptoprule
  \textsc{inf}  & \textsc{ptcp.m}  & \textsc{ptcp.f}  &  \textsc{ger}\\
  \midrule
  magljè & magljau & magljèda & magljòn\\
  \lspbottomrule  
  \end{tabularx}
  
  \medskip
  
 \begin{tabularx}{\textwidth}{p{1,7cm}lllll}
  \lsptoprule
\textsc{pers.pron} &\textsc{prs.ind} &\textsc{prs.subj} &\textsc{impf.ind} & \textsc{impf.subj} &\textsc{prf.ind}\\
 \midrule
ju &maglja & magli&magljava &magljavi & a magljau\\
té &magljas & maglias&magljavas &magljávias &as magljau\\
èl, èla &maglja &magli & magljava &magljavi &ò magljau\\
nus &magljáin & maglian    &magljavan &magliavian &vain magljau\\
vus &magljáis &magljas & magljavas &magliavias &vais magljau\\
èls, èlas& magljan &magljan &magljava &magliavian &òn magjliau\\
  \lspbottomrule
\end{tabularx}

\medskip

\begin{tabularx} {\textwidth}{p{2cm}XXlX}
 \lsptoprule
  \textsc{pers.pron} &\textsc{dir.cond} &  \textsc{indir.cond} & \textsc{fut}  &\textsc{imp}\\
\midrule
ju& magljás&magljassi &végn a magljè\\
té& magliassas&magljassias &végnas a magljè &maglia\\
èl, èla& magliás &magljassi &végn a magljè\\
nus& magljassan & magljássian &vagnin a magljè\\
vus& magljassas & magljássias &vagnis a magljè &magljái\\
èls, èlas& magljassan & magljassian&végnan a magljè\\
  \lspbottomrule
 \end{tabularx} 
\end{table}

The ending in -\textit{a} of the first person singular present and imperfect indicative is typical of Tuatschin Sursilvan. The standard ending in Sursilvan is -\textit{el} (\textit{jeu giavisch-el} `I wish'); however I found one example of \textit{-a} in the variety of Riein, a village situated in the Lumnezia valley.

\ea\label{}
\langinfo{Sursilvan}{Riein}{\citealt[327]{DRG4}}\\
\gll  Jeu giavisch-\textbf{a} bien cunfiert.  \\
     \textsc{1sg} wish-\textsc{prs.1sg} good consolation\\
\glt `My heartfelt sympathy.'
\z

This form was current in this local variety of Sursilvan for 1st person singular present and imperfect, as the following forms show: \textit{jeu astga} `I am allowed to', \textit{jeu gneva} `I used to come', \textit{jeu era} `I was', \textit{jeu suna} `I play (an instrument)', \textit{jeu sunava} `I used to play', and so on (examples taken out of the\textit{ Questiunari principal} of the DRG, recorded between 1900 and 1920; Ursin Lutz p.c., 2017/04/19).

In Medelin, the ending was also -\textit{a}.

\ea\label{}
\langinfo{Medelin}{}{\citealt[104]{Widmer1962}}\\
\gll   […] jeu \textbf{siemia} da lavinas […]. \\
     […] \textsc{1sg} dream.\textsc{prs.1sg} of avalanche.\textsc{pl} […]\\
\glt `[...] I dream of avalanches [...].'
\z

\subsubsection{Verbs with regular stem alternations}

\subsubsection{Irregular verbs}

\subsection{Usage of tense, aspect, and modal categories}
In this section, the usage of the non-finite categories past participle and gerund and all the finite categories will be analyzed, with the exception of the imperative, which will be treated in section xxx, below p. xxx.

\subsubsection{Past participle}
The past participle is used to form compound and doubly-compound tenses as well as passive voice; it is furthermore used attributively and predicatively and may also be nominalized. 

Some elements may intervene between the auxiliary verb and the past participle.

\ea\label{}
\langinfo{Tuatschin}{}{\DRG{4}{304}}\\
\gll Ùssa, quel dalla quaida ṣai \textbf{puspè} staus ain cul dét, mirai tschò!\\
   now \textsc{dem.m.sg} of.\textsc{def.art.f.sg} desire \textsc{be.prs.3sg} again \textsc{be.ptcp.m.sg.pred} in with finger.\textsc{m.sg} look.\textsc{imp.2p} here\\
\glt `Now the sweet-toothed has again stuck his finger into it, look here!'
\z


In compound tenses, if two verbs which need different auxiliaries are conjoined, either the subject or the subject and the auxiliary of the second verb may be omitted.

\ea\label{}
\langinfo{Tuatschin}{Rueras} {\citealt[69]{Büchli1966}}\\
\gll Ju \textbf{sun} \textbf{ius} avaun nuegl ət \textbf{a} \textbf{griu} ə bargiu […].\\
     \textsc{1sg}  be.\textsc{prs.1sg}  go.\textsc{ptcp.m.sg} before barn and have.\textsc{prs.1sg} shout.\textsc{ptcp} and cry.\textsc{ptcp}\\
\glt `I went in front of the barn and shouted and cried.'
\z


\ea\label{}
\langinfo{Tuatschin}{Rueras} {\citealt[68]{Büchli1966}}\\
\gll La damaun \textbf{essan} aun \textbf{lavai} baud ə \textbf{mirau} da nos tiers.\\
     \textsc{def.art.f.sg}  morning be.\textsc{3pl} still get.up.\textsc{ptcp.m.pl}  early and look.\textsc{ptcp} of \textsc{poss.1pl} animal.\textsc{m.pl}\\
\glt `In the morning we got up early and looked after our animals.'
\z

\ea\label{}
\langinfo{Tuatschin}{Selva}{\citealt[47]{Büchli1966}}\\
\gll El \textbf{ò} \textbf{priu} las duas sadialas gromma \textbf{ǝd} \textbf{è} \textbf{ius} òod téegia \textbf{ǝ} \textbf{svanius}.\\
    \textsc{3sg}  have.\textsc{prs.3sg} take.\textsc{ptcp} \textsc{def.art.f.pl} two.\textsc{f} bucket.\textsc{pl}  cream and \textsc{cop.prs.3sg} go.\textsc{ptcp.m.sg} out.of hut and disappear.\textsc{ptcp.m.sg}\\
\glt `He [the devil took the two buckets full of cream and left the hut and disappeared.'
\z

If the subject of a compound tense with the auxiliary èssar follows the verb, then the participle does not agree with the subject.

\ea\label{}
\langinfo{Tuatschin}{}{\citealt[13]{Büchli1966}}\\
\gll    Encunter sèera, cu ‘ls tiers èeran ain stavel, ṣai \textbf{vegniu} \textbf{ina} \textbf{nibla} \textbf{stgira} e dau in gron uradi. \\
    towards evening when \textsc{def.art.m.pl} animal.\textsc{pl} \textsc{cop.impf.3pl} in barn \textsc{cop.prs.3sg} come.\textsc{ptcp} \textsc{def.art.f.sg} cloud dark and give.\textsc{ptcp} \textsc{indef.art.m.sg} big storm\\
\glt `Towards evening, when the animals were in the barn, a dark cloud came and there was a big storm.'
\z

Absence of agreement

\ea\label{ex:1:}
\langinfo{Medelin}{Curaglia}{\DRG{1}{256}}\\
\gll  Per quel sei \textbf{vargiau} \textbf{als} \textbf{onns}, quel a mo als dis ple.\\
    for \textsc{dem.m.sg} \textsc{cop.prs.3sg} pass.\textsc{ptcp.m.sg} \textsc{def.art.m.pl} year.\textsc{pl}   \textsc{dem}  have.\textsc{prs.3sg} only \textsc{def.art.m.pl} day.\textsc{pl} any.more\\
\glt `For this man, the years are over, he has only some days left [to live].'
\z

\ea\label{}
\langinfo{Tuatschin}{Camischolas}{\DRG{3}{583}}\\
\gll  Al tétg da duas alas fatgs cun aissas betga splanadas […] ai sén lattas.\\
   \textsc{def.art.m.sg} roof of two side.\textsc{pl} make.\textsc{ptcp.m.sg.pred} with plank.\textsc{f.pl} \textsc{neg} plane.\textsc{ptcp.f.pl} […] \textsc{cop.prs.3sg} on slat.\textsc{f.pl.}\\
\glt `The two-sided roof made with planks that haven't been planed […] are on slats.'
\z


When the past participle is used attributively, the masculine singular form does not take the predicative -s if it has no complements; if it has complements, the participle is treated like a predicative adjective/participle.

\ea\label{}
\langinfo{Tuatschin}{Camischolas}{\DRG{3}{583}}\\
\gll  Al tétg da duas alas fatgs cun aissas betga splanadas […] ai sén lattas.\\
   \textsc{def.art.m.sg} roof of two side.\textsc{pl} make.\textsc{ptcp.m.sg.pred} with plank.\textsc{f.pl} \textsc{neg}plane.\textsc{ptcp.f.p.} […] \textsc{cop.prs.3sg} on slat.\textsc{f.pl.}\\
\glt `The two-sided roof made with planks that haven't been planed […] are on slats.'
\z




\subsubsection{Gerund}


\ea\label{}
\langinfo{Tuatschin}{Surrein} {\citealt[53]{Büchli1966}}\\
\gll    åls pasters udèevan adina \textbf{a} \textbf{vegnen} tiers.\\
   \textsc{def.art.m.pl} herdsman.\textsc{pl} hear.\textsc{impf.3pl} always \textsc{prep} come.\textsc{ger}  animal.\textsc{pl}\\
\glt `The herdsmen were always hearing cattle coming […].'
\z

\ea\label{}
\langinfo{Tuatschin}{Selva} {\citealt[28]{Büchli1966}}\\
\gll    ina sèera […] ò’ ‘ls pasters viu \textbf{ad} \textbf{en} las vaccas.\\
     \textsc{indef.art.f.sg} evening [...] have.\textsc{prs.3sg} \textsc{def.art.m.pl} herdsman.\textsc{pl} see.\textsc{ptcp} \textsc{prep} go.textsc{ger} \textsc{def.art.f.pl} cow.\textsc{pl} \\
\glt `[…] one evening the herdsmen saw the cows going away.'
\z

\ea\label{}
\langinfo{Tuatschin}{Camischolas} {\citealt[82]{Büchli1966}}\\
\gll   \textbf{Rǝturnond} gl um betg anavůůs dǝ misdé, o’ las zǝrclunzas tumiu […].\\
     come.back.\textsc{ger} \textsc{def.art.m.sg} man \textsc{neg} back of noon have.\textsc{prs.3pl} \textsc{def.art.f.pl} weeder.woman.\textsc{pl} be.afraid.\textsc{ptcp}\\
\glt `Since the man hadn’t come back by noon, the weeder women got afraid […].'
\z

\ea\label{}
\langinfo{Tuatschin}{Camischolas} {\citealt[88]{Büchli1966}}\\
\gll    \textbf{Mond} spell’ aua dǝ Segnas sé òn els udiu da tschella vard ǝnzatgi […].\\
     go.\textsc{ger} beside.\textsc{def.art.f.sg} water of  \textsc{pln} up have.\textsc{prs.3pl} \textsc{3pl} hear.\textsc{ptcp} of \textsc{dem} side somebody \\
\glt `When walking along the creek of Segnas they heard somebody on the other side […].'
\z

The gerund is also used as a modifier of the noun. In such cases, it behaves like an adjective and agrees with the noun it modifies, as in in schem lamentond ‘a moaning sigh (M.SG)’ (B 19), cristall-a-s tarlischont-a-s (cristal-F-PL shining-F-PL) ‘shining cristals’ (B 18), or.


\subsection{Infinitive}

The infinitive, which is the nominal and the citation form of the verb, can also be overtly nominalized by the definitive masculine singular article to form a subject sentence.
 
\ea\label{}
\langinfo{Tuatschin}{Camischolas}{\DRG{3}{584}}\\
\gll \textbf{Al} \textbf{dèrgiar} \textbf{giù} ai lu aun mal. Al Vageli Mon ai vagnus sut in caschnè.\\
  \textsc{def.art.m.sg} demolish.\textsc{inf} down \textsc{cop.prs.3sg} then still bad \textsc{def.art.m.sg} \textsc{pn} \textsc{pn} be.\textsc{prs.3sg} come.\textsc{ptcp.m.sg.pred} under \textsc{indef.art.m.sg} hayrack\\
\glt `Demolish [a hayrack] is indeed dangerous. Vigeli Monn came under a hayrack.'
\z

As example shows, the presence of the article is not necessary.

\ea\label{}
\langinfo{Medelin}{Curaglia}{\DRG{3}{585}}\\
\gll \textbf{Dèrgiar} \textbf{sé} èra prigulus.\\
    put.up.\textsc{inf} up \textsc{cop.impf.3sg} dangerous.\textsc{m.sg}\\
\glt `Put up [a hayrack] was dangerous.'
\z


\subsubsection{Present indicative}

\ea\label{}
\langinfo{Tuatschin}{Surrein}{\citealt[128]{Büchli1966}}\\
\gll    Ju sund ius sé Culmatsch ina dumengia. \\
\textsc{1sg} be.\textsc{prs.1sg} go.\textsc{ptpc.m.sg} up \textsc{pln} \textsc{indef.art.f.sg} Sunday\\
     \glt `One Sunday I went up to Culmatsch.'
\z


\subsubsection{Imperfect indicative}


\subsubsection{Perfect indicative}
The perfect is formed either with the auxiliary verb essar `be' or avai `have'. If the verb is conjugated with essar, the participle agrees with the subject.




Sometimes, the perfect is used where one would expect the imperfect because the verb does not refer to a situation with beginning, middle, and end, or, as in the following case, it has no inchoative reading.

\ea\label{}
\langinfo{Tuatschin}{Bugnei} {\citealt[139]{Büchli1966}}\\
\gll    El ò ǝncůnuschiu la familia, mů mai detg òora, tgi èeri \\
     \textsc{3sg} have.\textsc{prs.3sg} know.\textsc{ptcp} \textsc{def.art.f.sg} family but never tell.\textsc{ptcp} out who \textsc{cop.impf.sbjv.3sg}\\
\glt `He knew the family, but never said who they were.'
\z
\ea\label{}
\langinfo{Tuatschin}{Sedrun} {\citealt[103]{Büchli1966}}\\
\gll   ål buep ò schon ǝncůnuschiu ellas \\
     \textsc{def.art.m.sg} boy have.textsc{prs.3sg} already know.\textsc{ptcp} \textsc{3pl.f}\\
\glt `The boy already knew them [= the girls].'
\z

\ea\label{}
\langinfo{Tuatschin}{Cavorgia} {\citealt[53]{Büchli1966}}\\
\gll    La fumeglia d’ alp ò saviu nuet.\\
     \textsc{def.art.f.sg} farmhand.\textsc{coll} of alp have.\textsc{prs.3sg} know.\textsc{ptcp} nothing\\
\glt `The alp shepherds didn’t know anything.'
\z



\subsubsection{Pluperfect indicative}


\subsubsection{future}

\subsubsection{Doubly-compound tenses}

\subsubsection{Progressive aspect}
The progressive aspect is formed with the copula essar, the preposition vid ‘at’, the definite article masculine singular, and the infinitive of the verb.

\ea\label{}
\langinfo{Tuatschin}{Bugnei} {\citealt[132]{Büchli1966}}\\
\gll    Duas zǝrclunzas èeran vid ‘l zǝrclâ.\\
     two.\textsc{f} weeder.woman.\textsc{pl} \textsc{cop.impf.3pl} at \textsc{def.art.m.sg} weed.\textsc{inf}\\
\glt `Two weeder women were weeding […].'
\z




\subsubsection{Subjunctive}


Usage:

\ea\label{}
\langinfo{Tuatschin}{Ruèras}{\DRG{1}{393}}\\
\gll  Ju \textbf{tegn} tga quai \textbf{vegn}-\textbf{i} fatg pauc.  \\
    \textsc{1sg} hold.\textsc{prs.1sg} \textsc{comp} {dem} \textsc{pass.aux}-\textsc{prs.sbjv.3sg} make.\textsc{ptcp.grl} little\\
\glt `I think that this is not often done.'
\z


\ea\label{ex: }
\langinfo{Tuatschin}{}{\citealt[87]{Gadola1935}}\\
\gll  “[…] i o dau las siat.” “Lu cuschai, \textbf{tg}’in \textbf{aud-i}.”\\
     […] \textsc{expl} have.\textsc{prs.3sg} give.\textsc{ptcp} \textsc{def.art.f.pl} seven then be.quiet.\textsc{imp.2pl} \textsc{comp} hear-\textsc{prs.subj.3sg}\\
\glt `”[…] It has struck seven o’clock.” “Then be quiet, so that we can hear.”'
\z

\ea\label{}
\langinfo{Tuatschin}{}{\DRG{3}{560}}\\
\gll  Quèl \textbf{tratga} tg’èl \textbf{sai} sa po tgé!\\
    \textsc{dem} think.\textsc{prs.3sg} \textsc{comp=3sg} \textsc{cop.prs.sbjv.3sg} can.\textsc{prs.3sg} \textsc{interj} what \\
\glt `He thinks that he is something special!'
\z

\ea\label{}
\langinfo{Tuatschin}{}{\DRG{3}{582}}\\
\gll  Da mé \textbf{par}'\textbf{ai} tg'al aifer-piast da véjdar \textbf{èri} bétga schi lads.\\
    \textsc{dat} \textsc{1sg} seem.\textsc{prs.3sg=expl} \textsc{comp=def.art.m.sg} hayrack.field  of old.\textsc{m.sg} \textsc{cop.impf.subj.3sg} \textsc{neg} so wide.\textsc{m.sg.pred} \\
\glt `It seems to me that the surface of the hayrack of earlier times was not that wide.'
\z

\ea\label{}
\langinfo{}{}{\DRG{}{}}\\%dialect? where?
\gll  \textbf{Mira} tga quai lò \textbf{davjanti} bétg. \\
     look.\textsc{imper.2sg} \textsc{comp} \textsc{dem} there become.\textsc{pres.sbjv.3sg} \textsc{neg}\\
\glt `Make sure that this does not happen.'
\z

\ea\label{}
\langinfo{Tuatschin}{}{\DRG{5}{649}}\\
\gll Dius \textbf{banadèschi} a \textbf{carschjanti}!\\
     god bless.\textsc{prs.sbjv.3sg} and thrive.\textsc{prs.sbjv.3sg}\\
\glt `May God bless [it] and make [it] thrive!'
\z

\ea\label{}
\langinfo{Tuatschin}{}{\DRG{5}{777}}\\
\gll   \textbf{Avaun} \textbf{tgi} \textbf{végn} mal’ aura isan las vacas ṣgarṣcháival.\\
    before \textsc{comp.expl} come.\textsc{prs.ind.3sg} bad weather run.back.and.forth.\textsc{prs.3sg} \textsc{def.art.f.pl} cow.\textsc{pl} terrible.\textsc{m.sg} \\
\glt `Before bad weather comes, the cows run back and forth like mad.'
\z


\subsubsection{Direct and indirect conditional}







\subsubsection{Imperative}


\ea\label{}
\langinfo{Tuatschin}{}{\DRG{5}{649}}\\
\gll Dius \textbf{banadèschi} a \textbf{carschjanti}!\\
     god bless.\textsc{prs.sbjv.3sg} and thrive.\textsc{prs.sbjv.3sg}\\
\glt `May God bless [it] and make [it] thrive!'
\z


\subsubsection{}


\subsection{Particle verbs}
A particle verb is a verb that combines with an adverb to form a semantic unit. An example is\textit{ fá giù}, literally `make down', which means `make an appointment'.  The origin of this structure is controversial: it is considered either a genuine Romansh structure, or a loan from German or Swiss German. It is difficult to prove these hypotheses, because there are no Romansh texts before 1500 c.e., i.e. before Romansh was in heavy contact with German. \textit{fá giù}, however, is clearly a case of calque from Swiss German. In Swiss German, 'make an appointment` is [''ab'maxə]; the prefix \textit{ab}- has been interpreted as ['abə] `down', hence \textit{giù}, and  ['maxə] means 'do, make', which leads to \textit{fá giù}. 

There is an important difference between the German and the Romansh construction: In German, standard or Swiss, the particle is a verbal prefix which in simple tenses is located at the end of a sentence, as in

\ea\label{}
\langinfo{Swiss German}{}{own knowledge}\\
\gll  [ix \textbf{max} jedə ta:g mit im \textbf{ab}]\\
     \textsc{1sg}  make every day with him  \textsc{ptcl} \\
\glt `I make an appointment with him every day.'
\z

In such cases, the particle follows the verb in Tuatschin (and other Romansh varieties).

\ea\label{}
\langinfo{Swiss German}{}{own knowledge}\\
\gll  Ju \textbf{fétsch} \textbf{gjù} cun èl mintga dé.\\
     \textsc{1sg}  make.\textsc{prs.1sg}  \textsc{ptcl} with \textsc{3sg.m}  every day\\
\glt `I make an appointment with him every day.'
\z

However, in Tuatschin and other Romansh varieties, the particle is not immediately adjacent to the verb, since some words may intervene between the verb and the particle. These words are inverted subjects (pronouns or full noun phrases), the negator \textit{bétga} and its variants, the epistemic adverb \textit{schon} `certainly', and the temporal adverbs \textit{magari} `sometimes' and \textit{savens} `often'. %What about full NPs? 

\ea\label{}
\langinfo{Tuatschin}{Sadrún}{m13}\\
\gll   Damaun prèn \textbf{ju} sé èl.\\
     tomorrow take.\textsc{prs.1sg} \textsc{1sg} up \textsc{3sg}\\
\glt `Tomorrow I will lift him up.'
\z

\ea\label{}
\langinfo{Tuatschin}{}{\citealt[91]{Gadola1935}}\\
\gll  Té mu trafica usché vinavaun, epi sietta \textbf{il} \textbf{militer} giu tè in dé.\\
     \textsc{2sg} just be.up.to.\textsc{prs.3sg} so further and shoot.\textsc{prs.3sg} \textsc{def.art.m.sg} army down \textsc{2sg} \textsc{indef.art.m.sg} day\\
\glt `You just go on doing this way, and the army will shoot you down one day.' 
\z

\ea\label{}
\langinfo{Tuatschin}{Sadrún}{m13}\\
\gll   Damaun prèn ju \textbf{bégia} sé èl.\\
     tomorrow take.\textsc{prs.1sg} \textsc{neg} \textsc{1sg} up \textsc{3sg}\\
\glt `Tomorrow I won’t lift him up.'
\z

\ea\label{}
\langinfo{Tuatschin}{} {\citealt[51]{Berther1998}}\\
\gll  Al plé mal stun ju pal bien cazè. Al polisch crèscha \textbf{schòn} ensiamen.\\
     \textsc{def.art.m.sg}  most bad stay.\textsc{prs.1sg}  \textsc{1sg} for.\textsc{def.art.m.sg} good shoe \textsc{def.art.m.sg} thumb grow.\textsc{prs.3sg}  certainly together\\
\glt `I am very sorry for the shoe of good quality. My big toe will certainly knit together.'
\z

\ea\label{}
\langinfo{Tuatschin}{Sadrún}{m16}\\
\gll  Prèn \textbf{puspè} sé quaj! \\
    take.\textsc{imp.2sg} again up \textsc{dem}  \\
\glt `Lift this again! '
\z

\ea\label{}
\langinfo{Tuatschin}{Sadrún}{m13}\\
\gll  Ju prèn \textbf{magari} sé èl.  \\
     \textsc{1sg} take.\textsc{prs.1sg} sometimes up \textsc{3sg}\\
\glt `Sometimes I lift him up.'
\z

\ea\label{}
\langinfo{Tuatschin}{Sadrún}{m13}\\
\gll  Cun quèl fètschu \textbf{maj} gjù.  \\
    with \textsc{dem} make.\textsc{prs.1sg} \textsc{1sg} never down\\
\glt `With this person I never make an appointment.'
\z

\ea\label{}
\langinfo{Tuatschin}{Sadrún}{m13, f1}\\
\gll    Ju prèn \textbf{savèns} sé èl.\\
     \textsc{1sg} take.\textsc{prs.1sg} often up \textsc{3sg}\\
\glt `I often lift him up.'
\z

\ea\label{}
\langinfo{Tuatschin}{Sadrún}{f1}\\
\gll    Quèl prènd ju sé \textbf{savèns}.\\
     \textsc{dem} take.\textsc{prs.1sg} \textsc{1sg} up often\\
\glt `I often lift him up.'
\z

\ea\label{}
\langinfo{Tuatschin}{Sadrún}{m16}\\
\gll Èls prendan \textbf{spèrt} sé als ufauns.   \\
   \textsc{3pl} take.\textsc{prs.3pl} rapidly up \textsc{def.art.m.pl} child.\textsc{pl}  \\
\glt `They lift the children rapidly.'
\z

With \textit{spèrt} 'rapidly', the adverb may occur between the verb and its particle or not.

\ea\label{}
\langinfo{Tuatschin}{Sadrún}{m16}\\
\gll Els prendan sé \textbf{spèrt} als ufauns.   \\
   \textsc{3pl} take.\textsc{prs.3pl} up rapidly \textsc{def.art.m.pl} child.\textsc{pl}  \\
\glt `They lift the children rapidly.'
\z

\ea\label{}
\langinfo{Tuatschin}{Ruèras}{m11}\\
\gll Ju prèn \textbf{aun} sé \textbf{spèrt} agl ufaun.    \\
    \textsc{1sg} take.\textsc{prs.1sg} still up rapidly \textsc{def.art.m.sg} child \\
\glt `Right now, I’ll lift the child rapidly.'
\z

\ea\label{}
\langinfo{Tuatschin}{Ruèras}{m11}\\
\gll Ju prèn \textbf{spèrt} \textbf{aun} sé agl ufaun.    \\
    \textsc{1sg} take.\textsc{prs.1sg} rapidly still up \textsc{def.art.m.sg} child \\
\glt `Right now, I’ll lift the child rapidly.'
\z

\ea\label{}
\langinfo{Tuatschin}{Ruèras}{m11}\\
\gll Ju prèn \textbf{aun} \textbf{dabòt} sé agl ufaun.    \\
    \textsc{1sg} take.\textsc{prs.1sg} rapidly still up \textsc{def.art.m.sg} child \\
\glt `Right now, I’ll lift the child rapidly.'
\z

The adverb \textit{mintgataun} `sometimes', which is a synonym of \textit{magari}, may not intervene between the verb and the particle.% Or do some informants hesitate?

\ea\label{}
\langinfo{Tuatschin}{Sadrún}{m13}\\
\gll   *Ju prèn \textbf{mintgataun} sé èl. \\
     \textsc{1sg} take.\textsc{prs.1sg} sometimes up \textsc{3sg}\\
\glt `Sometimes I lift him up.'
\z




bét(t)ë nëvèn quaj vs *bét(t)ë quai nëvèn (CecHen 4: 203)

\ea\label{}
\langinfo{Medelin}{}{\DRG{3}{118}}\\
\gll  \textbf{Cavrognas} puspei \textbf{o} tut? \\
    nose.around.\textsc{prs.2sg} again out all\\
\glt `Do you nose around in everything again?'
\z


\subsection{Copulative verbs}

Change of states are formed with vegnir ‘come’ as auxiliary verb.

\ea\label{}
\langinfo{Tuatschin}{Bugnei} {\citealt[145]{Büchli1966}}\\
\gll    Te stůs stâ cò tůt persula ǝd i vegn unviern ǝ vegn fraid […].\\
    \textsc{2sg} must.\textsc{prs.2sg} stay here all alone.\textsc{f.sg} and \textsc{expl} come.\textsc{prs.3sg} winter and come.\textsc{prs.3sg} cold  \\
\glt `You must stay here alone, and winter is coming and it is getting cold.'
\z

\ea\label{}
\langinfo{Medelin}{Platta}{\DRG{5}{634}}\\
\gll  Èl antschavèva schòn a \textbf{vagní} \textbf{fraids}.\\
     \textsc{3sg} begin.\textsc{impf.3sg} already \textsc{comp}  become.\textsc{inf} cold.\textsc{m.sg.pred}\\
\glt `He [= the dead] was already beginning to become rigid.'
\z


\subsection{Existential verbs}
Existential constructions are formed with the expletive pronoun\textit{it} and the verbs \textit{èssar} `be', \textit{vai} `have', or \textit{dá} `give'.

\ea\label{}
\langinfo{Tuatschin}{Rueras} {\citealt[69]{Büchli1966}}\\
\gll  ǝt \textbf{i} \textbf{èera} vairamain dus vadials ain ina cadeina.\\
     and \textsc{expl} \textsc{exist.impf.3sg} really two calf.\textsc{pl} \textsc{loc} \textsc{indef.art.f.sg}  chain\\
\glt `[…] and there were really two calves attached with a chain.'
\z

\ea\label{}
\langinfo{Tuatschin}{Camischolas} {\citealt[82]{Büchli1966}}\\
\gll  I èer’ a Tschamůt ina matta fetg lůůṣcha.\\
     \textsc{expl} \textsc{exist.impf.3sg} in \textsc{tn} \textsc{indef.art.f.sg} girl very vain.\textsc{f.sg}\\
\glt ` There was a very vain girl in Tschamutt.'
\z

\ea\label{}
\langinfo{Tuatschin}{Bugnai}{\citealt[145]{Büchli1966}}\\
\gll    […] tůttas flurs tge \textbf{dèeva} lò […].\\
     […] all.\textsc{f.pl} flower.\textsc{pl} \textsc{rel} \textsc{exist.impf.3sg} there \\
\glt `[…] all the flowers that were there […].'
\z

\ea\label{}
\langinfo{Tuatschin}{}{\DRG{3}{464}}\\
\gll   Vé Sadrun \textbf{asai} in catsch.\\
   over.there \textsc{pln} \textsc{exist.prs.3sg} \textsc{def.art.m.sg} large.crowd \\
\glt `Over there in Sedrun, there is a large crowd.'
\z

\ea\label{}
\langinfo{Tuatschin}{Camischolas}{\DRG{3}{591}}\\
\gll   \textbf{I} \textbf{dat} è \textbf{tga} végn misch sils caschnès.\\
     \textsc{expl} \textsc{exist.prs.3sg} also \textsc{comp} get.\textsc{prs.3sg} mouldy on.\textsc{def.art.m.pl} hayrack.\textsc{pl}\\
\glt `It also happens that [the sheaves] get mouldy on top of the hayracks.'
\z


\subsection{Modal verbs}

\ea\label{}
\langinfo{Tuatschin}{Sedrun} {\citealt[106]{Büchli1966}}\\
\gll    «Gion, \textbf{vul} te betga gidâ me dǝ cargèe quella bůra?» «Ben ben, sců ju \textbf{pos}, vi ju schon gidâ.»\\
     \textsc{pn} want.\textsc{prs.2sg} \textsc{2sg} \textsc{neg} help.\textsc{inf} \textsc{1sg} of charge \textsc{dem.f.sg} block yes yes how \textsc{1sg} can.\textsc{prs.1sg} want.\textsc{prs.1sg} \textsc{1sg} certainly help.\textsc{inf}\\
\glt `«Gion, don’t you want to help me charge this block?» «Yes, sure, I will certainly help [you] as well as I can.»'
\z

\ea\label{}
\langinfo{Tuatschin}{}{\DRG{3}{376}}\\
\gll   Co \textbf{pon} ins cargè tschunconta vacas.\\
     here can.\textsc{prs.3sg} \textsc{gnr} charge fifty cow.\textsc{f.pl}\\
\glt `Here one can put to graze fifty cows.'
\z

\ea\label{}
\langinfo{Medelin}{}{\DRG{6}{441}}\\
\gll  Jeu \textbf{munglas} stá spèra fjuc. \\
    \textsc{1sg} should.\textsc{cond.1sg} stay next.to fire.\textsc{m.sg} \\
\glt `I should remain next to the fire [in order to watch it]. '
\z


\subsection{The construction \textit{vai} \textit{tga} 'have that'}

\ea\label{}
\langinfo{Tuataschin}{Ruèras}{\DRG{2}{397}}\\
\gll  Ju \textbf{a} \textbf{tga} fo bled.\\
     \textsc{1sg} have.\textsc{prs.3sg} \textsc{rel} do.\textsc{prs.3sg} sick\\
\glt `I feel sick.'
\z

\ea\label{}
\langinfo{Tuatschin}{}{\DRG{2}{215}}\\
\gll I brischa la cazzeta, i \textbf{vegn} \textbf{tga} suffla. \\
   \textsc{expl} burn.\textsc{prs.3sg} \textsc{def.art.f.sg} pot \textsc{expl} \textsc{fut.aux.prs.3sg} \textsc{rel} blow.\textsc{prs.3sg} \\
\glt `[The soot] on the pot is burning, it is getting stormy.'
\z



\section{Arguments of the verb} 
 
\subsection{Direct object}

\subsection{Indirect object}
 




\ea\label{}
\langinfo{Tuatschin}{Sèlva}{\citealt[26]{Büchli1966}}\\
\gll […] å la patruna ò dau aun ina \textbf{li} \textbf{zagrindǝra}.\\
    [...] and \textsc{def.art.f.sg} housewife have.\textsc{prs.3sg} give.\textsc{ptcp} still one.\textsc{f} \textsc{dat} gypsy.woman.\textsc{f.sg}\\
\glt `[…] and the houswife gave one more to the gypsy woman.'
\z

\ea\label{}
\langinfo{Tuatschin}{}{\DRG{1}{359}}\\
\gll Vais dau castognas \textbf{li} \textbf{empaladur}?   \\
     have.\textsc{prs.2pl} give.\textsc{ptcp.m.pl} chestnut.\textsc{f.pl} \textsc{dat} leader.\textsc{m.sg} \\
\glt `Did you give chestnuts to the leader [of the animals]?'
\z



\ea\label{}
\langinfo{Tuatschin}{Bugnai}{\citealt[143]{Büchli1966}}\\
\gll […] lu òn ins detg quai \textbf{li} \textbf{préir} […].\\
[…] then have.\textsc{prs.3pl} \textsc{gless} tell.\textsc{ptcp} \textsc{dem} \textsc{dat} priest\\
\glt `[…] then they told this to the priest […].'
\z

\ea\label{ex:1:}
\langinfo{Tuatschin}{}{\DRG{1}{163 }}\\
\gll  Quel dat ain alv \textbf{lis} \textbf{tiers}. \\
    \textsc{dem} give.\textsc{prs.3sg} in white \textsc{dat.art.m.pl} animal.\textsc{pl} \\
\glt `He mixes a lot of hay with the fodder. '
\z


 \ea\label{}
\langinfo{Tuatschin}{}{\citealt[199]{ASR1889}}\\
\gll Sche tots ratuns èn intenzionai uschê sco quels dus, lura vali betga tier els il proverbi, tge survescha \textbf{lis} \textbf{carstgauns} per zanur […].\\
    if all.\textsc{m.pl} \textsc{rat.pl} \textsc{cop.prs.3pl} intentioned.\textsc{pl} so like \textsc{dem.m.pl} two then apply.\textsc{prs.3sg.expl} \textsc{neg} by \textsc{3pl.m} \textsc{def.art.m.sg} proverb \textsc{rel} serve.\textsc{prs.3sg} \textsc{dat.pl} person.\textsc{m.pl} for dishonour\\
\glt `If all the rats were as well-intentioned as these two, the proverb which dishonours human beings wouldn’t apply to them […].'
\z

\ea\label{}
\langinfo{Tuatschin}{}{\DRG{1}{163}}\\
\gll  Quel dat ain alv \textbf{lis} \textbf{tiers}.  \\
     \textsc{dem.m.sg} give.\textsc{prs.3sg} in white \textsc{dat} animal.\textsc{m.pl}\\
\glt `He mixes hay with the fodder.'
\z

\ea\label{}
\langinfo{Tuatschin}{}{\DRG{2}{480}}\\
\gll   Las giuvnas duèssan ins tanai sen bratsch, a \textbf{lis} \textbf{véglias} dá cun in scanatsch.\\
    \textsc{def.art.f.pl} young.\textsc{pl} have.to.\textsc{cond.3sg} \textsc{gnr} hold.\textsc{inf} on arm and \textsc{dat.pl} old.\textsc{f.pl} give.\textsc{inf} with \textsc{indef.art.m.sg} log\\
\glt `The young women we should hold on our arms, and the old ones we should beat them with a log.'
\z





The definite dative article \textit{di/dis} or \textit{li/lis} is only used with full noun phrases. With personal pronouns \textit{di} or \textit{li} only functions as dative marker since it cannot be pluralized (cf. above xxx). This holds also for other pronouns that are definite, like \textit{tschel} 'that, the other'. In the following example, \textit{tschel} is pluralized but \textit{li} is not.

\ea\label{}
\langinfo{Tuatschin}{Sadrun}{\citealt[104]{Büchli1966}}\\
\gll  […] lu ò ‘l detg \textbf{li} \textbf{tschels}: […].\\
      […] then have.\textsc{prs.3sg} \textsc{3sg} say.\textsc{ptcp} \textsc{dat} \textsc{dem.m.pl}\\
\glt `[…] then he said to the others: […].'
\z


The allomorph of \textit{li}, \textit{di}, occurs much less than \textit{li} in the written documents; nevertheless it occurs in all documents except for Büchli (1966), and the native speakers I have consulted and who use the dative article only use \textit{di}.

\ea\label{ex:1:}
\langinfo{Tuatschin}{}{\DRG{5}{365}}\\
\gll  esser si dies \textbf{di} \textbf{vischnaunca}  \\
     \textsc{cop.inf} on back \textsc{dat} municipality\\
\glt `to become a burden on the municipality'
\z


\ea\label{}
\langinfo{Tuatschin}{Ruèras}{\citealt[8]{Valär2013b}}\\
\gll sen ʧa'mʊt ɔ l 'ɔndɐ dɐl tat daw dʊ'mɛnʥɐ vɐr'ʥɛːda kʊ lɐ 'ʦʊlvɐ d 'inɐ pɐ'naʎɐ 'ʃtʊrnɐ \textbf{diː} \textbf{fʊmǝ'ʨɛːza} a rʊt in dɛt […].\\
     on \textsc{pln} have.\textsc{prs.3sg} \textsc{def.art.f.sg} aunt of.\textsc{def.art.m.sg} grandfather give.\textsc{ptcp} Sunday pass.\textsc{ptcp.f.sg} with \textsc{def.art.f.sg} pestle of  \textsc{indef.art.f.sg} butter.churn crazy \textsc{dat} farm.girl and break.\textsc{ptcp.} \textsc{indef.art.m.sg} finger\\
\glt `Last Sunday in Tschamut the aunt of the grandfather hit the farm girl with the pestle of a butter churn and broke [her] a finger […].'
\z


\ea\label{}
\langinfo{Tuatschin}{}{\citealt[69]{Berther1998}}\\
\gll […] uonn dun iu ain mia dunna \textbf{di} \textbf{gediu}.\\
     […] this.year give.\textsc{prs.1sg} \textsc{1sg} in \textsc{poss.1sg.f} wife \textsc{dat} Jew \\
\glt `[…] this year I’ll give my wife to the Jew.'\footnote{Until more or less 150 years ago, Swiss Jews were only allowed to settle in two villages in the canton of Aargau. Until recently some of them worked as cattle dealers in the whole country.}
\z

\ea\label{}
\langinfo{Tuatschin}{}{\DRG{1}{366}}\\
\gll Quai da l’arveglia vegn dau \textbf{dis} \textbf{tgauras}.  \\
     \textsc{dem} of  \textsc{def.art.f.sg}=pea \textsc{pass.aux.prs.3sg} give.\textsc{ptcp} \textsc{daf} goat.\textsc{f.pl} \\
\glt `The straw of the peas is given to the goats.'
\z


If the noun precludes the use of the definite article, the marker \textit{da} is used. This is the case if, e.g., the noun is modified by an indefinite article (which is a zero-marker if the noun is plural), a quantifier, or a possessive determiner.

\ea\label{}
\langinfo{Tuatschin}{Sèlva}{\citealt[25]{Büchli1966}}\\
\gll […] ò i tuccau \textbf{d'} \textbf{ina} \textbf{matta} […].\\
      […] have.\textsc{prs.3sg} \textsc{expl} touch.\textsc{ptcp} \textsc{dat} \textsc{indef.art.f.sg} girl \\
\glt `[…] it was a girl's turn […].'
\z

\ea\label{}
\langinfo{Tuatschin}{}{\citealt[118]{Berther1998}}\\
\gll Betga mattai se scalins \textbf{da} \textbf{tgauras} \textbf{jastras}.\\
     \textsc{neg} put.\textsc{imp.2pl} up bell.\textsc{m.pl} \textsc{dat} goat.\textsc{pl} somebody.else’s.\textsc{pl}\\
\glt `Don’t put bells on somebody else’s goats.'
\z


\ea\label{}
\langinfo{Tuatschin}{Sadrun}{\citealt[104]{Büchli1966}}\\
\gll […] el dai dî \textbf{də} \textbf{negin} \textbf{carstgaun}, tg’ el vagi viu ellas cò […].\\
     […] \textsc{3pl}  should.\textsc{prs.sbjv.3sg} say.\textsc{inf} \textsc{dat} no person \textsc{comp} 3sg have.\textsc{prs.sbjv.3sg} see.\textsc{ptcp} \textsc{3pl.f} here\\
\glt `[…] he shouldn’t tell anybody that he had seen them here […].'
\z

\ea\label{}
\langinfo{Tuatschin}{Sèlva}{\citealt[25]{Büchli1966}}\\
\gll Perquai o ‘la dau tissi \textbf{dǝ} \textbf{siu} \textbf{piertg} […].\\
     therefore have.\textsc{prs.3sg} \textsc{3sg.f} give.\textsc{ptcp} poison \textsc{dat} \textsc{poss.3sg.m} pig \\
\glt ` Therefore she gave poison to her pig […].'
\z

\ea\label{}
\langinfo{Tuatschin}{}{\citealt[81]{Gartner1910}}\\
\gll kwaj ku'seʎ ɔ pla'ʒiw \textbf{al} \textbf{ʨawn}.\\
    \textsc{dem} advice have.\textsc{prs.3sg} like.\textsc{ptcp} \textsc{dat} dog\\
\glt `This advice the dog liked.'
\z

\ea\label{}
\langinfo{Tuatschin}{Sadrún}{m4, lines 239-240}\\
\gll   Quaj duvrava’l par dá \textbf{dis} \textbf{pòrs}, trúfals ansjaman par dá \textbf{als} \textbf{pòrs}. \\
%check whether DALS pors%
 \textsc{dem} use.\textsc{impf.3sg} \textsc{purp} give.\textsc{inf}  \textsc{dat.pl} pig.\textsc{pl} potato.\textsc{pl} together  \textsc{purp}  give.\textsc{inf} \textsc{dat.m.pl} pig.\textsc{pl} \\
\glt `This he used to give the pigs, together with potatoes to give the pigs.'
\z

It is not clear to me if the dative article is productive among those speakers who still use it; as example () shows, \textit{dá di/dis} is an expression meaning 'to feed' and it might well be that the dative article is only used in such cases.

\ea\label{}
\langinfo{Tuatschin}{Cavòrgja}{f1}\\
\gll «Ju a dau dis pòrs»? Da lèzas uras mintga familja vèva in piartg, dus, trais, a quaj mintgin savèva tgé ca què lèṣ di. Òz stuèssan nus mataj dí «Ju a dau da magljè dis pòrs» né «Ju a parvasjú més pòrs». \\
    \textsc{1sg} have.\textsc{prs.1sg} give.\textsc{ptcp} \textsc{dat.art.pl} pig.\textsc{pl} of \textsc{dem.f.pl} hour every family.\textsc{f.sg} have.\textsc{impf.3sg} \textsc{indef.art.m.sg} pig two three and  \textsc{dem} everybody know.\textsc{impf.3sg} what \textsc{comp}  \textsc{dem} want.\textsc{cond.3sg} say nowadays must.\textsc{cond.1pl} \textsc{1pl} probably  say.\textsc{inf} \textsc{1sg} have.\textsc{prs.1sg} give.\textsc{ptcp}  \textsc{comp} eat.\textsc{inf} \textsc{dat.art.pl} pig.\textsc{pl} or  \textsc{1sg} have.\textsc{prs.1sg} feed.\textsc{ptcp} \textsc{poss.1sg} pig.\textsc{pl}     \\
\glt `I gave to the pigs ? Formerly, every family had a pig, two, three, and this everybody knew what it meant. Nowadays we probably might have to say «I gave to eat to the pigs» or «I fed my pigs».'
\z

In Medelin, there are almost no traces of \textit{(a)li} or \textit{da} as dative markers; however, in the DRG materials some rare examples can be found. One of these is \textit{ali gjaval} `to the devil', a curse in which \textit{ali} is normally used in all current Sursilvan varieties.

\ea\label{}
\langinfo{Medelin}{}{\DRG{5}{215}}\\
\gll   ir \textbf{ali} \textbf{giaval} el tgil\\
    go.\textsc{inf} \textsc{dat} devil in.\textsc{def.art.m.sg} ass\\
\glt `go to hell' (literally `go into the devil’s ass')
\z


\ea\label{}
\langinfo{Medelin}{}{\DRG{4}{186}}\\
\gll  \textbf{Ada} \textbf{quel} vein nus cupanau giu crest’ oz.\\
     \textsc{io} \textsc{dem.m.sg} have.\textsc{prs.1pl} \textsc{1pl} fuck.\textsc{ptcp} down cockscomb\textsc{f.sg} today\\
\glt `Den haben wir heute tüchtig ausgescholten.'
\z

\textit{Da} as a dative marker is also reported for some Surmiran and Sutsilvan varieties:

\ea\label{}
\langinfo{Surmiran}{Marmorera}{\DRG{5}{19}}\\
\gll  Ja da detg \textbf{da} \textbf{mia} \textbf{sora} tgi la vegna no.\\
      \textsc{1sg} have.\textsc{prs.1sg} say.\textsc{ptcp} \textsc{dat} \textsc{poss.1sg} sister \textsc{comp} \textsc{3sg.f} come.\textsc{prs.sbjv.3sg} here\\
\glt `I told my sister to come here.'
\z


Ja do(u)m in mail da chel umfant. (DRG 5:19, Marmorera)

Ea il meas mattet, da quella povra gliat stoni gidar. (DRG 5: 19, Val Schons)





%semantic roles!



\subsection{Locative arguments}
Locatives may either be adverbs or adpositional phrases. These are formed with simple and complex prepositions as well as with circumpositions. Noun phrases within an adpositional phrase preclude the use of the definite article; however, if the place name is modified by another element, the definite article must be used.

In Tuatschin, as well as in other Romansh varieties, locative indications are very important.\footnote{See } When the speaker is located in the Tujetsch valley, he or she must indicate whether they are or go in direction out of the valley (\textit{ò} `out', going down the valley), in direction into the valley (\textit{ainta} `into', going up the valley), or over to a place (\textit{vi}), usually seen from the speech act place. If he or she are or go outside the valley, they must say whether they go up (\textit{sé}) or down (\textit{giù}).

\ea\label{}
\langinfo{Tuatschin}{Cavorgia}{\citealt[121]{Büchli1966}}\\
\gll å la notg ṣè ‘l giat vegnius \textbf{ainta} \textbf{letg} […].\\
     and \textsc{def.art.f.sg} night be.\textsc{prs.3sg} \textsc{def.art.m.sg} cat come.\textsc{ptcp.m.sg} into bed\\
\glt `And at night the cat came into [his] bed […].'
\z

\ea\label{}
\langinfo{Tuatschin}{Cavorgia}{\citealt[123]{Büchli1966}}\\
\gll Cu i òn purtau la bara \textbf{òo} \textbf{dǝ} \textbf{tgèesa}, mirav’ el \textbf{da} \textbf{fanestr}’ \textbf{òora}.\\
    when \textsc{3pl} have.\textsc{prs.3pl} carry.\textsc{ptcp} \textsc{def.art.f.sg} corpse out of house
 look.\textsc{impf.3sg} \textsc{3sg.m} of window out\\
\glt `When they carried the corpse out of the house, he was looking out of the window.'
\z

\ea\label{}
\langinfo{Tuatschin}{Bugnai}{\citealt[132]{Büchli1966}}\\
\gll Lu ṣen quels də Sedrun î \textbf{də} \textbf{las} \textbf{Cavorgias} \textbf{òora} å stai \textbf{òo} \textbf{Mustair} avaun chə 'ls zagrinders \\
       then \textsc{cop.prs.3pl} \textsc{dem.m.pl} of \textsc{pln} go.\textsc{ptcp.m.pl} of \textsc{def.art.f.pl} \textsc{pln} out and \textsc{cop.ptcp.m.pl} out \textsc{pln} before \textsc{comp} \textsc{def.art.m.pl} gypsy.\textsc{pl}\\
\glt `Then the people of Sedrun passed Las Cavorgias and were in  Mustair before the gypsies.' %avaun che + noun!
\z

\ea\label{}
\langinfo{Tuatschin}{Cavorgia}{\citealt[121]{Büchli1966}}\\
\gll […] ǝ mava \textbf{dǝ} la porta ǝ \textbf{dǝ} las rèmas \textbf{ain} ain clavau.\\
     […] and go.\textsc{impf.3sg} from \textsc{def.art.f.sg} door and from \textsc{def.art.f.pl} crack.\textsc{pl} in in barn \\
\glt `[…] and [the hay] came into the barn through the door and the cracks.'
\z

\ea\label{}
\langinfo{Tuatschin}{Tschamùt}{\citealt[13]{Büchli1966}}\\
      \gll Lura ṣè 'ls tiers vegni sůt ǝl Tgiern ain […].\\
     then \textsc{cop.prs.3pl} \textsc{def.art.m.pl} animal.\textsc{pl} come.\textsc{ptcp.m.pl} under \textsc{def.art.m.sg} \textsc{pln} in\\
\glt `Then the cattle came  … unter dem Bergkopf Tgiern daher gekommen'
\z


\ea\label{}
\langinfo{Tuatschin}{Bugnai}{\citealt[135]{Büchli1966}}\\
\gll Cu ‘l è vegnius a tgèesa la sèera \textbf{spèr} ina gronda prait-crap \textbf{vi}, ò ‘l schau dâ la segir \textbf{su} la prait-crap \textbf{giů} […].\\ % su als stimmhaft markieren
     when \textsc{3sg.m} \textsc{cop.prs.3sg} come.\textsc{ptcp.m.sg} to house.\textsc{f} \textsc{def.art.f.sg} afternoon next.to \textsc{indef.art.f.sg} big roc.face over have.\textsc{prs.3sg} \textsc{3sg.m} let.\textsc{ptcp} give.\textsc{inf} \textsc{def.art.f.sg} saw under \textsc{def.art.f.sg} rock.face down\\
\glt `When in the evening he came back home, passing a huge roc face, he let the saw fall down unter the rock face […].'
\z

\ea\label{}
\langinfo{Tuatschin}{Bugnai}{\citealt[139]{1966}}\\
\gll Lu ò ‘l viu \textbf{òoragiů} \textbf{dadòor} \textbf{Camischoolas} sen in prau ina familia tge sůlvèeva.\\
     then have.\textsc{prs.3sg} \textsc{3sg} see.\textsc{ptcp} out.down outside \textsc{pln} on \textsc{indef.art.m.sg} field \textsc{indef.art.f.sg} family \textsc{rel} have.breakfast.\textsc{impf.3sg}\\
\glt `Then he saw down there, in Camischolas, a family which was having breakfast on a field.'
\z


Some more examples are \textit{ain téegia} (Büchli 1966: 122), \textit{ain stiziun} (Büchli 1966: 123), \textit{ain nuegl} (B 66: 125), \textit{ain stiva} (B 66: 30), \textit{ain caplůta} (B 66: 45), a\textit{in caplůta dǝ Sontgaclau} (B 66: 45). 

However, if the noun phrase is modified by an adjective, the definite article must occur: \textit{ain la tgèesa veglia} (B 66: 30).


\begin{table}
\caption{Locatives I}
\label{locative1}
 \begin{tabular}{lllllll}
  \lsptoprule
            & & Bugnjaj & Sadrún & Camischùlas & Ruèras & Diani \\
  \midrule
  Bugnai  & →&   & ainta & ainta & aint & ainta\\
  Sadrun & → & ò  &  & ainta & aint(a) & ainta\\
Camischólas & →& ò & ò & & aint & ainta\\
Ruèras & →& ò & ò & ò & & vi\\
Diani & →& ò & ò & ò & ? &\\
  \lspbottomrule
 \end{tabular}
\end{table}

\begin{table}
\caption{locatives v (TarHen)}
\label{}
 \begin{tabular}{lllllll}
  \lsptoprule
     &       & Mumpé Tujètsch & Ségnas & Mustair & Surain/Sumvitg & Trùn\\ 
  \midrule
  Sadrun  & →  & ò &    ò  &    ò     & giù & giù\\
 & ← & & & ainta &   & sé \\
  \lspbottomrule
 \end{tabular}
\end{table}


\begin{table}
\caption{Locatives II}
\label{locative2}
 \begin{tabular}{lllllll}
  \lsptoprule
            & & Sèlva & Tschamùt & Ursèra & Caṣchinùta & Uri\\ 
  \midrule
  Sadrun  & →   & ainta    & sé/sén     &  & \\
  Surain  & →   &    &        & \\
  \lspbottomrule
 \end{tabular}
\end{table}

\begin{table}
\caption{Locatives III (TarHen)}
\label{loc3}
 \begin{tabular}{lllllll}
  \lsptoprule
            & & Surain & Cavòrgia & Val Mèdal & Curaglia & Mustair\\ 
  \midrule
  Sadrun  &    →&  vi  & giùn    & vin & vi & ò \\
  Surain  &   → &   &     &   \\
  \lspbottomrule
 \end{tabular}
\end{table}

\begin{table}
\caption{Locatives IV}
\label{locative4}
 \begin{tabular}{lllllll}
  \lsptoprule
            & & Surain/Sumvitg & Trùn & Gliòn & Cuera & Turitg\\ 
  \midrule
  Sadrùn  &    →&  &     &  \\
  Surain  &   → &   &     &   \\
  \lspbottomrule
 \end{tabular}
\end{table}

\begin{table}
\caption{Locatives V}
\label{locative5}
 \begin{tabular}{lllllll}
  \lsptoprule
            & & Italja & Tiaratudèstga  & Baviara & Austrja & Fròntscha\\ 
  \midrule
  Sadrùn  &    →& giù l' &  ain   &  òn & ò l' & ain\\
  Surain  &   → &   &     &   \\
  \lspbottomrule
 \end{tabular}
 \end{table}

I maunca aun las vals.



The preposition \textit{encùnter} governs dative (in all cases, or only with +human?) But \textit{encunter} ‘towards’ does not always trigger li : \textit{ǝncunt’ǝl tgamin} ‘towards the chimney’ (B 25), \textit{ǝncunter las Vals} ‘towards las Vals’ (B 41).

\ea\label{}
\langinfo{Tuatschin}{Ruèras}{\citealt[64]{Büchli1966}}\\
\gll Co ṣai	vegniu	ina femna \textbf{əncůnter} \textbf{li}	\textbf{quai} \textbf{pur} […].\\
     here \textsc{cop.prs.3sg} come.\textsc{ptcp} \textsc{indef.art.f.sg} woman towards \textsc{def.dat.art.sg} \textsc{dem} peasant \\
\glt `At this moment a woman came towards this peasant […].'
\z

\subsection{Temporal arguments}


\subsection{Modal arguments}

The comparative of bégn `well' can also be formed analytically by pli `more'.

\ea\label{}
\langinfo{Tuatschin}{}{\DRG{1}{296}}\\
\gll  I vo onz \textbf{ple} \textbf{begn}. \\
     \textsc{expl} go.\textsc{prs.3sg} rather more well\\
\glt `I feel rather better.'
\z


The preposition siantar governs dative

\ea\label{}
\langinfo{Tuatschin}{Bugnai}{\citealt[146]{Büchli1966}}\\
\gll   […] tgǝ negin fravi segl ǝntir můn sappi fâ \textbf{sianter} \textbf{li} \textbf{el}. \\
     […]  \textsc{comp} no smith on.\textsc{def.art.m.sg} whole world know.\textsc{prs.sbjv.3sg} make after \textsc{dat} \textsc{3sg}\\
\glt `[…] that no smith in the whole world would be able to make like him.'
\z


\subsection{Quantifiers}

\subsection{Purposive arguments}

\subsection{Further arguments}

Sociative is expressed by \textit{ensiamen cun} `together with'.

\ea\label{}
\langinfo{Tuatschin}{Ruèras}{\DRG{1}{602}}\\
\gll   La regina vegn cupanada; ella sgola e vo a spaz \textbf{ensiamen} \textbf{cun} in gries […]. \\
    \textsc{def.art.f.sg} queen  \textsc{pass.aux.prs.3sg} fertilize.\textsc{ptcp.f.sg} \textsc{3sg} fly.\textsc{prs.3sg} and go.\textsc{prs.3sg} for walk together with \textsc{indef.art.m.sg} drone \\
\glt `The queen is fertilized; she flies away and goes for a trip with a drone […].'
\z


\section{Negation}
The negator of the verb phrase is betg/betga.

Syntax of betg …. plé

\ea\label{}
\langinfo{Tuatschin}{Tschamut} {\citealt[12]{Büchli1966}}\\
\gll   el è \textbf{betga} ius \textbf{plé} cun quai catschadur. \\
      \textsc{3sg} be.\textsc{prs.3sg} \textsc{neg} go.\textsc{ptcp.m.sg} any.more with \textsc{dem.m.sg} hunter\\
\glt `[…] he didn’t go with this hunter any more.'
\z

\ea\label{}
\langinfo{Tuatschin}{Tschamut} {\citealt[15]{Büchli1966}}\\
\gll  schǝ vagi el \textbf{mu} in cazè \textbf{plé}.\\
     then have.\textsc{prs.sbjv.3sg} \textsc{3sg} only one.\textsc{m} shoe any.more \\
\glt `[…] then he would have only one shoe left.'
\z

\ea\label{}
\langinfo{Tuatschin}{Bugnei} {\citealt[147]{Büchli1966}}\\
\gll    ǝls paders savèevan \textbf{betga} spatgèe plé ditg \textbf{plé}\\
      \textsc{def.art.m.pl} Father.\textsc{pl} know.\textsc{impf.3pl} \textsc{neg} wait.\textsc{inf} more long any.more\\
\glt `The fathers couldn’t wait any longer […].'
\z

\ea\label{}
\langinfo{Tuatschin}{Cavorgia} {\citealt[119]{Büchli1966}}\\
\gll    Lura lèevan  i \textbf{betga} schèe fâ paster gron ella \textbf{plè} […].\\
     then want.\textsc{impf.3pl} \textsc{3pl} \textsc{neg} let.\textsc{inf} make.\textsc{inf} shepherd big \textsc{3sg.f} any.more\\
\glt `Then they didn’t want to let her be the main shepherdess […] any more.'
\z

\ea\label{ex:}
\langinfo{Tuatschin}{}{\citealt[69]{Berther2007}}\\
\gll   I tegnan \textbf{bétga} se schi bégn ils praus co \textbf{plé}. \\
     \textsc{3pl} hold.\textsc{prs.3pl} \textsc{neg} up so well \textsc{def.art.m.pl} field.\textsc{pl} here any.more \\
\glt `Here they don’t see to the fields well any more.'
\z

The following example illustrates \textit{mò … plé} `only ... more', which displays the same syntax as \textit{bétga plé}.

\ea\label{}
\langinfo{Medelin}{Curaglia}{\DRG[1]{256}}\\
\gll   Per quei sei vargiau als onns, quel a \textbf{mo} als dis \textbf{ple}.\\
    for \textsc{dem.m.sg} \textsc{cop.prs.3sg} pass.\textsc{ptcp.m.sg} \textsc{def.art.m.pl} day.{pl} \textsc{dem.m.sg} have.\textsc{prs.3sg} only \textsc{def.art.m.pl} day.{pl} any.more \\
\glt `For this [man] the years are over, he has only some more days [left to live].'
\z



There is no double negation after senza ‘without’.

\ea\label{}
\langinfo{Tuatschin}{Bugnei} {\citealt[147]{Büchli1966}}\\
\gll    ǝl giavel ò stiu î senza savai fâ zatgéi.\\
      \textsc{def.art.m.sg} devil have.\textsc{prs.3sg} must.\textsc{ptcp} go.\textsc{inf} without know do.\textsc{inf} something \\
\glt `The devil had to leave without being able to do anything.'
\z






