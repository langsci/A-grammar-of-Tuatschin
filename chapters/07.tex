\chapter{Word formation}

\section{Derivation}
Some derivational morphemes have already been treated: the non-finite verbal categories infinitive (§), past participle (§), and gerund (§ ).


za in zanua, zatgi, zatgei, zacù, zacú

-majn: punctuálmajn (Büchli 147, Bugnaj)


\subsection{Diminutive and augmentative}
The diminutive of nouns is formed with the suffix –\textit{èt/-èta}. The following example shows that the use of the diminutive does not preclude the use of pin `small'.

\ea\label{}
\langinfo{Tuatschin}{Rueras}{\citealt[62]{Büchli1966}}\\
\gll    Lò fùva in pin \textbf{laj-èt} cun pauc’ aua.\\
     there \textsc{exist.impf.3sg} \textsc{indef.art.m.sg} small lake-\textsc{dim} with little water \\
\glt `There was a small lake with little water.'
\z


fjuchèt (2289)

buébèt (Büchli 18, Tschamùt)

The augmentative suffix is \textit{-ún/-una}, as in \textit{ùmún} 'big man' or in \textit{tgèsuna} 'big house'.

\ea
\label{}
\langinfo{Tuatschín}{Sadrún}{m5}\\
\gll  Quaj è \textbf{dònun}. \\
\textsc{dem.unm} \textsc{cop.prs.3sg} pity.\textsc{m.sg.augm}\\
\glt `This is a real pity.'
\z

\ea
\label{}
\langinfo{Tuatschín}{Sadrún}{m4, l. 547f.}\\
\gll Sònda-duméngja vagnévan quèls ò cò a fagjévan \textbf{fjastunas}.\\
Saturday-Sunday come.\textsc{impf.3pl} \textsc{dem.m.pl} out here and do.\textsc{impf.3pl} party.\textsc{augm.f.pl}\\
\glt `During week-end they would come here and have big parties.'
\z

buébúna (2042)

raubuna `grosses Vermögen' (DRG 5: 109)

fomaz Bärenhunger (m5)
Ju a fòmaz. (m5)

{\color{red}
Examples cf. \DRG{1}{345--366} and in \DRG{4}{673--674}
}

\subsection{Adnominal derivational morphemes}
The following adnominal derivational sufffixes occur in the corpus: -ada/-èda, -èssar, 


scargjèda (chapter 9, line 1426)
santupada `meeting' (m5)

purèssar (chapter 9, line 373)

fòrèstalèssar, forest-ry, scòlarèssar, cuminèssar (darar)

scartèzja (2296), balèzja, luschèzja (2362)

gaglinam 

pupira

samudargjém (m4)

mal-: malsagidajval, malcuntjants (Büchli : 146, Bugnaj)

\section{Compounding of nouns}
Compounding can be done by joining two nouns with the preposition \textit{da} 'of', and also by juxtaposition of two nouns. Which strategy is used depends on the compound, and in some cases the two strategies may apply to the same nouns with different meanings.


\ea\label{ex::}
\langinfo{Tuatschín}{Cavòrgja}{m7}\\
\gll  In \textbf{tiar} \textbf{da} \textbf{tgèsa}, è `l gjat a `l tgaun, a \textbf{tiar-tgèsa} è sagir tùt quaj tga vò bigj’ ad alp. Als \textbf{tiars-tgèsa} èn atgnamajn cò, né sén majṣès, quaj è quèlas tgauras a nùrsas a pòrs.  \\
     \textsc{indef.art.m.sg} animal of house.\textsc{f.sg} \textsc{cop.prs.3sg} \textsc{def.art.m.sg} cat and \textsc{def.art.m.sg} dog  and animal.\textsc{m.sg}-house.\textsc{f.sg} \textsc{cop.prs.3sg} for\_sure all \textsc{dem.gl} \textsc{rel} go.\textsc{prs.3sg} \textsc{neg} to alpine\_pasture.\textsc{m.sg} \textsc{def.art.m.pl} animal.\textsc{pl}-house.\textsc{f.sg} \textsc{cop.prs.3pl} actually here or on assembly\_of\_houses.\textsc{m.sg} \textsc{dem.gl} \textsc{cop.prs.3sg} \textsc{dem.f.pl} goat.\textsc{pl} and sheep.\textsc{pl} and pig.\textsc{m.pl}\\
\glt `A "tiar da tgèsa" (a pet), these are cats and dogs, and "tiar tgèsa" are of course all those that do not go to the alpine pastures. The "tiars tgèsa" are actually here, or up in the assembly of houses, these are the goats, the sheep and the pigs.'
\z


\ea\label{ex:}
\langinfo{Tuatschín}{Sadrún}{m4, l.422ff.}\\
\gll[...] lò ani bagagjau gjù la… raquéntani… la crapa par bagagjè al \textbf{clutgè-basèlgja}.\\
{} there have.\textsc{prs.3pl.3pl} build.\textsc{ptcp.unm} down \textsc{def.art.f.sg} tell.\textsc{prs.3pl.3pl} \textsc{def.art.f.sg} stone.\textsc{coll} \textsc{purp} build.\textsc{inf}  \textsc{def.art.m.sg} tower.\textsc{m.sg}-church.\textsc{f.sg}  \\
\glt `[...] there they have removed, as they tell, the stones used to build the church tower [of Sedrun].'
\z

fil-sajda (872)

mòni-scúa (259)

baun-pégna (679)

lungatg-mùma (869)

pòrta-clavau ()

tgauvitg (1337)

patrún-basèlgja (1595)

pròcèssjún-basèlgja (1597)

ésch-stiva Büchli 1966: 30 (Selva)

crusch-fiar Büchli 1966: 134 (Bugnaj)

prajt-crap Büchli 1966: 135 (Bugnaj)

pòrta-basèlgja (Büchli 147 Bugnaj)

tètg-tégja (Büchli 122)

carschèn-matg (Büchli 16, Tschamùt) `waxing moon'

Nadal-nòtg (Büchli 19, Tschamùt)

