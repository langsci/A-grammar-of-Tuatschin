\chapter{Word formation}


\section{Diminutive and augmentative}
The diminutive of nouns is formed with the suffix –\textit{èt/-èta}. The following example shows that the use of the diminutive does not preclude the use of pin `small'.

\ea\label{ex:1:}
\langinfo{Tuatschin}{Rueras}{\citealt[62]{Büchli1966}}\\
\gll    Lò fůva in pign lai-et cun pauc’ awa.\\
     there \textsc{exist.impf.3sg} \textsc{indef.art.m.sg} small lake-\textsc{dim} with little water \\
\glt `There was a small lake with little water.'
\z

The augmentative suffix is \textit{-um/-uma}, as in \textit{ùmun} `huge man' or \textit{tgèsuna} `a big house'.

{\color{red}
Examples are given in \DRG{1}{345--366} and in \DRG{4}{673--674}
}