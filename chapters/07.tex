\chapter{Word formation}

\section{Reduplication}
In Tuatschin, reduplication only has an intensification function. Syntactic categories that may be reduplicated are attributive (\ref{ex:redadjattr1}) and predicative (\ref{ex:redadjpred1} and \ref{ex:redadjpred2}),  adjectives, adjectives used adverbially (\ref{ex:redadjpred3}) as well as adverbs modifying adjectives (\ref{ex:redadv1}) or used predicatively (\ref{ex:redadv2} and \ref{ex:redadv3}), or functioning as a discourse marker (\ref{ex:redadv4}).

\ea\label{ex:redadjattr1}
\langinfo{Tuatschín}{Sadrún}{m5, 1. 1199.}\\
\gll  Ála \textbf{véglja-véglja} tgèsa-parvènda, né?  \\
in.\textsc{def.art.f.sg} \textsc{red}\textasciitilde{old} presbytery right\\
\glt `At the very old presbytery, right?'
\z

\ea\label{ex:redadjpred1}
\langinfo{Tuatschin}{Surrain}{\citealt[128]{Büchli1966}}\\
\gll  […] i èra \textbf{sgtir-stgir} !\\
[…]  \textsc{expl} \textsc{cop.impf.3sg} \textsc{red}\textasciitilde{dark}\\
\glt `[…] it was pitch-dark.'
\z

\ea\label{ex:redadjpred2}
\langinfo{Tuatschín}{Ruèras}{m1, 1. 233ff.}\\
\gll    A quaj stèvnṣ èssar… \textbf{pulits-pulits} l’ jamna… tg’ al bap dètschi in frang a miaz.\\
and \textsc{dem.unm} must.\textsc{impf.1pl.1pl} \textsc{cop.inf} \textsc{red}\textasciitilde{well\_behaved}.\textsc{m.pl} \textsc{def.art.f.sg} week  \textsc{comp} \textsc{def.art.m.sg} father  give.\textsc{prs.sbjv.3sg} one.\textsc{m.sg} franc and half.\textsc{m.sg}\\
\glt `And we had to be … very well-behaved during the week … so that my father would give [us] one and a half francs.'
\z

\ea\label{ex:redadjpred3}
\langinfo{Tuatschín}{Sadrún}{m4, 1. 618ff.}\\
\gll  Al tat èr’ ajn a durméva lò grad sc’ in tajṣ, vèv’ udju \textbf{ṣchùbar-ṣchùbar} nuét.  \\
\textsc{def.art.m.sg} grandfather \textsc{cop.impf.3sg} up and sleep.\textsc{impf.3sg} there precisely like \textsc{indef.art.m.sg} badger have.\textsc{impf.3sg} hear.\textsc{ptcp.unm} \textsc{red}\textasciitilde{clean}.\textsc{adj.unm} nothing\\
\glt `My grandfather was up there and was sleeping like a log, he hadn’t heard anything at all.'
\z

\ea\label{ex:redadv1}
\langinfo{Tuatschín}{Sadrún}{m4, 1. 363f.}\\
\gll  Èl èr’ in tüp tga raṣdava bigja bjè, ju a gju \textbf{fétg-fétg} bian cun èl [...]. \\
\textsc{3sg} \textsc{cop.impf.3sg} \textsc{indef.art.m.sg} fellow \textsc{rel} speak.\textsc{impf.3sg} \textsc{neg} much \textsc{1sg} have.\textsc{prs.1sg} have.\textsc{ptcp.unm} \textsc{red}\textasciitilde{very} good.\textsc{unm} with \textsc{3sg.m}\\
\glt `He was a person who didn’t speak much, I went along very well with him [...].'
\z

\ea\label{ex:redadv2}
\langinfo{Tuatschin}{Tschamùt}{\citealt[18]{Büchli1966}}\\
\gll El ò mirau \textbf{antùrn-antùrn} […].\\
\textsc{3sg} have.\textsc{prs.3sg} look.\textsc{ptcp.unm} \textsc{red}\textasciitilde{around} \\
\glt `He looked around and around.'
\z

\ea\label{ex:redadv3}
\langinfo{Tuatschín}{Ruèras}{m1, 1. 299}\\
\gll    A sjantar surprju acòrds \textbf{adin-adina}.\\
and after take\_over.\textsc{ptcp.unm} piecework.\textsc{m.pl} \textsc{red}\textasciitilde{always} \\
\glt `And afterwards I took over piecework, always.'
\z

\ea\label{ex:redadv4}
\langinfo{Tuatschín}{Sèlva}{f, 1. 937}\\
\gll Quaj èra in’ jèda... brutal tiar nus, \textbf{bèn-bèn}.   \\
\textsc{dem.unm} \textsc{cop.impf.3sg} one.\textsc{f.sg} time terrible.\textsc{adj.unm} among \textsc{1pl} \textsc{red}\textasciitilde{really}\\
\glt `Once it was terrible among us, really.'
\z

\section{Compounding of nouns}
Compounding can be done by joining two nouns with the preposition \textit{da} 'of', and also by juxtaposition of two nouns (\ref{ex:clutgebaselgja1}), whereby the second noun modifies the first one. Which strategy is used depends on the compound, and in some cases the two strategies may apply to the same nouns with different meanings. This last point is best exemplified by (\ref{ex:tiartgèsa1}).


\ea\label{ex:tiartgèsa1}
\langinfo{Tuatschín}{Cavòrgja}{m7}\\
\gll  In \textbf{tiar} \textbf{da} \textbf{tgèsa}, è `l gjat a `l tgaun, a \textbf{tiar-tgèsa} è sagir tùt quaj tga vò bigj’ ad alp. Als \textbf{tiars-tgèsa} èn atgnamajn cò, né sén majṣès, quaj è quèlas tgauras a nùrsas a pòrs.  \\
\textsc{indef.art.m.sg} animal of house.\textsc{f.sg} \textsc{cop.prs.3sg} \textsc{def.art.m.sg} cat and \textsc{def.art.m.sg} dog  and animal.\textsc{m.sg}-house.\textsc{f.sg} \textsc{cop.prs.3sg} for\_sure all \textsc{dem.gl} \textsc{rel} go.\textsc{prs.3sg} \textsc{neg} to alpine\_pasture.\textsc{m.sg} \textsc{def.art.m.pl} animal.\textsc{pl}-house.\textsc{f.sg} \textsc{cop.prs.3pl} actually here or on assembly\_of\_houses.\textsc{m.sg} \textsc{dem.gl} \textsc{cop.prs.3sg} \textsc{dem.f.pl} goat.\textsc{pl} and sheep.\textsc{pl} and pig.\textsc{m.pl}\\
\glt `A "tiar da tgèsa" (a pet), these are cats and dogs, and "tiar tgèsa" are of course all those that do not go to the alpine pastures. The "tiars tgèsa" are actually here, or up in the assembly of houses, these are the goats, the sheep and the pigs.'
\z


\ea
\label{ex:clutgebaselgja1}
\langinfo{Tuatschín}{Sadrún}{m4, l.422ff.}\\
\gll[...] lò ani bagagjau gjù la… raquéntani… la crapa par bagagjè al \textbf{clutgè-basèlgja}.\\
{} there have.\textsc{prs.3pl.3pl} build.\textsc{ptcp.unm} down \textsc{def.art.f.sg} tell.\textsc{prs.3pl.3pl} \textsc{def.art.f.sg} stone.\textsc{coll} \textsc{purp} build.\textsc{inf}  \textsc{def.art.m.sg} tower.\textsc{m.sg}-church.\textsc{f.sg}  \\
\glt `[...] there they have removed, as they tell, the stones used to build the church tower [of Sedrun].'
\z

The compounding of nouns by juxtaposition is relatively frequent; further examples are \textit{baun-pégna} `oven bench', \textit{carschèn-matg} `waxing moon of May' \textit{crusch-fiar} `iron cross', \textit{ésch-stiva} `door of the living-room', \textit{fil-sajda} `silk thread', \textit{lungatg-mùma} `mother-tongue', \textit{mòni-scúa} `broomstick', \textit{patrún-basèlgja} `Church Patron', \textit{pòrta-basèlgja} `church door', \textit{pòrta-clavau} `barn door', \textit{prajt-crap} `rock face', \textit{pròcèsjún-basèlgja} `religious procession', \textit{tètg-tégja} `roof of the alpine hut', and \textit{tgau-vitg} `head of the village'. \textit{Nadal-nòtg} `Christmas Eve' has a different syntax: here, it is the first noun that modifies the second one, probably under influence of German \textit{Weihnachtsnacht}.

\section{Derivation}
Some derivational morphemes have already been treated: the non-finite verbal categories past participle (§ 4.1.2.1.1), gerund (§ 4.1.2.1.2), infinitive (§ 4.1.2.1.3), the adverbialiser -\textit{majn} (§ 4.2.2.3), and the causative -\textit{antá} (§ 5.5.3).


\subsection{Diminutive and augmentative}
The diminutive of nouns is formed with the suffix –\textit{èt/-èta}. The following example shows that the use of the diminutive does not preclude the use of \textit{pin} `small'.

\ea\label{}
\langinfo{Tuatschin}{Rueras}{\citealt[62]{Büchli1966}}\\
\gll    Lò fùva in pin \textbf{laj-èt} cun pauc’ aua.\\
     there \textsc{exist.impf.3sg} \textsc{indef.art.m.sg} small lake-\textsc{dim} with little water \\
\glt `There was a small lake with little water.'
\z

Further examples are \textit{fjuchèt} `little fire' and \textit{buébèt} `little boy'.

The augmentative suffix is \textit{-ún/-una}.

\ea
\label{}
\langinfo{Tuatschín}{Sadrún}{m5}\\
\gll  Quaj è \textbf{dònun}. \\
\textsc{dem.unm} \textsc{cop.prs.3sg} pity.\textsc{m.sg.augm}\\
\glt `This is a real pity.'
\z

\ea
\label{}
\langinfo{Tuatschín}{Sadrún}{m4, l. 547f.}\\
\gll Sònda-duméngja vagnévan quèls ò cò a fagjévan \textbf{fjastunas}.\\
Saturday-Sunday come.\textsc{impf.3pl} \textsc{dem.m.pl} out here and do.\textsc{impf.3pl} party.\textsc{augm.f.pl}\\
\glt `During week-end they would come here and have big parties.'
\z

Further examples are \textit{buébúna} `very tall girl', \textit{raubuna} `big assets', \textit{ùmún} 'big man', and \textit{tgèsuna} 'big house'.

Another augmentative is \textit{-az}, as in \textit{fòmaz} (> \textit{fòm} `hunger').

\ea
\label{}
\langinfo{Tuatschín}{Sadrún}{m5}\\
\gll Ju a \textbf{fòm-az}.\\
\textsc{1sg} have.\textsc{prs.1sg} hunger-\textsc{augm}\\
\glt `I am hungry as wolves.'
\z


\subsection{Adnominal derivational morphemes}
The derivational suffix which shows the most items in the corpus is -\textit{zjun}, which derives nouns from verbs, like \textit{complicazjun} `complication' < \textit{cumplicar} `complicate'. In some cases -\textit{zjun} derives a noun from a verb that is not used or has another meaning in Tuatschin. An example is \textit{vòtazjun} `votation', which is derived from standard Sursilvan \textit{votar}, but the verb that is used for `vote' is \textit{vuṣchá}, in Tuatschin as well as in normal Sursilvan speech. Another example is \textit{tradizjun} `tradition', which is derived from \textit{tradí}, but \textit{tradí} means `betray' and not `transmit'. This means that some of the nouns that are derived by -\textit{zjun} are learned words.

Some more examples of nouns derived by -\textit{zjun} are \textit{afèczjun} `affection', \textit{confadarazjun} `confederation', \textit{dirèczjun} `direction', \textit{fòrmazjun} `formation', \textit{habitazjun} `appartment', \textit{munizjun} `munition', \textit{òbligazjun} `obligation', \textit{réaczjun} `reaction', and \textit{tussègazjun} `poisoning'.

Another suffix that derives nouns from verb is the feminine ending of the past participle \textit{-ada/-èda}: \textit{cuṣchinada} `mixture of food' (< \textit{cuṣchiná} `cook'), \textit{santupada} `meeting' (< \textit{santupá} `meet'), \textit{satagljèda} `cut (to oneself)' (< \textit{satagljá} `cut oneself'), and \textit{scargjèda} `drowing down the animal from the alps'.

-\textit{ém} also derives nouns from verbs and insists on the repetition of the action, as \textit{samudargjém} `constant torturing oneself' (< \textit{samudargjè} `torture oneself').

Note that in santupada, satagjlèda, and samudargjém the reflexive prefix sa- is maintained, which shows that the reflexive verb forms a strong unit.

The suffixes -\textit{dad}, -\textit{èzja}, and -\textit{ira} derive nouns from adjectives. Examples are \textit{paupradad} `poverty' (< \textit{paupra} `poor (f.)), \textit{pussajvladad} `possibility' (< \textit{pussajvla} `possible' (f.), \textit{balèzja} `beauty' (< \textit{bials} `beautiful' (m.)), \textit{luṣchèzja} `proudness' (\textit{lùschs} `proud' (m.)), \textit{scartèzja} `lack' (\textit{scarts} `rare' (m.)), \textit{pupira} `poverty' (< \textit{paupars} `poor' (m.)), and \textit{tupira} `stupidity' (< \textit{tups} `stupid' (m.)).

The suffixes -\textit{am} and -\textit{èssar} (corresponding to the copula) derive nouns from noun. Examples are \textit{gaglinam} `flock of chickens' (< \textit{gagljina} `hen'), \textit{purèssar} `farming sector' (< \textit{pur} `farmer'), \textit{fòrèstalèssar} `forestry'(< \textit{fòrèstal} `forest ranger'), and \textit{scòlarèssar} `school sector' (< \textit{scùla} `school'). 

Prefixes are \textit{mal}- which usually modify adjectives: \textit{malcuntjants} `unsatisfied', \textit{malsagidajvals} `ungainly', and \textit{malsagirs} `unsure', as well as \textit{za}- (< \textit{ins sa} `one knows') which corresponds to English `any' or `some' with indefinite or other prounouns: \textit{zacù} `somehow', \textit{zacú} `somewhen, \textit{zanúa} `somewhere', \textit{zatgéj} `anything', and \textit{zatgi} `anybody'.



