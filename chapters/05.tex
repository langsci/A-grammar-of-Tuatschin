\chapter{Simple sentences}

\section{Declarative sentences}


\subsection{Order of the arguments}
Tuatschin has SVO word order (\ref{ex:sov1}), but if an element occurs before the subject, the subject is moved after the finite verb (\ref{ex:sov2}) and (\ref{ex:sov4}), which means that Tuatschin has subject inversion or, in other words, verb-second syntax.

\ea
\label{ex:sov1}
\langinfo{Tuatschín}{Sadrún}{m6, 1. 1304f.}\\
	\gll    [...] álṣò \texttt{[}nuṣ\texttt{]} \texttt{[}vajn gju\texttt{]} \texttt{[}ina pintga pjaglja\texttt{]} [...].\\
 {} well \textsc{1pl} have.\textsc{prs.1pl}  have.\textsc{ptcp.unm} \textsc{indef.art.f.sg} small salary \\
\glt `[...] well, we got a small salary [...].'
\z

\ea
\label{ex:sov2}
\langinfo{Tuatschín}{Sadrún}{m8, 1. 1434}\\
\gll \texttt{[}\textbf{Avaun} \textbf{in} \textbf{pèr} \textbf{jamnas}\texttt{]} … èr’ \texttt{[}\textbf{ju}\texttt{]} gjù Locarno [...].\\
before \textsc{indef.art.m.sg} pair week.\textsc{f.pl} {} \textsc{cop.impf.1sg} \textsc{1sg} down \textsc{pln}\\
\glt `Some weeks ago … I was in Locarno [...].'
\z

\ea
\label{ex:sov4}
\langinfo{Tuatschín}{Ruèras}{m10, 1. 988f.}\\
\gll  Api ah … va ju cumprau glj amprém mù in, a \texttt{[}\textbf{sjantar}\texttt{]} va \texttt{[}{\textbf{ju}}\texttt{]} … cumprau in zacún.\\
and eh {} have.\textsc{prs.1sg} \textsc{1sg} buy.\textsc{ptcp.unm} \textsc{def.art.m.sg} first only one.\textsc{m.sg} and after  have.\textsc{prs.1sg} \textsc{1sg} {} buy.\textsc{ptcp.unm}  \textsc{indef.art.m.sg} second\\
\glt `And ah ... at first I bought only one, but afterwards I … bought a second [one].'
\z


As shown in § 4.2.2 above, the indirect object usually precedes the direct object (\ref{ex:indir1}).

\ea
\label{ex:indir1}
\langinfo{Tuatschín}{Sadrún}{m6, 1. 1428f.}\\
\gll    “Quèl vès lu aun da pajè \texttt{[}da té\texttt{]} \texttt{[}al pustrètsch\texttt{]} dal piertg tga té vèvas partgirau.” \\
\textsc{dem.m.sg} have.\textsc{cond.3sg} then still \textsc{comp} pay.\textsc{inf} \textsc{dat} \textsc{2sg} \textsc{def.art.m.sg} money of.\textsc{def.art.m.sg} pig \textsc{rel} \textsc{2sg} have.\textsc{impf.2sg} look\_after.\textsc{ptcp.unm}\\
\glt `This one should still pay you the money of the pig you had looked after.'
\z

The coordinating conjunctions \textit{a} `and' (\ref{ex:aapi}), \textit{ábar} `but' (\ref{ex:sov3}), and \textit{dantaun} (\ref{ex:sov5}) `however' do not trigger subject inversion. In contrast, \textit{api} and its shorter form \textit{pi} `and, and then' do trigger subject inversion (\ref{ex:aapi}) and (\ref{ex:aapi1}).

\ea
\label{ex:sov3}
\langinfo{Tuatschín}{Sadrún}{f3, 1. 115ff.}\\
\gll Api vau détg, èls vagjan fatg ina tura tschèl’ jamna, \textbf{ábar} \textbf{èl} vaj quitau tga quaj sèj… .\\  
and\_then have.\textsc{prs.1sg} say.\textsc{ptcp.unm} \textsc{3pl.m} have.\textsc{prs.sbjv.3pl} make.\textsc{ptcp.unm} \textsc{indef.art.f.sg} tour  \textsc{dem.f.sg} week but \textsc{3sg.m} have.\textsc{prs.sbjv.3sg} worry.\textsc{m.sg} \textsc{comp} \textsc{dem.unm} \textsc{cop.prs.sbjv.3sg} \\
\glt `And then I said that they had made a tour that week, but that he sees to it that this be … .'
\z

\ea\label{ex:sov5}
\langinfo{Tuatschín}{} {\citealt[18]{Berther1998}}\\
\gll    Ju vagnès schon, \textbf{dantaun} \textbf{vus} \textbf{vajs} taun tschuf ajn tgèsa.\\
\textsc{1sg} come.\textsc{cond.1sg} really however \textsc{2pl} have.\textsc{prs.2pl} so.much dirt in house\\
\glt `Actually I would come, but you have so much dirt in your house.'
\z

\ea
\label{ex:aapi}
\langinfo{Tuatschín}{Sadrún}{f5, 1. 1256ff.}\\
\gll    Als gjaniturs fagèvan al pur, \textbf{ad} \textbf{ju} \textbf{èra} tschavrèr’ in tjams, ajn l’ antschata cun miu frá, in òn parsula, \textbf{api} \textbf{stavèv’} \textbf{ins} í culas tgauras tòca sé Nalps [...]. \\
\textsc{def.art.m.pl} parent.\textsc{pl} do.\textsc{impf.3pl} \textsc{def.art.m.sg} farmer and \textsc{1sg} \textsc{cop.impf.1sg} goatherd.\textsc{f.sg} \textsc{indef.art.m.sg} time in \textsc{def.art.f.sg} beginning with \textsc{poss.1sg.m.sg} brother one.\textsc{m.sg} year alone.\textsc{f.sg}  and  must.\textsc{impf} \textsc{gnr} go.\textsc{inf} with.\textsc{def.art.f.pl} goat.\textsc{pl} until up  \textsc{pln} \\
\glt `My parents were farmers, and I was a goatherd for a certain time, at the beginning with my brother, one year alone, and one had to go with the goats till Nalps [...].'
\z


\ea
\label{ex:aapi1}
\langinfo{Tuatschín}{Sadrún}{f8, l. 1458}\\
	\gll   \textbf{Pi} ṣè’ \textbf{l} vajramájn staus lò. \\
	then be.\textsc{prs.3sg} \textsc{3sg.m} true.\textsc{f.sg.adv} remain.\textsc{ptcp.m.sg} there\\
\glt `And the swan really remained there.'
\z



After subordinating conjunctions there is no subject inversion.

\ea
\label{}
\langinfo{Tuatschin}{Bugnaj} {\citealt[132]{Büchli1966}}\\
\gll    La buéba ò detg, \textbf{tg'} \textbf{èla} \textbf{ségi} bétga \textbf{ida} vòluntáriamajn, èla stuévi ira.\\
\textsc{def.art.f.sg} girl have.\textsc{prs.3sg} say.\textsc{ptcp.unm} \textsc{comp} \textsc{3sg} be.\textsc{prs.sbjv.3sg} \textsc{neg} go.\textsc{ptcp.f.sg} voluntary.\textsc{adv} \textsc{3sg.f} must.\textsc{impf.sbjv.3sg} go.\textsc{inf}\\
\glt `The girl said that she didn't go voluntarily, [but that] she had to go.'
\z

\ea
\label{}
\langinfo{Tuatschín}{Sadrún}{f3, 1.18ff.}\\
\gll  [...] api lura va ju in’ jèda talafònau dad èl, \textbf{prquaj} \textbf{tg'} \textbf{èl} \textbf{vèva} \textbf{tarmèz} in’ anunzja da mòrt [...]\\
{} and then have.\textsc{1sg}  \textsc{1sg} one.\textsc{f.sg} time call.\textsc{ptcp.unm} \textsc{dat} \textsc{3sg.m} because \textsc{comp} \textsc{3sg.m} have.\textsc{impf.3sg} send.\textsc{ptcp.unm} \textsc{indef.art.f.sg} announcement of death.\textsc{f.sg} [...]\\ 
\glt `[...] and then I phoned him once, because I should send a death notice [...].'
\z

\ea
\label{}
\langinfo{Tuatschín}{Sadrún}{m4, 1. 342f.}\\
\gll  [...] \textbf{avaun} \textbf{c’} \textbf{ju} \textbf{sùn} \textbf{staus} tial tat savévu da quaj nuét [...].\\
{} before \textsc{comp} \textsc{1sg} be.\textsc{prs.1sg} \textsc{cop.ptcp.m.sg} at.\textsc{def.art.m.sg} grandfather know.\textsc{impf.1sg.1sg} of \textsc{dem.unm} nothing\\
\glt `[...] before I stayed with my grandfather I didn’t know anything [...].'
\z

The rules concerning subject inversion are sometimes not taken into account. This is the case if the speaker wants to highlight a syntactic category.

\ea
\label{}
\langinfo{Tuatschín}{Ruèras}{m10, 1. 1065ff.}\\
\gll   A lur scha ’l vèṣ ussa pagljau par èx\underline{è}mpal in grép tga vès pudju bétar èl, scha \textbf{quèls} fùssan grad schulaj gjù ajl’ awa, ajl lac.\\
and then if \textsc{3sg.m} have.\textsc{cond.3sg} now hit.\textsc{ptcp.unm} for example.\textsc{m.sg} \textsc{indef.art.m.sg} rock \textsc{rel} have.\textsc{cond.3sg} can.\textsc{ptcp.unm} throw.\textsc{inf} \textsc{3sg.m} \textsc{corr} \textsc{dem.m.pl} be.\textsc{cond.3pl} immediately fall\_rapidly.\textsc{ptcp.m.pl} down into.\textsc{def.art.f.sg} water into.\textsc{def.art.m.sg} lake\\
\glt `And then if it [the load] hit a rock which could have thrown it down, the mules would have immediately fallen down rapidly into the water, into the lake.'
\z


The inverted subject is not immediately adjacent to the verb, since some elements may intervene between the verb and the subject.

\ea
\label{}
\langinfo{Tuatschín}{Sadrún}{m6, 1. 1318f.}\\
	\gll    Avaun nus \textbf{èra} \textbf{sagir} \textbf{al} \textbf{tgavrè} era schòn jus culas tgauras, lèz mava lu èra.\\
before \textsc{1pl} be.\textsc{impf.3sg} sure \textsc{def.art.m.sg} goatherd also already go.\textsc{ptcp.m.sg} with.\textsc{def.art.f.pl} goat.\textsc{pl} \textsc{dem.m.sg} go.\textsc{impf.3sg} then also\\
\glt `Before us the goatherd had certainly already gone with the goats, he also used to go.'
\z


Tuatschin is not a pro-drop language, but the omission of the subject may occur in subordinated and coordinated clauses if the subject can be recovered from the preceding context.

\ea
\label{}
\langinfo{Tuatschín}{Zarcúns}{m2, 1. 1566ff.}\\
\gll    Avaun ina fjasta mavan \textbf{aj}… tialas… gjufnas… par nègla, partgé \_ matévan sé sé la capjala… ina nègla.\\
before  \textsc{indef.art.f.sg} celebration go.\textsc{impf.3pl} \textsc{3pl} to.\textsc{def.art.f.pl}  young\_woman.\textsc{pl.} for carnation.\textsc{f.pl} because \textsc{subj}  put.\textsc{impf.3pl} up up  \textsc{def.art.f.sg} hat \textsc{indef.art.f.sg} carnation \\
\glt `Before a celebration they would go … to the … girls for carnations, because they would put … a carnation on their hat.'
\z

\ea
\label{}
\langinfo{Tuatschín}{Sadrún}{m5}\\
 \gll    Al sulèt intarèssánt è l’ ampréma sacùnda classa \textbf{nùca} \textbf{tga} \_ fòn la midada tial sursilvan […].\\
 \textsc{def.art.m.sg} only interesting \textsc{cop.prs.3sg} \textsc{def.art.f.sg} first second form where \textsc{rel} \textsc{subj} make.\textsc{prs.3pl} \textsc{def.art.f.sg} change towards.\textsc{def.art.m.sg} Sursilvan\\
 \glt `The only interesting thing is the first [and] second form where they [= the pupils] start switching towards Sursilvan [...].'
 \z
 

\section{Interrogative sentences}
Polar questions are characterized by rising intonation and subject inversion (\ref{ex:interr1}).

\ea
\label{ex:interr1}
\langinfo{Tuatschín}{Sadrún}{m8, l. 1447}\\
\gll «Gè, sùnd ju ajn tiu taritòri, distùrb’ ju té?»   \\
yes \textsc{cop.prs.1sg} \textsc{1sg} in \textsc{poss.2sg.m.sg} territory disturb.\textsc{prs.1sg} \textsc{1sg} \textsc{2sg} \\
\glt `Yes, am I in your territory, do I disturb you?'
\z

dacù `why' ()

Non-polar questions require the presence of an interrogative word and, like polar questions, exhibit subject inversion. Interrogative pronouns are \textit{cu/cura} `when', \textit{cù} `how', \textit{nòua} `where', \textit{partgé(j)} `why', \textit{tgéj} `what', \textit{tgi} `who'. The interrogative determiner is \textit{tgé(j)} `which, what'.


\ea
\label{}
\langinfo{Tuatschin}{Sadrún} {m5}\\
\gll \textbf{Tgi} ò malagjau quaj malètg?\\
who have.\textsc{prs.3sg} paint.\textsc{ptcp.unm} \textsc{dem.m.sg} picture\\
\glt `Who painted this picture?'
\z


\ea
\label{}
\langinfo{Tuatschin}{Cavòrgja} {\citealt[121]{Büchli1966}}\\
\gll \textbf{Tgéj} lajn fá avaun c' í á raschlá?\\
what want.\textsc{prs.1pl} do.\textsc{inf} before \textsc{rel} go.\textsc{inf} \textsc{comp} rake.\textsc{inf}\\
\glt `What shall we do before going to rake?'
\z

\ea
\label{}
\langinfo{Tuatschin}{Sadrún} {m5}\\
\gll \textbf{Nua} ajs staus?\\
where be.\textsc{prs.2sg} \textsc{cop.ptcp.m.sg}\\
\glt `Where have you been?'
\z

\ea
\label{}
\langinfo{Tuatschin}{Sadrún} {m5}\\
\gll \textbf{Cu} ṣè la vagnida?\\
when be.\textsc{prs.3sg} \textsc{3sg.f} come.\textsc{ptcp.f.sg}\\
\glt `When did she come?'
\z

\ea
\label{}
\langinfo{Tuatschin}{Sadrún} {m5}\\
\gll \textbf{Prtgéj} fas quaj?\\
why do.\textsc{prs.2sg} \textsc{dem.unm}\\
\glt `Why do you do this?'
\z

\ea
\label{}
\langinfo{Tuatschin}{Sadrún} {m5}\\
\gll \textbf{Tgéj} \textbf{tgamiṣcha} as cumprau?\\
which shirt.\textsc{f.sg} have.\textsc{prs.2sg} buy.\textsc{ptcp.unm}\\
\glt `Which shirt did you buy?'
\z



Indirect interrogative clauses will be treated in § 6.2.10 below.


\section{Imperative sentences}

\ea
\label{}
\langinfo{Tuatschín}{Cavòrgja}{m7, l.2164f.}\\
	\gll  «Ah uòn \textbf{nò} \textbf{té} ajnta Pardatsch.»\\
eh this\_year \textsc{go.imp.2sg} \textsc{2sg} into \textsc{pln}\\
\glt `« Ah, this year go to Pardatsch. »
\z


Imperative sentences may or may not lack a subject. In prohibitive sentences, the negator \textit{bétga} precedes the imperative.

\ea
\label{}
\langinfo{Tuatschín}{Sadrún}{me, l. 608f.}\\
\gll «\textbf{Té} \textbf{nò} lu vidòr ússa. Lò, quèsta sèra \textbf{dòrma} lu \textbf{bigja} ajn lò.»\\
\textsc{2sg} come.\textsc{imp.2sg} then down now there  \textsc{dem.f.sg} evening sleep.\textsc{imp.2sg} then \textsc{neg} in there\\
\glt `Come down here now. There, don’t sleep up there this evening.»'
\z

\ea
\label{}
\langinfo{Tuatschín}{Ruèras}{m1, l. 208}\\
\gll    \textbf{Vajas} quitaus cu vuṣ majṣ ṣur la lingja via… dal ṣùc, dal zùc.»\\
have.\textsc{imp.2pl} worry.\textsc{m.pl} when \textsc{2pl} go.\textsc{prs.2pl} over \textsc{def.art.f.sg} line way of.\textsc{def.art.m.sg} train.\textsc{m.sg} of.\textsc{def.art.m.sg} train\\
\glt `Be careful when you cross the railway line, the railway line.'
\z

\ea
\label{}
\langinfo{Tuatschín}{}{\DRG{2}{327}}\\
\gll  \textbf{Sapartgiraj} dals bétlars cu tgi vòn á tgavaj.  \\
     \textsc{refl}.beware.\textsc{imp.2pl} of.\textsc{def.art.m.pl} beggar.\textsc{pl} when \textsc{rel.3pl.comm} go.\textsc{prs.3sg} on horse\\
\glt `Beware of the beggars if they ride.'
\z

\ea
\label{}
\langinfo{Tuatschin}{}{\DRG{2}{503}}\\
\gll  \textbf{Bétg} tumaj!\\
     \textsc{neg} be.afraid.\textsc{imp.2pl}\\
\glt `Don’t be afraid!'
\z

The hortative is formed with the first person plural present of the verb \textit{vulaj} `want', \textit{lajn} `let's', and the infinitive.

\ea
\label{}
\langinfo{Tuatschín}{Sèlva}{f2, l. 942f.}\\
\gll Ad ùss stù ’l bunamajn vagní á mètar èls; quaj è tùt, \textbf{lajn} \textbf{dí}, fantasia [...]. \\
and now must.\textsc{prs.3sg} \textsc{3sg.m} really come.\textsc{inf} \textsc{purp} put.\textsc{inf} \textsc{3pl.m} \textsc{dem.unm} \textsc{cop.prs.3sg} all \textsc{imp.1pl} say.\textsc{inf} fantasy.\textsc{f.sg}\\
\glt `And now he must really come and put them [in the right place]; this is all, let’s say, fantasy [...].'
\z

\ea
\label{}
\langinfo{Tuatschín}{Cavòrgja}{f1}\\
\gll \textbf{Lajn} \textbf{còj} quèla tgarn.\\
\textsc{imp.1pl} cook.\textsc{inf} \textsc{dem.f.sg} meat\\
\glt `Let's cook this meat.'
\z

Optative:

\ea
\label{}
\langinfo{Tuatschín}{}{\DRG{5}{649}}\\
\gll Djus \textbf{banadèschi} a \textbf{carschjanti}!\\
god bless.\textsc{prs.sbjv.3sg} and thrive.\textsc{prs.sbjv.3sg}\\
\glt `May God bless [it] and make [it] thrive!'
\z


\section{Exclamative sentences}

\ea\label{}
\langinfo{Tuatschin}{}{\DRG{2}{215}}\\
\gll Jeusas, \textbf{co} quai brischa!   \\
   \textsc{excl} how \textsc{dem} burn.\textsc{prs.3sg}  \\
\glt `Jee, how it is burning [of a fire]!'
\z

\ea
\label{}
\langinfo{Tuatschín}{Sadrún}{f3; l. }\\
\gll Miu frá ṣchèva aun ér: «\textbf{Tgé} té pùs!»   \\
\textsc{poss.1sg.m.sg} brother say.\textsc{impf.3sg} still yesterday what \textsc{2sg} can.\textsc{prs.2sg} \\
\glt `My brother said not later than yesterday: «[Incredible that you still] have the strength [to do that]!»'
\z

\section{Voice}

\subsection{Reflexive}

In Tuatschin, reflexive voice is formed with the prefix \textit{sa-}, in all persons, tenses, and moods. The Sursilvan norm claims that the auxiliary verb be \textit{èssar} `be', and in the corpus it is mostly so (\ref{ex:refessar2} and \ref{ex:refessar3}); however, \textit{vaj} `have' is not rare (\ref{ex:refvaj1} - \ref{ex:refvaj3}).

\ea\label{ex:refessar1}
\langinfo{Tuatschin}{Sadrún}{m10; l. 1071f.}\\
\gll [...] ju [...] \textbf{sa-spruava} dad èssar ruassajvals [...].\\
{} \textsc{1sg} {}  \textsc{refl-}try.hard.\textsc{impf.1sg} \textsc{comp} \textsc{cop.inf} calm.\textsc{m.sg}\\
\glt `[...] I [...] tried hard to remain calm [...].'
\z

\ea\label{ex:refessar2}
\langinfo{Tuatschin}{Sadrún}{m4; l. 322ff.}\\
\gll [...] a lu \textbf{sùnd} ju \textbf{sa-dacidjus} da… raṣdá in pau ṣur da la... da mi’ ufaunza [...].   \\
{} and then  be.\textsc{prs.1sg}  \textsc{1sg}  \textsc{refl}-decide.\textsc{ptcp.m.sg}  \textsc{comp} talk.\textsc{inf} \textsc{indef.art.m.sg} little over of  \textsc{def.art.f.sg} of \textsc{poss.1sg.f.sg} childhood\\
\glt `[...] and then I have decided to ... talk a bit about ... my childhood [...].'
\z

\ea\label{ex:refessar3}
\langinfo{Tuatschin}{Sadrún}{m4; l. 357ff.}\\
\gll  Al tat, scù la mùma ò raquintau, \textbf{è}  \textbf{sa-prjuṣ} \textbf{ajn} quaj schi starmantús tg’ èl è curdauṣ gjùdajn ajn ina rùsna nundétga, ad ò lu stu í a fá cura, mass’ òns, a... sjantar lu  \textbf{sa-ravagnús} ábar maj pròpi stauṣ ajn gamba.  \\
\textsc{def.art.m.sg} grandfather as  \textsc{def.art.f.sg} mother have.\textsc{prs.3sg} tell.\textsc{ptcp.unm} be.\textsc{prs.3sg} \textsc{refl-}take.\textsc{ptcp.m.sg} in \textsc{dem.unm} so terrible.\textsc{adj.unm} \textsc{comp} \textsc{3sg} be.\textsc{prs.3sg} fall.\textsc{ptcp.m.sg} down\_into in \textsc{indef.art.f.sg} hole awful and have.\textsc{prs.3sg} then must.\textsc{ptcp.unm} go.\textsc{inf} \textsc{comp} make.\textsc{inf} treatment.\textsc{f.sg} many year.\textsc{m.pl} and after then \textsc{refl}-come\_again.\textsc{ptcp.m.sg} but never really \textsc{cop.ptcp.m.sg} in leg.\textsc{f.sg}\\
\glt `My grandfather, as my mother told, took this so seriously that he fell in an awful hole, and during many years he had to go to a health resort, and ... after that he recovered, but he never was really well.'
\z

\ea\label{ex:refvaj1}
\langinfo{Tuatschin}{Sadrún}{m4; l. 343ff.}\\
\gll  [...] avaun c’ ju sùn staus tial tat savévu da quaj nuét a \textbf{vèṣ} è bitga \textbf{sa-fatg} \textbf{ajn} zatgé spacjal.\\
{} before \textsc{comp} \textsc{1sg} be.\textsc{prs.1sg} \textsc{cop.ptcp.m.sg} at.\textsc{def.art.m.sg} grandfather know.\textsc{impf.1sg.1sg} of \textsc{dem.unm} nothing and have.\textsc{cond.1sg} also \textsc{neg} \textsc{refl-}do.\textsc{ptcp.unm} in something special.\textsc{m.sg}\\
\glt `[...] before I stayed with my grandfather I didn’t know anything and I wouldn’t have noticed anything either.'
\z

\ea\label{ex:refvaj2}
\langinfo{Tuatschin}{Sadrún}{m4; l. 631ff.}\\
	\gll Ju a maj gju pròblèm– èl \textbf{vès} maj \textbf{sa-vilau} cun mè né anzatgéj, ju a adina gju fétg ugèn al tat.   \\
	\textsc{1sg} have.\textsc{prs.1sg} never have.\textsc{ptcp.unm} problem.\textsc{m.sg} \textsc{3sg.m} have.\textsc{cond.3sg} never \textsc{refl-}get\_angry.\textsc{ptcp.unm} with \textsc{1sg} or something \textsc{1sg} have.\textsc{prs.1sg} always have.\textsc{ptcp.unm} very with\_pleasure \textsc{def.art.m.sg} grandfather\\
\glt `I have never had a problem – he would never have got angry at me or something like that, I have always been very fond of my grandfather.'
\z

\ea\label{ex:refvaj3}
\langinfo{Tujetsch}{}{\DRG{6}{321}}\\
\gll   Als fildiròms \textbf{òn} \textbf{sa-pagljau} ajnt.\\
    \textsc{def.art.m.pl} wire.\textsc{pl} have.\textsc{prs.3pl} \textsc{refl}-touch.\textsc{ptcp.unm} in \\
\glt `The wires touched each other.'
\z


\subsection{Reciprocal}

\ea\label{}
\langinfo{Tuatschín}{Camischùlas}{f6, l. 782f.}\\
\gll    Ad èlas duaṣ ábar ancònuschévan… \textbf{in}’ \textbf{l}’ \textbf{autra} ad ju lu halt bégja.\\
and \textsc{3pl.f} two.\textsc{f.pl} but know.\textsc{impf.3pl} one.\textsc{f.sg}  \textsc{def.art.f.sg} other and \textsc{1sg} then in\_fact \textsc{neg}\\
\glt `But these two already knew … each other but I didn’t.'
\z


\subsection{Causative}
Causative voice is formed with \textit{fá} ‘make’ and \textit{schè/schá} `have something done, let' followed by an infinitive. The causee is  adjacent to the second verb and its complements.

\ea\label{}
\langinfo{Tuatschín}{Sadrún}{m4, l.653f.}\\
\gll   [...] plaunsjú ṣèni vagní da \textbf{fá} \textbf{í} scha vèva ’l rùt in calum.\\
{} slowly be.\textsc{impf.3pl} come.\textsc{ptcp.m.pl} \textsc{comp} make.\textsc{inf} go.\textsc{inf} since have.\textsc{impf.3sg} \textsc{3sg.m} break.\textsc{ptcp.unm} \textsc{indef.art.m.sg} thigh \\
\glt `[...] they succeeded slowly in having him going [to the hospital] since he had broken a thigh.'
\z


\ea
\label{}
\langinfo{Tuatschín}{Sadrún}{m5}\\
\gll    Èl ò schau savaj \textbf{la} \textbf{règina} quaj.\\
     \textsc{3sg} have.\textsc{prs.3sg} let.\textsc{ptcp.unm} know.\textsc{inf} \textsc{def.art.f.sg} queen \textsc{dem}\\
\glt `He let the queen know this.'
\z


\ea
\label{}
\langinfo{Tuatschín}{Sadrún}{m4}\\
\gll Ju \textbf{fétsch} \textbf{fá} al cusunz in pèr tgautschas par èl.\\
\textsc{1sg} make.\textsc{prs.1sg} do.\textsc{inf} \textsc{def.art.m.sg} tailor \textsc{indef.art.m.sg} pair trousers.\textsc{m.pl} for \textsc{3sg.m}\\
\glt `I have the tailor make a pair of trousers for him.'
\z


\ea
\label{}
\langinfo{Tuatschín}{Ruèras}{f4, 1. 2001}\\
	\gll I \textbf{schèvan} \textbf{luvrá} fétg.\\
{}	\textsc{3pl} let.\textsc{impf.3pl} work.\textsc{inf} much\\
\glt `They had us work a lot.'
\z

\ea\label{}
\langinfo{Tuatschín}{Ruèras} {\citealt[62]{Büchli1966}}\\
\gll    […] i vèvan fatg vagní \textbf{al} \textbf{caplòn} \textbf{da} \textbf{Sèlva} par banadí la nibla […].\\
{}   \textsc{3pl} have.\textsc{impf.3pl} make.\textsc{ptcp.unm} come.\textsc{inf} \textsc{def.art.m.sg} chaplain of \textsc{pln} \textsc{purp} bless.\textsc{inf} \textsc{def.art.f.sg} cloud\\
\glt `They’d had the chaplain of Selva come in order to bless the cloud […].'
\z

\ea\label{}
\langinfo{Tuatschín}{Bugnaj} {\citealt[131]{Büchli1966}}\\
\gll     Als da Sadrún òn vulju fá stá anavùs \textbf{la} \textbf{buéba} […].\\
 \textsc{def.art.m.pl} of Sedrun have.\textsc{prs.3pl} want.\textsc{ptcp.unm} make.\textsc{inf} stay.\textsc{inf} back \textsc{def.art.f.sg} girl\\
\glt `The people of Sedrun wanted to have her remain there.'
\z

Standard Sursilvan possesses the derivational suffix -\textit{entar} which transforms a verb or another syntactic category into a causative. Tuatschin also possesses this suffix, -\textit{antá} in the spelling used in this grammar, but to a very reduced extent. Where Standard Sursilvan has \textit{cuschentar} `cause to be quiet', \textit{fughentar} `light a fire', or \textit{luchentar} (\textit{il tratsch}) `loosen (the soil)', Tuatschin has \textit{fá còschar}, \textit{dá fjuc}, and \textit{fá luc} (\textit{al tratsch}). In the corpus, the following factitive verbs with -\textit{antá} occur (\tabref{factanta}). These verbs can be derived from verbs, adjectives, or nouns.

\begin{table}
	\caption{factitive verbs}
	\label{factanta}
	\begin{tabular}{lllll}
		\lsptoprule
		\textit{bubrantá} &`make drunk' & < & \textit{bájbar} & `drink'\\
		\textit{buantá} & `water (animal)' & < & \textit{bájbar} & `drink'\\
		\textit{cuntantá} &`satisfy' & < & \textit{cuntjants} & `glad'\\
		\textit{durmantá} & `make sleep' & < & \textit{durmí} & `sleep'\\
		\textit{fimantá} &`smoke' & < & \textit{fém} & `smoke' (n.)\\
		\textit{luantá}	&`melt (tr.)' & < &\textit{luá} & `melt (itr.)'\\
		\textit{nagantá}	&`drown (tr.)' & < & \textit{nagá} & `drown (itr.)'\\
		\textit{schjantá} & `dry' & < & \textit{schétg} & `dry' (adj.)\\
		\textit{schlupantá} & `blow up'&	< & \textit{schlupá} & `explode'\\
		\textit{sagljantá} & `blow up' & < & \textit{saglí} & `run, jump'\\
		
		\lspbottomrule
	\end{tabular}
\end{table}

All these verbs show stem alteration: \textit{ju bubrjanta} `I make drunk' vs \textit{nus bubrantajn} `we make drunk'.

Further: ju cuntjanta, ju schlupjanta, ju schjanta, 

\subsection{Passive}
Stative passive is formed with the verb \textit{èssar} `be' and the participle, and dynamic passive with \textit{vagní} ‘come’ and the participle. In both cases the participle agrees with the subject if the patient precedes the passive construction.

 \ea\label{}
\langinfo{Tuatschín}{Bugnaj}{\citealt[132]{Büchli1966}}\\
\gll \textbf{Èla} ségi \textbf{vagnida} \textbf{tratga} cun starmantusa fòrza […].\\
\textsc{3sg} \textsc{cop.prs.sbjv.3sg} come.\textsc{ptcp.f.sg} pull.\textsc{ptcp.f.sg} with tremendous.\textsc{f.sg} power\\
\glt `[She said that] she had been pulled with tremendous power.'
\z

\ea\label{}
\langinfo{Tuatschín}{Camischùlas}{f6; l.694ff.}\\
\gll    [...] \textbf{nus} \textbf{vagnévan} pròpi \textbf{tanidas} a nus stèvan amprèndar a nus stèvan ṣchubargè a fá a tùt.\\
{} \textsc{1pl} \textsc{pass.aux.impf.3pl} really hold.\textsc{ptcp.f.pl} and \textsc{1pl} must.\textsc{impf.1pl} learn.\textsc{inf} and \textsc{1pl}  must.\textsc{impf.1pl} clean.\textsc{inf} and do.\textsc{inf} and all\\
\glt `[...] we were really kept [in a strict way] and we had to learn and we had to clean and do and everything.'
\z

\ea\label{}
\langinfo{Tuatschín}{Tschamùt} {\citealt[53]{Büchli1966}}\\
\gll    [...] \textbf{als} \textbf{tiars} èn \textbf{vagní} \textbf{pri} òd stával a \textbf{purtaj} navèn.\\
{}    \textsc{def.art.m.pl}  animal.\textsc{pl}  be.\textsc{prs.3pl}   come.\textsc{ptcp.m.pl}   take.\textsc{ptcp.m.pl}  out.of barn and bring.\textsc{ptcp.m.pl}  away\\
\glt `The animals were taken away from the barn and brought away.'
\z

If the patient follows the passive construction, there is no agreement.

\ea\label{}
\langinfo{Tuatschín}{Zarcúns}{m2; l. 1540ff.}\\
\gll    La cumpagnia da mats, lèzas vèvan mitg’ jèda, la cumpagnia da mats vèva mintg’ jèda, lu èra quaj… craju, sjat fjastas… tga \textbf{vagnéva}… \textbf{fátg} \textbf{parada}.\\
\textsc{def.art.f.sg} association of boy.\textsc{pl} \textsc{dem.f.pl} have.\textsc{impf.3pl} every.\textsc{f.sg} time \textsc{def.art.f.sg} association of boy.\textsc{pl} have.\textsc{impf.3pl} every.\textsc{f.sg} time then \textsc{cop.impf.3sg} \textsc{dem.unm} believe.\textsc{prs.1sg.1sg} seven celebration.\textsc{f.pl} \textsc{rel} \textsc{pass.aux.impf.3sg} do.\textsc{ptcp.unm} parade.\textsc{f.sg}\\
\glt `The young men's associations, they had every time, the young men's association had every time, then there were …, I believe, seven celebrations … when they would … prepare a parade.'
\z

\ea\label{}
\langinfo{Tuatschín}{Sadrún}{m4; l.606f.}\\
\gll A qu' \textbf{èra} schòn \textbf{dau} bjè \textbf{najv} ad èran bigj’ aun vagní vidò culs tiars.\\  
and \textsc{dem.unm} \textsc{pass.impf.3sg} already give.\textsc{ptcp.unm} much snow.\textsc{f.sg} and be.\textsc{impf.3pl} \textsc{neg} yet come.\textsc{ptcp.m.pl} down with.\textsc{def.art.m.pl} animal.\textsc{pl} \\
\glt `And there was already a lot of snow and they hadn’t come back down with the animals yet.'
\z

\ea\label{}
\langinfo{Tuatschín}{Sadrún}{f3; l.107ff.}\\
\gll Ad ùssa òni partju ajn quaj, al cantún ò circa trènta da quèls majnadistricts, \textbf{quèls} \textbf{èn} \textbf{partí} \textbf{ajn} ajn ragjúns, ad ju a la val Tujétsch a Musté.\\
and now have.\textsc{prs.3pl.3pl} divide.\textsc{ptcp.unm} in \textsc{dem.unm} \textsc{def.art.m.sg} canton  have.\textsc{prs.3sg} about thirty of \textsc{dem.m.pl} head\_of\_district.\textsc{pl} \textsc{dem.m.pl} \textsc{pass.aux.prs.3pl} divide.\textsc{ptcp.m.pl} in in region.\textsc{f.pl} and \textsc{1sg} have.\textsc{prs.1sg} \textsc{def.art.f.sg} valley \textsc{pln} and \textsc{pln}\\
\glt `And now they have divided that, the canton has about thirty of these heads of district, these are divided in regions, and I have the Tujetsch valley and Mustér.'
\z

\ea\label{}
\langinfo{Tuatschín}{Sadrún}{m4; l.447f.}\\
\gll Ins vèz’ aun tg’ \textbf{èra} \textbf{dau} vidajn \textbf{pùntgas} né \textbf{trádals} [...].   \\
\textsc{gnr} see.\textsc{prs.3sg} still \textsc{comp} \textsc{pass.aux.impf.3sg} give.\textsc{ptcp.unm} into chisel.\textsc{f.pl} or power\_drill.\textsc{m.pl}\\
\glt `One still can see that chisels or power drills had been used [...]w.'
\z

\ea\label{ex:1:}
\langinfo{Tuatschín}{Camischùlas} {\citealt[94]{Büchli1966}}\\
\gll    [...] a lò végn \textbf{i} fatg \textbf{mèssa}, sunau \textbf{orgla} a cantau \textbf{végljias} \textbf{canzuns} \textbf{romontschas} […].\\
 {}    and there \textsc{pass.aux.prs.3sg}  \textsc{expl} do.\textsc{ptcp.unm} mass play.\textsc{ptcp.m.sg} organ.\textsc{f.sg}  and sing.\textsc{ptcp.m.sg} old.\textsc{f.pl} song.\textsc{pl} Romansh.\textsc{pl}\\
\glt `[…] and there a mass is said, the organ is played, and old Romansh songs are sung […].'
\z

Place names are considered to have no gender, hence the use of the unmarked form of the participle.

\ea\label{}
\langinfo{Tuatschín}{Sadrún}{m4; l. 419f.}\\
\gll Ah… Nalps \textbf{è} \textbf{vagnú} \textbf{fraquantau} \textbf{ò} scù majṣès ad alps adina [...].\\
ah  \textsc{pln}  be.\textsc{prs.3sg} \textsc{pass.aux.ptcp.unm} visit.\textsc{ptcp.unm} out as assembly\_of\_houses and alp.\textsc{m.pl} always [...].\\
\glt `Ah … Nalps has always been visited as an assembly of houses and as pastures [...].'
\z

If the agent of a passive construction is mentioned, it is introduced by \textit{da}. Whether \textit{da} corresponds to the preposition or to the dative marker is not easy to decide. The only case where there is a difference between the two \textit{da}'s is the first person object pronoun which is either \textit{mè} (accusative and after prepositions) or \textit{mé} (dative marker). As noted above in § 3.5, some speakers prefer using the pronoun \textit{mé} (dative), whereas others use \textit{mè} (accusative) in order to introduce the agent of a passive construction. I'll consider \textit{da} in these cases as corresponding to the dative marker because in the Sursilvan variety of Musté, the difference is made between \textit{da maj} (preposition) and \textit{da mi} (dative).

\ea
\label{}
\langinfo{Tuatschín}{Sadrún}{m5; l. 1254ff.}\\
\gll [...] quèls mulissiars, quèls ah fagèvan lu quasi las préfatschèntas, né, né tg’ èran... cumissunaj \textbf{da} \textbf{quèls} ah \textbf{martgadònts} grònṣ da la bassa, né.\\
{} \textsc{dem.m.pl} negotiator.\textsc{pl} \textsc{dem.m.pl} eh do.\textsc{impf.3pl} then so\_to\_speak \textsc{def.art.f.pl}  intermediate\_trade.\textsc{pl} right or \textsc{comp} \textsc{pass.aux.impf.3pl} commission.\textsc{ptcp.m.pl} \textsc{dat} \textsc{dem.m.pl} eh businessman.\textsc{pl} big.\textsc{pl} of \textsc{def.art.f.sg} «lowlands» right \\
\glt `[...] These negotiators, they would do so to speak the intermediate trade, or they were ... commissioned by the big businessmen from outside the Grisons, right?'
\z

\ea
\label{}
\langinfo{Tuatschín}{Sadrún}{m1}\\
\gll Al baghètg è vagnuṣ bagagjauṣ \textbf{da} \textbf{mju} \textbf{auc}.\\
\textsc{def.art.m.sg} building be.\textsc{prs.3sg} \textsc{pass.aux.ptcp.m.sg} build.\textsc{ptcp.m.sg} \textsc{dat} \textsc{poss.1sg.m.sg} uncle\\
\glt `The building has been built by my uncle.'
\z