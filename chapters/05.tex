\chapter{Simple sentences}

\section{Declarative sentences}


\section{Order of the arguments}

No subject inversion:

\ea\label{ex:1:}
\langinfo{Tuatschin}{} {\citealt[18]{Berther1998}}\\
\gll    Ju vegness schon, \textbf{dentaun} \textbf{vus} \textbf{vais} taun tschuf ain tgèsa.\\
      \textsc{1sg} come.\textsc{cond.1sg} really however \textsc{2pl} have.\textsc{prs.2pl} so.much dirt in house\\
\glt `Actually I would come, but you have so much dirt in your house.'
\z
 

\section{Interrogative sentences}



\ea\label{ex:1:}
\langinfo{Tuatschin}{Bugnei} {\citealt[132]{Büchli1966}}\\
\gll    La bueba ò detg, tg' ella segi betga ida voluntariamain, ella stuevi ira.\\
     \textsc{def.art.f.sg} girl have.\textsc{prs.3sg} say.\textsc{ptcp} \textsc{comp}         \textsc{3sg} be.\textsc{prs.sbjv.3sg} \textsc{neg} go.\textsc{ptcp.f.sg} voluntarily      \textsc{3sg.f}  must.\textsc{impf.sbjv.3sg} go.\textsc{inf}\\
\glt `The girl said that she didn't go voluntarily, [but that] she had to go.'
\z
 
    
\ea\label{ex:1:}
\langinfo{Tuatschin}{Sedrun} {\citealt[103]{Büchli1966}}\\
\gll    El ò dumandau ellas, pǝrtgei ellas segien betg idas a messa.\\
     \textsc{3sg.m} have.\textsc{prs.3sg} ask.\textsc{ptcp} \textsc{3pl.f} why  \textsc{3pl.f} be.\textsc{prs.sbjv.3pl} \textsc{neg} go.\textsc{ptcp.3pl.f} to mass\\
\glt `He asked them why they didn’t go to mass.'
\z

\ea\label{ex:1:}
\langinfo{Tuatschin}{Sedrun} {\citealt[105]{Büchli1966}}\\
\gll    ǝ damonda ců i vondi.\\
     and ask.\textsc{prs.3sg} how  \textsc{expl} go.\textsc{prs.sbjv.3sg}\\
\glt `[…] and he asked how he was.'
\z




\section{Imperative sentences}

\ea\label{}
\langinfo{Tuataschin}{}{\DRG{2}{327}}\\
\gll  Sapartgirai dals betlers cu tgi von a tgavai.  \\
     \textsc{refl}.beware.\textsc{imp.2pl} of.\textsc{def.art.m.pl} beggar.\textsc{pl} when \textsc{rel.3pl.comm} go.\textsc{prs.3sg} on horse\\
\glt `Beware of the beggars if they ride.'
\z

\ea\label{}
\langinfo{Tuatschin}{}{\DRG{2}{503}}\\
\gll  Betg tumai!\\
     \textsc{neg}  be.afraid.\textsc{imp.2pl}\\
\glt `Don’t be afraid!'
\z



\section{Exclamative sentences}

\ea\label{}
\langinfo{Tuatschin}{}{\DRG{2}{215}}\\
\gll Jeusas, \textbf{co} quai brischa!   \\
   \textsc{excl} how \textsc{dem} burn.\textsc{prs.3sg}  \\
\glt `Jee, how it is burning [of a fire]!'
\z


\section{Voice}

\subsection{Reflexive}

Reflexive verbs are formed by the prefix sa-. The Sursilvan norm claims that the auxiliary verb be èssar `be'; however, vai `have' is not rare.

\ea\label{}
\langinfo{Tujetsch}{}{\DRG{6}{321}}\\
\gll   Als fildiròms \textbf{an} \textbf{sa-pagliau} aint.\\
    \textsc{def.art.m.pl} wire.\textsc{pl} have.\textsc{prs.3pl} \textsc{refl}-touch.\textsc{ptcp.m.sg} in \\
\glt `The wires touched each other. '
\z


\subsection{Reciprocal}

\subsection{Causative}
Causative voice is formed with the verb fà ‘make’. The causee is located after the second verb.
%whether this is a subject or an object will have to be tested with personal pronouns.%

\ea\label{}
\langinfo{Tuatschin}{Sadrún}{m16}\\
\gll    Èl ò schau savay \textbf{la règin}a quaj.\\
     \textsc{3sg} have.\textsc{prs.3sg} let.\textsc{ptcp} know.\textsc{inf} \textsc{def.art.f.sg} queen \textsc{dem}\\
\glt `He let the queen know this.'
\z

\ea\label{ex:1:}
\langinfo{Tuatschin}{Rueras} {\citealt[62]{Büchli1966}}\\
\gll    […] i vèevan fatg vegnî \textbf{ål caplòn dǝ Selva} pǝr benedî la nibla […].\\
    [...]   \textsc{3pl} have.\textsc{impf.3pl} make.\textsc{ptcp} come.\textsc{inf} \textsc{def.art.m.sg} chaplain of \textsc{pln} \textsc{purp} bless.\textsc{inf} \textsc{def.art.f.sg} cloud\\
\glt `They’d had the chaplain of Selva come in order to bless the cloud […].'
\z

\ea\label{ex:1:}
\langinfo{Tuatschin}{Bugnei} {\citealt[131]{Büchli1966}}\\
\gll     l də Sedrun òn vuliu fâ stâ anavůůs la bueba […].\\ %check def.art.m.pl
   \textsc{def.art.m.pl} of Sedrun have.\textsc{prs.3pl} want.\textsc{ptcp} make.\textsc{inf} stay.\textsc{inf} back \textsc{def.art.f.sg} girl\\
\glt `The people of Sedrun wanted to have her remain there.'
\z

-antá




\subsection{Passive}
Stative passive is formed with the verb \textit{esser} `be' and the participle, and dynamic passive with \textit{vegnì} ‘come’ and the participle. In both cases the participle agrees with the subject if the patient is in subject position. %stative passive: look for a verb that is conjugated with 'avai'.

 \ea\label{}
\langinfo{Tuatschin}{Bugnai}{\citealt[132]{Büchli1966}}\\
\gll Ella segi \textbf{vegnida} \textbf{tratga} cun stermentusa forza […].\\
     \textsc{3sg} \textsc{cop.prs.sbjv.3sg} come.\textsc{ptcp.f.sg} pull.\textsc{ptcp.f.sg} with tremendous.\textsc{f.sg} power\\
\glt `[She said that] she had been pulled with tremendous power.'
\z


\ea\label{ex:1:}
\langinfo{Tuatschin}{Tschamut} {\citealt[53]{Büchli1966}}\\
\gll    ǝls tiers en \textbf{vegni} \textbf{pri} òod stavel ǝ \textbf{purtai} naven.\\
     \textsc{def.art.m.pl}  animal.\textsc{pl}  be.\textsc{prs.3pl}   come.\textsc{ptcp.m.pl}   take.\textsc{ptcp.m.pl}  out.of barn and bring.\textsc{ptcp.m.pl}  away\\
\glt `The animals were taken away from the barn and brought away.'
\z

If the patient remains in object position, i.e. if there is an impersonal passive construction which involves the expletive pronoun \textit{i} in subject position, the participle does not agree with the patient.


\ea\label{ex:1:}
\langinfo{Tuatschin}{Camischolas} {\citealt[94]{Büchli1966}}\\
\gll    ǝ lò vegn \textbf{i} fatg \textbf{messa}, sun-au \textbf{orgla} ǝ cant-au \textbf{veglia-s} \textbf{canzun-s} \textbf{romontscha-s} […].\\
     and there \textsc{pass.aux.prs.3sg}  \textsc{expl}  do.\textsc{ptcp.m.sg} mass play-\textsc{ptcp.m.sg}  organ.\textsc{f.sg}  and sing-\textsc{ptcp.m.sg} old.\textsc{f}-\textsc{pl}   song-\textsc{pl} romansh-\textsc{pl}\\
\glt `[…] and there a mass is said, the organ is played, and old Romansh songs are sung […].'
\z


\ea\label{ex:1:}
\langinfo{Tuatschin}{Rueras}{\citealt[65]{Büchli1966}}\\
\gll    Co \textbf{ṣai} vegniu \textbf{ina} \textbf{femna} dǝ Méidel […].\\
     here be.\textsc{prs.3sg.expl} come.\textsc{m.sg}  \textsc{indef.art.f.sg} woman of  \textsc{pln} \\
\glt `There came a woman from Meidel […].'
\z


