\chapter{Noun phrase}

\section{The noun}

\subsection{Gender}
Tuatschin differentiates two genders, masculine and feminine, which are not restricted to natural gender, but natural gender and grammatical gender usually correspond.

Natural gender of humans and animates is either differentiated by two different words, or – exclusively with human nouns – the suffix \textit{-a} (sometimes with the infix -\textit{èss}-) is added to the masculine form (\tabref{tab:nouns:gendis}).

\begin{table}
\caption{Natural gender distinctions}
\label{tab:nouns:gendis}
\begin{tabular}{lllll}
 \lsptoprule
  \midrule
  bap & `father' & vs & mùma & `mother'\\
  béadi & `grandson` & vs & béáditga & `granddaughter' \citet{DRG1: 60}\\ %cf. beaditga
buap& `boy' & vs & buaba &`girl'\\
mastral & `senior official' & vs & mastarlèssa & `wife of the senior official' \\
prenzi & `prince' & vs & princèssa & `princess'\\
sir & `father-in-law' & vs & sira & `mother-in-law'\\
tgaun & `dog' & vs & cògna & `bitch'\\
tgéjt & `rooster' &vs& gaglina & `hen'\\
vadí & `calf' & vs & vadjala & `female calf'\\
\lspbottomrule
\end{tabular}
\end{table}

Some feminine counterparts of masculine animals which are listed in \citet{Spescha1989} are not in use in Tuatschin, as for example \textit{cavalla} `mare', \textit{utschala} `female bird', \textit{purschala} `sow'. In the case of \textit{píartg} 'pig', the feminine counterpart is only used in a metaphoric sense: \textit{ina pòrtga} `a dirty woman'.

in taur tscharva `deer' , ina vaca tscharva `hind'
in bùc tgamus `chamois buck', ina tgaura tgamùs `'
in bùc tgavríal, ina tgaura cavrial

máscal, fèmna (schémja)


Some inanimate masculine nouns have a feminine singular counterpart which usually refers to collective or generic entities which cannot be pluralized or counted. Compare: in \textit{tgern} ‘a/one horn’, \textit{tschun corns} ‘five horns’, \textit{la corna} ‘(the) horns’. In the case of paired body terms the collective noun refers to the two entities, like \textit{ganugl} ‘knee’ vs. \textit{la ganuglia} ‘the (two) knees’. Some more examples are


 \ea\label{}
\langinfo{Tuatschin} {Selva} {\citealt[26]{Büchli1966}}\\
\gll    Ell’ ò vuliu vender \textbf{maila} […].\\
    \textsc{3sg.f} have.\textsc{prs.3sg} want.\textsc{ptcp} sell.\textsc{inf} apple.\textsc{coll}\\
\glt `She wanted to sell apples.'
\z

\ea\label{ex:1:adj}
\langinfo{Tuatschin}{Selva} {\citealt[53]{Büchli1966}}\\
\gll «Ò la \textbf{paira} èe pais?» La mumma ò detg: «Na, la \textbf{paira} ò betga pais.» ǝ lu ò la feglia detg: «Schǝ la \textbf{paira} ò betga pais, vai ju magliau in rusp.»\\
have.{\prs}.3{\sg} \textsc{def}.{\art}.\textsc{f}.{\sg} pear.\textsc{coll} also foot.\textsc{pl} \textsc{def}.{\art}.\textsc{f}.{\sg} mother have.{\prs}.3{\sg} say.{\ptcp} no \textsc{def}.{\art}.\textsc{f}.{\sg} pear.\textsc{coll} have.{\prs}.3{\sg} \textsc{neg} foot.{\pl} and then have.{\prs}.3{\sg} \textsc{def}.{\art}.\textsc{f}.{\sg} daughter say.{\ptcp} if \textsc{def}.{\art}.\textsc{f}.{\sg} pear.\textsc{coll} have.{\prs}.3{\sg} \textsc{neg} foot.{\pl} have.1{\sg} 1{\sg}. eat.{\ptcp} \textsc{indef}.\textsc{art}.\textsc{m}.\textsc{sg}  toad\\
\glt `« Do pears have feet ? » The mother said : « No, pears do not have feet. » Then the daughter said : « If pears do not have feet, then I have eaten a toad. »'
\z

\ea\label{}
\langinfo{Tuatschin}{}{\DRG{6}{697}}\\
\gll  Al mèxgiar fimjanta cun \textbf{ògna} […].  \\
    \textsc{def.art.m.sg} butcher smoke.\textsc{prs.3sg} with alder.\textsc{coll} \\
\glt `The butcher smokes with alder wood.'
\z

còtgal, còtgla
trajs pèrs majla, trajs pèra calzès
curnagl curnaglja 'Bergdohle'

fìap, fòpa: in fiep plé grònd.



\subsection{Number}
Plural is formed adding \textit{-s} to the stem of the noun, respectively to any part of the noun phrase, whether the stem ends in a vowel or in a consonant: \textit{tgèsa} (f.) `house' vs. \textit{tgèsas} `houses' or \textit{rusp} (m.) `toad' vs. \textit{rusps} `toads'. If the noun ends in an \textit{-s}, there is no differentiation between singular and plural. % find example

There are some irregular plurals which are listed in \tabref{irregplI},  \tabref{irregplII}, and \tabref{irregplIII}. 


\begin{table}
\caption{Nouns: irregular plural I}
\label{irregplI}
 \begin{tabular}{llllll}
  \lsptoprule
   -ì & -jalts & & -éigls & -ólts & \\
  \midrule
\textit{aní} & \textit{anjalts} & `ring' & \textit{anṣéjgl} & \textit{anṣólts} & `kid'\\
\textit{castí} & \textit{castjalts} & `castle' & \textit{catschéjgl} & \textit{catschólts} & `sock'\\
\textit{cuntí} & \textit{cuntjalts} & 'knife' & \textit{spéjgl} & \textit{spòglts} & 'bobbin'\\
\textit{flagí}  &   \textit{flagjalts} & `flail' \\
\textit{iṣchí} & \textit{iṣchjalts} & `maple tree'\\
  \textit{martí} & \textit{martjalts} &`hammer'\\
  \textit{purschí} & \textit{purschalts} & `piglet'\\
  \textit{rassplí} & \textit{rasspljalts} & `pencil'\\
  \textit{rastí} & \textit{rastjalts} & `rake'\\
\textit{utschí} & \textit{utschalts}  &`bird'\\
\textit{vadí} & \textit{vadjalts} & `calf'\\
  \lspbottomrule
 \end{tabular}
\end{table}

Masculine nouns with the diphthong /ie/ or /ej/ in the stem change the diphthong to o, whereby the nouns starting with palatal \textit{tg} /ʨ/depalatalize to \textit{c} /k/.


\begin{table}
\caption{Nouns: irregular plural II} 
\label{irregplII}
 \begin{tabular}{llllll}
  \lsptoprule
   -ìa & -ò & & other & ò & \\
  \midrule
críac & cròcs & `plough' & éjf & òfs & `egg'\\
fíap & fòps & `hollow' & tgéit & còts \footnote{DRG 3:595 offers \textit{tgéit} vs. \textit{tgéits}, i.e. a regular plural. Some informants also indicate \textit{tgéits} as the plural form of \textit{tgéit}.} & `rooster'\\
íart  & òrts & `garden' & tgérn & còrns & `horn'\\
  ías & òs & `bone' \\
píartg & pòrs & `pig'\\
tgaubríacal & tgaubròcals & 'somersault' & tagljér & tagliòrs & `plate'\\
  \lspbottomrule
 \end{tabular}
\end{table}

Note, however, that the plural of \textit{tjéjt} `rooster' is often \textit{tgéjts} and not \textit{còts} (TarHen 2017 03 14 I, sec 356).

\begin{table}
\caption{Nouns: irregular plural III}
\label{irregplIII}
 \begin{tabular}{lll}
  \lsptoprule
singular & plural \\
  \midrule
bóf & bós & `ox'\\
cumandamèn & cumandamaints & `commandment'\\
dé & dis & `day'\\
líuc  & lògans & `place'\\ % ni liug?
tgavaj & tgavals & `horse'\\
tgavégl & tgavéiglts & `hair'\\ 
trutg & truigls & `narrow path'\\
ùm & ùmans & `man'\\
  \lspbottomrule
 \end{tabular}
\end{table}

Furthermore, some monosyllabic masculine nouns having the diphthong /iə/ convert this falling diphthong in a rising one: \textit{culìar} vs. \textit{culjars} `collar', \textit{falìan} vs. \textit{faljans} `spider', \textit{fíar} /fiər/ iron vs. \textit{fjars} /fjars/ or \textit{palíat} vs. \textit{paljats} `arrow', \textit{ṣchíarl} vs. \textit{ṣchjarls} `kind of basket', \textit{tíarm} /tiərm/ boundary stone vs. \textit{tjarms} /tjarms/, \textit{unvíarn} vs. \textit{unvjarns} `winter'.


Compound nouns:

\ea\label{ex: }
\langinfo{Tuatschin}{}{\citealt[87]{Gadola1935}}\\
\gll  […] ju vai era piu ansiamen ils quéns de tschels dus \textbf{tgau}\textbf{-s}-tegia.  \\
    […] \textsc{1sg} have.\textsc{prs.1sg} also take.\textsc{ptcp} together \textsc{def.art.m.pl} bill.\textsc{pl} of \textsc{dem.m.pl} two head-\textsc{pl}-alp.hut \\
\glt `[…] I have also assembled the bills of the other two heads of the alp-huts.'
\z


\section{Determiners and pronouns}
The determiners all precede the noun they modify and distinguish number and gender, but no case. An exception is the definite dative article, which does not distinguish gender, but which attributes case to the noun phrase, as its name indicates.

\subsection{Articles}

\subsubsection{Definite article}

\begin{table}
\caption{Definite article}
\label{tab:1:defart}
 \begin{tabular}{llll}
  \lsptoprule
   \textsc{m.sg}   &  \textsc{m.pl} & \textsc{f.sg} & \textsc{f.pl}\\ 
  \midrule
  al, gl, l  & als, as, ls & la, l &  las\\
    \lspbottomrule
 \end{tabular}
\end{table}

l (post-verbal and after vowel) (B 104, B 64)

ɐz (C 162)

ls (28) (after vowel)

In combination with the preposition \textit{da} ‘of’ and a place name, the definite article is used to form demonyms.

\ea\label{}
\langinfo{Tujetsch}{Camischolas}{\citealt[94]{Büchli1966}}\\
\gll    \textbf{ǝls} \textbf{dǝ} \textbf{Tujetsch} tegnan aut quai liug […].\\
    \textsc{def.art.m.pl} of \textsc{pln} hold.\textsc{prs.3pl} high \textsc{dem} place\\
\glt `The people of Tujetsch uphold this place […].'
\z

This construction is not restricted to the habitants of villages or towns, but to any habitable place.

\ea\label{}
\langinfo{Tuatschin}{}{\citealt[]{Büchli1966}}\\%page number!
\gll […] òn \textbf{əls} \textbf{də} \textbf{tgèesa} detg.\\
[…] have.\textsc{prs.3pl} \textsc{def.art.m.pl} of house say.\textsc{ptcp}\\
\glt `[…] said those at home.'
\z



\subsubsection{Indefinite article}
The indefinite article singular is identical to the numeral \textit{in} (m.)/\textit{ina} (f.) ‘one’; there is no plural indefinite article; plural indefinites are bare noun phrases.

\ea\label{}
\langinfo{Tuatschin}{Rueras}{\citealt[66]{Büchli1966}}\\
\gll Ingnèeda vèev’ \textbf{in} pur \textbf{in} stauschbena.\\
     once have.\textsc{impf.3sg} \textsc{indef.art.m.sg} peasant \textsc{indef.art.m.sg} wheelbarrow \\
\glt `Once a peasant had a wheelbarrow.'
\z


Like the definite article, the indefinite article is used for demonyms.

\ea\label{}
\langinfo{Tuatschin}{Selva}{\citealt[52]{Büchli1966}}\\
\gll   In dǝ Méidel è ius cul tren giů Cuèera.\\
     one of \textsc{pln} be.\textsc{prs.3sg} go.\textsc{ptcp.m.sg} with train down \textsc{pln}\\
\glt `A person from Medel went to Cuera by train.'
\z

\subsubsection{Dative article} 

Until approximately the 1960, the dative article \textit{di} or \textit{li} was in common use.\footnote{Some indications concerning the dative article in all Romansh varieties can be found in \citet{Linder1987}.} Nowadays it is obsolescent; spontaneous production are very rare in the corpus and were exclusively produced by elder people.

The dative article is a definite article; it distinguishes number but not gender. Its forms are \textit{di} (sg) and \textit{dis} (pl) or \textit{li} and \textit{lis}.  Whereas \textit{li, lis} were widespread in other Sursilvan dialects, as well as in further  Romansh varieties such as Sutsilvan and Surmiran,  \textit{di, dis} was a genuine Tuatschin form.

\textit{Di} and \textit{li} were also used for dative personal pronouns, however without making a difference between gender and number (see xxx).

\ea\label{}
\langinfo{Tuatschin}{Rueras} {\citealt[64]{Büchli1966}}\\
\gll  Co ṣai	vegniu	ina	femna əncůnter \textbf{li} \textbf{quai}   \textbf{pur} […].\\
     here	be.\textsc{prs}.3\textsc{sg}  come.\textsc{ptcp}	\textsc{indef}.\textsc{art}.\textsc{f}.\textsc{sg} woman	towards	\textsc{dat}.\textsc{art}.\textsc{sg} \textsc{dem} peasant\\
\glt `At this moment a woman came towards this peasant […].'
\z


\ea\label{}
\langinfo{Tuatschin}{Tschamut} {\citealt[12]{Büchli1966}}\\
\gll  Quai è curdau sé li gliut.\\
     \textsc{dem} be.\textsc{prs.3sg} fall.\textsc{ptcp} up \textsc{dat} people.\textsc{f.sg}\\
\glt `People noticed this.'
\z

\ea\label{}
\langinfo{Tuatschin}{}{\citealt[199]{ASR1889}}\\
\gll  Sche tots ratuns èn intenzionai uschê sco quels dus, lura vali betga tier els il proverbi, tge survescha \textbf{li-s} \textbf{carstgaun-s} per zanur […].\\
     if all.\textsc{m.pl} rat.\textsc{pl} \textsc{cop.prs.3pl} intentioned.\textsc{pl} so like \textsc{dem.m.pl} two then apply.\textsc{prs.3sg.expl}  \textsc{neg}  by \textsc{3pl.m}  \textsc{def.art.m.sg} proverb \textsc{rel}  serve.\textsc{prs.3sg}  \textsc{dat}-\textsc{pl}  person-\textsc{m.pl} for dishonour\\
\glt `If all the rats were as well-intentioned as these two, the proverb which dishonours human beings wouldn’t apply to them […].'
\z

\ea\label{}
\langinfo{Tuatschin}{Bugnai} {\citealt[147]{Büchli1966}}\\
\gll El vèeva aber raquintau li-s pader-s quai tg’ el vèeva fatg giu cul giavel […]..\\
     \textsc{3sg} have.\textsc{impf.3sg} but tell.\textsc{ptcp} \textsc{dat-pl} Father-\textsc{pl} \textsc{dem} \textsc{rel} \textsc{3sg} have.\textsc{impf.3sg} make.\textsc{ptcp}  down with.\textsc{3sg} devil\\
\glt `But he had told the Fathers what he had settled with the devil.'%Quei exempel sai jeu duvrar forsa per mussar si la posiziun dad 'aber'.
\z

The following examples have been uttered spontaneously by my consultants.

Example xxx shows the simultaneous occurrence of di and a + definite article in the same utterance.

\ea\label{}
\langinfo{Tuatschin}{Sadrún} {m12}\\
\gll   Quaj duvrava’l par dá \textbf{dis} \textbf{pòrs}, trúfals ansjaman par dá \textbf{als} \textbf{pòrs} \\
    \textsc{dem} use.\textsc{impf=3sg} \textsc{purp} give.\textsc{inf} \textsc{dat.art.pl} pig.\textsc{pl} potato.\textsc{pl} together \textsc{purp} give.\textsc{inf} to.\textsc{def.art.m.pl} pig.\textsc{pl} \\
\glt `This he used in order to give to the pigs, potatoes together [with nettles] in order to give to the pigs. '
\z

\subsection{Demonstratives}
There are three series of demonstratives: the \textit{quèl}-series ({\tabref{demquel}), which has deictic demonstrative as well as anaphoric functions, the \textit{lèz}-series ({\tabref{demlez}}), which is exclusively an\-a\-phor\-ic, and the \textit{quèst}-series ({\tabref{demquest}}), which is only used as a determiner with temporal complements of the verb.

\begin{table}
\caption{Demonstratives: the \textit{quèl}-series}
\label{demquel}
 \begin{tabular}{llllll}
  \lsptoprule
            & \textsc{m.sg} & \textsc{m.pl} & \textsc{f.sg} & \textsc{f.pl} & \textsc{genderless}\\
  \midrule
  determiner  & quai &  quèls  & quèla  & quèlas\\
  pronoun  & quèl & quèls & quèla & quèlas & quai\\
  \lspbottomrule
 \end{tabular}
\end{table}

\ea\label{}
\langinfo{Medelin}{}{\DRG{4}{485}}\\
\gll    \textbf{Quel} lavura betga bia ple, mo la continua fa aun dabia.\\
     \textsc{dem.m.sg} work.\textsc{prs.3sg} \textsc{neg} much any.more but \textsc{def.art.f.sg} perseverance make.\textsc{pres.3sg} still much\\
\glt `This one doesn’t work any more but his perseverance still makes a lot.'
\z


The genderless demonstrative \textit{quai} may refer anaphorically to a precedent sentence %example
or to a noun. When it refers to nouns, it may have singular or plural reference; if it is the subject of the sentence, the copula agrees with \textit{quai} and not with the predicative noun.

\ea\label{}
\langinfo{Tuatschin}{}{\DRG{4}{376}}\\
\gll \textbf{Quai} \textbf{e} dètgas cumars.\\
\textsc{dem.gless} \textsc{cop.prs.3sg} real.\textsc{f.pl} chatterbox.\textsc{pl}\\
\glt `These are real chatterboxes.'
\z

\ea\label{}
\langinfo{Tuatschin}{}{\citealt[28]{Gartner1910}}\\
\gll ɛ kwaj mʋʃʨəs? – kwaj ɛ furmikləs […].\\
     \textsc{cop.prs.3sg} \textsc{dem.gless} fly.\textsc{pl} -  \textsc{dem.gless} \textsc{cop.prs.3sg} ant.\textsc{pl}\\
\glt `Are these flies? – They are ants […].'
\z

\ea\label{ex:1:}
\langinfo{Tuatschin}{}{\citealt[15]{Berther2007}}\\
\gll Uardei co se quels dus, quai è dus vairs lumps!\\
     look.\textsc{imp.hon} how \textsc{cop.prs.3sg} \textsc{\textbf{dem.m.pl}} two \textsc{dem.grl} \textsc{cop.prs.3sg} two real.\textsc{m.pl} rascal.\textsc{pl}  \\
\glt `Look how these two are, these are two real rascals!'
\z

In Medelin, there is one occurrence of the dative marker \textit{ada} used with \textit{quel}.

\ea\label{}
\langinfo{Medelin}{}{\DRG{6}{528}}\\
\gll  \textbf{Adaquel} vai jeu furau oz. \\
    \textsc{dat.dem.m.sg} have.\textsc{prs.1sg} \textsc{1sg} sink.\textsc{ptcp} today \\
\glt `Him I have beaten today.'
\z


\begin{table}
\caption{Demonstratives: the \textit{lez}-series}
\label{demlez}
 \begin{tabular}{llllll}
  \lsptoprule
            & \textsc{m.sg} & \textsc{m.pl} & \textsc{f.sg} & \textsc{f.pl} & \textsc{genderless}\\
  \midrule
  determiner  & glièz &  lèz  & lèza  & lèzas\\
  pronoun  & lèz & lèz & lèza & lèzas & glièz \\
  \lspbottomrule
 \end{tabular}
\end{table}

\ea\label{}
\langinfo{Tuatschin}{Bugnai}{\citealt[132]{Büchli1966}}\\
\gll Lu ṣen ai î sen claustra å fâ vegnî giů gl avat. \textbf{Lez} è vegniu giů […].\\
     then \textsc{cop.prs.3pl} \textsc{3pl} go.\textsc{ptcp.m.3pl} on monastery \textsc{purp} make come down \textsc{def.art.m.sg} abbot \textsc{dem} \textsc{cop.prs.3sg} come.\textsc{ptpc.m.sg} down\\
\glt `Then they went up to the monastery and made the abbot come down. He came down […].'
\z

\ea\label{}
\langinfo{Tuatschin}{}{\citealt[85]{Berther1998}}\\
\gll «Da tgei as lu semiau?» «\textbf{Gliez} vi ju schon di da té.»\\
     of what have.\textsc{prs.2sg} then dream.\textsc{ptcp} \textsc{dem} want.\textsc{prs.1sg} \textsc{1sg} 
 certainly tell.\textsc{inf} \textsc{dat} \textsc{2sg}\\
\glt `«What did you dream of then?» «That I will tell you, of course.»'
\z

Gliez is not restricted as to syntactic functions.

\ea\label{}
\langinfo{Tuatschin}{}{\citealt[91]{Gadola1935}}\\
\gll Vus Sep Flurin duvreis nuéta selamentar pervia de \textbf{gliez} […].\\
    \textsc{2sg.hon} \textsc{pn} \textsc{pn} need.\textsc{prs.2pl} nothing \textsc{refl.}complain.\textsc{inf} because of \textsc{dem} \\
\glt `You, Sep Flurin, need not complain about that […].'
\z


The following examples illustrate the anaphoric use of the demonstrative. %The question is whether this is OK for modern speakers or whether quai can/must be replaced by lez.

\ea\label{}
\langinfo{Tuatschin}{Rueras}{\citealt[64]{Büchli1966}}\\
\gll    Gl atun sissura ṣè \textbf{quai} \textbf{pur} de Ruèeras ius a féira giů Ligiaun.\\
     \textsc{def.art.m.sg} autumn after be.\textsc{prs.3sg} \textsc{dem.m.sg} peasant of \textsc{pln} go.\textsc{ptcp.m.sg} to fair down \textsc{pln}\\
\glt `The autumn after that this peasant went down to Lugano to the fair.'
\z

\ea\label{}
\langinfo{Tuatschin}{Rueras}{\citealt[65]{Büchli1966}}\\
\gll    \textbf{Quai} è  schabegiau òo Zarcuns […].\\
    \textsc{dem.gl} be.\textsc{prs.3sg}  happen.\textsc{ptcp} out Zarcuns\\
\glt `This happened in Zarcuns […].' 
\z

\ea\label{}
\langinfo{Tuatschin}{Rueras}{\citealt[66]{Büchli1966}}\\
\gll    Ingnèeda vèev’ in pur in stauschbena. \textbf{Quel} vegnèeva naven la notg.\\
     once have.\textsc{impf.3sg} \textsc{indef.art.m.sg} peasant \textsc{def.art.m.sg} wheelbarrow \textsc{dem.m.sg} come.\textsc{impf.3sg} away \textsc{def.art.f.sg} night\\
\glt `Once a peasant had a wheelbarrow. This used to disappear during night.'
\z

\ea\label{}
\langinfo{Tuatschin}{Sedrun}{\citealt{Büchli1966}}\\
\gll    El vèeva dǝ cargèe ina bůra, mů \textbf{quella} èera gǝlada ai la naiv […].\\
    \textsc{3sg} have.\textsc{impf.3sg} to carry \textsc{indef.art.f.sg} block but \textsc{dem.f.sg} \textsc{cop.impf.3sg} freeze.\textsc{ptcp.f.sg} in \textsc{def.art.f.sg} snow\\
\glt `He had to carry a block (of wood), but it was frozen in the snow […]. '
\z

\ea\label{ex:}
\langinfo{Tuatschin}{}{\citealt[88]{Gadola1935}}\\
\gll Quai daien \textbf{quels} dus giuvens \textbf{lo} fa […].\\
    \textsc{dem} have.to.\textsc{prs.3pl} \textsc{dem.m.pl} two young.\textsc{m.pl} there do.\textsc{inf}\\
\glt `This the two young [men] over there should do [...].'
\z

\ea\label{}
\langinfo{Tuatschin}{Camischolas}{\DRG{3}{577}}\\
\gll   Als caschnès da mò dus pòsts, \textbf{quèls} numnávani giainas.\\
     \textsc{def.art.m.pl} hayrack.\textsc{pl} of only two post.\textsc{pl} \textsc{dem.m.pl} call.\textsc{impf.3pl.3pl.comm} "giainas"\\
\glt `The hayracks of only two posts, these were called “giainas”.'
\z

\ea\label{}
\langinfo{Tuatschin}{Camischolas}{\DRG{3}{379}}\\
\gll  Quai èra cèrts sèrvituts. \textbf{Glièz} sa ju mai tg’i a dau històrias parvi da quai. \\
    \textsc{dem} \textsc{cop.impf.3sg} certain.\textsc{m.pl} servitude.\textsc{pl} \textsc{dem.grl} know.\textsc{prs.1sg} never \textsc{comp=expl} have.\textsc{prs.3sg} give.\textsc{ptcp} story.\textsc{f.pl} because of \textsc{dem} \\
\glt `These were certain servitudes. I don’t know at all whether there were problems because of that.'
\z

Anaphoric use of the quaj-series with place names:

a lu vajn nuS durmju sé ál hòspiz da quaj Grimsel (ca. 568)


In order to determine more precisely the emplacement of the referent, the demonstratives can be modified by the locative adverbs \textit{cò} `here' and \textit{lò} `there', whereby the adverb is located after the noun in case of a demonstrative determiner.

\ea\label{}
\langinfo{Medelin}{}{\DRG{3}{506}}\\
\gll   \textbf{Quèl} \textbf{lò} a bétg bja ṣùt ils tgavéils.\\
     \textsc{dem.m.sg} there have.\textsc{prs.3sg} \textsc{neg} much under \textsc{def.art.m.sg} hair.\textsc{pl}\\
\glt `This one over there doesn't have much under his hair.'
\z



The \textit{quest}-series is only used when modifying a temporal noun.

\ea\label{}
\langinfo{Tuatschin}{}{\citealt[86]{Gadola1935}}\\
\gll Co survegnis vus il dètg urden \textbf{questa} \textbf{sèra}.\\
  here get.\textsc{prs.2pl} \textsc{2pl} \textsc{def.art.m.sg} dètg urden \textsc{dem.f.sg} evening\\
\glt `Here you get the … this evening.'
\z

\ea\label{}
\langinfo{Tuatschin}{}{\citealt[91]{Gadola1935}}\\
\gll    Cons signals os té fatg giu \textbf{questa} \textbf{stad} sél Calmut?\\
  how.many signal.\textsc{pl} have.\textsc{prs.2sg} \textsc{2sg} make.\textsc{ptcp} down \textsc{dem.f.sg} summer on.\textsc{def.art.m.sg} \textsc{pln}\\
\glt `How many signals did you make this summer on the Calmut ?'
\z

The \textit{tschèl}-series.
\textit{Tschèl} is often best translated by \textit{the other}.

\ea\label{}
\langinfo{Tuataschin}{}{\DRG{4}{598}}\\
\gll Ina fò an cuṣchina a tschella fò sé ils létgs.\\
  one make.\textsc{prs.3sg} in kitschen and \textsc{dem.f.sg} make.\textsc{prs.3sg} up \textsc{def.art.m.pl} bed.\textsc{pl}\\
\glt `One works in the kitchen and the other makes the beds.'
\z

\subsubsection{Difference between the \textit{quèl}- and the \textit{lèz}-series}
As mentioned before, the \textit{quèl}-series may refer deictically and anaphorically to the referent, whereas the \textit{lèz}-series only refers anaphorically to the referent. Therefore the question arises as to the difference between the two series in the domain of anaphora.

There are at least two domains where \textit{lèz} is preferred over \textit{quèl}: with topicalized subjects which are located outside the sentence, and with the preposition cun 'with'.  

\ea\label{}
\langinfo{Tuatschin}{Sadrún}{m4}\\
\gll a lu al \textbf{tat}, \textbf{lèz} pinava tíar la tschajna  \\
  and then \textsc{def.art.m.sg} grandfather \textsc{dem} prepare.\textsc{impf.3sg} to \textsc{def.art.f.sg} dinner  \\
\glt `[...] and then my grandfather, he used to prepare dinner [...].'
\z

\ea\label{}
\langinfo{Tuatschin}{Sadrún}{m4}\\
\gll Ju a antupau al \textbf{Gíari}, a lu sùndju juṣ ád alp cun \textbf{lèz}.\\
  \textsc{1sg} have.\textsc{prs.1sg} meet.\textsc{ptcp} \textsc{def.art.m.sg} \textsc{pn} and then be.\textsc{prs.1sg.1sg} go.\textsc{ptcp.m.sg} to alp with \textsc{dem.m.sg}   \\
\glt `I met Gieri, and then I went to the alp with him.'
\z



\subsection{Possessives}


\begin{table}
\caption{Possessive determiners}
\label{possdet}
 \begin{tabular}{lllll}
  \lsptoprule
& \textsc{m.sg} & \textsc{f.sg}  & \textsc{f.sg}  & \textsc{f.pl}\\
  \midrule
\textsc{1sg}  & miu  &   mès &  & \\
\textsc{2sg} & tiu & & tia\\
\textsc{3sg} & siu & séis\\
\textsc{1pl} & nias & nós, nòs?\\
\textsc{2pl} & vias \\
\textsc{3pl}	& siu & séis \\
   \lspbottomrule
 \end{tabular}
\end{table}

\ea\label{}
\langinfo{Tuatschin}{Camischolas}{\citealt[90]{Büchli1966}}\\
\gll   […] e privava uschéia əls purets də \textbf{siu} \textbf{fatg}.\\
     [...] and deprive.\textsc{impf.3sg} so \textsc{def.art.m.pl} small.peasant.\textsc{pl} of \textsc{poss.3pl} property\\
\glt `[…] and used to deprive this way the small peasants of their property.'
\z

\ea\label{}
\langinfo{Tuatschin}{Camischolas}{\citealt[88]{Büchi1966}}\\
\gll    Co òn \textbf{séis} \textbf{schnuzs} ǝntschiet a bargèe […].\\
     here have.\textsc{prs.3pl} \textsc{poss.det.3pl.m} moustache.\textsc{pl} begin.\textsc{ptcp} to burn.\textsc{inf}\\
\glt `At this moment their moustaches began to burn […].'
\z

\ea\label{}
\langinfo{Tuatschin}{Camischolas}{\citealt[90]{Büchli1966}}\\
\gll    El vèeva numnadamain mez tratsch dǝ siu iert ain seis cazès […].\\
\textsc{3sg} have.\textsc{impf.3sg} namely put.\textsc{ptcp} soil of \textsc{poss.3sg} garden \textsc{loc} \textsc{poss.3sg} shoe.\textsc{pl}\\
\glt `He had namely put some soil of his garden into his shoes […].'
\z



Normally, the possessive determiners precede the noun, but in proverbs, sayings and so on, they may follow it.

\ea\label{}
\langinfo{Tuatschin}{}{\citealt[100]{Büchli1966}}\\
\gll Cůsch bucca tia, schǝ cuschan bucca tůttas.\\
     keep.quiet.\textsc{imp} mouth \textsc{poss.2sg} then keep.quiet.\textsc{prs.3pl} mouth all.\textsc{f.pl}\\
\glt `Keep quiet, then everybody will keep quiet.'
\z


\subsection{Indefinites}
\ea\label{ex:}
\langinfo{Tuatschin}{}{\citealt[69]{Berther2007}}\\
\gll  Avaun tschien onns pliravan nos buns vegls da pudai fugir gl’ unviern naven da Selva ed ussa, \textbf{tut} \textbf{tge} vut sta luora […].  \\
     ago hundred year.\textsc{m.pl} complain.\textsc{impf.3pl} \textsc{poss.1pl.m} good.\textsc{pl} old.\textsc{pl} \textsc{comp} be.able.\textsc{inf} escape \textsc{def.art.m.sg} winter from of \textsc{tpn} and now all \textsc{rel} want.\textsc{prs.3sg} stay there.out \\
\glt `One hundred years ago our good old [people] used to complain because they wanted to escape from winter away from Selva, and now everybody wants to stay there.'
\z

\ea\label{}
\langinfo{Tuatschin}{}{\DRG{2}{491}}\\
\gll  Os è \textbf{tùt} \textbf{tga} vò cun braintas da stors. \\
     now \textsc{cop.prs.3sg} all \textsc{rel} go.\textsc{prs.3sg} with basket.\textsc{pl} of sheet.metal\\
\glt `Now everybody goes with a basket made of sheet metal.'
\z

\ea\label{}
\langinfo{Medelin}{Curaglia}{\DRG{3}{585}}\\
\gll Caschnès ancuntar clavau \textbf{mintga} \textbf{pur} \textbf{tg}'a in gron ni in pin.\\
     hayrack.\textsc{m.pl} against barn every peasant \textsc{rel}=have.\textsc{prs.3sg} \textsc{indef.art.m.sg} big or \textsc{def.art.m.sg} small\\
\glt `Hayracks put against the wall of a barn every peasant has a big or a small one.'
\z


\textit{Adatgi} as a dative indefinite pronoun is found in the following example. %obsolete?

\ea\label{}
\langinfo{Tuataschin}{}{\DRG{2}{154}}\\
\gll  \textbf{Adatgi} plai barba, \textbf{adatgi} barbis, \textbf{adatgi} giutta, \textbf{adatgi} ris. \\
    \textsc{dat}.some please.\textsc{prs.3sg} beard,  \textsc{dat}.some moustache \textsc{dat}.some pearl.barley \textsc{dat}.some rice \\
\glt `Some like beards, some moustaches, some pearl barley, some rice.'
\z


\ea\label{}
\langinfo{Tuatschin}{Camischolas}{\DRG{3}{580}}\\
\gll   Tut als posts vèvan \textbf{anzatgé} \textbf{in} crap sutain.\\
     all \textsc{def.art.m.pl} have.\textsc{prs.3pl} something \textsc{indef.art.m.sg} stone under.in\\
\glt `All hayrack pals had some stone under them.'
\z





\subsection{Quantifiers}

\subsubsection{Numerals}
Cardinal numbers are found in \tabref{tab:cardnum1}.


\begin{table}
\caption{Cardinal numerals (first part)} 
\label{tab:cardnum1}
 \begin{tabular}{llllll}
  \lsptoprule
  \midrule
&&0&nula\\
1&in&11&indisch&21&véntgín\\
2&dus&12&dódisch&22&véntgadús\\
3&trajs, traja&13&trèdisch&\\
4&quatar&14&quitòrdisch&\\
5&tschun&15&quindisch\\
6&sis&16&sédasch, sédisch\\
7&sjat&17&gisját\\
8&òtg&19& ṣchotg\\ %check whether schotsch
9&nóv&19& ṣchèniv\\
10&déjsch&20&végn\\
  \lspbottomrule
 \end{tabular}
\end{table}

\ea\label{}
\langinfo{Tujetsch}{}{\DRG{6}{728}}\\
\gll  \textbf{In} uffaun è pauc, \textbf{dus} è drètg, \textbf{trais} è strètg, \textbf{quátar} è fula, \textbf{tschun} è paluna a sbaluna.\\
    one child \textsc{cop.prs.3sg} little two \textsc{cop.prs.3sg} all.right three  \textsc{cop.prs.3sg} just.enough four  \textsc{cop.prs.3sg} crowd five  \textsc{cop.prs.3sg} pile and  collapse.\textsc{prs.3sg}\\
\glt `One child is little, two are all right, three are just enough, four are a crowd, five are a lot that collapses.'
\z


The cardinal numeral \textit{traj(a)} is used with collective nouns:%check dua and other combinations of traja

\ea\label{}
\langinfo{Tuatschin}{}{\citealt[28]{Gartner1910}}\\
\gll    ɛləs ɔn \textbf{traj} \textbf{pɛra} ʨɔmbəs […].\\
    \textsc{3pl.f} have.\textsc{prs.3pl} three.\textsc{coll} pair.\textsc{coll} leg.\textsc{pl}\\
\glt `They [the ants] have three pairs of legs.'
\z

The fractions occuring in the corpus are miɐʦ, mjaʦɐ ‘half’, ɐnti:r ‘whole’.

domasdus [dɔmɐs'duːs] (C 143), tomasdus [tɔmɐʃ'duːs] (C 169) num. tous les deux

Ordinal numbers have special forms from 1-4; from 5 onwards they take the suffix \textit{-aval / -avla}: \textit{parmè / parmèra} 'first', \textit{sécund / sécunda} 'second', \textit{tiarz / tjarza} 'third', \textit{quart / quarta} 'fourth', \textit{tschunaval / tschunavla} 'fifth', \textit{déiṣchaval / déiṣchavla} 'tenth', \textit{ véntgadusavel / véntgadusavla} 'twenty-second', and so on.

Instead of sécund, sécunda `second', one finds also zacund.

\ea\label{}
\langinfo{Tujetsch}{}{\DRG{6}{421}}\\
\gll  dá cun flugjals la \textbf{zécunda} jèda.\\
    give.\textsc{inf} with flail.\textsc{pl} \textsc{def.art.f.sg} second.\textsc{f.sg} time \\
\glt `beat with flails for the second time'
\z


\subsubsection{Other quantifiers}

\ea\label{}
\langinfo{Tuatschin}{}{\DRG{1}{546}}\\
\gll  Quai frust o dau uonn \textbf{aungataun} fain. \\
    \textsc{dem.m.sg}	meadow have.\textsc{prs.3sg} give.\textsc{ptcp} this.year as.much hay \\
\glt `This meadow has produced as much hay [as last year].'
\z

Instead of \textit{aungatáun} it is also possible to say \textit{aun ignjèda taun} `again once as.much'.

In rare cases, a quantifier may be nominalized, as \textit{bjè} `much, many' in example \REF{quantnom}.

\ea\label{quantnom}
\langinfo{Tuatschin}{}{\DRG{2}{386}}\\
\gll    \textbf{Al} \textbf{biè} fò bétga plain.\\
     \textsc{def.art.m.sg} much make.\textsc{prs.3sg} \textsc{neg} full\\
\glt `A big quantity does not fill the stomach.'
\z

\subsubsection{The construction \textit{tùt tga} and others}\footnote{See \citet{Linder1987}}

\ea\label{}
\langinfo{Medelin}{Curaglia}{\DRG{6}{319}}\\
\gll  Cu la mata fila stòpa, \textbf{tùt} \textbf{tga} vut tanéi sé ròca. \\
    when \textsc{def.art.f.sg} girl spin.\textsc{prs.3sg} tow all.\textsc{m.sg} \textsc{rel} want.\textsc{prs.3sg} hold.\textsc{inf} up distaff \\
\glt `When a girl spins tow, all [the boys] want to hold the distaff.'
\z


\section{The adjective}
\subsection{Forms of the adjective}
The adjective distinguishes two genders, masculine and feminine, and two numbers : singular and plural. Singular is unmarked, but plural gets the \textit{-s} suffix.

All adjectives display four different forms, but with two different distributions. The adjectives with no stem alternation have 
(1) one form for masculine attributive singular, 
(2) one form for masculine predicative singular as well as attributive and predicative plural, 
(3) feminine singular, and 
(4) feminine plural. 
An example is \textit{alv} 'white' (\tabref{tab:adj:nostemchange}).

\begin{table}
\caption{Adjectives without stem changes}
\label{tab:adj:nostemchange}
 \begin{tabular}{ll} % add l for every additional column or remove as necessary
  \lsptoprule
  \midrule
 in cùdisch \textit{alv} & `a white book'\\
Quai cùdisch è \textit{alvs}. & `This book is white.'\\
sis cùdischs \textit{alvs} & `six white books'\\
Quèls cùdischs èn \textit{alvs}. & `These books are white.'\\
ina flur \textit{alva} & `a white flower'\\
Quèla flur è \textit{alva}. & `This flower is white.'\\
sis flurs \textit{alvas} & `six white flowers'\\
Quèlas flurs èn \textit{alvas}. & `These flowers are white.'\\
  \lspbottomrule
 \end{tabular}
\end{table}

But adjectives that display stem alterations have (1) one form for masculine singular attributive and genderless, which is always singular, (2) one form for masculine singular and plural predicative, (3) one form for feminine singular attributive and predicative, and (4) one form for feminine plural attributive and predicative. An example is \textit{bi} 'beautiful' (see \tabref{tab:adj:stemalternations}).

\begin{table}\
\caption{Adjectives with stem alternations}
\label{tab:adj:stemalternations}
 \begin{tabular}{ll} % add l for every additional column or remove as necessary
  \lsptoprule
   \midrule
  in \textit{bi} dé & `a beautiful day'\\
Mazá è bégia \textit{bi}. &`To kill is not nice.'\\
Quai cùdisch è \textit{bials}. & `This book is beautiful.'\\
Quèls cùdischs èn \textit{bials}. & `These books are beautiful.'\\
Quai è ina \textit{biala} flur. & `This is a beautiful flower.'\\
Quèla flur è fétg \textit{biala}. &`This flower is very beautiful.'\\
\textit{bialas} flurs & `beautiful flowers'\\
Las flurs sén quai prau èn \textit{bialas}. &`The flowers on this meadow are beautiful.'\\
  \lspbottomrule
 \end{tabular}
\end{table}

The adjectives which show a change in the stem are shown in \tabref{tab:list:adj:stemalternations}.\footnote{Some adjectives with stem alternations which are listed in \citet{Spescha1989} are not used in Tuatschin. These are \textit{detschiert}, \textit{detscharta} `resolute', \textit{tanien}, \textit{tanienta} `such'; for \textit{stiert}, \textit{storta} and uiersch, uiarscha the adjective crùtsch, crùtscha is used; however, \textit{ina stòrta} `a bend' is used; as for \textit{ierfan}, \textit{orfna} `orphan', only the masculine form \textit{íarfan} is used as a noun for both genders. Furtherore, \textit{tiest}, \textit{tosta} `dried' is only used in \textit{majla tòsta} `dried apples' or \textit{pèra tòsta} `dried pears'.}



\begin{table}
\caption{List of adjectives with stem alternations}
\label{tab:list:adj:stemalternations}
 \begin{tabular}{ll} % add l for every additional column or remove as necessary
  \lsptoprule
  \midrule
 agjan, agjans, atgna, atgnas &`own'\\
bi, bials, biala, bialas &`beautiful'\\
bian, buns, buna, bunas &  `good'\\
cavíartg, cavòrtgs, cavòrtga, cavòrtgas& `hollow'\\
grías, gròs, gròssa, gròssas &`big'\\
íastar, jastars, jastra, jastras& `foreign'\\
matgíart, macôrts, macòrta, macòrtas &`ugly'\\
míart, mòrts, mòrta, mòrtas &`dead'\\
míaz, mjasa & `half'\\
níabal, nòbels, nòbla, nòblas & `noble'\\
níaf, nófs, nóva, nóvas &`new'\\ % niaf ni niav?
pin (pign), pins, pintga, pintgas &`small, little’\\
schlíat, schljats, schljata, schljatas& `bad\\ 
sògn, sògns, sòntga, sòntgas &`holy'\\
tgétschen, còtgens, còtgna, còtgnas& `red'\\
tíarz, tjarzs, tjarza, tjarzas &`third'\\
taun, taunta & `so many'\\
tschíac, tschocs, tschòca, tschòcas &`blind'\\
  \lspbottomrule
 \end{tabular}
\end{table}


\ea\label{ex:1:adj}
\langinfo{Camischolas}{Tuatschin}{\citealt[82]{Büchli1966}}\\
\gll lu o’ ‘llas \textbf{anflau} el \textbf{morts} spel badugn giů.\\
     then have.{\prs}.3{\pl} 3\textsc{f}.{\pl} find.{\ptcp} 3\textsc{m}.{\sg}  dead.\textsc{m}.{\sg}  next.\textsc{def}.{\art}.\textsc{m}.{\sg} birch down\\
\glt '[…] then they found him dead next to the birch.'
\z

\ea\label{}
\langinfo{Tuatschin}{Cavorgia}{\citealt[119]{Büchli1966}}\\
\gll    Ella schèeva \textbf{crescher} l’ jarva schi \textbf{biala} […].\\
    \textsc{3sg.f} let.\textsc{impf.3sg} grow.\textsc{inf} \textsc{def.art.f.sg} grass so beautiful.\textsc{f.sg} \\
\glt `She used to let the grass grow so beautiful […].'
\z
 


\subsection{Degrees of comparison}
Linder (1987: 233-50) shows that for some adjectives  the superlative is expressed in the same way as the positive in all Romansh written varieties. This also holds for Tuatschin.

\ea\label{}
\langinfo{Tuatschin}{Cavorgia}{\citealt[120]{Büchli1966}}\\
\gll    Lu ṣai vegnu giů la lavina […] ǝd ò mez sůt la gronda part d’l vitg.\\
     then \textsc{cop.prs.3sg} come.\textsc{ptcp} down \textsc{def.art.f.sg} avalanche […] and have.\textsc{prs.3sg} put.\textsc{ptcp} under \textsc{def.art.f.sg} big part of=\textsc{def.art.m.sg} village\\
\glt `Then the avalanche came down […] and buried the biggest part of the village.'
\z

\ea\label{}
\langinfo{Tuatschin}{}{\citealt[61]{Gartner1910}}\\
\gll    […] al ˈʥuvǝn ajn ˈʃʨafa da l ˈura, kwɛl ɔ ɛl beʨ ǝmflaw \\
  [...] \textsc{def.art.m.sg} young in box of \textsc{def.art.f.sg} clock \textsc{dem.m.sg} have.\textsc{prs.3sg} \textsc{3sg} \textsc{neg} find.\textsc{ptcp}\\
\glt `[…] the youngest [goat] in the box of the clock, this he didn’t find.'
\z



\subsection{Modifiers of the adjectives}

\subsection{Adjectives as adverbs}
A special case is bien `good', which is sometimes mistaken for bégn `well'.

\ea\label{}
\langinfo{Tuatschin}{}{\DRG{2}{624}}\\
\gll  In vjántar gròn maglia \textbf{bien}. \\
     \textsc{def.art.m.sg} stomach big eat.\textsc{prs.3sg} good\\
\glt `A big belly eats well.'
\z



\subsection{Absence of agreement}
If the predicative adjective is left-dislocated in order to topicalize it, it does not agree with the subject.

\ea\label{}
\langinfo{Tuatschin}{Sadrún}{m5, 1, sjantar 72}\\
\gll    parquáj ṣè \textbf{impurtònt} lura \textbf{la} \textbf{gramática} tga té fas […]. \\
     therefore \textsc{cop.prs.3sg} important then \textsc{def.art.f.sg} grammar \textsc{rel} \textsc{2sg} make.\textsc{prs.2sg}\\
\glt `[…] therefore the grammar you write is important […].'
\z

This is also the case with the causative verb \textit{schè} 'let'. The causee, be it a noun or a pronoun, always follows the adjective.

\ea\label{}
\langinfo{Tuatschin}{Sadrún}{m4}\\
\gll Ju a schau \textbf{avíart} agl ésch.   \\
 \textsc{1sg} have.\textsc{prs.1sg} leave.\textsc{ptcp} open.\textsc{gl} \textsc{def.art.m.sg} door\\
\glt `I left the door open.'
\z

\ea\label{}
\langinfo{Tuatschin}{Ruèras}{\citealt[64]{Büchli1966}}\\
\gll    ǝl pur ò […] schau \textbf{liber} l’ uelp.\\
     \textsc{def.art.m.sg} peasant have.\textsc{prs.3sg} […] leave.\textsc{ptcp} free.\textsc{m.sg} \textsc{def.art.f.sg} fox\\
\glt `The peasant let […] the fox free.'
\z

\ea\label{}
\langinfo{Tuatschin}{Sadrún}{\citealt[105]{Büchli1966}}\\
\gll   […] quella èera gelada […] tg' èera numpussaivel li el də \textbf{fâ} \textbf{liber} ella […].\\
      […] \textsc{dem.f.sg} \textsc{cop.impf.3sg} freeze.\textsc{ptcp.3sg.f} […] \textsc{comp} \textsc{cop.impf.3sg} impossible \textsc{dat} \textsc{3sg.m} to make free \textsc{3sg.f}\\
\glt `[…] that [block] was frozen […] so that it was impossible for him to get it free […].'
\z

If a pluralized noun is non-referential, the verb does not agree in number with the subject, i.e. it occurs in its singular form, and the genderless form of the adjective is used.

\ea\label{}
\langinfo{Tuatschin}{Sadrún}{m4}\\
\gll Nùrsas \textbf{è} \textbf{sgarschajval}.   \\
 sheep.\textsc{f.pl} \textsc{cop.prs.3sg} horrible.\textsc{gl} \\
\glt `Sheep are horrible.'
\z

\ea\label{}
\langinfo{Tuatschin}{Sadrún}{m4}\\
\gll Vacanzas \textbf{è} \textbf{bi}.   \\
holiday.\textsc{pl} \textsc{cop.prs.3sg} nice.\textsc{gl}   \\
\glt `Holidays are nice.'
\z


\ea\label{}
\langinfo{Tuatschin}{Sadrún}{\citealt[106]{Büchli1966}}\\
\gll    Pǝrquai sǝtila ‘l òo \textbf{bluts} […].\\
     therefore \textsc{refl}.pull.\textsc{prs.3sg} \textsc{3sg} out naked.\textsc{m.sg}\\
\glt `Therefore he took off all his clothes […].'
\z

\ea\label{}
\langinfo{Tuatschin}{Surain}{\citealt[129]{Büchli1966}}\\
\gll    […] schagliůc vegn ju manizzaus əmpaglia schi \textbf{manéedels} […].\\
     [..] otherwise \textsc{pass.aux.prs.1sg} \textsc{sg} chop.\textsc{ptcp.m.sg} damaged so fine.\textsc{m.sg}\\
\glt `[…] otherwise I get completely chopped into such fine pieces […].'
\z
 
 \ea\label{ex:}
\langinfo{Tuatschin}{}{\citealt[31]{Berther2007}}\\
\gll  Mia melna tscho o unflau in tec \textbf{ina} \textbf{tgomba} \textbf{davus}.  \\
     \textsc{poss.f.1sg} yellow.\textsc{f} here have.\textsc{prs.3sg} swell.\textsc{ptcp}  \textsc{indef.art.m.sg} bit \textsc{indef.art.f.sg} leg back\\
\glt `My yellow [cow] has a swollen back leg.'
\z


Al quinau vèva mal ina tgòmba. (ca. 588)

\ea\label{}
\langinfo{Tuatschin}{Cavòrgja}{f1}\\
\gll Nus stgavan ir \textbf{líbar} tùt \textbf{als} \textbf{pòrs}.   \\
 \textsc{1pl}  be.allowed.\textsc{impf.1pl}  go.\textsc{inf} free all  \textsc{def.art.m.pl} pig.\textsc{pl} \\
\glt `We were allowed to let all the pigs go around freely.'
\z




\section{Noun phrases and prepositional phrases modifying a noun}

\ea\label{}
\langinfo{Tuatschin}{}{\DRG{3}{253}}\\
\gll  Sco’ls \textbf{dis} \textbf{tgaun} von ain, voni ora.  \\
     as=\textsc{def.art.m.pl} day.\textsc{pl} dog go.\textsc{prs.3pl} in go.\textsc{prs.3pl.comm} out\\
\glt `As the dog days come, they also go. '
\z


Some nouns, which express a high degree of a certain quality, are used like qualifying adjectives.

\ea\label{}
\langinfo{Tuatschin}{}{\DRG{2}{639}}\\
\gll  Quai è \textbf{buantat} vaca da latg. \\
     \textsc{dem} \textsc{cop.prs.3sg} good.quality cow of milk\\
\glt `This is an excellent milk cow.'
\z

\ea\label{}
\langinfo{Tuatschin}{}{\DRG{3}{204}}\\
\gll  calira \textbf{da} \textbf{barsar} \textbf{vivs} \\
     heat \textsc{comp} burn alive.\textsc{m.sg.pred}\\
\glt `terrible heat'
\z


\ea\label{}
\langinfo{Tuatschin}{}{\DRG{2}{723-724}}\\
\gll   butéglia \textbf{da} \textbf{scadá} \textbf{pais} \\
    bottle \textsc{comp} warm.\textsc{inf} foot.\textsc{pl} \\
\glt `hot-water bottle'
\z


\section{Bare noun phrases}

\section{Personal pronouns}
\tabref{tab:perspron} shows the paradigm of the personal pronouns which is closest to the corresponding Standard Sursilvan forms.

\begin{table}
\caption{Personal pronouns}
\label{tab:perspron}
 \begin{tabular}{llll}
  \lsptoprule
  subject & direct object & indirect object & after preposition\\
  \midrule
ju & mè & da mé & cun mè \\
té & té & da té & cun té\\
èl & èl	& dad èl & cun èl \\
èla & èla & dad èla & cun èla\\
nus & nus & da nus & cun nus\\
vus & vus & da vus & cun vus\\
èls & èls & dad èls & cun èls\\
èlas & èlas & dad èlas & cun èlas\\
 \lspbottomrule
 \end{tabular}
\end{table}


Honorific pronouns:

In the first part of the 20th century, the honorific pronoun for priests was \textit{Els}, which corresponds to the third person plural masculine form. The verb agrees in number with this pronoun.



mé vs. mè:

Tarcisi: mè, da mè té, da té; sén té, cun mè, cun té.


\ea\label{}
\langinfo{Tuataschin}{}{\citealt[85]{Gadola1935}}\\
\gll In auter onn scheis vus era ir ad alp \textbf{mè}.\\
 \textsc{indef.art.m.sg} other year let.\textsc{prs.2pl} \textsc{2pl} also go.\textsc{inf} to alp \textsc{1sg}\\
\glt `Another year you’ll let me too go to … .'
\z

té vs. tè:

\ea\label{}
\langinfo{Tuatschin}{}{\citealt[91]{Gadola1935}}\\
\gll    Té mu trafica usché vinavaun, epi sietta il militer giu \textbf{tè} in dé.\\
  \textsc{2sg} only do.\textsc{imper.2sg} so further and.then shoot.\textsc{prs.3sg} \textsc{def.art.m.sg} army down \textsc{2sg} one.\textsc{m.sg} day\\
\glt `Just go on doing these things, and the army will shoot you  down one day.'
\z

syntactic function of \textit{ins}

Direct object:
\ea\label{}
\langinfo{Tuatschin}{Selva}{\citealt[46]{Büchli1966}}\\
\gll  […] tg’ ǝl giavel segi adina spèeras laghegiaus pǝr cudizzã \textit{ins} […].\\ %check cudizzã
     [...]  \textsc{comp} \textsc{def.art.m.sg} devil \textsc{cop.prs.sbj.3sg} always next lie.in.wait.\textsc{ptpc} \textsc{purp} tease.\textsc{inf} \textsc{gnr}\\
\glt `[…] that the devil is always next [to us], lying in wait in order to tease us.'
\z

Uses of i/ai.

\ea\label{}
\langinfo{Tuatschin}{}{\DRG{3}{639}}\\
\gll  Schi òn bétg als còrns, schi mòrdan \textbf{ai}.  \\
     if have.\textsc{prs.3sg} \textsc{neg} \textsc{def.art.m.pl} horn.\textsc{pl} \textsc{corr} bite.\textsc{prs.3pl} \textsc{3pl.sbj}\\
\glt `If they [the goats] don’t have horns, they bite.'
\z


\ea\label{}
\langinfo{Tuatschin}{Bugnai}{\citealt[136]{Büchli1966}}\\
\gll Lu ṣen \textbf{ai} î sen claustra […].\\
     then \textsc{cop.prs.3pl} \textsc{3pl.sbj}  go.\textsc{ptcp.m.3pl} on monastery […]\\
\glt `Then they went up to the monastery […].'
\z

\ea\label{}
\langinfo{Tuatschin}{Ruèras}{\citealt[67]{Büchli1966}}\\
\gll  […] nos tiers. \textbf{I} mavan adina giů ǝncůnter las  plauncas  dǝ l’ Ondadusa […].\\
     […] \textsc{poss.1pl} animal.\textsc{pl} \textsc{3pl.sbj} go.\textsc{impf.3pl} always down towards \textsc{def.art.f.pl} slope.\textsc{pl} of \textsc{def.art.f.sg} \textsc{pln}\\
\glt `[…] our animals. They would always to down towards the slopes of the Ondadusa.'
\z

The pronoun \textit{ai} is also used for 3sg object or after preposition.

\ea\label{}
\langinfo{Medelin}{}{\DRG{5}{566}}\\
\gll Il vadí è cò giu, va giu par \textbf{ai}.\\
\textsc{def.art.m.sg} calf.\textsc{m.sg} \textsc{cop.prs.3sg} here down go.\textsc{imper.2sg} down for \textsc{3sg}\\
\glt `The calf if down here, go down for it.'
\z


dative vs. accusative: Es kann sich in diesen beiden Tälern (Tuj., Meldel) nicht um den Ersatz des Dativs durch den Akkusativ me, te handeln, da der Dativ me klar von Akkusativ mE unterschieden wird. (Widmer 1959: 118)

The current dative marker for all persons is \textit{da/dad}, but until approximatively the 1960, \textit{da/dad} was used for the first and second persons, singular and plural, and for the third persons \textit{di} or \textit{li} was used in both Tuatschin regions. Note that, in contrast to the correspondent dative article, the pronoun marker does not differentiate number (see below xxx). Examples x-y illustrate the current usage, and examples z-a the earlier usage.

\ea\label{}
\langinfo{Tuatschin}{Surain}{\citealt[128f.]{Büchli1966}}\\
\gll    Mů quellas òn scumandau \textbf{da} \textbf{me}, dǝ dî òora, tgi ellas segien […].\\
     but \textsc{demf.pl} have.\textsc{prs.3pl} forbid.\textsc{ptcp} \textsc{dat} \textsc{1sg} \textsc{comp} say.\textsc{inf} out who \textsc{3pl} \textsc{cop.prs.sbjv.3pl} \\
\glt `But these [girls] forbade me to tell who they were […].'
\z

\ea\label{}
\langinfo{Tuatschin}{}{\DRG{3}{499}}\\
\gll   \textbf{Da} \textbf{té} dess ins rúmpar ain la cavaza.\\
     \textsc{dat} \textsc{2sg} have.to.\textsc{cond.3sg} \textsc{gnr} break in \textsc{def.art.f.sg} skull\\
\glt `One should break your head.'
\z


\ea\label{}
\langinfo{Tuatschin}{Tschamut} {\citealt[14]{Büchli1966}}\\
\gll  ǝd el ò encůnuschiu siu cuntî ǝ detg quai \textbf{li} \textbf{ella}. \\
     and 3\textsc{sg}.\textsc{m}  have.\textsc{prs}.3\textsc{pl}  know.\textsc{ptcp}  \textsc{poss}.3\textsc{sg}  knife and say.\textsc{ptcp}  \textsc{dem}  \textsc{dat}  3\textsc{sg}.\textsc{f}\\
\glt `And he recognized his knife and told her that.'
\z

 
\ea\label{}
\langinfo{Tuatschin}{Tschamut} {\citealt[14]{Büchli1966}}\\
\gll  ǝd ella ò piau \textbf{li} \textbf{el} tůt-s tier-s per engraziâ \textbf{li} \textbf{el}.\\
     and 3\textsc{sg}.\textsc{f} have.\textsc{prs}.3\textsc{sg} pay.\textsc{ptcp} \textsc{dat} 3\textsc{sg}.\textsc{m} all-\textsc{m}.\textsc{pl} animal-\textsc{pl} \textsc{purp} thank.\textsc{inf} \textsc{dat} 3\textsc{sg}.\textsc{m}\\
\glt `And she paid him all the animals in order to thank him.'
\z

\ea\label{}
\langinfo{Tuatschin}{Bugnei} {\citealt[53]{Büchli1966}}\\
\gll  əd ò detg tg' əl vegli esser gidaivels \textbf{li} \textbf{el} […].\\
     and have.\textsc{prs}.3\textsc{sg} say.\textsc{ptcp} \textsc{comp} 3\textsc{sg}.\textsc{m} want.\textsc{prs}.\textsc{sbjv}.3\textsc{sg} be.\textsc{inf} helpful.\textsc{m}.\textsc{sg} \textsc{dat} \textsc{3sg}.\textsc{m}\\
\glt `[…] and said that he wanted to help him […].'
\z

\ea\label{}
\langinfo{Tuatschin}{Bugnai}{\citealt[145]{Büchli1966}}\\
\gll    […] ǝ lu ṣai vegniu ǝndǝmain \textbf{ad} \textbf{el} quella Nossadunna sell’ alp, […].\\
     […]  and then \textsc{cop.prs.3sg.expl} come.\textsc{ptcp} mind {dat} \textsc{3sg} \textsc{dem.f.sg} Virgin on.\textsc{def.art.f.sg} alp \\
\glt `[…] and then that holy Virgin on the alp came into his mind.'
\z


\ea\label{}
\langinfo{Tuatschin}{Ruèras}{\citealt[65]{Büchli1966}}\\
\gll  Ju əngrazia pərsianter ə dun \textbf{də} \textbf{Vus} əl cuntî pər in survetsch begn fatg.\\
    \textsc{1sg} thank.\textsc{prs.1sg} therefore and give.\textsc{prs.1sg} \textsc{dat} \textsc{2sg.hon} \textsc{def.art.m.sg} knife for \textsc{indef.art.m.sg} service well do.\textsc{ptcp}\\
\glt `I thank [you] for it and give you the knife for a well done service.'
\z

 \ea\label{}
\langinfo{Tuatschin}{Selva} {\citealt[53]{Büchli1966}}\\
\gll  La sèera ò in pur dau suttetg \textbf{li} \textbf{ellas} […].\\
     \textsc{def}.\textsc{art}.\textsc{f}.\textsc{sg} evening have.\textsc{prs}.3\textsc{sg} \textsc{indef}.\textsc{art}.\textsc{m}.\textsc{sg} peasant give.\textsc{ptcp} accommodation \textsc{dat} 3\textsc{pl}.\textsc{f}\\
\glt `In the evening a peasant offered them accommodation […].'
\z
 

These forms of the dative marker are found neither in the \textit{Dicziunari Rumantsch Grischun} nor in \citet{Widmer1959}, nor in my own corpus.

In the examples offered in \citet{Gartner1910}, \textit{da/dad} is usually used for first and second persons, but see example \REF{gart1}, where da is used with \textit{mé} and a with \textit{té}. With third persons, only cases with \textit{a/ad} occur, which corresponds to the standard Sursilvan way of dative marking.

\ea\label{gart1}
\langinfo{Tuatschin}{}{\citealt[96]{Gartner1910}}\\
\gll abər ɛn'ʃtaʎ rɔju tɛ ʨa te musjɛs \textbf{da} \textbf{me} kɔ te su'pɔrtəs kwɛləs ʨ in ɔ faʨ \textbf{a} \textbf{te} […].\\
     but instead ask.\textsc{prs.1sg.1sg} \textsc{2sg} \textsc{comp} \textsc{2sg} show.\textsc{prs.sbjv.2sg} \textsc{dat} \textsc{1sg} how \textsc{2sg} stand.\textsc{prs.2sg} \textsc{dem.f.pl} \textsc{rel} \textsc{gnr} have.\textsc{prs.3sg} do.\textsc{ptcp} \textsc{dat} \textsc{2sg}\\
\glt `[…] but instead I ask you to show me how you stand this [offence] they did to you […].'
\z
 
 \ea\label{}
\langinfo{Tuatschin}{}{\citealt[81]{Gartner1910}}\\
\gll  […] ʨ ɛls ve'ɲɛsən a fa enʦa'ʨe da l mal da te.\\
     […]  \textsc{comp} \textsc{3pl} come.\textsc{cond.3pl} to make.\textsc{inf} something of \textsc{def.art.m.sg} bad \textsc{dat} \textsc{2sg}\\
\glt `[…] that they would end up doing some harm to you.'
\z

\ea\label{}
\langinfo{Tuatschin}{}{\citealt[22]{Gartner1910}}\\
\gll 'grɔma a pa'ʒɛda mawnk aj mai \textbf{da} \textbf{nus}.\\
     cream and butter lack.\textsc{prs.3sg} \textsc{expl} never \textsc{dat} \textsc{1pl}\\
\glt `We never lack cream and butter.'
\z


\ea\label{}
\langinfo{Tuatschin}{}{\citealt[18]{Gartner1910}}\\
\gll 'awda kwaj ʥad griʃ \textbf{da} \textbf{vus}?\\
     belong.\textsc{prs.3sg} \textsc{dem.m.sg} cat grey \textsc{dat} \textsc{2pl}\\
\glt `Does this grey cat belong to you?'
\z


\ea\label{}
\langinfo{Tuatschin}{}{\citealt[33]{Gartner1910}}\\
\gll le'ʥi kwɛl a daj glajti anavɔs ɛl \textbf{ad} \textbf{ɛl}.\\
     read.\textsc{imp.2pl} \textsc{dem.m.sg} and give.\textsc{imp.2pl} soon back \textsc{3sg} \textsc{dat} \textsc{3sg}\\
\glt `Read it and give it soon back to him.'
\z

 
\ea\label{}
\langinfo{Tuatschin}{}{\citealt[86]{Gartner1910}}\\
\gll ad ɛl ɔ partiw \textbf{ad} \textbf{ɛlts} la rawba.\\
     and \textsc{3sg.m} have.\textsc{prs.3sg} distribute.\textsc{ptcp} \textsc{dat} \textsc{3pl.m} \textsc{def.art.f.sg} fortune\\
\glt `And he distributed his fortune among them.'
\z

Gadola:

\ea\label{}
\langinfo{Tuatschin}{}{\citealt[90]{Gadola1935}}\\
\gll  A \textbf{da} \textbf{té} duess ins dar vin anzanu’ auter, té ragner! \\
     and \textsc{dat} \textsc{2sg} have.to.\textsc{cond.3sg} \textsc{gnr} give.\textsc{inf} wine somewhere else \textsc{2sg} miser \\
\glt `And to you one should give wine somewhere else, you miser!'
\z

The only occurrence of the form \textit{ada} of the dative article is is to be found in the DRG; however, there it is only given in parenthesis as an alternative to \textit{da}.

\ea\label{}
\langinfo{Tuatschin}{}{\DRG{3}{462}}\\
\gll  Da ina castrada \textbf{(a)da} quella rusna.\\
     give.\textsc{imp.2sg} \textsc{def.art.f.sg} act.of.joining \textsc{dat} \textsc{dem.f.sg} hole\\
\glt `Tie together this hole with a cord. '
\z


Syntax:

\ea\label{ex:}
\langinfo{Tuatschin}{}{\citealt[85]{Gadola1935}}\\
\gll  In auter onn scheis vus era ir ad alp \textbf{mè}. \\
     \textsc{indef.art.m.sg} other year let.\textsc{prs.2pl} \textsc{2pl} also go.\textsc{inf} to alp \textsc{1sg}\\
\glt `Another year you will also let me go to alp.'
\z


The data for Medelin in the DRG are relatively scarse.

\ea\label{}
\langinfo{Medelin}{Curaglia}{\citealt[118]{Widmer1959}}\\
\gll  lɐ buéba pɔrtɐ da \textbf{te} ɐls kəcɛwls\\
    \textsc{def.art.f.sg} girl bring.\textsc{prs.3sg} \textsc{dat} \textsc{2sg} \textsc{def.art.m.pl} sock.\textsc{pl}\\
\glt `The girl brings you the socks.'
\z

\ea\label{}
\langinfo{Medelin}{Platta}{\citealt[118]{Widmer1959}}\\
\gll   yeu dun \textbf{dɐ} \textbf{té} lɐ tolɐ də kɐfé \\
     \textsc{1sg} give.\textsc{prs.1sg} \textsc{dat} \textsc{2sg} \textsc{def.art.f.sg}  pot of coffee\\
\glt `I give you the coffeepot.'
\z


\ea\label{}
\langinfo{Medelin}{}{\DRG{5}{255}}\\
\gll    Ṣchai \textbf{dad} \textbf{el}!\\
    tell.\textsc{imp.2pl} \textsc{dat} \textsc{3sg}\\
\glt `Tell him!'
\z

In the three texts published in Widmer (1962), only the standard Sursilvan dative marker \textit{a} occurs.

\ea\label{}
\langinfo{Medelin}{}{\citealt[101]{Widmer1962}}\\
\gll Suenter tgins a tratg o magnuc di ins \textbf{al} \textbf{latg} schireun, quel e seuns. \\
    after \textsc{comp=gnr} have.\textsc{prs.3sg} pull out cheese say.\textsc{prs.3sg} \textsc{gnr} \textsc{dat} milk  whey \textsc{dem.m.sg} \textsc{cop.prs.3sg} healhy.\textsc{m.sg} \\
\glt `After having taken out cheese, the milk is called `whey'.'
\z


\ea\label{}
\langinfo{Medelin}{}{\citealt[100]{Widmer1962}}\\
\gll  “Uss onda Barla vessan nus bugen, sche vus raquintassas ignada \textbf{a} \textbf{nus} sur il chischa, quei lessen nus bugen saveir.” “O quei e fetg sempel, que vi jeu schon dir \textbf{a} \textbf{vus}.\\
     now aunt \textsc{pn} have.\textsc{cond.1pl} \textsc{1pl} with.pleasure if \textsc{2sg.hon} tell.\textsc{cond.2pl} once \textsc{dat} \textsc{1pl} over \textsc{def.art.m.sg} make.cheese.\textsc{inf} \textsc{dem} want.\textsc{cond.1pl} \textsc{1pl} with.pleasure know.\textsc{inf} oh \textsc{dem} \textsc{cop.prs.3sg} very simple \textsc{dem} want.\textsc{prs.1sg} \textsc{1sg} certainly say.\textsc{inf} \textsc{dat} \textsc{2pl} \\
\glt ` “Now aunt Barla, we would very much like if you could tell us once about making cheese, this we would like to know.” “Oh, this is very simple, this I will certainly tell you.” '
\z

\ea\label{}
\langinfo{Medelin}{}{\citealt[103]{Widmer1962}}\\
\gll   Quella dun jeu \textbf{als} \textbf{pors}.\\
    \textsc{dem.f.sg} give.\textsc{prs.1sg} \textsc{1sg} \textsc{dat.def.art.m.pl} pig.\textsc{pl}\\
\glt `This one I give the pigs.'
\z



\section{Expletive pronoun} 
 
\ea\label{}
\langinfo{Tuatschin}{}{PaBä}\\ 
\gll Cò ò \textbf{i} bétg óf-s.\\
    here \textsc{exist}.\textsc{prs}.3\textsc{sg} \textsc{expl}  \textsc{neg} egg-\textsc{pl}\\
\glt 'There are no eggs here.'
\z

\ea\label{}
\langinfo{Tuatschin}{Bugnai}{\citealt[145]{Büchli1966}}\\
\gll […] \textbf{i} vegn unviern ǝ vegn fraid. \\
     […] \textsc{expl} come.\textsc{prs.3sg} winter and come.\textsc{prs.3sg} cold \\
\glt `Winter is coming and it is getting cold.'
\z

\ea\label{}
\langinfo{Tuatschin}{}{\DRG{2}{215}}\\
\gll \textbf{I} brischa la cazzeta, \textbf{i} vegn tga suffla. \\
   \textsc{expl} burn.\textsc{prs.3sg} \textsc{def.art.f.sg} pot \textsc{expl} \textsc{fut.aux.prs.3sg} \textsc{rel} blow.\textsc{prs.3sg} \\
\glt `[The soot in] the pot is burning, it is getting stormy.'
\z



\section{Relative clauses}

subject relative clauses

\ea\label{}
\langinfo{Medelin}{Platta}{\DRG{4}{389}}\\
\gll Nus vein in lungatg \textbf{tga} campogna ampau culla taliana.\\
     \textsc{1pl} have.\textsc{prs.1pl} \textsc{indef.art.m.sg} language \textsc{rel} accompany.\textsc{prs.3sg} little with.\textsc{def.art.f.sg} italian\\
\glt `We have a language that is a little bit similar to Italian [e.g., the Italian dialect of Val Blenio].'
\z

If the antecedent is \textit{tgi} `who' or \textit{tgé} `what', the relative pronoun is realized \textit{ca}.

\ea\label{}
\langinfo{Tuatschin}{}{\DRG{6}{449}}\\
\gll   \textbf{Tgi} {ch}’è da stròm dai sapartgirá dal fjuc.\\
     who \textsc{rel=cop.prs.3sg} of straw must.\textsc{prs.3sg} \textsc{refl}.be.on.one’s.guard of=\textsc{def.art.m.sg} fire\\
\glt `A person who is made of straw should be on her guard against fire.'
\z


locative: absence of subject in the rel clause

\ea\label{}
\langinfo{Tuatschin}{Sedrun}{TarHen, 1. 216}\\
\gll    Al sulèt intarèssant è l’ ampréma sacùnda classa \textbf{nòca} \textbf{tga} fòn la midada tíal sursilvan […].\\
     \textsc{def.art.m.sg} only interesting \textsc{cop.prs.3sg} \textsc{def.art.f.sg} first second form where \textsc{rel} make.\textsc{prs.3pl} \textsc{def.art.f.sg} change towards.\textsc{def.art.m.sg} Sursilvan\\
\glt `The only interesting thing is the first [and] second form where they start switching  towards Sursilvan.'
\z
   

Instrumental


\ea\label{}
\langinfo{Tuatschin}{Rueras} {\citealt[8f.]{Valär2013b}}\\
\gll    […] cʊ‘ʎɛːdɐ ʨ ins lɐ‘vajɐ nʊǝt ɐlz djants dɐ rʊj. \\
     […] clotted.milk \textsc{rel} \textsc{gnr} damage.\textsc{prs.3sg} textsc{neg} \textsc{def.art.m.pl} tooth.\textsc{pl} \textsc{comp} gnaw.\textsc{inf} \\
\glt `[…] clotted milk with which you do not damage your teeth when you gnaw at it.'
\z

Locative

Locative relative clauses are formed with nʊkɐ ʨǝ, where nʊkɐ functions as the antecedent of the clause. nʊkɐ alredy contains a complementizer (kɐ) –  see standard Survilvan nua che ‘where that’, also realized nʊ kɐ – but in Tuatschin, kɐ forms an unanalyzable unit with nʊ.

\ea\label{}
\langinfo{}{}{\citealt[145]{Büchli1966}}\\
\gll   […] sell’ alp \textbf{nůcca} \textbf{tg}’ el èera staus dǝ buǝp […]. \\
     […]  on.\textsc{def.art.f.sg} alp where \textsc{rel} \textsc{3sg} \textsc{cop.impf.3sg} cop.\textsc{ptcp.3sg.m} as boy\\
\glt `[…] on the alp where he had been as a boy. '
\z

However, there are also examples where \textbf{nʊkɐ} is used without \textit{ʨə}.

\ea\label{}
\langinfo{Tuatschin}{Bugnai}{\citealt[134]{Büchli1966}}\\
\gll    […] la damaun èera la crappa giů, nocca la baselgia stat oz sen quai prau.\\
     […] \textsc{def.art.f.sg} morning \textsc{cop.impf.3sg} \textsc{def.art.m.sg} stone.\textsc{coll} down \textsc{rel.loc} \textsc{def.art.f.sg} church stay.\textsc{prs.3sg} today on \textsc{dem.m.sg} field\\
\glt `[…] in the morning the stones where down there, where the church is located nowadays.'
\z

Orthographically nʊkɐ may also be written in two words.

\ea\label{}
\langinfo{Tuatschin}{Bugnai}{\citealt[134]{Büchli1966}}\\
\gll    ə lu en i secussegliai sen vəschnaunca dɛ bagagèe ella giů lò, nu' ch' ella stat oz \\
     and then \textsc{cop.prs.3pl} \textsc{3pl} \textsc{refl}.discuss.\textsc{ptcp.m.pl} on municipality to build.\textsc{inf} \textsc{3sg} down there where \textsc{rel} \textsc{3sg} stay.\textsc{prs.3sg} today\\
\glt `And then they discussed [the problem] in the municipality [and decided] to build it [= the church] where it is located nowadays.'
\z

\section{Generic noun phrases}
Generic subject noun phrases are formed with the definite article singular or plural, as well as with the indefinite article singular or plural.

\ea\label{}
\langinfo{Tuatschin}{}{\DRG{3}{519}}\\
\gll  \textbf{Las} \textbf{tgauras} èn las sulètas fèmnas tga portan barba.\\
     \textsc{def.art.f.pl} goat.\textsc{pl} \textsc{cop.prs.3pl} \textsc{def.art.f.pl} only.\textsc{pl} woman.\textsc{pl} \textsc{rel} carry.\textsc{prs.3pl} beard\\
\glt `Goats are the only women that have a beard.'
\z

\ea\label{}
\langinfo{Tuatschin}{}{\DRG{3}{630}}\\
\gll \textbf{Las} \textbf{tgautgas} fon bétg al ùm. \\
  \textsc{def.art.f.pl} trousers.\textsc{pl} make.\textsc{prs.3pl} \textsc{neg} \textsc{def.art.m.sg} man \\
\glt `Trousers do not make a man.'
\z

\ea\label{}
\langinfo{Tuatschin}{}{\DRG{6}{437}}\\
\gll  \textbf{In} \textbf{bogn} da flucs til’ò als tissis.  \\
    \textsc{indef.art.m.sg} bath of dry.hay pull.\textsc{prs.3sg}=out \textsc{def.art.m.pl} poison.\textsc{pl} \\
\glt `A bath of dried hay flowers pulls out the poisons.'
\z

\ea\label{}
\langinfo{Tujersch}{}{\DRG{6}{432}}\\
\gll  \textbf{Ufauns} \textbf{ pigns} maglian la flur ad antardan la lavur. \\
     child.\textsc{m.pl} small.\textsc{pl} eat.\textsc{prs.3pl} \textsc{def.art.f.sg} flower and delay.\textsc{prs.3pl} \textsc{def.art.f.sg} work\\
\glt `Small children consume power and delay the work.'
\z



