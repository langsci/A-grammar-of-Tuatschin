\chapter{Noun phrase}

\section{The noun}

\subsection{Gender}
Tuatschin differentiates two genders, masculine and feminine, which are not restricted to natural gender, but natural gender and grammatical gender usually correspond.

Natural gender of humans and animates is either differentiated by two different words, or – exclusively with human nouns – the suffix \textit{-a} (sometimes with the infix -\textit{èss}-) is added to the masculine form (\tabref{tab:nouns:gendis}).

\begin{table}
\caption{Natural gender distinctions}
\label{tab:nouns:gendis}
\begin{tabular}{lllll}
 \lsptoprule
  \textit{bab} & `father' & vs & \textit{mùma} & `mother'\\
  \textit{bé\underline{á}di} & `grandson` & vs & \textit{bé\underline{á}dia} & `granddaughter'\footnote{\DRG{1}{60} notes the form \textit{beáditga}, which my consultants do not know.}\\
\textit{buéb}& `boy' & vs & \textit{buéba} &`girl'\\
\textit{frá} & `brother' & vs & \textit{sòra} & `sister'\\
\textit{tat} & `grandfather' & vs & \textit{tata} & `grandmother'\\
\textit{mastral} & `senior official' & vs & \textit{mastarlèssa} & `wife of the senior official' \\
\textit{prénci} & `prince' & vs & \textit{princèssa} & `princess'\\
\textit{sir} & `father-in-law' & vs & \textit{sira} & `mother-in-law'\\
\textit{tgaun} & `dog' & vs & \textit{cògna} & `bitch'\\
\textit{tgéjt} & `rooster' &vs& \textit{gaglina} & `hen'\\
\textit{vadí} & `calf' & vs & \textit{vadjala} & `female calf'\\
\lspbottomrule
\end{tabular}
\end{table}

Some feminine counterparts of masculine animals which are listed in \citet[239f.]{Spescha1989} are not in use in Tuatschín, as for example \textit{cavalla} `mare', \textit{utschala} `female bird', \textit{purschala} `sow'. In the case of \textit{piartg} `pig', the feminine counterpart is only used in a metaphoric sense: \textit{ina pòrtga} `a dirty girl'.

Some animals take \textit{taur} `bull' and \textit{vaca} `cow' to distinguish male from female, as e.g. \textit{in taur tscharva} `a deer', \textit{ina vaca tscharva} `a hind', and still others use \textit{bùc} `buck' and \textit{tgaura} `goat' for the same purpose: \textit{in bùc tgam\underline{ù}s} `a chamois buck', \textit{ina tgaura tgam\underline{ù}s} `a female chamois', or \textit{in bùc cavrial} `a male roe deer', \textit{ina tgaura cavrial} `a female roe deer'. Where there are no such nouns differentiating natural gender, \textit{fèmna} `woman; female' and \textit{máscal} `male' are used if necessary.



\subsection{Number}
Singular is not marked, and plural is formed by adding \textit{-s} to the stem of the noun, respectively to any part of the noun phrase (except for masculine plural participles, which take \textit{-i}), whether the stem ends in a vowel or in a consonant: \textit{tgèsa} (f.) `house' vs \textit{tgèsas} `houses', or \textit{rusp} (m.) `toad' vs \textit{rusps} `toads'. If the noun ends in an \textit{-s}, there is no differentiation between singular and plural. In these cases the suffix \textit{-s} is purely orthographic as in (\ref{ex:plurals1}). 

\ea
\label{ex:plurals1}
\langinfo{Tuatschín}{Sadrún}{m6, l. 1461f.}\\
\gll [...] nuṣ vèvan naginṣ raps \textbf{sèzs}.\\
{} \textsc{1pl} have.\textsc{impf.1pl}  no.\textsc{m.pl} cent.\textsc{pl} self.\textsc{m.pl}\\
\glt `[...] we didn’t have any money of our own.'
\z

An <s> is written after <z>, but not after <s>, as in \textit{patarnòs}, which is either singular or plural according to context.


There are some irregular plurals which are listed in \tabref{irregplI},  \tabref{irregplII},  \tabref{irregplIII}, \tabref{irregplIV}, and \tabref{irregplV}. 

\begin{table}
\caption{Nouns: irregular plural I}
\label{irregplI}
 \begin{tabular}{llll}
  \lsptoprule
   \textit{-í} & > & \textit{-jalts/-alts}  \\
  \midrule
\textit{aní} && \textit{anjalts} & `ring'\\
\textit{castí} && \textit{castjalts} & `castle' \\
\textit{cuntí} && \textit{cuntjalts} & 'knife' \\
\textit{flagí}  &&   \textit{flagjalts} & `flail' \\
\textit{iṣchí} && \textit{iṣchjalts} & `maple tree'\\
\textit{martí} && \textit{martjalts} &`hammer'\\
\textit{purschí} && \textit{purschalts} & `piglet'\\
\textit{ras-plí} && \textit{ras-pljalts} & `pencil'\\
\textit{rastí} && \textit{rastjalts} & `rake'\\
\textit{utschí} && \textit{utschalts}  &`bird'\\
\textit{vadí} && \textit{vadjalts} & `calf'\\
  \lspbottomrule
 \end{tabular}
\end{table}

\begin{table}
\caption{Nouns: irregular plural II}
\label{irregplII}
 \begin{tabular}{llll}
  \lsptoprule
    \textit{-éigls} & > & \textit{-ùlts}\\
  \midrule
\textit{anṣéjgl} && \textit{anṣùlts} & `kid'\\
\textit{catschéjgl} && \textit{catschùlts} & `sock'\\
\textit{cavréjgl}\footnote{The form \textit{cavrial} is also used.} && \textit{cavrùlts} & `roe deer'\\
\textit{spéjgl} && \textit{spùglts} & 'bobbin'\\
 \lspbottomrule
 \end{tabular}
\end{table}

The ending \textit{-lts } is also realized \textit{-ls}, whithout \textit{t}: \textit{cuntjals} 'knives' or \textit{catschòls} `socks'.

Masculine nouns with the diphthong /je/ or /ej/ in the stem change the diphtong to <ò> /ɔ/, whereby the nouns starting with palatal <tg> /ʨ/ depalatalize to <c> /k/.


\begin{table}
\caption{Nouns: irregular plural III} 
\label{irregplIII}
 \begin{tabular}{llll}
  \lsptoprule
   \textit{-ia-} & > & \textit{-ò- / -ù-} \\
  \midrule
\textit{criac} && \textit{cròcs} & `plough'\\
\textit{fiap} && \textit{fòps} & `hollow'\\
\textit{iart}  && \textit{òrts} & `garden'\\
\textit{ias} && \textit{òs} & `bone' \\
\textit{piartg} && \textit{pòrs} & `pig'\\
\textit{riavan} && \textit{rùvans} & `slope'\\
\textit{tgaubriacal} && \textit{tgaubr\underline{ò}cals} & `somersault'  \\
  \lspbottomrule
 \end{tabular}
\end{table}

\begin{table}
	\caption{Nouns: irregular plural IV} 
	\label{irregplIV}
	\begin{tabular}{llll}
		\lsptoprule
		 \textit{-é-} & > & \textit{ò-} & \\
		\midrule
		 \textit{tgérn} & & \textit{còrns} & `horn'\\
		 \textit{tagljér} & & \textit{tagliòrs} & `plate'\\
		\lspbottomrule
	\end{tabular}
\end{table}

Note that in contrast to standard Sursilvan the plural of \textit{tgéjt} `rooster' and \textit{tgiarp} `body' is usually \textit{tgéjts} and \textit{tgiarps} and not \textit{còts} and \textit{còrps} .\footnote{See also \DRG{3}{595}.}


\begin{table}
\caption{Nouns: irregular plural V}
\label{irregplV}
 \begin{tabular}{lll}
  \lsptoprule
singular & plural \\
  \midrule
\textit{bùf} & \textit{bùs} & `ox'\\
\textit{buéb} & \textit{buéts} & `boy'\\
\textit{cumandamèn} & \textit{cumandamajnts} & `commandment'\\
\textit{dé} & \textit{dis} & `day'\\
\textit{éjf} & \textit{ùfs} & `egg'\\
\textit{ljuc}  & \textit{lògans} & `place'\\
\textit{tgavaj} & \textit{tgavals} & `horse'\\
\textit{tgavégl} & \textit{tgavéiglts} & `hair'\\ 
\textit{trutg} & \textit{truigls} & `narrow path'\\
\textit{ùm} & \textit{ùmans} & `man'\\
  \lspbottomrule
 \end{tabular}
\end{table}

Some monosyllabic masculine nouns having the falling diphthong /iə/ convert it in a rising diphthong /ja/: \textit{culiar} vs \textit{culjars} `collar', \textit{falian} vs \textit{faljans} `spider', \textit{fiar} vs \textit{fjars} `iron', or \textit{paliat} v. \textit{paljats} `arrow', \textit{ṣchiarl} vs \textit{ṣchjarls} `kind of basket', \textit{stiarl} vs \textit{stjarls} 'one year old calf', \textit{tiarm} vs \textit{tjarms} `boundary stone', and \textit{unviarn} vs \textit{unvjarns} `winter'.

In compound nouns it is usually the second noun which modifies the first noun. In such cases, it is only the modified noun, the head noun, which is pluralized.


\ea\label{ex: }
\langinfo{Tuatschín}{}{\citealt[87]{Gadola1935}}\\
\gll  […] ju vaj\footnotemark{} èra prju ansjaman als quéns da tschèls dus \textbf{tgau}\textbf{-s}-tègja.  \\
    {} \textsc{1sg} have.\textsc{prs.1sg} also take.\textsc{ptcp.unm} together \textsc{def.art.m.pl} bill.\textsc{pl} of \textsc{dem.m.pl} two head-\textsc{pl}-alpine\_hut.\textsc{f.sg} \\
\glt `[…] I have also assembled the bills of the other two heads of the alpine huts.'\footnotetext{The form \textit{vaj} is incorrect in this position and should be replaced by \textit{a}. \textit{Vaj} is only used with subject inversion, \textit{va(j) ju} `have I'.}
\z

\subsection{Collective nouns}
Some inanimate masculine nouns have a feminine singular counterpart which usually refers to collective or generic entities which cannot be pluralized or counted. Compare: \textit{ in tgèrn} `a/one horn’ vs \textit{tschun còrns} `five horns’ vs \textit{la còrna} `the horns’. The noun \textit{pèra} `pair (collective)' is only used with paired terms as \textit{cazès} `shoes'; with other entities, \textit{pèrs} `pairs' is used, as examples (\ref{np1}) and (\ref{np2}) show. 

\ea\label{np1}
\langinfo{Tuatschín} {Sadrún}{m4}\\
\gll    \textbf{traja} \textbf{pèra} cazès \\
three pair.\textsc{coll} shoe.\textsc{pl}\\
\glt `three pairs of shoes'
\z

\ea\label{np2}
\langinfo{Tuatschín} {Sadrún} {m4}\\
\gll    \textbf{trajs} \textbf{pèrs} majla\\
three pair.\textsc{pl} apple.\textsc{coll}\\
\glt  three pairs of apples'
\z

Some more examples of collective nouns are \textit{blòc/blòca} `block', \textit{còtgal/còtgla} `charcoal',\textit{} \textit{crap/crapa} `stone', \textit{curnagl/curnaglja} `Alpine chough', \textit{fégl/féglj}a `leaf', \textit{grép/grépa} `rock', \textit{lèn/lèna} `wood', \textit{majl}/ \textit{majla} `apple', and \textit{pajr/pajra} `pear'. In the case of \textit{fiap} `hollow', \textit{fòpa} refers, not to a collective noun, but to a bigger hollow.

The following examples illustrate collective nouns in context.

\ea\label{}
\langinfo{Tuatschín} {Selva} {\citealt[26]{Büchli1966}}\\
\gll    Èl’ ò vulju vèndar \textbf{majla} […].\\ 
\textsc{3sg.f} have.\textsc{prs.3sg} want.\textsc{ptcp.unm} sell.\textsc{inf} apple.\textsc{coll}\\
\glt `She wanted to sell apples.'
\z

\ea\label{ex:1:adj}
\langinfo{Tuatschín}{Selva} {\citealt[53]{Büchli1966}}\\
\gll «Ò la \textbf{pajra} è pajs?» La mùma ò détg: «Na, la \textbf{pajra} ò bétga pajs.» A lu ò la féglja détg: «Scha la \textbf{pajra} ò bétga pajs, vaj ju magljau in rusp.»\\
have.{\prs}.3{\sg} \textsc{def}.{\art}.\textsc{f}.{\sg} pear.\textsc{coll} also foot.\textsc{m.pl} \textsc{def.art.f.sg} mother have.{\prs}.3{\sg} say.\textsc{ptcp.unm} no \textsc{def.art.f.sg}  pear.\textsc{coll} have.{\prs.3sg} \textsc{neg} foot.\textsc{m.pl} and then have.{\prs.3sg} \textsc{def.art.f.sg} daughter say.\textsc{ptcp.unm} if \textsc{def.art.f.sg} pear.\textsc{coll} have.\textsc{prs.3sg} \textsc{neg} foot.\textsc{m.pl} have.\textsc{prs.1sg} \textsc{1sg} eat.\textsc{ptcp.unm} \textsc{indef.art.m.sg} toad\\
\glt `«Do pears have feet?» The mother said: «No, pears do not have feet.» Then the daughter said: «If pears do not have feet, then I have eaten a toad.»'
\z

\ea
\label{}
\langinfo{Tuatschín} {Ruèras}{m10, l. 997f.}\\
\gll  A lu mávani gl unviarn gjùn Cavòrgja a trèvan sé \textbf{lèna} culs méls.\\
and then go.\textsc{impf.3pl.3pl} \textsc{def.art.m.sg} winter down\_in \textsc{pln} and pull.\textsc{impf.3pl} up wood.\textsc{coll} with.\textsc{def.art.m.pl} mule.\textsc{pl} \\
\glt `And then during winter they used to go down to Cavorgia and transport wood with the mules up [to Surrein].'
\z

\ea\label{}
\langinfo{Tuatschín}{}{\DRG{6}{697}}\\
\gll  Al mècsgjar fimjanta cun \textbf{ògna} [...].\\
\textsc{def.art.m.sg} butcher smoke.\textsc{prs.3sg} with alder.\textsc{coll} \\
\glt `The butcher smokes with alder wood [...].'
\z

In standard Sursilvan some masculine paired body terms have a collective form which refers to both entities, as for example \textit{in ṣchanugl} `one knee’ vs \textit{la ṣchanuglia} `the (two) knees’. In Tuatschin, this is not the case, since the feminine form corresponds to the feminine singular.

\ea
\label{}
\langinfo{Tuatschín} {Sadrún}{m4, l. 468ff.}\\
\gll   Basta, ah, par vagní cò sén quaj Pardatsch al, al tat vèva aun fatg ina satagljèda, \textbf{ina} \textbf{ganùglja} vèva `l tagljau sé [...]. \\
enough eh \textsc{purp} come.\textsc{inf} here on \textsc{dem.m.sg} \textsc{pln} \textsc{def.art.m.sg} \textsc{def.art.m.sg} grandfather have.\textsc{impf.3sg} in\_addition make.\textsc{ptcp.unm} \textsc{indef.art.f.sg} \textsc{refl.}cut.\textsc{ptcp.f.sg} \textsc{indef.art.f.sg} knee have.\textsc{impf.3sg} \textsc{3sg.m} cut.\textsc{ptcp.unm} up\\
\glt `Enough, eh, in order to come back to Pardatsch, my grandfather, in addition, had had a cut, he cut his knee [...].'
\z

\subsection{Bare noun phrases}
Bare nouns or noun phrases, in the sense of noun phrases without determiners, obligatorily occur with mass nouns and in indefinite plural object noun phrases. Bare nouns are very frequent in prepositional phrases, mostly with a locative meaning (see § 4.2.3 below). Example (\ref{ex:baremass}) illustrates mass nouns, (\ref{ex:bareindefpl}) an indefinite plural object noun phrase, and (\ref{ex:barepp1}) a locative prepositional phrase.

\ea
\label{ex:baremass}
\langinfo{Tuatschín}{Cavòrgja}{m7, l. 2269f.}\\
	\gll [...] ins duvrava \textbf{magnùc} a \textbf{tgarnpiartg}.   \\
{} \textsc{gnr} need.\textsc{impf.3sg} cheese.\textsc{m.sg} and bacon.\textsc{m.sg} \\
\glt `[my father would bring here, in addition to the lunch] one needed, also cheese and bacon.'
\z

\ea
\label{ex:bareindefpl}
\langinfo{Tuatschín}{Sadrún}{m9, l. 1826f.}\\
	\gll [...] ábar ùṣ anflá \textbf{capitaniṣ} è bigja schi sémpal.\\
{} but now find.\textsc{inf} captain.\textsc{m.pl} \textsc{cop.prs.3sg} \textsc{neg} so simple.\textsc{adj.unm}\\
\glt `[...] but nowadays to find captains is not so easy.'
\z

\ea
\label{ex:barepp1}
\langinfo{Tuatschín}{Sadrún}{f3}\\
\gll Ùsa vòm ju \textbf{á} \textbf{tgèsa}.\\
now go.\textsc{prs.1sg} \textsc{1sg} to house.\textsc{f.sg}\\
\glt `Now I am going home.'
\z

In rare cases, a bare noun may function as a direct object (\ref{ex:bareobjnpsg}).

\ea
\label{ex:bareobjnpsg}
\langinfo{Tuatschín}{Camischùlas}{f6, l. 764}\\
\gll Pi vajn nus méz \textbf{svagljarín} [...].\\
then have.\textsc{prs.1pl} \textsc{1pl} put.\textsc{ptcp.unm} alarm\_clock.\textsc{m.sg}\\
\glt `Then we set the alarm clock [...].'
\z

After the negator \textit{bétga} a bare noun or noun phrase is frequent.

\ea
\label{}
\langinfo{Tuatschín}{Ruèras}{f4, l. 2063}\\
	\gll Ábar quaj dèva bétga \textbf{discussjun}.\\
but \textsc{dem.unm} give.\textsc{impf.3sg} \textsc{neg} discussion.\textsc{f.sg}\\
\glt `But there were no discussions.'
\z

\subsection{Conjoining of nouns}
Nouns are joined by \textit{a} `and' and \textit{né} `or'; \textit{né ... né} is used for `neither ... nor'.

\ea
\label{}
\langinfo{Tuatschín}{Ruèras} {m3, l. 2187f.}\\
\gll  [...] \textbf{al}  \textbf{purtgè} \textbf{a} \textbf{‘l} \textbf{tarsél} vèvan dad èssar ajn … ajn tégja [...].\\
 {} \textsc{def.art.m.sg} swineherd and \textsc{def.art.m.sg} assistant have.\textsc{impf.3pl} \textsc{comp} \textsc{cop.inf} in {} in alpine\_hut.\textsc{f.sg} \\
\glt `[...] the swineherd and the assistant had to stay in ... in the alpine hut [...].'
\z

\ea
\label{}
\langinfo{Tuatschín}{Sadrún} {m4, l. 452f.}\\
\gll Ins vèz’ aun tg’ èra dau vidajn \textbf{pùntgas} \textbf{né} \textbf{trádals} [...].\\
\textsc{gnr} see.\textsc{prs.3sg} still \textsc{comp} \textsc{pass.aux.impf.3sg} give.\textsc{ptcp.unm} into chisel.\textsc{f.pl} or power\_drill.\textsc{m.pl}\\
\glt `One still can see that chisels or power drills had been used [...].'
\z

\ea
\label{}
\langinfo{Tuatschín}{Sèlva}{\citealt[47]{Büchli1966}}\\
\gll A lu ò `l signún gju \textbf{né} \textbf{gròma} \textbf{né} \textbf{latg}.\\
and then have.\textsc{prs.3sg} \textsc{def.art.m.sg} alpine have.\textsc{ptcp.unm} neither cream.\textsc{f.sg} nor milk.\textsc{m.sg}\\
\glt And then the didn't have cream nor milk.'
\z


\section{Determiners and pronouns}
The determiners all precede the noun they modify and distinguish number and gender but not case. An exception is the definite dative article, which distinguishes number but not gender, and which attributes case to the noun phrase as its name indicates. The definite dative article is obsolescent.

\subsection{Articles}

\subsubsection{Definite article}
The definite article distinguishes between masculine and feminine on the one hand, and between singular and plural on the other, yielding a system of four terms.

\begin{table}
\caption{Definite article}
\label{tab:1:defart}
 \begin{tabular}{llll}
  \lsptoprule
   \textsc{m.sg}   &  \textsc{m.pl} & \textsc{f.sg} & \textsc{f.pl}\\ 
  \midrule
  \textit{al},\textit{agl/gl}, \textit{`l}, \textit{l'} & \textit{als/alṣ}, \textit{`ls/`lṣ} & \textit{la}, \textit{l'} &  \textit{las/laṣ}\\
\lspbottomrule
\end{tabular}
\end{table}

The masculine form \textit{al} is used before a noun or a nominalized adjective that starts with a consonant (\ref{ex:artmsg1}) and \textit{agl/gl} with a noun that starts with a vowel (\ref{ex:artmsg2}) and (\ref{ex:artmsg3}). \textit{`l} is used after a word that ends with a vowel (\ref{ex:artmsg4}) and \textit{l'} before an adjective or a determiner that precedes a masculine noun (\ref{ex:artmsg5}).

\ea\label{ex:artmsg1}
\langinfo{Tuatschín}{Sadrún}{f3, l. 6}\\
\gll   [...] \textbf{al} \textbf{gròn} èra racrut, a tschèlṣ duṣ ajn amprèndissadi. \\
[...] \textsc{def.art.m.sg} big \textsc{cop.impf.3sg} recruit.\textsc{m.sg} and \textsc{dem.m.pl} two.\textsc{m.pl} in apprenticeship.\textsc{m.sg}\\
\glt `[...] the oldest was a recruit, and the other two [were] in an apprenticeship.'
\z

\ea\label{ex:artmsg2}
\langinfo{Tuatschín}{Camischùlas}{f6, l. 735f.}\\
\gll A lu ajn ajn tgòmbra èri ina tga vèva survgnú \textbf{agl} \textbf{avrél} plaza.\\
and then in in room.\textsc{f.sg} \textsc{exist.impf.3sg.expl} one.\textsc{f.sg} \textsc{rel} have.\textsc{impf.3sg} get.\textsc{ptcp.unm} \textsc{def.art.m.sg} April job.\textsc{f.sg}\\
\glt `And then in our room there was one [girl] that had got a job in April.'
\z

\ea
\label{ex:artmsg3}
\langinfo{Tuataschin}{Sadrún}{m6, l. 881f.}\\
\gll [...] qu' è stau \textbf{gl} \textbf{òn} mili a a sjat tschian a tòns [...].\\
[...] \textsc{dem.unm} be.\textsc{prs.3sg} be.\textsc{ptcp.unm} \textsc{def.art.m.sg} year thousand and and seven hundred and so\_many\\
\glt `[...] this was in 1700 and something [...].'
\z

\ea\label{ex:artmsg4}
\langinfo{Tuatschín}{Sadrún}{m1, l. 176}\\
\gll A lu \textbf{ò} \textbf{`l} bab détg sé pr mè [...].\\
and then have.\textsc{prs.3sg} \textsc{def.art.m.sg} father  say.\textsc{ptcp.unm} up to \textsc{1sg}\\
\glt `And then my father said to me [...]'
\z

\ea\label{ex:artmsg5}
\langinfo{Tuatschín}{Sadrún}{f3, l. 68f.}\\
\gll  \textbf{L}’ \textbf{autar} \textbf{dé} va ju gju la lubiantscha dad í vidajn [...].  \\
 \textsc{def.art.m.sg} other.\textsc{m.sg} day have.\textsc{prs.1sg} \textsc{1sg} have.\textsc{ptcp.unm} \textsc{def.art.f.sg} permission \textsc{comp} go.\textsc{inf} in\\
\glt `The day after I got the permission to go there [...].'
\z

There is, however, one exception. Before the quantifier \textit{antir} `whole', it is not \textit{l'} but \textit{gl} that is used (\ref{ex:artmsg6}).

\ea\label{ex:artmsg6}
\langinfo{Tuatschín}{Sadrún}{m5}\\
\gll  Nus vajn luvrau \textbf{gl'} \textbf{antir} \textbf{dé}.  \\
\textsc{1pl} have.\textsc{prs.1sg} work.\textsc{ptcp.unm} \textsc{def.art.m.sg} whole.\textsc{m.sg} day\\
\glt `We worked the whole day.'
\z

The masculine plural form \textit{als/alṣ} is used before consonant (\ref{ex:artmpl1}) and \textit{ls/lṣ} before vowel. The form \textit{'ls/'lṣ} occurs after a word ending in a vowel (\ref{ex:artmpl2}).

\ea
\label{ex:artmpl1}
\langinfo{Tuatschín}{Sadrún}{m9, l. 1805ff.}\\
\gll A quaj òzaldé \textbf{als} \textbf{gjuvans} tga végnan ò da scùla sadatan bétga gjù schi fétg cun in téc plé végls.   \\
and \textsc{dem.unm} nowadays \textsc{def.art.m.pl} young.\textsc{pl} \textsc{rel} come.\textsc{prs.3pl} out of school.\textsc{f.sg} \textsc{refl}.give.\textsc{prs.3pl} \textsc{neg} down so often with i\textsc{ndef.art.m.sg} bit more old.\textsc{m.pl}\\
\glt `And nowadays the young people who come out of school do not want to have to do so much with those who are a bit older.'
\z

\ea
\label{ex:artmpl2}
\langinfo{Tuatschín}{Surajn}{f5, l. 1301f.}\\
\gll [...] \textbf{cu} \textbf{`ls} purs vagnévan [...].\\
{} when \textsc{def.art.m.pl} farmer.\textsc{pl} come.\textsc{impf.3pl}\\
\glt `[...] when the farmers would come [...].'
\z

The feminine forms \textit{la} and \textit{las} occur before consonant; the forms \textit{l'} and \textit{laṣ} occur before vowel.

\ea\label{}
\langinfo{Tuatschín}{Surajn}{f3, l. 47f.}\\
\gll A sjantar... òni ampustau tùt nùfs ... pr \textbf{l'} \textbf{antira} \textbf{val}.\\
and after have.\textsc{prs.3pl.3pl} order.\textsc{ptcp.unm} all new.\textsc{m.pl} {} for \textsc{def.art.f.sg} whole valley\\
\glt `And then … they ordered all new … for the entire valley.'
\z

In combination with the preposition \textit{da} ‘of’ and a place name, the definite article is used to form demonyms.

\ea
\label{}
\langinfo{Tuatschín}{Camischùlas}{\citealt[94]{Büchli1966}}\\
\gll    \textbf{Als} \textbf{da} \textbf{Tujétsch} tégnan aut quaj ljuc […].\\
    \textsc{def.art.m.pl} of \textsc{pln} hold.\textsc{prs.3pl} high.\textsc{unm} \textsc{dem.m.sg} place\\
\glt `The people of Tujetsch uphold this place […].'
\z

This construction is not restricted to the inhabitants of villages or towns, but occurs with any habitable place.

\ea\label{}
\langinfo{Tuatschín}{Ruèras}{\citealt[68]{Büchli1966}}\\
\gll […] òn \textbf{als} \textbf{da} \textbf{tgèsa} détg.\\
{} have.\textsc{prs.3pl} \textsc{def.art.m.pl} of house say.\textsc{ptcp.unm}\\
\glt `[…] said those at home.'
\z


\subsubsection{Indefinite article}
The indefinite article singular is identical to the numeral \textit{in} (m.)/\textit{ina} (f.) ‘one’. The feminine form \textit{ina} is realizied \textit{in'} before a word starting with a vowel. There is no plural indefinite article; plural indefinite noun phrases are bare.

The indefinite article is used to introduce into discourse entities which are not known to the hearer or reader  (\ref{indefart1}). 

\ea\label{indefart1}
\langinfo{Tuatschín}{Ruèras}{\citealt[66]{Büchli1966}}\\
\gll \textbf{In'} \textbf{jèda} vèv’ \textbf{in} pur \textbf{in} stauschbèna.\\
     \textsc{indef.art.f.sg} time have.\textsc{impf.3sg} \textsc{indef.art.m.sg} peasant \textsc{indef.art.m.sg} wheelbarrow \\
\glt `Once\footnote{Tuatschín speakers usually write \textit{in' jèda} in one word: \textit{ignèda}.} a peasant had a wheelbarrow.'
\z

Like the definite article, the indefinite article is used for demonyms.

\ea\label{}
\langinfo{Tuatschín}{Sèlva}{\citealt[52]{Büchli1966}}\\
\gll   In da Méjdal è jus cul trèn gjù Cuéra.\\
     one.\textsc{m.sg} of \textsc{pln} be.\textsc{prs.3sg} go.\textsc{ptcp.m.sg} with.\textsc{def.art.m.sg} train down \textsc{pln}\\
\glt `A person from Medel went down to Cuera by train.'
\z

\subsubsection{Definite dative article} 
Until approximately the 1960, the dative article \textit{di} or \textit{li} was in common use.\footnote{Some indications concerning the dative article in all Romansh varieties can be found in \citet{Linder1987}, and \citet{Maurer2017} looks at the definitive dative article in Tuatschin from a diachronic perspective.} Nowadays it is obsolescent; spontaneous productions are rare in the corpus and were exclusively produced by elder people.

The dative article is a definite article; it distinguishes number but not gender. Its forms are \textit{di} (sg) and \textit{dis} (pl) or \textit{li} and \textit{lis}.  Whereas \textit{li, lis} were widespread in other Sursilvan dialects as well as in further  Romansh varieties such as Sutsilvan and Surmiran, \textit{di, dis} was a genuine Tuatschin form.

\textit{Di} and \textit{li} were also used for marking dative case with personal pronouns, however without differentiating gender and number (see below § 3.6.1).

\ea\label{}
\langinfo{Tuatschín}{Tschamùt} {\citealt[12]{Büchli1966}}\\
\gll  Quaj è curdau sé \textbf{li} \textbf{gljut}.\\
     \textsc{dem.unm} be.\textsc{prs.3sg} fall.\textsc{ptcp.unm} up \textsc{dat.art.sg} people.\textsc{f}\\
\glt `People noticed this.'
\z

\ea\label{ex:1:}
\langinfo{Tuatschín}{}{\DRG{5}{365}}\\
\gll  èssar sé dias \textbf{di} \textbf{vischnaunca}\\
\textsc{cop.inf} on back \textsc{dat.art.sg} municipality.\textsc{f}\\
\glt `to become a burden on the municipality'
\z

\ea\label{}
\langinfo{Tuatschín}{Bugnaj}{\citealt[143]{Büchli1966}}\\
\gll […] lu òn inṣ détg quaj \textbf{li} \textbf{préjr} […].\\
{} then have.\textsc{prs.3sg.euph} \textsc{gnr} tell.\textsc{ptcp.unm} \textsc{dem.unm} \textsc{dat.art.sg} priest.\textsc{m}\\
\glt `[…] then they told this to the priest […].'
\z

\ea\label{}
\langinfo{Tuatschín}{}{\citealt[69]{Berther1998}}\\
\gll […] uòn dùn ju ajn mia dùna \textbf{di} \textbf{gjadju}.\\
{} this\_year give.\textsc{prs.1sg} \textsc{1sg} in \textsc{poss.1sg.f.sg} wife \textsc{dat.art.sg} Jew.\textsc{m} \\
\glt `[…] this year I’ll give my wife to the Jew.'\footnote{Until more or less 150 years ago, Swiss Jews were only allowed to settle in two villages in the canton of Aargau. Until recently some of them worked as cattle dealers in the whole country.}
\z

\ea\label{}
\langinfo{Tuatschín}{}{\DRG{2}{480}}\\
\gll   Las gjufnas duèssan ins tanaj sén bratsch, a \textbf{lis} \textbf{végljas} dá cun in scanatsch.\\
\textsc{def.art.f.pl} young.\textsc{pl} must.\textsc{cond.3sg.euph} \textsc{gnr} hold.\textsc{inf} on arm.\textsc{m.sg} and \textsc{dat.art.pl} old.\textsc{f.pl} give.\textsc{inf} with \textsc{indef.art.m.sg} log\\
\glt `The young women we should hold on our arms, and the old ones we should beat with a log.'
\z

\ea\label{}
\langinfo{Tuatschín}{}{\citealt[199]{ASR1889}}\\
\gll  Scha tùts ratunṣ èn intanzjònaj usché scù quèlṣ dus, lura vali bétga tiar èlṣ al proverbi tga survèscha \textbf{lis} \textbf{carstgauns} par zanur […].\\
     if all.\textsc{m.pl} rat.\textsc{pl} \textsc{cop.prs.3pl} benevolent.\textsc{m.pl} so like \textsc{dem.m.pl} two then apply.\textsc{prs.3sg.expl}  \textsc{neg}  by \textsc{3pl.m}  \textsc{def.art.m.sg} proverb \textsc{rel}  serve.\textsc{prs.3sg}  \textsc{dat.art.pl} person.\textsc{m.pl} for dishonour\\
\glt `If all the rats were as benevolent as these two, the proverb which dishonours human beings would not apply to them […].'
\z

\ea\label{}
\langinfo{Tuatschín}{}{\DRG{1}{366}}\\
\gll Quaj da l’ arvéglja végn dau \textbf{dis} \textbf{tgauras}.  \\
\textsc{dem.unm} of \textsc{def.art.f.sg} pea \textsc{pass.aux.prs.3sg} give.\textsc{ptcp.unm} \textsc{dat.art.pl} goat.\textsc{f.pl} \\
\glt `The straw of the peas is given to the goats.'
\z

The dative article was also used after the preposition \textit{ancúntar} `towards' and \textit{sjantar} `after'.

\ea\label{}
\langinfo{Tuatschín}{Ruèras} {\citealt[64]{Büchli1966}}\\
\gll  Cò ṣaj vagnú ina	fèmna ancùntar \textbf{li} \textbf{quaj} \textbf{pur} […].\\
here be.\textsc{prs}.3\textsc{sg}  come.\textsc{ptcp.unm}	\textsc{indef}.\textsc{art}.\textsc{f}.\textsc{sg} woman towards	\textsc{dat}.\textsc{art}.\textsc{sg} \textsc{dem.m.sg} peasant\\
\glt `At this moment a woman came towards this peasant […].'
\z

\ea\label{}
\langinfo{Tuatschín}{Bugnaj}{\citealt[146]{Büchli1966}}\\
\gll   […] tga nagín fravi ségl antir mùn sapi fá \textbf{sjantar} \textbf{li} \textbf{èl}. \\
{}  \textsc{comp} no.\textsc{m.sg} smith on.\textsc{def.art.m.sg} whole world know.\textsc{prs.sbjv.3sg} do.\textsc{inf} after \textsc{dat.sg} \textsc{3sg.m}\\
\glt `[…] that no smith in the whole world would be able to make [things] like him.'
\z

If the noun precludes the use of the definite article, the marker \textit{da} was used. This is the case if, e.g., the noun is modified by an indefinite article (\ref{ex:dat:indef:1}) (which is zero-marked if the noun is plural (\ref{ex:dat:indef:2})), a quantifier (\ref{ex:dat:indef:3}), or a possessive determiner (\ref{ex:dat:indef:4}).

\ea\label{ex:dat:indef:1}
\langinfo{Tuatschín}{Sèlva}{\citealt[25]{Büchli1966}}\\
\gll […] òi tucau \textbf{d'} \textbf{ina} \textbf{mata} […].\\
{} have.\textsc{prs.3sg.expl} touch.\textsc{ptcp.unm} \textsc{dat} \textsc{indef.art.f.sg} girl \\
\glt `[…] it was a girl's turn […].'
\z

\ea\label{ex:dat:indef:2}
\langinfo{Tuatschín}{}{\citealt[118]{Berther1998}}\\
\gll Bétga mataj sé scalíns \textbf{da} \textbf{tgauras} \textbf{jastras}.\\
\textsc{neg} put.\textsc{imp.2pl} up bell.\textsc{m.pl} \textsc{dat} goat.\textsc{f.pl} somebody\_else’s.\textsc{pl}\\
\glt `Don’t put bells on somebody else’s goats.'
\z

\ea\label{ex:dat:indef:3}
\langinfo{Tuatschín}{Sadrún}{\citealt[104]{Büchli1966}}\\
\gll […] èl daj dí \textbf{da} \textbf{nagín} \textbf{carstgaun} tg’ èl vagi vju èlas cò […].\\
{} \textsc{3ssg.m} should.\textsc{prs.sbjv.3sg} say.\textsc{inf} \textsc{dat} no.\textsc{m.sg} person \textsc{comp} \textsc{3sg.m} have.\textsc{prs.sbjv.3sg} see.\textsc{ptcp.unm} \textsc{3pl.f} here\\
\glt `[…] he shouldn’t tell anybody that he had seen them here […].'
\z

\ea\label{ex:dat:indef:4}
\langinfo{Tuatschín}{Sèlva}{\citealt[25]{Büchli1966}}\\
\gll Parquaj ò ‘la dau tissi \textbf{da} \textbf{sju} \textbf{piartg} […].\\
therefore have.\textsc{prs.3sg} \textsc{3sg.f} give.\textsc{ptcp.unm} poison.\textsc{m.sg} \textsc{dat} \textsc{poss.3sg.m.sg} pig \\
\glt `Therefore she gave poison to her pig […].'
\z

The following examples of the definite dative article have been uttered spontaneously by my consultants. There are only examples with \textit{di/dis}.

\ea\label{datart7}
\langinfo{Tuatschín}{Sadrún} {f3, l. 145}\\
\gll  Di B.A.W a quaj vòi \textbf{di} \textbf{cantún}.\\
\textsc{dat.sg} B.A.W. and \textsc{dem.unm} go.\textsc{prs.3sg.expl} \textsc{dat.art.sg} canton.\textsc{m}\\
\glt `To the B.A.W., and then it goes to the canton.'
\z

\ea\label{datart8}
\langinfo{Tuatschín}{Sadrún} {f3, l. 132ff.}\\
\gll A zacú ò da quaj … ṣaj vagnú … tgé … quèls majnadistricts … òn la còmpat\underline{è}nza da … dí \textbf{dis} \textbf{vaschnauncas} … tgé i òn da fá [...].\\
and sometime out of \textsc{dem.unm} {} be.\textsc{prs.3sg} come.\textsc{ptcp.unm} {} \textsc{comp} {} \textsc{dem.m.pl} head\_of\_district.\textsc{pl} {} have.\textsc{prs.3sg} \textsc{def.art.f.sg} competence \textsc{comp} {} say.\textsc{inf}  \textsc{dat.art.pl} municipality.\textsc{f.pl} {} what \textsc{3pl} have.\textsc{prs.3pl}  \textsc{comp} do.\textsc{inf} \\
\glt `And sometime the result of this was that the heads of district have the authority to tell the municipalities what they have to do [...].'
\z

\ea
\label{datart9}
\langinfo{Tuatschín}{Sadrún} {m4, l. 401f.}\\
\gll  Ju sùn juṣ in tjamṣ a plaṣchèva da mé ṣchùbar nuét api vau détg \textbf{di} \textbf{mùma} in dé [...]. \\
 \textsc{1sg} be.\textsc{prs.1sg} go.\textsc{ptcp.m.sg} \textsc{indef.art.m.sg} time and please.\textsc{impf.3sg} \textsc{dat} \textsc{1sg} clean.\textsc{adj.unm} nothing and have.\textsc{prs.1sg.1sg} say.\textsc{ptcp.unm} \textsc{dat.art.sg} mother.\textsc{f} \textsc{indef.art.m.sg} day\\
\glt `I went [to nursery school] for a certain time and I didn’t like it at all and one day I said to my mother [...].'
\z

Example (\ref{datart10}) shows the simultaneous occurrence of \textit{di} and the standard Sursilvan construction \textit{a + definite article} in the same utterance.

\ea\label{datart10}
\langinfo{Tuatschín}{Sadrún} {m4, l. 580f.}\\
\gll   Quaj duvrava `l par dá \textbf{dis} \textbf{pòrs}, trúfals ansjaman par dá \textbf{áls} \textbf{pòrs}. \\
\textsc{dem.unm} use.\textsc{impf.3sg} \textsc{3sg.m} \textsc{purp} give.\textsc{inf} \textsc{dat.pl} pig.\textsc{m.pl} potato.\textsc{m.pl} together \textsc{purp} give.\textsc{inf}  \textsc{dat.def.art.m.pl} pig.\textsc{pl} \\
\glt `This he used to give the pigs, potatoes together [with nettles] to give the pigs. '
\z
 
It is not clear to me to which extent the definite dative article is productive among those native speakers who still use it. When I asked one consultant what she thinks about a sentence like \textit{Ju a dau dis pòrs}, literally `I have given to the pigs', her answer was:

\ea\label{datart11}
\langinfo{Tuatschín}{Cavòrgja}{f1}\\
\gll «Ju a dau \textbf{dis} pòrs»? Da lèzas uras mintga familja vèva in piartg, dus, trais, a quaj mintgín savèva tgé ca què lèṣ dí. Òz stuèssan nus mataj dí: «Ju a dau da magljè \textbf{dis} pòrs» né «Ju a parvaṣju méjs pòrs». \\
\textsc{1sg} have.\textsc{prs.1sg} give.\textsc{ptcp.unm} \textsc{dat.art.pl} pig.\textsc{m.pl} of \textsc{dem.f.pl} hour.\textsc{pl} every family.\textsc{f.sg} have.\textsc{impf.3sg} \textsc{indef.art.m.sg} pig two three and  \textsc{dem.unm} everybody know.\textsc{impf.3sg} what \textsc{comp}  \textsc{dem.unm} want.\textsc{cond.3sg} say.\textsc{inf} nowadays must.\textsc{cond.1pl} \textsc{1pl} probably  say.\textsc{inf} \textsc{1sg} have.\textsc{prs.1sg} give.\textsc{ptcp.unm}  \textsc{comp} eat.\textsc{inf} \textsc{dat.art.pl} pig.\textsc{m.pl} or  \textsc{1sg} have.\textsc{prs.1sg} feed.\textsc{ptcp.unm} \textsc{poss.1sg.3pl.m} pig.\textsc{pl}\\
\glt `"I have given to the pigs"? Formerly, every family had a pig, two, three, and everybody knew what it meant. Nowadays we probably might have to say: «I gave to eat to the pigs» or «I fed my pigs».'
\z

It is interesting to note that the consultant uses \textit{dá da magljè \textbf{dis} pòrs} as a modern way of saying things. In my view, this shows that the marker \textit{di/dis} is still a part of the grammar of some older people.

More examples of dative marking will be given below in § 4.2.2 about indirect objects.

\subsection{Demonstratives}
There are four series of demonstratives: the \textit{quèl}-series (\tabref{demquel}), which has deictic as well as anaphoric functions, the \textit{lèz}-series ({\tabref{demlez}}), which is exclusively anaphoric, the \textit{quèst}-series ({\tabref{demquest}}), which is only used as a determiner with temporal complements of the verb, and the \textit{tschèl}-series  ({\tabref{demtschel}}), which is used in contrast to the \textit{quèl}-series in the sense of `the other'. The demonstratives are not restricted to any syntactic function.

In the demonstrative paradigms, there is one syncretism: the masculine singular determiner\footnote{\textit{Quaj} is also used as a form that may determine place names (see example (\ref{ex:dem.unm.pln}) in this section), but this is not the case of the other three demonstratives.} and the pronoun which is unmarked for gender have the same form: \textit{quaj}, \textit{gljèz}, and \textit{tschaj}.\footnote{There is no such syncretism in the \textit{quèst}-series since in this paradigm there are no pronouns.} This parallels the syncretism in the domain of the adjective: the masculine singular attributive adjective has the same form as the predicative adjective whose antecedent has no gender (see below § 3.3.1).

\subsubsection{The \textit{quèl}-series}

The demonstratives of the \textit{quèl}-series are the only ones which have a deictic function. However, they do not give any information about distance from the place of speech; in order to distinguish proximal from distal, the adverbs \textit{cò} `here' and \textit{lò} `there' are optionally used and are located at the end of the noun phrase.

\begin{table}
\caption{Demonstratives: the \textit{quèl}-series}
\label{demquel}
 \begin{tabular}{llllll}
  \lsptoprule
   gender         & \textsc{m.sg} & \textsc{m.pl} & \textsc{f.sg} & \textsc{f.pl} & \textsc{unmarked}\\
  \midrule
  determiner  & \textit{quaj} &  \textit{quèls}  & \textit{quèla}  & \textit{quèlas} & \textit{quaj}\\
  pronoun  & \textit{quèl} & \textit{quèls} & \textit{quèla} & \textit{quèlas} & \textit{quaj}\\
  \lspbottomrule
 \end{tabular}
\end{table}

\tabref{demquel} shows that there is only one difference between the determiners and the pronouns: the masculine singular form. 


Examples (\ref{ex:quaj:anaph1}) and (\ref{ex:quaj:anaph2}) illustrate the deictic function of the determiners of the \textit{quaj}-series.

\ea
\label{ex:quaj:anaph1}
\langinfo{Tuatschín}{Sadrún} {m4, l. 324f.}\\
\gll [...] vau prju \textbf{quaj} cùdisch \textbf{cò} da la mütòlògia [...].\\
{} have.\textsc{prs.1sg} take.\textsc{ptcp.unm} \textsc{dem.m.sg} book here of \textsc{def.art.f.sg} mythology\\
\glt `[...] I took this book of mythology [...]¨.'\footnote{The speaker was pointing at the book.}
\z

\ea
\label{ex:quaj:anaph2}
\langinfo{Tuatschín}{}{\citealt[88]{Gadola1935}}\\
\gll Quaj dajan \textbf{quèls} dus gjuvans \textbf{lò} fá […].\\
    \textsc{dem.unm} must.\textsc{prs.3pl} \textsc{dem.m.pl} two.\textsc{m.pl} young.\textsc{pl} there do.\textsc{inf}\\
\glt `This the two young [men] over there should do [...].'
\z

The following examples illustrates the anaphoric function of the determiners of the \textit{quaj}-series.

\ea
\label{}
\langinfo{Tuatschín}{Ruèras} {m1, l. 254f.}\\
\gll A quaj èra mù \textbf{in} \textbf{ganc} tras. A \textbf{quaj} \textbf{gang} udéva dad òmaṣdús.\\
and \textsc{dem.unm} \textsc{exist.impf.3sg} only one.\textsc{m.sg} corridor through and \textsc{dem.m.sg} corridor belong.\textsc{impf.3sg} \textsc{dat} both.\textsc{m.pl}\\
\glt ‘And there was only one corridor. And this corridor belonged to both [families].’
\z

\ea
\label{}
\langinfo{Tuatschín}{Cavòrgja} {m7, l. 2320f.}\\
\gll [...] nuṣ vajn mùgnṣch a fatg tùt \textbf{quèlas} \textbf{lavurs}.\\
{} \textsc{1pl} have.\textsc{prs.1pl} milk.\textsc{ptcp.unm} and do.\textsc{ptcp.unm} all \textsc{dem.f.pl} work.\textsc{pl}\\
\glt `[...] we milked and did all these works.'
\z

Examples (\ref{ex:dempronquel1}) and (\ref{ex:dempronquel2}) illustrate the anaphoric function of the demonstrative pronouns of the \textit{quèl}-series.

\ea
\label{ex:dempronquel1}
\langinfo{Tuatschín}{Ruèras}{\citealt[66]{Büchli1966}}\\
\gll    In' jèda vèv’ in pur in stauschbèna. \textbf{Quèl} vagnéva navèn la notg.\\
     \textsc{def.art.f.sg} time have.\textsc{impf.3sg} \textsc{indef.art.m.sg} farmer \textsc{indef.art.m.sg} wheelbarrow \textsc{dem.m.sg} come.\textsc{impf.3sg} away \textsc{def.art.f.sg} night\\
\glt `Once a farmer had a wheelbarrow. This used to disappear during night.'
\z

\ea\label{ex:dempronquel2}
\langinfo{Tuatschín}{Sadrún}{\citealt{Büchli1966}}\\
\gll    Èl vèva da cargè ina bùra, mù \textbf{quèla} èra gjalada ajla naiv […].\\
    \textsc{3sg.m} have.\textsc{impf.3sg} \textsc{comp} carry.\textsc{inf} \textsc{indef.art.f.sg} block but \textsc{dem.f.sg} \textsc{cop.impf.3sg} freeze.\textsc{ptcp.f.sg} in.\textsc{def.art.f.sg} snow\\
\glt `He had to carry a block (of wood), but it was frozen in the snow […]. '
\z

In § 3.3.1 it will be shown that place names are treated as entities without gender since they trigger the use of the unmarked form of the adjective or the participle in predicative function (see examples \ref{ex:adj.unm5} and \ref{ex:ptcp.unm1} below). Therefore, in (\ref{ex:dem.unm.pln}) I interpret \textit{quaj} modifying the place name \textit{Pardatsch} as unmarked for gender but not as masculine singular.

\ea
\label{ex:dem.unm.pln}
\langinfo{Tuatschín}{Cavòrgja} {m7, l. 2248f.}\\
	\gll A grad ajnta \textbf{quaj} \textbf{Pardatsch} èran nus quátar, tschun buéts.   \\
and precisely in \textsc{dem.unm} \textsc{pln} \textsc{cop.impf.1pl} \textsc{1pl} four five boy.\textsc{m.pl}\\
\glt `And precisely in Pardatsch we were four, five boys.'
\z


The unmarked demonstrative pronoun \textit{quaj} refers anaphorically to a precedent sentence or cataphorically to a noun.

\ea
\label{}
\langinfo{Tuatschín}{Sadrún} {f3, l. 86ff.}\\
\gll [...] stuèv’ í  séls bauns, ah, sél Krüzlipass, Pas dlas Cruschs, cun muossavias, a vèva sjat da quèlas da purtá sé.  A \textbf{quaj} èra pasanca, api vau tartgau basta.  \\
{} must.\textsc{impf.1sg} go.\textsc{inf} on.\textsc{def.art.m.pl} bench ah on.\textsc{def.art.m.sg} \textsc{pln} pass of.\textsc{def.art.f.pl} cross.\textsc{pl} with signpost.\textsc{f.pl} and have.\textsc{impf.1sg} seven of  \textsc{dem.f.pl} \textsc{comp} carry.\textsc{inf} up and \textsc{dem.unm} \textsc{cop.impf.3sg} very\_heavy and have.\textsc{prs.1sg} think.\textsc{ptcp.unm} enough \\
\glt `[...] [I] had to go on the benches, ah, on the Krüzlipass, Pass dallas Cruschs, with signposts, and I had seven of them to carry up. And this was terribly heavy, and then I thought [it was] enough.'
\z

When \textit{quaj} as the subject of the sentence refers cataphorically to nouns, the copula agrees with \textit{quaj} and not with the predicative noun, which may be singular or plural, feminine or masculine.

\ea
\label{ex:quaj:agrwithsubj}
\langinfo{Tuatschín}{}{\DRG{4}{376}}\\
\gll \textbf{Quaj} \textbf{è} dètgas cumars.\\
\textsc{dem.unm} \textsc{cop.prs.3sg} real.\textsc{f.pl} chatterbox.\textsc{pl}\\
\glt `These are real chatterboxes.'
\z

\ea\
label{}
\langinfo{Tuatschín}{}{\citealt[28]{Gartner1910}}\\
\gll \textbf{ɛ} \textbf{kwaj} mʋʃʨəs? – \textbf{kwaj} \textbf{ɛ} furmikləs […].\\
     \textsc{cop.prs.3sg} \textsc{dem.unm} fly.\textsc{f.pl} \textsc{}  \textsc{dem.unm} \textsc{cop.prs.3sg} ant.\textsc{f.pl}\\
\glt `Are these flies? – They are ants […].'
\z

\ea
\label{}
\langinfo{Tuatschín}{}{\citealt[15]{Berther2007}}\\
\gll Uardaj cò sé quèls dus, \textbf{quaj} \textbf{è} dus vajrs lumps!\\
     look.\textsc{imp.pol} here up \textsc{dem.m.pl} two \textsc{dem.unm} \textsc{cop.prs.3sg} two real.\textsc{m.pl} rascal.\textsc{pl}  \\
\glt `Look at these two up there, these are two real rascals!'
\z

\ea
\label{}
\langinfo{Tuatschín}{Ruèras} {m1, l. 307}\\
\gll    [...] \textbf{qu}’ \textbf{èra} \textbf{stau} gròndas lavurs [...].\\
{} \textsc{dem.unm} be.\textsc{impf.3sg} \textsc{cop.ptcp.unm} big.\textsc{f.pl} work.\textsc{pl}\\
\glt `[...] this had been  hard work [...].'
\z

In the corpus, there are some occurrences of \textit{què} as an allomorph of \textit{quaj}:

\ea
\label{}
\langinfo{Tuatschín}{Camischùlas} {f6, l. 831f.}\\
\gll    [...] quaj vagnéva raṣdau cò ajn \textbf{què} \textbf{cantún} mù ròmòntsch.\\
{} \textsc{dem.unm} \textsc{pass.aux.impf.3sg} speak.\textsc{ptcp.unm} here in  \textsc{dem.m.sg} corner only Romansh.\textsc{m.sg}\\
\glt `[...] in that corner only Romansh was spoken.'
\z

\subsubsection{The \textit{lèz}-series}

The \textit{lèz}-series has exclusively anaphoric functions. In the corpus there is only one example with a determiner of the \textit{lèz}-series, (\ref{ex:lez1}).

\begin{table}
\caption{Demonstratives: the \textit{lez}-series}
\label{demlez}
 \begin{tabular}{llllll}
  \lsptoprule
         gender   & \textsc{m.sg} & \textsc{m.pl} & \textsc{f.sg} & \textsc{f.pl} & \textsc{unmarked}\\
  \midrule
  determiner  & \textit{gljèz} &  \textit{lèz}  & \textit{lèza}  & \textit{lèzas} & ---\\
  pronoun  & \textit{lèz} & \textit{lèz} & \textit{lèza} & \textit{lèzas} & \textit{gljèz} \\
  \lspbottomrule
 \end{tabular}
\end{table}

\ea
\label{ex:lez1}
\langinfo{Tuatschín}{Camischùlas} {f6, l. 806ff.}\\
\gll […] scha nus mòndjan a séjṣjan spèr la sòr’ Andréa, \textbf{lèza} savèva ròm\underline{ò}ntsch. Scha nus séjṣjan \textbf{sén} \textbf{lèza} \textbf{majṣa}, astgjan nus raṣdá ram\underline{ò}ntsch […].\\
{} if \textsc{1pl} go.\textsc{prs.sbjv.1pl} and sit.\textsc{prs.sbjv.1pl} next\_to \textsc{def.art.f.sg} sister.\textsc{f.sg} \textsc{pn} \textsc{dem.f.sg} know.\textsc{impf.3sg} Romansh.\textsc{m.sg} if \textsc{1pl} sit.\textsc{prs.sbjv.1pl} on \textsc{dem.f.sg} table be\_allowed.\textsc{prs.sbjv.1pl} \textsc{1pl} speak.\textsc{inf} Romansh.\textsc{m.sg}\\
\glt `[…] if we went to sit next to Sister Andrea, she knew Romansh. If we sat at that table, we would be allowed to speak Romansh [...].'
\z

There are two reasons why in (\ref{ex:lez1}) the noun lèza refers to is mentioned. On the one hand, \textit{majṣa} is mentioned at a certain distance (line 800), and on the other if \textit{lèza} were used without \textit{majṣa}, the sentence would be ambiguous and \textit{lèza} could be interpreted as referring to \textit{sòr' Andrea}.
 
 But pronouns of the \textit{lèz}-series are relatively frequent and occur with all syntactic functions. In (\ref{ex:lez2}), \textit{lèz} functions as a subject, in (\ref{ex:lez3}) as a direct object, in (\ref{ex:lez4}) \textit{lèzs} functions as an indirect object; in (\ref{ex:lez5}) and (\ref{ex:lez6}) \textit{lèz} and \textit{gljèz} occur in prepositional phrases.
 
\ea
\label{ex:lez2}
\langinfo{Tuatschín}{Bugnaj}{\citealt[132]{Büchli1966}}\\
\gll Lu ṣèn aj i sén claustra á fá vagní gjù \textbf{gl} \textbf{avat}. \textbf{Lèz} è vagnús gjù […].\\
     then \textsc{cop.prs.3pl} \textsc{3pl} go.\textsc{ptcp.m.3pl} on monastery \textsc{purp} make.\textsc{inf} come.\textsc{inf} down \textsc{def.art.m.sg} abbot \textsc{dem.m.pl} \textsc{cop.prs.3sg} come.\textsc{ptpc.m.sg} down\\
\glt `Then they went up to the monastery and made the abbot come down. He came down […].'
\z

\ea
\label{ex:lez3}
\langinfo{Tuatschín}{Ruèras}{m3, l. 2116f.}\\
	\gll [...] al signún a tauntaṣ a tauntas, a té as da gidá \textbf{lèz}.\\
{} \textsc{def.art.m.sg} dairyman have\textsc{.prs.3sg} so\_many.\textsc{f.pl} and so\_many.\textsc{f.pl} and \textsc{2sg} have.\textsc{prs.2sg} \textsc{comp} help.\textsc{inf} \textsc{dem.m.sg}	\\
\glt `[...] the dairyman has so and so many [cows], and you have to help him.'
\z

\ea
\label{ex:lez4}
\langinfo{Tuatschín}{Ruèras}{m3, l. 2217f.}\\
\gll A lur' vèv’ inṣ da dá … da magljè \textbf{da} \textbf{lèzs} [...].\\
and then have.\textsc{impf.3sg} \textsc{gnr} \textsc{comp} give.\textsc{inf} {} \textsc{comp} eat.\textsc{inf} \textsc{dat} \textsc{dem.m.pl}\\
\glt `And then one had to feed ... feed them [= the pigs] [...].'
\z

\ea\label{ex:lez5}
\langinfo{Tuatschín}{Sadrún} {m10, l. 1088ff.}\\
\gll  [...] a lu va ju rantau in mél vi dal autar,  álṣò vi [...] dal … cavèstar \textbf{da} \textbf{lèz}, álṣò vi dal … dal bast \textbf{da} \textbf{lèz} scù inṣ di [...].  \\
{...} and then have.\textsc{1sg} \textsc{1sg} bind.\textsc{ptcp.unm} one.\textsc{m.sg} mule over of.\textsc{def.art.m.sg} other well over {} of.\textsc{def.art.m.sg} {} bridle of \textsc{dem.m.sg} that\_is\_to\_say on of.\textsc{def.art.m.sg} {} of.\textsc{def.art.m.sg} packsaddle of \textsc{dem.m.sg} as \textsc{gnr}  say.\textsc{prs.3sg} \\
\glt `[...] and then I bound one mule to the other, well, to [...] the bridle of this one, that is to say to the the packsaddle of that one as one says [...].'
\z

\ea
\label{ex:lez6}
\langinfo{Tuatschín}{}{\citealt[91]{Gadola1935}}\\
\gll Vuṣ Ṣèp Flurín duvrajs nuéta salamantá \textbf{parvia} \textbf{da} \textbf{gljèz} […].\\
\textsc{2sg.pol} \textsc{pn} \textsc{pn} need.\textsc{prs.2sg.pol} \textsc{neg} \textsc{refl.}complain.\textsc{inf} because of \textsc{dem.unm} \\
\glt `You, Sep Flurin, need not complain about that […].'
\z

The pronoun \textit{gljèz} does not refer to entities that have gender; in (\ref{ex:lez7}) and (\ref{ex:lez8}) it refers to object clauses.


\ea
\label{ex:lez7}
\langinfo{Tuatschín}{Sadrún} {m4, l. 452ff.}\\
\gll Ins vèz’ aun tg’ èra dau vidajn pùntgas né trádals; sch’ i sitavan \textbf{gljèz} sau bétg.\\
\textsc{gnr} see.\textsc{prs.3sg} still \textsc{comp} \textsc{pass.aux.impf.3sg} give.\textsc{ptcp.unm} into chisel.\textsc{f.pl} or power\_drill.\textsc{m.pl} whether \textsc{3pl} blow\_up.\textsc{impf.3pl} \textsc{dem.unm} know.\textsc{prs.1sg.1sg} \textsc{neg}\\
\glt `One still can see that chisels or power drills had been used; whether they would blow up I don’t know.'
\z

\ea
\label{ex:lez8}
\langinfo{Tuatschín}{}{\citealt[85]{Berther1998}}\\
\gll «Da tgéj as lu samjau?» «\textbf{Gljèz} vi ju schòn dí da té.»\\
     of what have.\textsc{prs.2sg} then dream.\textsc{ptcp.unm} \textsc{dem.unm} want.\textsc{prs.1sg} \textsc{1sg} 
 certainly tell.\textsc{inf} \textsc{dat} \textsc{2sg}\\
\glt `«What did you dream of then?» «That I will tell you, of course.»'
\z

In some rare cases, \textit{gljèz} is used cataphorically (\ref{ex:cataph1}).

\ea\label{ex:cataph1}
\langinfo{Tuatschín}{Camischùlas}{\DRG{3}{379}}\\
\gll  Quaj èra cèrts sèrvituts. \textbf{Gljèz} sa ju maj tg’ i ò dau històrias parví da quaj. \\
    \textsc{dem.unm} \textsc{cop.impf.3sg} certain.\textsc{m.pl} constraint.\textsc{pl} \textsc{dem.unm}  know.\textsc{prs.1sg} \textsc{1sg} never \textsc{comp} \textsc{expl} have.\textsc{prs.3sg} give.\textsc{ptcp.unm} story.\textsc{f.pl} because of \textsc{dem.unm} \\
\glt `These were certain constraints. I don’t know at all whether there were problems because of that.'
\z

As mentioned before, the \textit{quèl}-series may refer deictically and anaphorically to the referent, whereas the \textit{lèz}-series only refers anaphorically to it. Therefore the question arises as to the difference between the two series in the domain of anaphora.

It has not been possible to obtain an explanation for the difference between these two series from my consultants. However, there are at least two domains where \textit{lèz} is preferred over \textit{quèl}: with topicalized subjects which are located outside the sentence ((\ref{demtop4}); see also (\ref{ex:lez2}) above), and with the preposition \textit{cun} 'with' (\ref{demtop5}). But the \textit{quèl}-series is not excluded from these domains, as (\ref{demtop1}) and (\ref{demtop3}) show.

\ea
\label{demtop4}
\langinfo{Tuatschín}{Sadrún}{m4, L. 511f.}\\
\gll [...] a lu al \textbf{tat}, \textbf{lèz} pinava tiar la tschajna [...].  \\
  {} and then \textsc{def.art.m.sg} grandfather \textsc{dem.m.sg} prepare.\textsc{impf.3sg} to \textsc{def.art.f.sg} dinner  \\
\glt `[...] and then my grandfather, he would prepare dinner [...].'
\z

\ea\label{demtop1}
\langinfo{Tuatschín}{Camischolas}{\DRG{3}{577}}\\
\gll   Als caschnès da mù dus pòsts, \textbf{quèls} numnávani gjajnas.\\
     \textsc{def.art.m.pl} hay\_rack.\textsc{pl} of only two.\textsc{m.pl} post.\textsc{pl} \textsc{dem.m.pl} call.\textsc{impf.3pl.3pl} “gjajnas”.\textsc{f.pl}\\
\glt `The hay racks of only two posts were called \textit{giainas}.'
\z

\ea
\label{demtop5}
\langinfo{Tuatschín}{Sadrún}{m4}\\
\gll Ju a antupau al \textbf{Giari}, a lu sùnd ju juṣ ád alp \textbf{cun} \textbf{lèz}.\\
\textsc{1sg} have.\textsc{prs.1sg} meet.\textsc{ptcp.unm} \textsc{def.art.m.sg} \textsc{pn} and then be.\textsc{prs.1sg} \textsc{1sg} go.\textsc{ptcp.m.sg} to alp with \textsc{dem.m.sg}  \\
\glt `I met Gieri, and then I went to the alpine pasture with him.'
\z

\ea\label{demtop3}
\langinfo{Tuatschín}{Sadrún}{f3, l. 65f.}\\
\gll Ju stòpi ir' á fá gjù \textbf{cun} \textbf{quèls} sé la tégja dal, da Majgals [...].\\
\textsc{1sg} must.\textsc{prs.sbjv.1sg} go.\textsc{inf} \textsc{comp} make.\textsc{inf} down with \textsc{dem.m.pl} up \textsc{def.art.f.sg} alpine\_hut of.\textsc{def.art.m.sg} of \textsc{pln}\\
\glt `I should go and make an appointment with those up there in the alpine hut of the, of Maighels [...].'
\z

\subsubsection{The \textit{quèst}-series}
The \textit{quest}-series is only used when modifying a temporal noun which includes the time of speech; it only functions as a determiner.

\begin{table}
\caption{Demonstratives: the \textit{quest}-series}
\label{demquest}
 \begin{tabular}{llllll}
  \lsptoprule
      gender      & \textsc{m.sg} & \textsc{m.pl} & \textsc{f.sg} & \textsc{f.pl}\\
  \midrule
  determiner  & \textit{quèst} &  \textit{quèsts}  & \textit{quèsta}  & \textit{quèstas}\\
  
  \lspbottomrule
 \end{tabular}
\end{table}

\ea\label{ex:questjamna}
\langinfo{Tuatschín}{Sadrún}{f3, l. 111}\\
\gll Ùsa \textbf{quèst}' \textbf{jamna} vau fatg gròndas turas [...].'\\
now \textsc{dem.f.sg} week have.\textsc{prs.1sg.1sg} do.\textsc{ptcp.unm} big.\textsc{f.pl} tour.\textsc{pl}\\
\glt `Now this week I did long tours [...].'
\z

\ea\label{}
\langinfo{Tuatschín}{Sadrún}{m4, l. 614}\\
\gll Lò, \textbf{quèsta} \textbf{sèra} dòrma lu bigja ajn lò.\\
there \textsc{dem.f.sg} evening sleep.\textsc{imp.2sg} then  \textsc{neg} in there.\\
\glt `Don't sleep up there this evening.'
\z

Temporal nouns may also be modified by the \textit{quèl}- and the \textit{tschèl}-series, but then they exclude the time of speech: \textit{quèsta stad} `this summer' vs \textit{quèla stad} `that summer', or \textit{quèst' jamna} `this week' (\ref{ex:questjamna}) vs \textit{quèl' jamna} `that week'  (\ref{ex:queljamna}).

\ea\label{ex:queljamna}
\langinfo{Tuatschín}{Sadrún}{f3, l. 115f.}\\
\gll Api vau détg èlṣ vagjan fatg ina tura \textbf{tschèl}’ \textbf{jamna} [...].\\  
and\_then have.\textsc{prs.1sg} say.\textsc{ptcp.unm} \textsc{3pl.m} have.\textsc{prs.sbjv.3pl} make.\textsc{ptcp.unm} \textsc{indef.art.f.sg} tour  \textsc{dem.f.sg} week\\
\glt `And then I said that they had done a tour that week [...].'
\z

In the case of `this year', it is usually rendered either by \textit{uòn}, rarely by  \textit{quèst òn}, but \textit{quaj òn} is used to refer to a year that does not include speech time.

\ea\label{}
\langinfo{Tuatschín}{Cavòrgja}{m8, l. 1539f.}\\
\gll [...] \textbf{quaj} \textbf{òn} vèvan nuṣ da fá sé … ina … fòrmazjun nòva [...].\\
{} \textsc{dem.m.sg} year have.\textsc{impf.1pl} \textsc{1pl} \textsc{comp} make.\textsc{inf} up {} \textsc{indef.art.f.sg} {} lineup  new.\textsc{f.sg} [...].\\
\glt `[...] that year we had to do ... a .. new lineup ... [...].'
\z


\subsubsection{The \textit{tschèl}-series}
The \textit{tschèl}-series is usually used explicitly or implicitly in contrast to the \textit{quèl}-series or to the numeral \textit{in/ina} `one' and is best translated by `the other'.\footnote{To use \textit{autar} `other' instead of \textit{tschèl} in such constructions is felt to be more Sursilvan than Tuatschin by the native speakers I have consulted.}

\begin{table}
\caption{Demonstratives: the \textit{tschèl}-series}
\label{demtschel}
 \begin{tabular}{llllll}
  \lsptoprule
    gender        & \textsc{m.sg} & \textsc{m.pl} & \textsc{f.sg} & \textsc{f.pl} & \textsc{unmarked}\\
  \midrule
  determiner  & \textit{tschaj} &  \textit{tschèls}  & \textit{tschèla}  & \textit{tschèlas} & ---\\
  pronoun  & \textit{tschèl} & \textit{tschèls} & \textit{tschèla} & \textit{tschèlas} & \textit{tschaj}\\
  \lspbottomrule
 \end{tabular}
\end{table}

\ea\label{}
\langinfo{Tuataschin}{}{\DRG{4}{598}}\\
\gll \textbf{Ina} fò ajn cuṣchina a \textbf{tschèla} fò sé als létgs.\\
  one.\textsc{f.sg} do.\textsc{prs.3sg} in kitchen.\textsc{f.sg} and \textsc{dem.f.sg} make.\textsc{prs.3sg} up \textsc{def.art.m.pl} bed.\textsc{pl}\\
\glt `One works in the kitchen and the other makes beds.'
\z

\ea\label{}
\langinfo{Tuatschín}{Ruèras}{m1, l. 254ff.}\\
\gll    A quaj gang udéva dad òmaṣdús.  In èra da \textbf{quèla} \textbf{familja}, ad in da \textbf{tschèla}.\\
and \textsc{dem.m.sg} corridor belong.\textsc{impf.3sg} \textsc{dat}  both.\textsc{m.pl} one.\textsc{m.sg} \textsc{cop.impf.3sg} \textsc{dat}  \textsc{dem.f.sg} family and one.\textsc{m.sg} \textsc{dat} \textsc{dem.f.sg}\\
\glt `And this corridor belonged to both [families]. One belonged to this family, and one to the other.'
\z

\ea\label{}
\langinfo{Tuatschín}{Sadrún}{m5}\\
\gll Da \textbf{quèla} vart \textbf{cò} vajn nus la tgèsa còmun\underline{a}la, a da \textbf{tschèla} vart la basèlgja.\\
on \textsc{dem.f.sg} side here have.\textsc{prs.1pl} \textsc{1pl} \textsc{def.art.f.sg} house communal and on \textsc{dem.f.sg} side \textsc{def.art.f.sg} church\\
\glt`On this side here we have the community hall and on the other side the church.'
\z

\ea\label{}
\langinfo{Tuatschín}{Ruèras}{m1, l. 297}\\
\gll [...] vèv' è plaṣchaj vi da \textbf{quaj} vi da \textbf{tschaj} [...].\\
{} have.\textsc{impf.1sg} also pleasure.\textsc{m.sg} over of \textsc{dem.unm} over of \textsc{dem.unm}\\
\glt `[...] [I] also enjoyed doing this and that [...].'
\z

The unmarked pronouns \textit{quaj} and \textit{tschaj}  also fulfil other functions than purely anaphoric ones. In (\ref{ex:quaj:1}), \textit{quaj} has a temporal, in (\ref{ex:tschaj:1}), \textit{tschaj} has a consecutive function, and in (\ref{ex:tschaj:2}) \textit{tschaj} means `otherwise'.

\ea\label{ex:quaj:1}
\langinfo{Tuatschín}{Ruèras} {m1, l. 199ff.}\\
\gll    Quaj èr’ in bal, a quaj dèv’ ins cun in fist, a dèvan quaj ála  landstr\underline{ò}s, \textbf{quaj} mava fòrsa in autò gl antiar sjantarmjazdé, áutar nuét.\\
\textsc{dem.unm} \textsc{cop.impf.3sg} \textsc{def.art.m.sg} ball and \textsc{dem.unm} give.\textsc{impf.3sg} \textsc{gnr} with \textsc{def.art.m.sg} stick and give.\textsc{impf.1pl} \textsc{dem.unm} in.\textsc{def.art.f.sg} main\_road.\textsc{f.sg} \textsc{dem.unm} go.\textsc{impf.3sg} maybe \textsc{indef.art.m.sg} car \textsc{def.art.m.sg} whole afternoon other.\textsc{unm} nothing\\
\glt `This was a ball, and you played it with a stick, and we played this on the main road, at that time only one car would pass by during the whole afternoon, nothing else.
\z

\ea
\label{ex:tschaj:1}
\langinfo{Tuatschín}{Ruèras} {m1, l. 168ff.}\\
\gll    Ad ju èra, vèva … in buéb, in frá è sadaṣgrazjaus tgu vèv’ òtg majns ála Val Milá, ála grépa, ad vèva aun duas sòras, a \textbf{tschaj} èra buép parsul. \\
and \textsc{1sg} \textsc{cop.impf.1sg} have.\textsc{impf.1sg} {} \textsc{indef.art.m.sg} boy \textsc{indef.art.m.sg} brother be.\textsc{prs.3sg} \textsc{refl}.have\_accident.\textsc{ptcp.m.sg} \textsc{rel.1sg} have.\textsc{impf.1sg} eight month.\textsc{m.pl} in.\textsc{def.art.f.sg} valley \textsc{pln} in.\textsc{def.art.f.sg} rock.\textsc{coll} and have.\textsc{impf.1sg} in\_addition two.\textsc{f.pl} sister.\textsc{pl} and \textsc{dem.unm}  \textsc{cop.impf.1sg} boy.\textsc{m.sg} alone.\textsc{m.sg}\\
\glt `And I was, had … a boy, a brother had and accident when I was eight months old, in the Val Milá, in the rocks, and in addition I had two sisters, hence I was the only boy.'
\z 

\ea
\label{ex:tschaj:2}
\langinfo{Tuatschín}{Ruèras} {f4, l. 1936}\\
	\gll  Ábar \textbf{tschaj} sùnd ju adina stada cò tiar purs.\\
but \textsc{dem.unm} be.\textsc{prs.1sg} \textsc{1sg} always \textsc{cop.ptcp.f.sg} here by farmer.\textsc{m.pl}\\
\glt `But otherwise I have always been here working for farmers.'
\z

It is not always clear what function these pronouns have. In (\ref{ex:tschel1}), \textit{quaj} may have a causal (`therefore') or a purposive (`in order to do this') function.

\ea
\label{ex:tschel1}
\langinfo{Tuatschín}{Sadrún}{m4, l. 678f.}\\
\gll  Èl durméva bjè, \textbf{quaj} mava `l sél baun-pégna [...].\\
\textsc{3sg.m} sleep.\textsc{impf.3sg} a\_lot \textsc{dem.unm} go.\textsc{impf.3sg} \textsc{3sg.m} on.\textsc{def.art.m.sg} bench.\textsc{m.sg}-oven.\textsc{f.sg}\\
\glt `He slept a lot and used to go [and sit] on the oven bench [...].'
\z

\subsection{Possessives}
The possessive paradigm shows one case of syncretism: the possessive determiners of the third persons have the same form whether the possessor is singular or plural. The standard Sursilvan form \textit{lur} `their' for third person plural masculine and feminine is not used.

\begin{table}
	\caption{Possessive determiners}
	\label{possdet}
	\begin{tabular}{lllll}
		\lsptoprule
		& \textsc{m.sg} & \textsc{m.pl}  & \textsc{f.sg}  & \textsc{f.pl}\\
		\midrule
		\textsc{1sg}  & \textit{mju}\footnote{In the corpus, there is one occurrence of the standard Sursilvan form /\textit{miw}/ (l.114).}  & \textit{méjṣ} &  \textit{mia} & \textit{miaṣ}\\
		\textsc{2sg} & \textit{tju} & \textit{téjṣ} & \textit{tia} & \textit{tiaṣ}\\
		\textsc{3sg} & \textit{sju} & \textit{séjṣ} & \textit{sia} & \textit{siaṣ}\\
		\textsc{1pl} & \textit{niaṣ} & \textit{nòṣ} & \textit{nòssa} & \textit{nòssaṣ}\\
		\textsc{2pl} & \textit{viaṣ}  & \textit{vòṣ} & \textit{vòssa} & \textit{vòssaṣ}\\
		\textsc{3pl}	& \textit{sju} & \textit{séjṣ}  & \textit{sia} & \textit{siaṣ}\\
		\lspbottomrule
	\end{tabular}
\end{table}

\ea
\label{}
\langinfo{Tuatschín}{Sèlva}{f2, l. 968ff.}\\
\gll [...] cu quèl antschavèv’ ajn cun \textbf{séjs} \textbf{spérts} a tùt quaj tg’ èra.\\
{} when \textsc{dem.m.sg} begin.\textsc{impf.3sg} in with \textsc{poss.3.m.pl} spirit.\textsc{pl}  and all \textsc{dem.unm} \textsc{rel} \textsc{exist.impf.3sg}  \\
\glt `[...] when he started with his spirits and everything that was there.'
\z

\ea
\label{}
\langinfo{Tuatschín}{Camischùlas}{\citealt[90]{Büchli1966}}\\
\gll   […] a privava uschéja als purèts da \textbf{sju} \textbf{fatg}.\\
     {} and deprive.\textsc{impf.3sg} so \textsc{def.art.m.pl} small\_peasant.\textsc{pl} of \textsc{poss.3.m.sg} property\\
\glt `[…] and used in this way to deprive the small peasants of their property.'
\z

The feminine singular possessives \textit{mia} and \textit{tia} usually lose their final \textit{-a} when they precede a noun that starts with a vowel, as in \textit{ti' ònda} `your aunt', or as in (\ref{mi'1}). In the corpus, there are also some cases of the masculine form \textit{mju} that is pronounced \textit{mi'} before vowel (\ref{mi'2}).

\ea
\label{mi'1}
\langinfo{Tuatschín}{Sadrún}{m4, l. 326f.}\\
\gll Lu sùnd ju sadacidjus da … raṣdá in pau ṣur da la ... da \textbf{mi’} \textbf{ufaunza} [...]. \\
then  be.\textsc{prs.1sg} \textsc{1sg}  \textsc{refl}.decide.\textsc{ptcp.m.sg} \textsc{comp} {} talk.\textsc{inf} \textsc{indef.art.m.sg} little over of  \textsc{def.art.f.sg} {} of  \textsc{poss.1sg.f.sg} childhood\\
\glt `Then I decided to ... talk a bit about ... my childhood [..].'
\z

\ea
\label{mi'2}
\langinfo{Tuatschín}{Sadrún}{f3, l. 8}\\
\gll  A… \textbf{mi'} \textbf{ùm} fagèva gè survigiládar [...].  \\
and \textsc{poss.1sg.m.sg} man do.\textsc{impf.3sg} in\_fact supervisor.\textsc{m.sg}\\
\glt `And … in fact, my husband was a supervisor [...].'
\z

Normally, the possessive determiners precede the noun, but in proverbs or sayings they may follow it.

\ea\label{}
\langinfo{Tuatschín}{}{\citealt[100]{Büchli1966}}\\
\gll Cùsch \textbf{buca} \textbf{tia}, scha cùschan buca tùtas.\\
keep\_quiet.\textsc{imp.2sg} mouth.\textsc{f.sg} \textsc{poss.2sg} then keep\_quiet.\textsc{prs.3pl} mouth.\textsc{f.sg} all.\textsc{f.pl}\\
\glt `Keep quiet, then everybody will keep quiet.'
\z

The possessive pronouns take different forms according to whether they occur as the predicate or in another function, as e.g. subject.

\ea\label{}
\langinfo{Tuatschín}{Sadrún}{m5}\\
\gll Quaj cùdisch è \textbf{méjs} / \textbf{téjs} / \textbf{séjs}.\\
\textsc{dem.m.sg} book \textsc{cop.3sg} \textsc{poss.1sg.m.sg} {} \textsc{poss.2sg.m.sg} {} \textsc{poss.3sg.m.sg} \\
\glt `This book is mine / yours / his.'
\z

\ea
\label{}
\langinfo{Tuatschín}{Sadrún}{m5}\\
\gll \textbf{Al mju} / \textbf{Al tju} / \textbf{Al sju} è sén cruna.\\
\textsc{poss.1sg.m.sg} {} \textsc{poss.2sg.m.sg} {} \textsc{poss.3sg.m}.sg \textsc{cop.3sg} on bookshelf.\textsc{m.sg}   \\
\glt `Mine / Yours / His / Hers is on the bookshelf.'
\z


\begin{table}
	\caption{Possessive pronouns}
	\label{posspron}
	\begin{tabular}{lllllllll}
		\lsptoprule
		& \textsc{m.sg} && \textsc{m.pl}  && \textsc{f.sg}  && \textsc{f.pl}\\
		& nom. & pred. & nom. & pred. & nom. & pred. & nom. & pred.\\
		\midrule
		\textsc{1sg}  & \textit{al mju}  & \textit{méjṣ} & \textit{als méjṣ} & \textit{méjṣ} & \textit{la mia} & \textit{mia} & \textit{las miaṣ} & \textit{miaṣ}\\
		\textsc{2sg} & \textit{al tju} & \textit{téjṣ} & \textit{als téjṣ} &\textit{téjṣ} & \textit{la tia} & \textit{tia} & \textit{las tiaṣ} & \textit{tiaṣ}\\
		\textsc{3sg} & \textit{al sju} & \textit{séjṣ} & \textit{als séjṣ}& \textit{séjṣ} & \textit{la sia} & \textit{sia} & \textit{las siaṣ} & \textit{siaṣ}\\
		\textsc{1pl} & \textit{al niaṣ} & \textit{nòṣ} & \textit{als nòṣ} & \textit{nòṣ}  & \textit{la nòssa} & \textit{nòssa} & \textit{las nòssaṣ} & \textit{nòssaṣ}\\
		\textsc{2pl} & \textit{al viaṣ}  & \textit{vòṣ} & \textit{als vòṣ} & \textit{vòṣ} & \textit{la vòssa} & \textit{vòssa} & \textit{las vòssaṣ} & \textit{vòssaṣ}\\
		\textsc{3pl} & \textit{al sju} & \textit{séjṣ} & \textit{als séjṣ} & \textit{séjṣ} & \textit{la sia} & \textit{sia} & \textit{las siaṣ} & \textit{siaṣ}\\
		\lspbottomrule
	\end{tabular}
\end{table}

Regarding third person plural predicative, some speakers prefer to use \textit{dad èls} `of them' instead of \textit{séjs} `theirs', using \textit{séjs} only for singular (`his').

\ea
\label{}
\langinfo{Tuatschín}{Sadrún}{m5}\\
\gll Quaj cùdiṣch è \textbf{dad} \textbf{èls}.\\
\textsc{dem.m.sg} book \textsc{cop.prs.3sg} of \textsc{3pl.m}\\
\glt `This book is theirs.'
\z




\subsection{Indefinites}
The indefinite determiners are \textit{mintga} (+\textsc{C}) / \textit{mintg'} (+\textsc{V})\footnote{Note that /j/ counts as vowel.} `every', \textit{nagín} (m) / \textit{nagina} (f) `no', and \textit{tùt + def.art} / \textit{tùta + bare noun} `all'. \textit{Mintga}, \textit{tùt}, and \textit{tùta} are invariable.

\ea\label{}
\langinfo{Tuatschín}{Surajn}{f5, l. 1313f.}\\
\gll Na na, lu vagnévan nuṣ anav\underline{ù}s, api vèva \textbf{mintg’} \textbf{jamna} zatgí da dá marjanda … da nus [...].\\
no no then come.\textsc{impf.1pl} \textsc{1pl} back and have.\textsc{impf.3sg} every.\textsc{f.sg}  week somebody \textsc{comp} give.\textsc{inf} meal.\textsc{f.sg} {} \textsc{dat} \textsc{1pl}\\
\glt `No, no, we would come back, and then every week there was somebody who would give us a meal [...].'
\z

\ea
\label{}
\langinfo{Tuatschín}{Sadrún}{m6, l. 1327f.}\\
\gll    A nus mavan culs pòrs sé Valtgèva, \textbf{mintga} \textbf{dé} sé a gjù, ju savès raquintá da té quaj.\\
and \textsc{1pl}  go.\textsc{impf.1pl} with.\textsc{def.art.m.pl} pig.\textsc{pl} up \textsc{pln} every day.\textsc{m.sg} up and down  \textsc{1sg}  can.\textsc{cond.1sg}  tell.\textsc{inf}  \textsc{dat}  \textsc{2sg} \textsc{dem.unm}\\
\glt `And we would go up to Valtgèva with the pigs, every day up and down, I could tell you about that.'
\z

\ea
\label{}
\langinfo{Tuatschín}{Cavòrgja}{m7, l. 2230f.}\\
\gll La stat … èra \textbf{nagina} \textbf{scùla}.  \\
\textsc{def.art.f.sg} summer {} \textsc{exist.impf.3sg} no.\textsc{f.sg} school \\
\glt `During summer ... there was simply no school.'
\z

\ea
\label{}
\langinfo{Tuatschín}{Ruèras}{m1, l. 2732}\\
\gll Partgé nuṣ vèvan \textbf{nagíns} \textbf{talafòns}, \textbf{nagín} \textbf{rádjò}, nuét.\\
because \textsc{1pl} have.\textsc{impf.1pl} no.\textsc{m.pl} phone.\textsc{pl} no.\textsc{m.sg} radio nothing\\
\glt `Because we had no phones, no radio, nothing.'
\z

\ea
\label{}
\langinfo{Tuatschín}{Sadrún}{f3, l. 56ff.}\\
\gll  Quaj è pròpi in ljuc ... nù tg’ i vagnéva schau \textbf{tùt} \textbf{la} \textbf{munizjun} tg’ i vèva, sigir.\\
\textsc{dem.unm} \textsc{cop. prs.3sg} exactly \textsc{indef.art.m.sg} place {} where \textsc{rel} \textsc{expl} \textsc{pass.aux.impf.3sg} leave.\textsc{ptcp.unm} all \textsc{def.art.f.sg} munition \textsc{rel} \textsc{expl} have.\textsc{impf.3sg} sure.\textsc{adj.unm}\\
\glt `This is exactly a place ... where they stored all the munition, for sure.'
\z

\ea\label{}
\langinfo{Tuatschín}{Surajn}{f5, l. 1303f.}\\
\gll  [...] api stèvan nus vagní á tgèsa ád ... á múlgjar \textbf{tùt} \textbf{las} \textbf{tgauras}.\\
{} and must.\textsc{impf.1pl} \textsc{1pl} come.\textsc{inf} to house.\textsc{f.sg} and {} \textsc{purp} milk.\textsc{inf} all \textsc{def.art.f.pl} goat.\textsc{pl}\\
\glt `[...] and then we had to go home and milk all the goats.'
\z

\ea
\label{}
\langinfo{Tuatschín}{Sadrún}{f3, l. 41f.}\\
\gll A sjantar òi gju nùm … i [...] midjan ò \textbf{tùt} \textbf{als} \textbf{muossavias}.\\
and after have.\textsc{prs.3sg.expl} have.\textsc{ptcp.unm} name.\textsc{m.sg} {} \textsc{expl}  {} change.\textsc{prs.sbjv.3pl} out all \textsc{def.art.m.pl} signpost.\textsc{pl}\\
\glt `And after this, one had to … [...] they would replace all the signposts.'
\z

As mentioned above, \textit{tùta} is used with bare nouns, is invariable and restricted to singular reference.

\ea\label{}
\langinfo{Tuatschín}{Sadrún}{m6, l. 1368ff.}\\
	\gll    Ajn \textbf{tùta} \textbf{cas} mia, mia mùma a la mùma da mju còl\underline{è}ga tg’ èra è cun mé … vèvan stju gidá nus [...].\\
in all case.\textsc{m.sg} \textsc{poss.1sg.f.sg}  \textsc{poss.1sg.f.sg} mother and \textsc{def.art.f.sg} mother of  \textsc{poss.1sg.m.sg} mate \textsc{rel}  \textsc{cop.impf.3sg} also with \textsc{1sg} {}  have.\textsc{impf.3pl} must.\textsc{ptcp.unm} help.\textsc{inf} \textsc{1pl}\\
\glt `Anyhow my, my mother and the mother of my mate who was with me … had had to help us [...].'
\z

\ea\label{}
\langinfo{Tuatschín}{Ruèras}{f4, l. 2045f.}\\
	\gll [..] api lu mavan ins \textbf{tùta} \textbf{stat}, lò vèva da fá quaj.\\
{} and then go.\textsc{impf.3sg.euph} \textsc{gnr} whole summer.\textsc{f.sg} there have.\textsc{impf.1sg} \textsc{comp} do.\textsc{inf} \textsc{dem.unm}\\
\glt `[...] and then we would go for the whole summer, there I had to do that.'
\z

In \citet[62]{Büchli1966} there is an occurrence of \textit{tùts} (\ref{ex:tutstiars}), i. e of variable \textit{tùt}, but this is an obsolescent construction. In my oral corpus, there is no example of this construction.

\ea
\label{ex:tutstiars}
\langinfo{Camischùlas}{Tuatschín}{\citealt[14]{Büchli1966}}\\
\gll Ad èla ò piau li èl \textbf{tùts} \textbf{tiars} [...].\\
and \textsc{3sg.f} have.\textsc{prs.3sg} pay.\textsc{ptcp.unm} \textsc{dat} \textsc{3sg.m.} all\textsc{.m.pl} animal.\textsc{pl}\\
\glt `And she paid him all the animals [...].'
\z

The indefinite pronouns are \textit{anzatgé(j) / zatgé(j)} `something', \textit{x-zatgé(j)} `something, anything', \textit{finadín (m) / finadina (f)} `everybody (without exception)', \textit{mintgín} `everybody', \textit{nagín} `nobody', \textit{nuét} `nothing',
 \textit{tùt} `everything', \textit{tùtas} (f) / \textit{tùts} (m) `all', and \textit{zatgí} `somebody'.
 

\ea
\label{}
\langinfo{Tuatschín}{Camischùlas}{f6, l. 740ff.}\\
\gll    Api èra la sòra òra uschéja … avaun niaṣ ésch ad ò spatgau a spatgau tòca la audi \textbf{anzatgéj} [...].\\
and \textsc{cop.impf.3sg} \textsc{def.art.f.sg} nun out so {} in\_front\_of \textsc{poss.1pl.m.sg} door and have.\textsc{prs.3sg} wait.\textsc{ptcp.unm} and wait.\textsc{ptcp.unm} until \textsc{3sg.f} hear.\textsc{prs.sbjv.3sg} something\\
\glt `And then the nun was out [on the corridor] like this ... in front of our door, waiting and waiting until she would hear something  [...].'
\z

\ea
\label{}
\langinfo{Tuatschín}{Camischùlas}{f6, l. 753ff.}\\
\gll [...] api ṣaj stau in’ ura da ṣchubargè né fá ò cul fiar né \textbf{x-zatgéj} luvrá palas sòras [...].\\
{} and be.\textsc{prs.3sg} \textsc{cop.ptcp.unm} one.\textsc{f.sg} hour \textsc{comp} clean.\textsc{inf} or do.\textsc{inf} out with.\textsc{def.art.m.sg} iron or anything do.\textsc{inf} for.\textsc{def.art.f.pl} nun.\textsc{pl} \\
\glt `[...] and then we had to clean for one hour or iron or do something else for the nuns [...].'
\z

\ea
\label{}
\langinfo{Tuatschín}{Ruèras}{m1, l. 243}\\
\gll Tùts stuèvan spargnè. \textbf{Finadín}. Vèvan nagíns… réhs.\\
all.\textsc{m.pl} must.\textsc{impf.3pl} save.\textsc{inf} everyone have.\textsc{impf.1pl} no.\textsc{m.pl} rich.\textsc{pl}\\
\glt `Everyone had to save. Absolutely everyone. We had no … rich people.'
\z

\ea
\label{}
\langinfo{Tuatschín}{Ruèras}{m3, l. 2102f.}\\
	\gll Quaj … ins stèva vagní … \textbf{mintgín} sén sju quántum [...].\\
\textsc{dem.unm} {} \textsc{gnr} must.\textsc{impf.3sg} come.\textsc{inf} {} everyone.\textsc{m.sg} on \textsc{poss.3sg.m.sg} amount\\
\glt `This ... one had to reach ... everyone their amount [...].'
\z

\ea
\label{}
\langinfo{Tuatschín}{Sadrún}{f3, l. 91}\\
\gll  Cò angj\underline{ù} va ju la finala \textbf{nagíns}.  \\
here in\_down have.\textsc{prs.1sg} \textsc{1sg} \textsc{def.art.f.sg} end no.\textsc{m.pl}  \\
\glt `In the end I don’t have any down here.'
\z

\ea
\label{}
\langinfo{Tuatschín}{Sadrún}{f3, l. 59}\\
\gll  Ad ùṣ è quaj \textbf{nuét} dal tùt.   \\
and now \textsc{cop.prs.3sg} \textsc{dem.unm} nothing of.\textsc{def.art.m.sg} all\\
\glt `And now there is nothing of all that [left].'
\z

\ea
\label{}
\langinfo{Tuatschín}{Sadrún}{f3, l. 131}\\
\gll Laṣ vaschnauncas lajn í \textbf{tùt} ajn décad\underline{è}nza.\\
\textsc{def.art.f.pl} municipality.\textsc{pl} let.\textsc{prs.3pl} go.\textsc{inf} all in decadence.\textsc{f.sg}  \\
\glt `The municipalities let everything go into decline.'
\z

\ea
\label{}
\langinfo{Tuatschín}{Ruèras}{m1, l. 243}\\
\gll    \textbf{Tùts} stuèvan spargnè.\\
all.\textsc{m.pl} must.\textsc{impf.3pl} save.\textsc{inf}\\
\glt `Everyone had to save.'
\z

\ea
\label{}
\langinfo{Tuatschín}{Surajn}{f5, l. 1313f.}\\
 \gll Na na, lu vagnévan nuṣ anav\underline{ù}s, api vèva mintg’ jamna \textbf{zatgí} da dá marjanda … da nus [...].\\
 no no then come.\textsc{impf.1pl} \textsc{1pl} back and have.\textsc{impf.3sg} every.\textsc{f.sg}  week somebody \textsc{comp} give.\textsc{inf} meal.\textsc{f.sg} {} \textsc{dat} \textsc{1pl}\\
 \glt `No no, we would then come back, and then every week there was somebody who would give us a meal [...].'
 \z 
 
\textit{Adatgí} as a dative indefinite pronoun is found in the following example; note that nowadays this form is obsolete.

\ea\label{}
\langinfo{Tuatschín}{}{\DRG{2}{154}}\\
\gll  \textbf{Adatgí} plaj barba, \textbf{adatgí} barbís, \textbf{adatgí} gjùta, \textbf{adatgí} ris. \\
\textsc{dat}.some please.\textsc{prs.3sg} beard.\textsc{f.sg}  \textsc{dat}.some moustache.\textsc{m.sg} \textsc{dat}.some pearl\_barley.\textsc{f.sg} \textsc{dat}.some rice.\textsc{m.sg} \\
\glt `Some like beards, some moustaches, some pearl barley, some rice.'
\z

The combination of \textit{anzatgé} `something' with \textit{in / ina} `indefinite article' does not function as a quantifier, but has a comparative meaning, best translated by `a kind of', as in (\ref{ex:antzatgéin}).

\ea\label{ex:antzatgéin}
\langinfo{Tuatschín}{Camischùlas}{\DRG{3}{580}}\\
\gll   Tùt als pòsts vèvan \textbf{anzatgé} \textbf{in} crap ṣutajn.\\
all \textsc{def.art.m.pl} pal.\textsc{pl} have.\textsc{impf.3pl} something \textsc{indef.art.m.sg} stone under\_in\\
\glt `All hay rack pals had something like a stone under them.'
\z

\subsection{Quantifiers}

\subsubsection{Numerals}
Cardinal numbers are found in \tabref{tab:cardnum1} and \tabref{tab:cardnum2}.


\begin{table}
\caption{Cardinal numerals (first part)} 
\label{tab:cardnum1}
 \begin{tabular}{llllll}
  \lsptoprule
 &&0&\textit{nula}\\
1&\textit{in}&11&\textit{indiṣch}&21&\textit{véntgín}\\
2&\textit{duṣ, duaṣ}&12&\textit{dùdiṣch}&22&\textit{véntgadúṣ}\\
3&\textit{trajṣ}&13&\textit{trèdiṣch}&28&\textit{véntg\underline{ò}tg}\\
4&\textit{quátar}&14&\textit{quitòrdiṣch}&30 & \textit{trènta}\\
5&\textit{tschun}&15&\textit{quindiṣch} & 40 & \textit{curònta}\\
6&\textit{siṣ}&16&\textit{sédaṣch/sédiṣch}& 50 & \textit{tschuncònta}\\
7&\textit{sjat}&17&\textit{gisját} & 60 & \textit{sissònta}\\
8&\textit{òtg}&19& \textit{ṣchòtg} & 70 & \textit{sjatònta}\\
9&\textit{nùv}&19& \textit{ṣchèniv} & 80 & \textit{òtgònta}\\
10&\textit{déjṣch}&20&\textit{végn} & 90 & \textit{navònta}\\
  \lspbottomrule
 \end{tabular}
\end{table}

\begin{table}
	\caption{Cardinal numerals (second part)} 
	\label{tab:cardnum2}
	\begin{tabular}{llll}
		\lsptoprule
		100&\textit{tschian} & 1.000 & \textit{mili}/\textit{méli}\\
		200&\textit{dúatschian} & 2.000 & \textit{duamili}\\
		300&\textit{tráj(a)tschian} & 3.000 & \textit{trajamili}\\
		400&\textit{quátartschian} & 4.000 &  \textit{quátermili}\\
		1.000 & \textit{in maljún} & 2.000.000 & \textit{dus maljúnṣ}\\
		\lspbottomrule
	\end{tabular}
\end{table}

From one hundred onwards, the hundreds are linked by the conjunction \textit{a / ad} `and' from one to twenty: \textit{tschian ad òtg} `108', \textit{duatschian a ṣchèniv} `219', \textit{sjattschian a végn} `720', but \textit{nùftschian trèntadús} `932'.

The numerals that follow \textit{mili/méli} `thousand' and \textit{maljún} 'million' are also linked by \textit{a} `and': \textit{méli ad òtgtschian òtgòntasjat} `1887' (line 332f.) or \textit{dus maljúnṣ a trajatschian a végn} `2.320'.

\ea
\label{}
\langinfo{Tuatschín}{}{\DRG{6}{728}}\\
\gll  \textbf{In} ufaun è pauc, \textbf{duṣ} è drètg, \textbf{trajṣ} è strètg, \textbf{quátar} è fula, \textbf{tschun} è paluna a ṣbaluna.\\
one.\textsc{m.sg} child \textsc{cop.prs.3sg} little two.\textsc{m.pl} \textsc{cop.prs.3sg} all\_right three \textsc{cop.prs.3sg} narrow.\textsc{adj.unm} four \textsc{cop.prs.3sg} crowd.\textsc{f.sg} five \textsc{cop.prs.3sg} pile.\textsc{f.sg} and  collapse.\textsc{prs.3sg}\\
\glt `One child is little, two are all right, three are narrow, four are a crowd, five are a lot that collapses.'
\z

\ea\label{}
\langinfo{Tuatschín}{Sadrún}{f3, l. 94ff.}\\
\gll  [...] quaj va\footnotemark{} sé sén \textbf{dua} \textit{mili} \textbf{a} \textbf{trajtschian} a taunts mètarṣ ṣur mar. \\
{} \textsc{dem.unm} go.\textsc{prs.3sg} up on two thousand and three\_hundred and  so\_many.\textsc{m.pl} metre.\textsc{pl} above sea.\textsc{f.sg}\\
\glt `[...] this goes up until 2'300 metres or so above sea level.'\footnotetext{\textit{va} is a standard Sursilvan form; the Tuatschin form is \textit{vò}.}
\z

The cardinal numerals \textit{dua} and  \textit{traj(a)} are used with \textit{tschian} `hundred' and \textit{mili} `thousand' (see \tabref{tab:cardnum2}), and also with some collective nouns.

\ea
\label{}
\langinfo{Tuatschín}{Sadrún}{m5}\\
\gll Èl ò cumprau \textbf{dua/traja} \textbf{pèra} cazès.\\
\textsc{3sg.m} have.\textsc{prs.3sg} \textsc{ptcp.unm} two/three pair.\textsc{coll} shoe.\textsc{m.pl}\\
\glt `He bought two / three pairs of shoes.'
\z

\ea
\label{}
\langinfo{Tuatschín}{}{\citealt[28]{Gartner1910}}\\
\gll    ɛləz ɔn \textbf{traj} \textbf{pɛra} ʨɔmbəs […].\\
    \textsc{3pl.f} have.\textsc{prs.3pl} three pair.\textsc{coll} leg.\textsc{pl}\\
\glt `They [the ants] have three pairs of legs.'
\z

The fractions occurring in the corpus are \textit{quart} `quart', \textit{miaz}, \textit{mjasa} ‘half’, \textit{antir, -a} ‘whole’; `both' is rendered by \textit{(d)òmaṣdús/(d)òmaṣdúas}.

\ea\label{}
\langinfo{Tuatschín}{Sadrún}{m4, l. 446f.}\\
\gll A quaj è ina rùsna, ò tgé pù qual' èssar … \textbf{in} \textbf{mètar} \textbf{a} \textbf{miaz} … lada [...].\\
and \textsc{dem.unm} \textsc{cop.prs.3sg} \textsc{indef.art.f.sg} hole oh what  can.\textsc{prs.3sg} \textsc{dem.f.sg} be.\textsc{inf} {} one.\textsc{m.sg} metre and half {} large.\textsc{f.sg}\\
\glt `And there is a cave, oh how big may it be …, one and a half metres … wide [...].'
\z

\ea
\label{}
\langinfo{Tuatschín}{Ruèras} {m3, l. 2013}\\
	\gll Quaj è \textbf{in}’ ur’ \textbf{a} \textbf{mjasa} par vièdi.  \\
\textsc{dem.unm} \textsc{cop.prs.3sg} one.\textsc{f.sg} hour and half.\textsc{f.sg} for trip.\textsc{m.sg}\\
\glt `It takes one and a half hours per trip.'
\z

\ea
\label{}
\langinfo{Tuatschín}{Sadrún}{f3, l. 47f.}\\
\gll  A sjantar ... òni ampustau tùt nùfs … pr \textbf{l’} \textbf{antira} \textbf{val}.  \\
and after {} have.\textsc{prs.3pl.3pl} order.\textsc{ptcp.unm} all new.\textsc{m.pl} {} for \textsc{def.art.f.sg} whole valley \\
\glt `And then ... they ordered all new ones … for the entire valley.'
\z

\ea\label{}
\langinfo{Tuatschín}{Ruèras}{m10, l. 1091f.}\\
\gll  [...] api ah sùnd jus cun \textbf{dòmaṣdús}.  \\
{} and eh be.\textsc{prs.1sg} go.\textsc{ptcp.m.sg} with both.\textsc{m.sg} \\
\glt `[...] and then eh I left with both [mules].'
\z

Ordinal numbers have special forms from 1-4; from 5 onwards they take the suffix \textit{-aval /  -avla}: \textit{amprém, ampréma} `first', \textit{sacund,-a} / \textit{zacund,-a} `second', \textit{tiarz, tjarza} `third', \textit{quart, -a} `fourth', \textit{tschunaval, tschunavla}  `fifth', \textit{déjiṣchaval, déjṣcha-vla} `tenth', \textit{ véntgaduṣavel, véntgaduṣavla} `twenty-second', and so on. `Last' is rendered by \textit{davùs, davùsa}.

\ea\label{}
\langinfo{Tuatschín}{}{\DRG{6}{421}}\\
\gll  dá cun flugjals la \textbf{zacunda} jèda.\\
    give.\textsc{inf} with flail.\textsc{m.pl} \textsc{def.art.f.sg} second time \\
\glt `beat with flails for the second time'
\z

\subsubsection{Other quantifiers}
Quantifying determiners are \textit{ampau} `a bit', \textit{aungatáun} `as much as, once as much', \textit{anqual / inqual} `some', \textit{bjè} `many', \textit{massa / mass'} `many, lots', and \textit{zatgé(j) / zitgé(j)} `some' (literally `something'). All these determiners are invariable except \textit{bjè}, which is invariable or which agrees in gender and number with the noun it modifies.


\ea\label{}
\langinfo{Tuatschín}{Ruèras}{m2, l. 1550f.}\\
\gll    [...] vignévan quels lu vi a dèvan … matévan ajnagjù \textbf{ampau} \textbf{raps} [...].\\
 {} come.\textsc{impf.3pl} \textsc{dem.m.pl} then over and give.\textsc{impf.3pl} {} put.\textsc{impf.3sg} into\_and\_down a\_bit cent.\textsc{m.pl}\\
\glt `[...] they would come over and give ... put into the piggy bank some cents [...].'
\z

\ea\label{}
\langinfo{Tuatschín}{}{\DRG{6}{546}}\\
\gll  Quaj frust ò dau uòn \textbf{aungatáun} \textbf{fajn}. \\
\textsc{dem.m.sg} meadow have.\textsc{prs.3sg} give.\textsc{ptcp.unm} this\_year as\_much hay \\
\glt `This meadow has produced as much hay [as last year].'
\z

Instead of \textit{aungatáun} it is also possible to say \textit{aun in jèda taun} `still one time as much'.

\ea
\label{}
\langinfo{Tuatschín}{Sadrún}{f3, l. 162}\\
\gll  \textbf{Anqual} \textbf{jèda} … drùvi halt … da dí.\\
some time.\textsc{f.sg} {} must.\textsc{prs.3sg.expl} just {} \textsc{comp} say.\textsc{inf}\\
\glt `It is sometimes … just necessary … to say [it].'
\z

\ea
\label{}
\langinfo{Tuatschín}{Sadrún}{m4, l. 537f.}\\
\gll  [...] a lò vòu schòn ah … gju \textbf{inqual} \textbf{tèma}.  \\
{} and there have.\textsc{prs.1sg.1sg} really eh {} have.\textsc{ptcp.unm} some fear.\textsc{f.sg} \\
\glt `[...] and there I was eh sometimes afraid.'
\z

\ea
\label{}
\langinfo{Tuatschín}{Sadrún}{m4, l. 364}\\
\gll  [...] ad ò lu stu í á fá cura, \textbf{mass}’ \textbf{òns} [...].\\
{} and have.\textsc{prs.3sg} then must.\textsc{ptcp.unm} go.\textsc{inf} \textsc{comp} make.\textsc{inf} treatment.\textsc{f.sg} many year.\textsc{m.pl}\\
\glt `[...] and for many years he had to go to a health resort [...].'
\z

\ea
\label{}
\langinfo{Tuatschín}{Zarcúns}{m2, l. 1526f.}\\
\gll    [...] qu’ èra \textbf{massa} \textbf{gjuvantétgna} cò ála val. \\
{} \textsc{dem.unm} \textsc{cop.impf.3sg} lots youth.\textsc{f.sg}  here in.\textsc{def.art.f.sg} valley\\
\glt `[...] there were a lot of young people here in the valley.'
\z

\ea
\label{}
\langinfo{Tuatschín}{Sadrún}{m4, l. 437}\\
\gll Ad i aun èra lò, \textbf{zatgé} \textbf{rastònza} ṣè aun lò [...].\\
and \textsc{expl} still \textsc{cop.impf.3sg} there some remnant.\textsc{f.sg} \textsc{cop.prs.3sg} still there\\
\glt `And there is also, there still are some remnants there [...].'
\z

\ea\label{}
\langinfo{Tuatschín}{Sadrún}{m4, l. 471f.}\\
\gll  Lèdṣ vèva lu dau \textbf{zatgéj} \textbf{étg} dad úndṣchar ajn [...].\\
\textsc{dem.m.sg} have.\textsc{impf.3sg} then  give.\textsc{ptcp.unm} some ointment.\textsc{m.sg} \textsc{comp} oil.\textsc{inf} in\\
\glt `He had given [him] some ointment to rub in [...].'
\z

As mentioned above, the determiner \textit{bjè} is either invariable or takes the plural forms \textit{bjèras} (f) or \textit{bjèrs} (m).

\ea
\label{}
\langinfo{Tuatschín}{Ruèras}{m10, l. 995f.}\\
\gll  Lu mávani bjè gjùn Cavòrgja, cò da quaj gròn uaul vagnéva \textbf{bjè} \textbf{lèna}.\\
then go.\textsc{impf.3pl.3pl} much down\_in \textsc{pln} here from \textsc{dem.m.sg} big forest come.\textsc{impf.3sg} much wood.\textsc{coll}  \\
\glt `Then they often went down to Cavorgia, there came much wood from that big forest there.'
\z

\ea
\label{}
\langinfo{Tuatschín}{Sadrún}{m4, l. 394ff.}\\
\gll Quaj èra ina munièssa da Gljòn, la sòra Paulina, quèla ò dau \textbf{bjè} \textbf{òns} scùlèta cò.\\
\textsc{dem.unm} \textsc{cop.impf.3sg} \textsc{indef.art.f.sg} nun from \textsc{pln} \textsc{def.art.f.sg} Sister \textsc{pn} \textsc{dem.f.sg}  have.\textsc{prs.3sg} give.\textsc{ptcp.unm} many year.\textsc{m.pl} nursery\_school.\textsc{f.sg} here\\
\glt `That was a nun from Glion, Sister Paulina, she taught for many years at the nursery school here.'
\z

\ea
\label{}
\langinfo{Tuatschín}{Sadrún}{m5}\\
\gll \textbf{Bjèras} \textbf{fèmnas} òn zambargjau.\\
many.\textsc{f.pl} woman.\textsc{pl} have.\textsc{prs.3pl} do\_crafting.\textsc{ptcp.unm}\\
\glt `Many women did crafting.'
\z

In comparative (\ref{ex:plebje}) and superlative (\ref{ex:laplebje}) constructions, \textit{bjè} is treated like an adjective (see § 3.3.2 below). 

\ea
\label{ex:plebje}
\langinfo{Tuatschín}{Sadrún}{m6, l. 908ff.}\\
	\gll [...] la banadiczjún, lèza vèva \textbf{plé} \textbf{bjè} \textbf{fòrza} sélas stréjas.\\
{} \textsc{def.art.f.sg} blessing \textsc{dem.f.sg} have.\textsc{impf.3sg} much more power.\textsc{f.sg} on.\textsc{def.art.f.pl} witch.\textsc{pl}\\
\glt `[...] the blessing had much more power over the witches.'
\z

\ea
\label{ex:laplebje}
\langinfo{Tuatschín}{Ruèras}{m3, l. 2197f.}\\
\gll [...] quaj duvrava \textbf{la} \textbf{plé} \textbf{bjè} \textbf{lèna} … quaj stèva buglí.  \\
{} \textsc{dem.unm} need.\textsc{impf.3sg} \textsc{def.art.f.sg} more much firewood.\textsc{coll} {} \textsc{dem.unm} must.\textsc{impf.3sg} boil.\textsc{inf}\\
\glt `[...] that required the highest quantity of firewood ... the whey had to boil.'
\z

As a pronoun, \textit{bjè} takes the form \textit{bjèrs} (m) and \textit{bjèras} (f).

\ea
\label{}
\langinfo{Tuatschín}{Zarcúns}{m2, l. 1628f.}\\
	\gll    [...] \textbf{bjèras} schavan lu è bétga vagní ajn [...].\\
{} many.\textsc{f.pl} let.\textsc{impf.3pl} then also \textsc{neg} come.\textsc{inf} in\\
\glt `[...] many [young women] wouldn’t let [the young man] come in [...].'
\z

\ea
\label{}
\langinfo{Tuatschín}{Sadrún}{m4, l. 487ff.}\\
\gll  [...] ábar dals, méjs gjaniturs vajn nuṣ ùssa bégja détg Vus, ábar i éra \textbf{bjèrs} tga… òn détg tòcan, gè práctisch adina Vus dals gjaniturs.  \\
{} but \textsc{dat.def.art.m.pl} \textsc{poss.1sg.m.pl} parent.\textsc{pl} have.\textsc{prs.1pl} \textsc{1pl} now \textsc{neg} say.\textsc{ptcp.unm} \textsc{2pl} but \textsc{expl} \textsc{exist.impf.3sg}  many.\textsc{m.pl} \textsc{rel} have.\textsc{3pl} say.\textsc{ptcp.unm} until yes practically always \textsc{2pl} \textsc{dat.def.art.m.pl} parent.\textsc{pl}\\
\glt `[...] but to the, my parents we now never said \textit{Vus}, but there were many who have said until, well practically always \textit{Vus} to their parents.'
\z

Non-countable quantifying nouns such as \textit{in téc} `a bit', \textit{in tsch\underline{ù}pal} `a lot' as well as countable quantifying nouns such as \textit{ina butèglja} `a bottle' or \textit{duas butègljas} `two bottles' are used without partitive preposition.

\ea
\label{}
\langinfo{Tuatschín}{Sadrún}{m4, l. 566ff.}\\
\gll  Ò lò vòu fòrza schòn è survagnú \textbf{in} \textbf{téc} \textbf{quajda} d' í par crapa [...].\\
down there  have.\textsc{prs.1sg.1sg} maybe really also get.\textsc{ptcp.unm} \textsc{indef.art.m.sg} bit desire.\textsc{f.sg} \textsc{comp} go.\textsc{inf} for stone.\textsc{coll} \\
\glt `Out there I might have started enjoying looking for stones a bit [...].'
\z

\ea
\label{}
\langinfo{Tuatschín}{Ruèras}{m10, l. 1178f.}\\
\gll  Ad uschéja vès ju, savès ju raquintá \textbf{in} \textbf{tsch\underline{ù}pal} \textbf{èvènimajnts} tg’ èn schabagjaj cun quèls méls.\\
and so have.\textsc{cond.1sg} \textsc{1sg} can.\textsc{cond.1sg} \textsc{1sg} tell.\textsc{inf} \textsc{indef.art.m.sg} lot incident.\textsc{m.pl} \textsc{rel} be.\textsc{prs.3pl} happen.\textsc{ptcp.m.pl} with \textsc{dem.m.pl} mule.\textsc{pl}\\
\glt `And so I would, I could recount a lot of incidents that happened with these mules.'
\z

\ea
\label{}
\langinfo{Tuatschín}{Zarcúns}{m2, l. 1575f.}\\
\gll    Quèl tga gartagjava … survagnév’ ina \textbf{butèglja} \textbf{vin}.\\
\textsc{dem.m.sg} \textsc{rel} succeed.\textsc{impf.3sg} {} get.\textsc{impf.3sg} \textsc{indef.art.f.sg} bottle wine.\textsc{m.sg}\\
\glt `The person who succeeded would get a bottle of wine.'
\z

Further examples of countable quantifying nouns occurring in the corpus are \textit{in glas aua} 'a glass of water' (l. 2078), \textit{in matg flurs} `a bunch of flowers' (l. 1820), \textit{in pèr jamna}s `a couple of weeks' (l. 1471), \textit{mju quantum vacas} `my amount of cows' (l. 2073f.), \textit{in gròn tòc prau} `a big piece of meadow' (\citet[121]{Büchli1966}).

Furthermore, \textit{bjè} is used adverbially (\ref{ex:bjeadv}) meaning `often'; it can also be nominalised and corresponds to `much, many' (\ref{quantnom}) or `mostly' (\ref{ex:albje2}).

\ea
\label{ex:bjeadv}
\langinfo{Tuatschín}{Ruèras}{m1, l. 293ff.}\\
\gll    A gju quèl al plé gròn plaṣchaj … da … surprèndar lavurs da maridur a da májstar, a mava plé \textbf{bjè} sén gljèz.\\
and have.\textsc{ptcp.unm} \textsc{dem.m.sg} \textsc{def.art.m.sg} more big.\textsc{m.sg.unm} pleasure {} of {} take\_over.\textsc{inf} job.\textsc{f.pl} of bricklayer.\textsc{m.sg} and of joiner.\textsc{m.sg} and  go.\textsc{impf.1sg} more often on \textsc{dem.unm}\\
\glt `And had the greatest pleasure … to take over bricklayers’s or joiners’ jobs, and I did more often that [kind of work.]'
\z

\ea\label{quantnom}
\langinfo{Tuatschín}{}{\DRG{2}{386}}\\
\gll    \textbf{Al} \textbf{bjè} fò bétga plajn.\\
     \textsc{def.art.m.sg} much make.\textsc{prs.3sg} \textsc{neg} full\\
\glt `A big quantity does not fill the stomach.'
\z

\ea
\label{ex:albje2}
\langinfo{Tuatschín}{Sadrún}{m9, l. 1904f.}\\
	\gll [...] \textbf{al} \textbf{bjè} mavan nus sémplamajn lò nuca nuṣ vèvan còlègs [...].\\
{} \textsc{def.art.m.sg} much go.i\textsc{mpf.1pl} \textsc{1pl} simple.\textsc{f.sg.adv} there where.\textsc{rel} \textsc{1pl} have.\textsc{impf.1pl} mate.\textsc{m.pl}\\
\glt `[...] we would mostly go simply where we had friends [...].'
\z




\subsubsection{The construction \textit{tùt tga} and similar}
The construction with an indefinite noun or noun phrase and similar followed by the relative pronoun \textit{tga} is common to all Romansh varieties except for Puter, and has been described by \citet[185-204]{Linder1987}.

In the oral corpus, only three exampes have been found.

\ea
\label{magaritga}
\langinfo{Tuatschín}{Sadrún}{m4, l. 555}\\
\gll A  \textbf{magari} \textbf{tga} pudèvan lu bigja … vidajn [...]. \\
and sometimes \textsc{rel} can.\textsc{impf.3pl} then \textsc{neg} {} in\\
\glt `Now sometimes they couldn’t manage to come … into [the \textit{stiva} and sleep on hay] [...].'
\z

\ea
\label{}
\langinfo{Tuatschín}{Sadrún}{m3, l. 2063f.}\\
	\gll Quaj \textbf{nagín} \textbf{tga} [...] lèva bétga fá.\\
	 \textsc{dem.unm} nobody \textsc{rel} {}  want.\textsc{impf.3sg} \textsc{neg} do.\textsc{inf}\\
\glt `Nobody [...] would refuse to do it.'
\z

\ea
\label{}
\langinfo{Tuatschín}{Sadrún}{m3, l. 2075f.}\\
	\gll [...] quaj \textbf{tùt} \textbf{tga} mungèva [...].  \\
{} \textsc{dem.unm} all \textsc{rel} milk.\textsc{impf.3sg}\\
\glt `[...] here everybody had to milk [...].'
\z

What is special about this construction is that the \textit{tga}-phrase looks like a relative clause, but in fact this is not the case. In (\ref{magaritga}), \textit{magari} `sometimes' is not the antecedent of a relative clause, but is a time adverb that belongs to the main clause.

It has not yet been possible to determine the exact function of this construction, but in any case it introduces new information and emphasizes the situation referred to (see \citet[195-198]{Linder1987}). However, it is not a general topic construction since it is limited to a few words, especially to indefinites in subject function (\tabref{indefinitetga}).

\begin{table}
	\caption{List of indefinites and others + \textit{tga}}
	\label{indefinitetga}
	\begin{tabular}{ll}
		\lsptoprule
				\textit{bétg in tga} & `not one'\\
		\textit{bjèrs tga} & `many'\\
		\textit{gnang in tga} & `not even one'\\
		\textit{magari tga} & `sometimes'\\
		\textit{mintga N tga} & `every'\\
		\textit{mintgín tga} & `everybody'\\
		\textit{nagín tga} & `nobody'\\
		\textit{paucs tga} & `not many'\\
		\textit{tùt tga} & `everything, everybody'\\
		\textit{tùtas tga} & `all (f.)'\\
		\textit{tùts tga} & 'all (m.)'\\
		\lspbottomrule
	\end{tabular}
\end{table}

For example, it is not possible to say

\ea\label{}
\langinfo{Tuatschín}{Sadrún}{m5}\\
\gll \textbf{Al} \textbf{Gjòn} *\textbf{tg}' ò angulau als raps.\\
\textsc{def.art.m.sg} \textsc{pn} \textsc{rel} have.\textsc{prs.3sg} steal.\textsc{ptcp.unm} \textsc{def.art.m.pl} cent.\textsc{pl}\\
\glt `It is Gion who stole the money.'
\z

But one can find a construction involving the expletive pronoun \textit{i} and the copula which has the same functions and the same restrictions, and of which the \textit{tùt tga} construction could be an ellipsis (see also  \citet[201]{Linder1987}).

\ea\label{}
\langinfo{Tuatschín}{Sadrún}{m5}\\
\gll \textit{I} \textbf{è} \textit{gnang} \textit{in} \textbf{tg}' è vagnús tiar nus.\\
\textsc{expl} \textsc{cop.prs.3sg} not\_even one.\textsc{m.sg} \textsc{rel} be.\textsc{prs.3sg} come.\textsc{ptcp.m.sg} to \textsc{1sg}\\
\glt `Not a single person came to see us.'
\z

The following examples of the construction with \textit{tga} are either elicited or taken from written sources. The elicited forms were accepted by all the native speakers that have been consulted.

\ea
\label{}
\langinfo{Tuatschín}{Sadrún}{m5}\\
\gll \textbf{Gnang} \textbf{in} \textbf{tg}' è vagnús tiar nus.\\
not\_even one.\textsc{m.sg} \textsc{rel} be.\textsc{prs.3sg} come.\textsc{ptcp.m.sg} to \textsc{1sg}\\
\glt `Not a single person came to see us.'
\z

\ea
\label{}
\langinfo{Tuatschín}{Sadrún}{m5}\\
\gll Á quèla fjasta \textbf{mintgín} \textbf{tg}' è vagnús.   \\
to \textsc{dem.f.sg} party everyone \textsc{rel} be.\textsc{prs.3sg} come.\textsc{ptcp.m.sg} \\
\glt `Everybody came to this party.'
\z

\ea
\label{}
\langinfo{Tuatschín}{}{\citealt[69]{Berther2007}}\\
\gll  Avaun tschian òns pliravan nòṣ bunṣ véglṣ da pudaj fugí gl unviarn navèn da Sèlva ad ùssa, \textbf{tùt} \textbf{tga} vut stá luòra […].\\
ago hundred year.\textsc{m.pl} complain.\textsc{impf.3pl} \textsc{poss.1pl.m.pl} good.\textsc{pl} old.\textsc{pl} \textsc{comp} be\_able.\textsc{inf} escape.\textsc{inf} \textsc{def.art.m.sg} winter from of \textsc{pln} and now all \textsc{rel} want.\textsc{prs.3sg} stay there\_out \\
\glt `One hundred years ago our good old [people] used to complain because they wanted to escape from winter away from Selva, and now everybody wants to stay there.'
\z

\ea
\label{}
\langinfo{Tuatschín}{}{\DRG{2}{491}}\\
\gll  Ùṣ è \textbf{tùt} \textbf{tga} vò cun brajntas da stùrs.\\
now \textsc{cop.prs.3sg} all \textsc{rel} go.\textsc{prs.3sg} with basket.\textsc{pl} of sheet\_metal\\
\glt `Now everybody goes with a basket made of sheet metal.'
\z

If \textit{tùt} has a plural reference, the agreement is syntactic, which means that verbs, adjectives, and participles occur in their singular form, the latter two in their unmarked form as in (\ref{tùtplur}).

\ea\label{tùtplur}
\langinfo{Tuatschín}{Sadrún}{m5}\\
\gll  Quasi \textbf{tùt} tg' \textbf{è} \textbf{vagnú} á fjasta.\\
almost all \textsc{rel} be.\textsc{prs.3sg} come.\textsc{ptcp.unm} to party.\textsc{f.sg}\\
\glt `Almost everyone came to the party.'
\z


\section{The adjective}
\subsection{Forms of the adjective}
The adjective distinguishes two genders and two numbers: masculine, feminine, singular, and plural. Singular is unmarked except for masculine predicative adjectives, but plural gets the suffix \textit{-s}.

All adjectives display four forms, whereby a single form may fulfil different functions:

\begin{itemize}
\item one form for masculine attributive singular and for predicative adjectives that are unmarked for gender,
\item one form for masculine singular predicative as well as masculine plural attributive and predicative,
\item one form for feminine singular attributive and predicative, and
\item one form for feminine plural attributive and predicative.
\end{itemize}

The distribution of the adjectives according to whether they occur in predicative or in attributive function have a slightly different distribution in the domain of masculine adjectives (\tabref{tab:adj:attrandpred}).

\begin{table}
	\caption{Forms of the attributive and predicative adjectives}
	\label{tab:adj:attrandpred}
	\begin{tabular}{llllll}
		\lsptoprule
		& masculine && feminine && unmarked \\
		&		attributive & predicative& attributive & predicative & predicative \\
		\midrule
		\textsc{sg} & \textit{-Ø} & \textit{-s} & \textit{-a} & \textit{-a} & \textit{-Ø}\\
		\textsc{pl} & \textit{-s} & \textit{-s} & \textit{-as} & \textit{-as}\\
		\lspbottomrule
	\end{tabular}
\end{table}

Adjectives may show stem alterations  (\tabref{tab:adj:stemalterations}) or not (\tabref{tab:adj:nostemalterations}). A list of adjectives with stem alteration is given in \tabref{tab:list:adj:stemalterations}.\footnote{Some adjectives with stem alterations which are listed in \citet[282f.]{Spescha1989} are not used in Tuatschin. These are \textit{detschiert}, \textit{detscharta} `resolute', \textit{tanien}, \textit{tanienta} `such'; for \textit{stiert}, \textit{storta} and \textit{uiersch}, \textit{uiarscha}, both `crooked', the adjective \textit{crùtsch, -a} is used; however, \textit{ina stòrta} `a bend' exists. As for \textit{ierfan}, \textit{orfna} `orphan', only the masculine form \textit{iarfan} is used as a noun for both genders. Furtherore, \textit{tiest}, \textit{tosta} `dried' is only used in \textit{majla tòsta} `dried apples' or \textit{pèra tòsta} `dried pears'.}

\begin{table}
\caption{Adjectives without stem alterations}
\label{tab:adj:nostemalterations}
 \begin{tabular}{ll}
  \lsptoprule
 \textit{in cùdisch} \textbf{\textit{alv}} & `a white book'\\
\textit{Quaj è \textbf{alv}}. & `This is white.'\\
\textit{Quai cùdisch} è \textbf{\textit{alvs}}. & `This book is white.'\\
\textit{sis cùdischs} \textbf{\textit{alvs}} & `six white books'\\
\textit{Quèls cùdischs èn} \textbf{\textit{alvs}}. & `These books are white.'\\
\textit{ina flur} \textbf{\textit{alva}} & `a white flower'\\
\textit{Quèla flur è} \textbf{\textit{alva}}. & `This flower is white.'\\
\textit{sis flurs \textbf{alvas}} & `six white flowers'\\
\textit{Quèlas flurs èn} \textbf{\textit{alvas}}. & `These flowers are white.'\\
  \lspbottomrule
 \end{tabular}
\end{table}

\begin{table}\
\caption{Adjectives with stem alterations}
\label{tab:adj:stemalterations}
 \begin{tabular}{ll}
  \lsptoprule
    \textit{in \textbf{bi} dé} & `a beautiful day'\\
\textit{Mazá è bégja \textbf{bi}}. &`To kill is not nice.'\\
\textit{Quai cùdisch è} \textbf{\textit{bjals}}\textit{.} & `This book is beautiful.'\\
\textit{sis \textbf{bials} cùdischs} & `six beautiful books'\\
\textit{Quèls cùdischs èn} \textbf{\textit{bjals}}\textit{.} & `These books are beautiful.'\\
\textit{Quai è ina} \textbf{\textit{bjala}} \textit{flur.} & `This is a beautiful flower.'\\
\textit{Quèla flur è fétg \textbf{bjala}. }&`This flower is very beautiful.'\\
\textit{Quaj è} \textbf{\textit{bjalas}} \textit{flurs} & `These are beautiful flowers'\\
\textit{Las flurs sén quai prau èn \textbf{bjalas}}. &`The flowers on this meadow are beautiful.'\\
  \lspbottomrule
 \end{tabular}
\end{table}

\begin{table}
\caption{List of adjectives with stem alterations}
\label{tab:list:adj:stemalterations}
 \begin{tabular}{ll}
  \lsptoprule
 \textit{agjan, agjanṣ, atgna, atgnaṣ} &`own'\\
\textit{bi, bialṣ, biala, bialaṣ} &`beautiful'\\
\textit{bian, bunṣ, buna, bunaṣ} &  `good'\\
\textit{caviartg, cavòrtgs, cavòrtga, cavòrtgaṣ}& `hollow'\\
\textit{griaṣ, gròṣ, gròssa, gròssaṣ} &`big'\\
\textit{íastar, jastarṣ, jastra, jastraṣ}& `foreign'\\
\textit{matgiart, macòrts, macòrta, macòrtaṣ} &`ugly'\\
\textit{miart, mòrts, mòrta, mòrtaṣ} &`dead'\\
\textit{miaz, mjasa} & `half'\\
\textit{niabal, nòbelṣ, nòbla, nòblaṣ} & `noble'\\
\textit{néjv, nùfs, nóva, nóvaṣ} &`new'\\
\textit{pin, pinṣ, pintga, pintgaṣ} &`small, little’\\
\textit{schliat, schljats, schljata, schljataṣ}& `bad\\ 
\textit{sògn, sògnṣ, sòntga, sòntgaṣ} &`holy'\\
\textit{tgétschen, còtgenṣ, còtgna, còtgnaṣ}& `red'\\
\textit{tiarz, tjarzs, tjarza, tjarzaṣ}&`third'\\
\textit{tschiac, tschòcs, tschòca, tschòcaṣ}&`blind'\\
\textit{ziap, zòps, zòpa, zòpaṣ} & `limp'\\
  \lspbottomrule
 \end{tabular}
\end{table}

The adjectives ending in -\textit{al} lose their reduced vowel in the feminine form, as in \textit{pussajval} (m) vs \textit{pussajvla} (f) `possible', or \textit{ṣgarṣchajval} (m) vs \textit{ṣgarṣchajvla} `terrible'.


The predicative forms of the adjective do not only occur with copulative verbs, but also if the adjective refers to a physical or mental state of the noun it refers to, as in examples (\ref{ex:adj.agr1}) - (\ref{ex:adj.agr6}).

\ea
\label{ex:adj.agr1}
\langinfo{Camischùlas}{Tuatschín}{\citealt[82]{Büchli1966}}\\
\gll [...] lu ò las \textbf{anflau} èl \textbf{mòrts} spèl badugn gjù.\\
    {} then have.{\prs.3sg} \textsc{3pl.f} find.\textsc{ptcp.unm} \textsc{3sg.m} dead.\textsc{m.sg} next.\textsc{def.art.m.sg}. birch down\\
\glt `[…] then they found him dead next to the birch.'
\z

\ea
\label{schèwithagr}
\langinfo{Tuatschín}{Cavorgia}{\citealt[119]{Büchli1966}}\\
\gll    Èla \textbf{schèva} \textbf{crèschar} l’ jarva schi \textbf{bjala} […].\\
    \textsc{3sg.f} let.\textsc{impf.3sg} grow.\textsc{inf} \textsc{def.art.f.sg} grass so beautiful.\textsc{f.sg} \\
\glt `She used to let the grass grow so beautiful […].'
\z
 
 \ea\label{Ex:adj.agr3}
 \langinfo{Tuatschín}{Sadrún}{m5}\\
 \gll  Té vas \textbf{zòps}.\\
 \textsc{2sg} go.\textsc{prs.2sg} limp.\textsc{m.sg} \textsc{} \textsc{} \textsc{} \textsc{} \textsc{} \\
 \glt `You are limping.'
 \z
 
 \ea
 \label{ex:adj.agr4}
 \langinfo{Tuatschín}{Sadrún}{m10}\\
 \gll Èl è turnaus \textbf{cuntjants} á tgèsa.\\
 \textsc{3sg.m} be.\textsc{prs.3sg} come\_back.\textsc{ptcp.m.sg} happy.\textsc{m.sg} to home.\textsc{f.sg} \\
 \glt `He came back home happy.'
 \z
 
 \ea\label{ex:adj.agr5}
 \langinfo{Tuatschín}{Sadrún}{\citealt[106]{Büchli1966}}\\
 \gll    Parquaj satila ‘l ò \textbf{bluts} […].\\
 therefore \textsc{refl}.pull.\textsc{prs.3sg} \textsc{3sg.m} out naked.\textsc{m.sg}\\
 \glt `Therefore he took off all his clothes […].'
 \z
 
 \ea\label{ex:adj.agr6}
 \langinfo{Tuatschín}{Surain}{\citealt[129]{Büchli1966}}\\
 \gll    […] schagljùc végn ju manizaus ampaglja schi \textbf{manédels} […].\\
 {} otherwise \textsc{pass.aux.prs.1sg} \textsc{1sg} chop.\textsc{ptcp.m.sg} damaged so fine.\textsc{m.sg}\\
 \glt `[…] otherwise I get completely chopped into such fine pieces […].'
 \z
 
 In infinitive sentences with a copulative verb (\ref{ex:adjpred:inf}), the adjective gets the masculine singular predicative \textit{-s} in spite of the fact that it has a generic, not a masculine referent.
 
 \ea
 \label{ex:adjpred:inf}
 \langinfo{Tuatschín}{Sadrún}{m4}\\
 \gll Èssar \textbf{mazauns} è bigja bi.\\
 \textsc{cop.inf} ill.\textsc{m.sg} \textsc{cop.prs.3sg} \textsc{neg} nice.\textsc{adj.unm}\\
 \glt `Being ill is not nice.'
 \z
 
 The generic pronoun \textit{ins} triggers the use of the masculine singular form of the adjective in predicative position (\ref{ex:ins:predadj1}).
 
 \ea
 \label{ex:ins:predadj1}
 \langinfo{Tuatschín}{Ruèras} {f4, l. 1986f.}\\
 \gll  [...] èr’ \textbf{ins} \textbf{trésts} ajn in cèrt sèn [...].\\
 {} \textsc{cop.impf.3sg} \textsc{gnr} sad.\textsc{m.sg} in \textsc{indef.art.m.sg} certain sense\\
 \glt `[...] one felt sad in a certain sense [...].'
 \z
 
 The morphologically unmarked form occurs in predicative function if the subject it refers to has no gender, as e.g the demonstrative \textit{quaj} `this', place names, generic noun phrases, or nonfinite subject clauses, as in (\ref{ex:adj.unm1}) - (\ref{ex:adj.unm5}). The same holds for the predicative past participle, as in (\ref{ex:ptcp.unm1}).

\ea\label{ex:adj.unm1}
\langinfo{Tuatschín}{Sadrún}{m4, l. 416}\\
\gll  […] \textbf{quaj} èra bigja grat schi \textbf{sémpal}.  \\
{} \textsc{dem.unm} \textsc{cop.impf.3sg} \textsc{neg} exactly so easy.\textsc{adj.unm}\\
\glt `[...] this was not exactly that simple.'
\z

\ea\label{ex:adj.unm2}
\langinfo{Tuatschín}{Sadrún}{m4}\\
\gll \textbf{Nùrsas} è \textbf{ṣgarṣchajval}.\\
sheep.\textsc{f.pl} \textsc{cop.prs.3sg} horrible.\textsc{adj.unm}\\
\glt `Sheep are horrible.'
\z

\ea\label{ex:adj.unm3}
\langinfo{Tuatschín}{Sadrún}{m4}\\
\gll \textbf{Vacanzas} è \textbf{bi}.\\
holiday.\textsc{pl} \textsc{cop.prs.3sg} nice.\textsc{adj.unm}\\
\glt `Holidays are nice.'
\z

\ea\label{ex:adj.unm4}
\langinfo{Tuatschín}{Sadrún}{m5}\\
\gll  \textbf{Vjagè} è \textbf{bi}.\\
travel.\textsc{inf} \textsc{cop.prs.3sg} beautiful.\textsc{adj.unm}\\
\glt `Travelling is great.'
\z

\ea\label{ex:adj.unm5}
\langinfo{Tuatschín}{Sadrún}{m4}\\
\gll \textbf{Gljòn} è \textbf{pin}.\\
\textsc{pln} \textsc{cop.prs.3sg} small.\textsc{adj.unm}\\
\glt `Glion is small.'
\z

\ea\label{ex:ptcp.unm1}
\langinfo{Tuatschín}{Sadrún}{m4, l. 424f.}\\
\gll  Ah… \textbf{Nalbṣ} è \textbf{vagnú} \textbf{fraquantau} ò scù ah majṣès ad alps adina […].  \\
eh \textsc{pln} be.\textsc{prs.3sg} \textsc{pass.aux.ptcp.m.unm} visit.\textsc{ptcp.m.unm} out as eh assembly\_of\_house.\textsc{m.sg} and alpine\_pasture.\textsc{f.pl} always\\
\glt `Eh … Nalps has always been visited as an assembly of houses and as pastures […].'
\z

The unmarked form is also used with pronouns that have no gender like \textit{zatgéj} `something' in (\ref{ex:zatgéjunm}).


\ea\label{ex:zatgéjunm}
\langinfo{Tuatschín}{Ruèras}{m10, l. 1136f.}\\
\gll  A galòpau, galòpau, galòpau, \textbf{zatgéj} \textbf{sgarṣchajval}, nuṣ èssan vagní da tanaj èls pér gjù ṣur Sèlva.  \\ 
and gallop.\textsc{ptcp.unm} gallop.\textsc{ptcp.unm} gallop.\textsc{ptcp.unm} something terrible.\textsc{adj.unm} \textsc{1pl} be.\textsc{prs.1pl} come.\textsc{ptcp.m.pl}   \textsc{comp} hold.\textsc{inf} \textsc{3pl.m} only down above \textsc{pln}  \\
\glt `And [the mules] galloped, galloped, galloped, this was horrible, we only managed to hold onto them above Selva.'
\z

\subsection{Degrees of comparison of adjectives and adverbs}
In this section, the degrees of comparisons of adverbs will be included for convenience.

The positive is built with \textit{usché} `so' and \textit{scù} `as, like'. The comparative is constructed with \textit{plé} `more' and \textit{tga} `that'. Only the comparative of superiority is used; the comparative of inferiority could be constructed, but is not in use. The superlative is built like the comparative, but with the definite article in addition.

\begin{table}
	\caption{Degrees of comparison}
	\label{}
\begin{tabular}{llllllll}
	\lsptoprule
	positive & \textit{Ju} & \textit{sùn} & \textbf{\textit{usché}} & \textit{gròns} & \textbf{\textit{scù}} & \textit{té}.\\
comparative & \textit{Ju} & \textit{sùn} & \textbf{\textit{plé}} & \textit{\textit{gròns}} & \textit{\textbf{tga}}\footnote{Rarely \textit{tgé}} & \textit{té}.\\
superlative & \textit{Ju} & \textit{sùn} & \textbf{\textit{al plé gròn}} & \textit{da tùts}.\\
	\lspbottomrule
\end{tabular}
\end{table}

In the corpus, most comparatives are left without a compared element (\ref{ex:compwithoutcompared}); an example with the compared element is (\ref{ex:compwithcompared}).

\ea
\label{ex:compwithoutcompared}
\langinfo{Tuatschín}{Sadrún}{m9, l. 1848ff.}\\
	\gll [...] tgé c’ ins vèza bjè è \textbf{als} \textbf{plé} \textbf{passaj}, als pènṣionaj vèz’ ins bjè sén pista [...].\\
{} what \textsc{rel} \textsc{gnr} see.\textsc{prs.3sg} much \textsc{cop.prs.3sg} \textsc{def.art.m.pl} more old.\textsc{m.pl} \textsc{def.art.m.sg} retired.\textsc{pl} see.\textsc{prs.3sg} \textsc{gnr} much on slope.\textsc{f.sg}\\
\glt `[...] what one often sees are older people, one can see a lot of retired people on the slopes [...].'
\z

\ea
\label{ex:compwithcompared}
\langinfo{Tuatschín}{Ruèras}{m1, l. 188ff.}\\
\gll    [...] ò fatg scùlas vinavaun … stada in téc \textbf{plé} \textbf{pardèrta} \textbf{tgé} \textbf{quaj} tgu èra.\\
{} have.\textsc{prs.3sg} make.\textsc{ptcp.unm} school.\textsc{f.pl} further {} \textsc{cop.ptcp.f.sg} \textsc{indef.m.sg} bit more  clever.\textsc{f.sg} than \textsc{dem.unm} \textsc{rel.1sg} \textsc{cop.impf.1sg} \\
\glt `[...] [she] kept going to school … was a little bit cleverer than what I was.'
\z

Although adjectives may precede or follow the noun they modify, with the superlative the prenominal syntax is preferred with short adjectives, even with the adjectives that always follow the noun as is the case of colour adjectives (\ref{ex:supco}).

\ea
\label{l}
\langinfo{Tuatschín}{Ruèras}{m3 l. 2040f.}\\
	\gll Èl’ èra \textbf{la} \textbf{plé} \textbf{grònda} \textbf{buéba} dal vitg.  \\
\textsc{3sg.f} \textsc{cop.impf.3sg} \textsc{def.art.f.sg} more tall.\textsc{f.sg} girl of.\textsc{def.art.m.sg} village	\\
\glt `She was the tallest girl of the village.'
\z

\ea
\label{ex:supco}
\langinfo{Tuatschín}{Sadrún}{m5}\\
\gll Gljòn è \textbf{al} \textbf{plé} \textbf{vèrd} \textbf{martgau} da la Sursèlva.\\
\textsc{pln} \textsc{cop.prs.3sg} \textsc{def.art.m.sg} more green city of \textsc{def.art.f.sg} \textsc{pln}\\
\glt `Glion is the greenest city of the Surselva.'
\z

\ea
\label{}
\langinfo{Tuatschín}{Sadrún}{m5, l. 1250f.}\\
\gll Quaj èra atgnamajn \textbf{la} \textbf{fòntauna} \textbf{la} \textbf{plé} \textbf{impurtònta} da hanlètg [...].\\
\textsc{dem.unm} \textsc{cop.impf.3sg} actually  \textsc{def.art.f.sg} source \textsc{def.art.f.sg} most important of business.\textsc{m.sg}\\
\glt `This was actually the most important source of  business [...].'
\z

There are some synthetic comparatives and superlatives: \textit{bian} `good', \textit{mégljar} `better', and  \textit{al mégljar} `the best', as well as \textit{schliat} `bad', \textit{mèndar} `worse', and \textit{al mèndar} `the worst'.

\ea
\label{}
\langinfo{Tuatschín}{Tschamùt}{\citealt[20]{Büchli1966}}\\
\gll A suéntar ṣè `l staus \textbf{mégljars} [...].\\
and after be\textsc{.prs.3sg} \textsc{3sg.m} \textsc{cop.ptcp.unm} better.\textsc{m.sg}\\
\glt`And after [that] he behaved better [towards the animals] [...].'
\z


\citet[233-50]{Linder1987} shows that for some adjectives (or substantivized adjectives, see (\ref{ex:tga:3})) the superlative is formed without \textit{plé} `more' in all Romansh written varieties. This also holds for Tuatschín.

\ea
\label{}
\langinfo{Tuatschín}{Cavòrgja}{\citealt[120]{Büchli1966}}\\
\gll    Lu ṣai vagnú gjù la lavina […] ad ò méz ṣùt \textbf{la} \textbf{gronda} \textbf{part} dl vitg.\\
then \textsc{cop.prs.3sg} come.\textsc{ptcp.unm} down \textsc{def.art.f.sg} avalanche […] and have.\textsc{prs.3sg} put.\textsc{ptcp.unm} under \textsc{def.art.f.sg} big part of.\textsc{def.art.m.sg} village\\
\glt `Then the avalanche came down […] and buried the biggest part of the village.'
\z

\ea\label{ex:tga:3}
\langinfo{Tuatschín}{Sadrún}{f3, l. 6f.}\\
\gll   Álṣò i èran… grad, grat, \textbf{al} \textbf{gròn} èra racrut, a tschèlṣ duṣ ajn amprèndissadi. \\
well \textsc{3pl} \textsc{cop.impf.3pl} just just \textsc{def.art.m.sg} big \textsc{cop.impf.3sg} recruit.\textsc{m.sg} and  \textsc{dem.m.pl} two.\textsc{pl} in apprenticeship.\textsc{m.sg} \\
\glt `Well, they were … just, the oldest was a recruit, and the other two [were] in an apprenticeship.'\\
\z

\ea\label{}
\langinfo{Tuatschín}{}{\citealt[61]{Gartner1910}}\\
\gll    […] al \textbf{gjuvan} ajn stgafa da l' ura, quèl ò èl bétg anflau.\\
  {} \textsc{def.art.m.sg} young in box of \textsc{def.art.f.sg} clock \textsc{dem.m.sg} have.\textsc{prs.3sg} \textsc{3sg.m} \textsc{neg} find.\textsc{ptcp.unm}\\
\glt `[…] the youngest [goat] in the clock box, this one he didn’t find.'
\z

\ea\label{}
\langinfo{Tuatschín}{Ruèras}{f7, l. 1747f.}\\
	\gll  [...] mia \textbf{féglja} \textbf{pintg}’ è plétost [...] pintga.   \\
{} \textsc{poss.1sg.f.sg} daughter young \textsc{cop.prs.3sg} rather {} small.\textsc{f.sg}\\
\glt `[...] my youngest daughter is rather [...] short.'
\z

A possible way of forming the elative is using the prefix \textit{u-}, which is a loan from colloquial Swiss German.

\ea\label{}
\langinfo{Tuatschín}{Sadrún}{f3, l. 73f.}\\
\gll  Quaj èra stau zatgé nùndétg, ábar stau \textbf{u-bjals} mùm\underline{è}nts.\\
\textsc{dem.unm} be.\textsc{impf.3sg} \textsc{cop.ptcp.unm} something incredible.\textsc{m.sg} but \textsc{cop.ptcp.unm}  \textsc{elat}-beautiful.\textsc{m.pl} moment.\textsc{pl} \\
\glt `This was something incredible, but these were very beautiful moments.'
\z

\ea\label{}
\langinfo{Tuatschín}{Camischùlas}{f6, l. 723f.}\\
\gll    A las sòras savèvan tga nus trajs nus vagjan adina \textbf{u-léjgar} [...].\\
and \textsc{def.art.f.pl} nun.\textsc{pl} know.\textsc{impf.3pl} \textsc{comp} \textsc{1pl} three \textsc{1pl} have.\textsc{prs.sbjv.1pl} always \textsc{elat}-funny.\textsc{adj.unm}\\
\glt `And the nuns knew that the three of us, we always had fun [...].'
\z

Another way of forming the elative is using a noun that is mostly derived from an adjective, with different suffixes: \textit{bials / biala} `beautiful' → \textit{balèzja} `beauty', \textit{buns / buna} `good' → \textit{buantád} `good quality', \textit{pasar} `weigh' → \textit{pasanca} `heavy load', \textit{paupers / paupra} `poor' → \textit{pupira} `poverty', \textit{tups / tupa} `stupid' → \textit{tupira}, `stupidity', or \textit{scarts / scarta} `scarce' → \textit{scartèzja} `scarcity'. These elatives are used attributively and predicatively.

 
\ea\label{}
\langinfo{Tuatschín}{Ruèras}{f7, l. 1744f.}\\
	\gll  [...] i èra fétg tgaud a la mar èra \textbf{balèzja} gjù Sardégna.\\
{} \textsc{expl} \textsc{cop.impf.3sg} very warm.\textsc{adj.unm} and \textsc{def.art.f.sg} sea \textsc{cop.impf.3sg} beauty.\textsc{f.sg.elat} down \textsc{pln}\\
\glt `Well, it was ... it was very war and the sea was beautiful in Sardinia.'
\z

\ea\label{}
\langinfo{Tuatschín}{}{\DRG{2}{639}}\\
\gll  Quaj è \textbf{buantád} vaca da latg. \\
\textsc{dem.unm} \textsc{cop.prs.3sg} good\_quality.\textsc{f.sg.elat} cow of milk\\
\glt `This is an excellent milk cow.'
\z

\ea
\label{}
\langinfo{Tuatschín}{Sadrún}{f3, l. 89f.}\\
\gll  A quaj èra \textbf{pasanca}, api vau tartgau basta.\\
and \textsc{dem.unm} \textsc{cop.impf.3sg} heavy\_load.\textsc{f.sg.elat} and have.\textsc{prs.1sg} think.\textsc{ptcp.unm} enough  \\
\glt `And this was terribly heavy, and then I thought [it was] enough.'
\z

\ea
\label{}
\langinfo{Tuatschín}{Ruèras}{m4, l. 2164}\\
	\gll  Quèla vèva còrna usché davùsòra, \textbf{pasanca} \textbf{tiar} [...].  \\
\textsc{dem.f.sg} have.\textsc{impf.3sg} horn.\textsc{coll} so back\_out heavy\_load.\textsc{f.sg.elat} animal.\textsc{m.sg} \\
\glt `This one had horns that had grown backwards, a very heavy animal [...].'
\z

\ea\label{}
\langinfo{Tuatschín}{Ruèras}{m1, l. 375f.}\\
\gll   Ábar i èra … \textbf{pupira}, quèls spargnavan è starmantús [...].\\
but \textsc{expl}  \textsc{cop.impf.3sg} poverty.\textsc{f.sg.elat} {} \textsc{dem.m.pl} save.\textsc{impf.3pl} also terrible.\textsc{unm}\\
\glt `But this was ... real poverty, they would save as much as they could [...].'
\z

\ea
\label{}
\langinfo{Tuatschín}{Camischùlas}{f6, l. 861f.}\\
\gll    Quaj fùs stau \textbf{tupira} par mè da stuaj raṣdá ajn tgòmbra cun tschèlas ròmòntschas … tudèstg.\\
\textsc{dem.unm} \textsc{cop.cond.3sg} \textsc{cop.ptcp.unm} stupidity.\textsc{f.sg.elat}  for \textsc{1sg} \textsc{comp} must.\textsc{inf} speak.\textsc{inf} in room.\textsc{f.sg} with \textsc{dem.f.pl} Romansh.\textsc{pl} {} German.\textsc{m.sg}\\
\glt `It would have been very stupid for me if I'd had to speak German in the room with the other Romansh room-mates.'
\z

\ea
\label{}
\langinfo{Tuatschín}{Cavòrgja}{m7, l. 2296}\\
\gll Quaj èra \textbf{scartèzja}.   \\
\textsc{dem.unm} \textsc{cop.impf.3sg} scarcity.\textsc{f.sg.elat}\\
\glt `The cervelats were very scarce.'
\z

\subsection{Intensifiers}
The intensifiers of the adjective occurring in the corpus are \textit{dètg} `fairly', \textit{fétg} `very', \textit{mèmi/mèmja} `too', \textit{pulit} `very', \textit{réjsch} `brand-', \textit{schi} `so', \textit{in téc} `a bit', \textit{tùt} `completely', and \textit{ualti} `quite'. These intensifiers are all invariable.

\ea
\label{}
\langinfo{Tuatschín}{Sadrún}{m4, l. 643f.}\\
\gll A lu ṣaj ... èl è lu saravagnús \textbf{dètg} \textbf{stupèn} [...].   \\
and then be.\textsc{prs.3sg} {} \textsc{3sg.m} be.\textsc{prs.3sg} then \textsc{refl}.recover.\textsc{ptcp.m.sg} fairly  excellent.\textsc{adj.unm}\\
\glt `And then, he recovered perfectly well [...].'
\z

\ea
\label{}
\langinfo{Tuatschín}{Sadrún}{m4, l. 336f.}\\
\gll Quaj è stau ina ... \textbf{fétg} \textbf{grònda} familja, èls òn gju indiṣch ufauns [...].   \\
\textsc{dem.unm}  be.\textsc{prs.3sg}  \textsc{cop.ptcp.unm} \textsc{indef.art.f.sg} {} very big family \textsc{3pl.m} have.\textsc{prs.3pl} have.\textsc{ptcp.unm} eleven child.\textsc{m.pl}\\
\glt `This was a ... very big family, they had eleven children [...].'
\z

\ea
\label{}
\langinfo{Tuatschín}{Ruèras}{m3, l. 2070f.}\\
\gll  Ábar quaj vagnéva dau ṣura in pènṣum \textbf{mèmi} \textbf{gròn} [...].  \\
but \textsc{dem.um} \textsc{pass.aux.impf.3sg} give.\textsc{ptcp.unm} up \textsc{indef.art.m.sg} homework too big\\
\glt `But they would give us too much homework [...].'
\z

\ea
\label{}
\langinfo{Tuatschín}{Ruèras}{m3, l. 2206f.}\\
	\gll  A pér cu quaj èra fatg scha èra la scòtga \textbf{mèmi} \textbf{tgauda} [...].  \\
and only when \textsc{dem.unm} \textsc{pass.aux.impf.3sg} do.\textsc{ptcp.unm} \textsc{corr} \textsc{cop.impf.3sg} \textsc{def.art.f.sg} whey too hot\\
\glt `And only when this was done, the whey was too hot [...].'
\z

\ea
\label{}
\langinfo{Tuatschín}{Sadrún}{m, l. 1388f.}\\
\gll [...] a qu’ èra stau par èla \textbf{mèmja} \textbf{hèfti}.\\
{} and \textsc{dem.unm} be.\textsc{impf.3sg} \textsc{cop.ptcp.unm} for \textsc{3sg.f} too violent.\textsc{unm} \\
\glt `[...] and this had been too violent for her.'
\z

\ea
\label{}
\langinfo{Tuatschín}{Sadrún}{m6, l. 876}\\
\gll    A qu’ è pròpi ina … \textbf{pulit} \textbf{grònda} plata [...].\\
and \textsc{dem.unm} \textsc{cop.prs.3sg} really \textsc{indef.art.f.sg} {} very big slab\\
\glt `And this really is a … very big slab [...].'
\z

\ea
\label{ex:réjsch}
\langinfo{Tuatschín}{Cavòrgja}{m7, l. 2355ff.}\\
\gll A prènd èl ò dal sacadòs in cuntí \textbf{réjsch} \textbf{néjv} da moni mèlan … èxàct al madèm.\\
and take.\textsc{prs.3sg} \textsc{3sg.m} out of.\textsc{def.art.m.sg} backpack \textsc{indef.art.m.sg} knife brand new.\textsc{m.sg} of handle.\textsc{m.sg} yellow {} exact.\textsc{adj.unm} \textsc{def.art.m.sg} same\\
\glt `And he takes a brand-new knife with a yellow handle out of the backpack ... exactly the same.'
\z

In the corpus, \textit{réjsch} only modifies \textit{néiv} `new' (\ref{ex:réjsch}), but \citet[931]{Decurtins2012} cites further \textit{resch tschietschen} `flame red' and \textit{resch bletsch} `soaking wet' for standard Sursilvan.

\ea
\label{}
\langinfo{Tuatschín}{Sadrún}{m8, l. 1483ff.}\\
\gll Api grad ajn quèl mumèn vagnév’ in cégn … gròn né– vi datiar ad èra lò usché \textbf{in} \textbf{téc} \textbf{dòminant}.\\
and exactly in \textsc{dem.m.sg} moment come.\textsc{impf.3sg} \textsc{indef.art.m.sg} swan {} big.\textsc{m.sg.unm} or over next\_to and \textsc{cop.impf.3sg} there so \textsc{indef.art.m.sg} bit dominant.\textsc{adj.unm}\\
\glt `And precisely at that moment a big swan… was coming to the place where I was, a bit a dominant one.'
\z

\ea\label{}
\langinfo{Tuatschín}{Sadrún}{m6, l. 1385f.}\\
\gll    [...] ju sa aun bégn tg’ èl’ èra vagnida \textbf{tùt} \textbf{còtschna} [...].\\
{} \textsc{1sg} know.\textsc{prs.1sg} still well \textsc{rel} \textsc{3sg.f} be.\textsc{impf.3sg} become.\textsc{ptcp.f.sg} completely red.\textsc{f.sg} \\
\glt `[...] I still remember very well that she had turned completely red [...].'
\z

\ea\label{}
\langinfo{Tuatschín}{Sadrún}{m4, l. 370f.}\\
\gll [...] qu’ è lu stau in tjams \textbf{ualti} \textbf{dir} cunzún par la tata [...]. \\
{} \textsc{dem.unm} be.\textsc{prs.3sg} then \textsc{cop.ptcp.unm} \textsc{indef.art.m.sg} time quite hard.\textsc{m.sg} especially for \textsc{def.art.f.sg} grandmother \\
\glt `[...] this has then been a very hard time especially for my grandmother [...].'
\z

\ea
\label{}
\langinfo{Tuatschín}{Ruèras}{m1, l. 280f.}\\
\gll    Èl vèva \textbf{schi} \textbf{gròn} plaṣchaj, quèl bunamajn mava ajn cul tgau ajl rádjò.\\
\textsc{3sg.m} have.\textsc{impf.3sg} so big.\textsc{m.sg} pleasure \textsc{dem.m.sg} almost go.\textsc{impf.3sg} into with.\textsc{def.art.m.sg} head in.\textsc{def.art.m.sg} radio\\
\glt `He had such a great pleasure, he almost would go into the radio with his head.'
\z


\subsection{Adjectives in adverbial function}
The modal adverbs derived from adjectives by the suffix -\textit{majn} (see § 4.2.5 below) is not as widespread as in other Romance varieties, probably due to Swiss German influence. Therefore, many adjectives in their unmarked form are used as adverbs.

\ea\label{}
\langinfo{Tuatschín}{Ruèras}{m1, l. 191f.}\\
\gll [...] ina tg’ ò luvrau \textbf{stédi}.\\
{} one.\textsc{f.sg} \textsc{rel} have.\textsc{prs.3sg} work.\textsc{ptcp.unm} diligent.\textsc{adj.unm} \\
\glt `[...] one who always worked hard.'
\z

\ea\label{}
\langinfo{Tuatschín}{Ruèras}{m1, l. 375f.}\\
\gll   Ábar i èra … pupira, quèls spargnavan è \textbf{starmantús} [...].\\
but \textsc{expl}  \textsc{cop.impf.3sg} {} poverty.\textsc{f.sg} \textsc{dem.m.pl} save.\textsc{impf.3pl} also terrible.\textsc{adj.unm}\\
\glt `But this was ... real poverty, they would save as much as they could [...].'
\z

\ea\label{}
\langinfo{Tuatschín}{Ruèras}{m10, l. 1075ff.}\\
\gll  [...] álṣò òr dal grép òni fatg ina pintga …  sènda tg’ ins sò ira ah á paj \textbf{flòt}.\\
{} this\_is\_to\_say out of.\textsc{def.art.m.sg} rock have.\textsc{prs.3pl.3pl} make.\textsc{ptcp.unm} \textsc{indef.art.f.sg} small {} path \textsc{rel} \textsc{gnr} can.\textsc{prs.3sg} go.\textsc{inf} eh on foot.\textsc{m.sg} easy.\textsc{adj.unm} \\
\glt `[...] this is to say out of the rock they made a small … path through which one could easily go eh on foot.'
\z

A special case is \textit{bian} `good', which is sometimes mistaken for \textit{bégn} `well', as in example {\ref{bianbégn}}.

\ea\label{bianbégn}
\langinfo{Tuatschín}{}{\DRG{2}{624}}\\
\gll  In vjantar gròn maglja \textbf{bian}. \\
     \textsc{def.art.m.sg} stomach big eat.\textsc{prs.3sg} good.\textsc{adj.unm} \\
\glt `A big belly eats well.'
\z

\subsection{Position of the attributive adjective}
The position of the adjective in standard Sursilvan has been analysed in \citet{Winzap1981}. According to this author, the position of the adjective depends mostly on syntactic, stylistic, rhythmic, and semantic criteria (\citet[1]{Winzap1981}). He proposes, among others, the following rules, which are more tendencies than strict rules (\citet[3ff.]{Winzap1981}): 

\begin{itemize}
\item (1) If the adjective functions as the head of an adjective phrase, it usually occurs after the noun.
\item (2) But if the adjective is modified by a degree word such as \textit{ualti} `quite', it may precede the noun.
\item (3) If the noun is followed by a complement, the adjective may precede the head noun. It follows it only in case there is no doubt about which noun it modifies.
\item (4) Monosyllabic adjectives usually precede the noun.
\item (5) Inversely, polysyllabic adjectives tend to follow the noun.
\item (6) Monosyllabic adjectives follow the noun if the noun also consists of only one syllable.
\item (7) If the adjective has a descriptive function, it precedes the noun and is not stressed.
\item (8) If the adjective has a distinctive function, it follows the noun and is stressed.
\item (9) Some adjectives have different meanings according to their position, as \textit{criu} `raw' in \textit{ina criua sort} `a cruel destiny' vs \textit{schambun criu} `raw ham'.
\end{itemize}	

Note that (1) - (6) are syntactic, whereas (7) - (9) are semantic criteria. No examples have been found for rules (2), (3), and (6).

It is very improbable that Tuatschin differs from standard Sursilvan in this respect, and since the ten criteria mentioned by Winzap are best considered tendencies, I will illustrate each of these without further analysing this issue since it would be beyond the scope of this grammar.

Examples (\ref{ex:ualtidir}) and (\ref{ex:pensummemigron}) illustrate rule (1), since the adjective which is the head of an adjective phrase follows the noun.
 
\ea
\label{ex:ualtidir}
\langinfo{Tuatschín}{Sadrún}{m4, l. 370f.}\\
\gll  [...] qu’ è lu stau in tjams \textbf{ualti} \textbf{dir} \textbf{cunzún} \textbf{par} \textbf{la} \textbf{tata} [...]. \\
{} \textsc{dem.unm} be.\textsc{prs.3sg} then \textsc{cop.ptcp.unm} \textsc{indef.art.m.sg} time quite hard especially for \textsc{def.art.f.sg} grandmother [...].\\
\glt `[...] this was then a very hard time, especially for my grandmother [...].'
\z

\ea
\label{ex:pensummemigron}
\langinfo{Tuatschín}{Ruèras}{m3, l. 2070f.}\\
\gll  Ábar quaj vagnéva dau ṣura in \textbf{pènṣum} \textbf{mèmi} \textbf{gròn} [...].  \\
but \textsc{dem.um} \textsc{pass.aux.impf.3sg} give.\textsc{ptcp.unm} up \textsc{indef.art.m.sg} homework too big\\
\glt `But they would give us too much homework [...].'
\z

Example (\ref{ex:pensummemigron}) furthermore shows that this rule also apply to monosyllabic adjectives like \textit{gròn} `big' that usually precede the noun.

Examples (\ref{ex:carufaun}) and (\ref{ex:grejvmumen}) illustrate rule (4) (monosyllabic adjectives precede the noun). It must, however, be emphasized that the feminine counterparts of masculine monosyllabic adjectives are polysyllabic and that the rule applies to them as well (\ref{ex:grondaslavurs}).

\ea
\label{ex:carufaun}
\langinfo{Tuatschín}{Sadrún}{m5}\\
\gll Quaj è ualti tgèr, ábar quaj è in \textbf{car} ufaun.\\
\textsc{dem.unm} \textsc{cop.prs.3sg} quite expensive.\textsc{adj.unm} but \textsc{dem.unm} \textsc{cop.prs.3sg} \textsc{indef.art.m.sg} dear.\textsc{m.sg.unm} child\\
\glt `This is quite expensive, but this is a dear child.'
\z 

\ea
\label{ex:grejvmumen}
\langinfo{Tuatschín}{Ruèras}{m1, l. 287f.}\\
\gll    Ad èra stau in \textbf{gréjv} \textbf{mùm\underline{è}n}, cun sissòntanùv òns.\\
and be.\textsc{impf.3sg} \textsc{cop.ptcp.unm} \textsc{indef.art.m.sg} difficult.\textsc{m.sg} moment with sixty-nine year.\textsc{m.pl} \\
\glt `And this had been a difficult moment, with sixty-nine years.'
\z 

\ea
\label{ex:grondaslavurs}
\langinfo{Tuatschín}{Ruèras}{m1, l. 307}\\
\gll [...] qu’ èra stau \textbf{gròndas} \textbf{lavurs} [...].\\
{} \textsc{dem.unm} be.\textsc{impf.3sg} \textsc{cop.ptcp.unm} big.\textsc{f.pl} work.\textsc{pl}\\
\glt `[...] this had been hard work [...].'
\z

Rule (5) (polysyllabic adjectives follow the noun) is illustrated by (\ref{ex:rusnanundetga}).

\ea
\label{ex:rusnanundetga}
\langinfo{Tuatschín}{Sadrún}{m5, l. 363f.}\\
\gll  [...] èl è curdauṣ gjùdajn ajn \textbf{ina} \textbf{rùsna} \textbf{nùndétga} [...].  \\
{} \textsc{3sg.m} be.\textsc{prs.3sg} fall.\textsc{ptcp.m.sg} down\_into in \textsc{indef.art.f.sg} hole awful\\
\glt `[...] he fell in an awful hole [...].'
\z

Rule (7) (the adjective precedes the noun if it has a descriptive function) is shown in (\ref{ex:intarassantslogans}).

\ea
\label{ex:intarassantslogans}
\langinfo{Tuatschín}{Sadrún}{m5, l. 425f.}\\
\gll [...] lu ṣè aun dus trajs \textbf{intarassants} \textbf{lògans} ajn cò [...].\\
 {} then \textsc{exist.prs.3sg} still two three interesting place.\textsc{m.pl} in here\\
\glt `[...] there were furthermore two or three interesting places up there [...].'
\z

(\ref{ex:intarassantslogans}) also shows that rule (7) overrides rule (5) according to which polysyllabic adjectives tend to follow the noun.

Example (\ref{ex:armaulsgrons}) illustrates rule (8) (post-nominal position because it has a distinctive function). As a matter of fact, the big animals are opposed to the goats, pigs, and hens which are small animals.

\ea
\label{ex:armaulsgrons}
\langinfo{Tuatschín}{Sadrún}{m5, l. 378ff.}\\
\gll   [...] tgéj pudévan èls vaj, déjsch quindiṣch \textbf{armaulṣ} \textbf{gronṣ} api lu aun ... tgauras sòu tga vèvan a … pòrṣ a gaglinaṣ [...]. \\
[...] what can.\textsc{impf.3pl} \textsc{3pl.m} have.\textsc{inf} ten fifteen animal.\textsc{m.pl} big.\textsc{pl} and then besides {} goat.\textsc{f.pl} know.\textsc{prs.1sg.1sg} \textsc{comp} have.\textsc{impf.3pl} and {} pig.\textsc{m.pl} and hen.\textsc{f.pl}\\
\glt `[...] what could they have, maybe ten, fifteen big animals and then also goats I know they had, and … pigs and hens [...].'
\z

This contrast is also found in \textit{pástar gròn} `main shepherd (literally `big shepherd')', vs \textit{pástar pin} `second shepherd (literally `small shepherd')', \textit{gròn} and \textit{pin} being adjectives that normally precede the noun.


Rule (9) (different meaning according to the position of the adjective) can be illustrated by \textit{néjv} / \textit{nùva} `new'.

\ea
\label{ex:neivauto}
\langinfo{Tuatschín}{Sadrún}{m5}\\
\gll Ju a cumprau in \textbf{néjv} \textbf{autò}.\\
\textsc{1sg} have\textsc{.prs.1sg} buy.\textsc{ptcp.unm} \textsc{indef.art.m.sg} new car\\
\glt `I bought a new car.'
\z

\ea
\label{ex:autoneif}
\langinfo{Tuatschín}{Sadrún}{m5}\\
\gll Ju a cumprau in \textbf{autò} \textbf{néjf}.\\
\textsc{1sg} have\textsc{.prs.1sg} buy.\textsc{ptcp.unm} \textsc{indef.art.m.sg} car new\\
\glt `I bought a new car.'
\z

In (\ref{ex:neivauto}) \textit{néjv autò} refers to a car that replaces an old one and which could be a second-hand car, whereas in (\ref{ex:autoneif}) \textit{autò néjf} refers to a brand-new car. This rule applies also to (\ref{ex:nufsplajds}) as opposed to (\ref{ex:cuntirejschnejv}) or (\ref{ex:formazjun}).

\ea
\label{ex:nufsplajds}
\langinfo{Tuatschín}{Ruèras}{m3, l. 1991}\\
\gll A vuṣ vajṣ anflau ò ampau \textbf{nùfs} \textbf{plajds}?    \\
and \textsc{2pl} have.\textsc{prs.2pl} find.\textsc{ptcp.unm} out a\_little new.\textsc{m.pl} word.\textsc{pl}\\
\glt `And did you find out some new words?'
\z


\ea
\label{ex:cuntirejschnejv}
\langinfo{Tuatscín}{Cavòrgja}{m7, l. 2355ff.}\\
\gll A prènd èl ò dal sacadòs \textbf{in} \textbf{cuntí} \textbf{réjsch} \textbf{néjv} \textbf{da} \textbf{mòni} \textbf{mèlan} [...].\\
and take.\textsc{prs.3sg} \textsc{3sg.m} out of.\textsc{def.art.m.sg} backpack \textsc{indef.art.m.sg} knife brand new.\textsc{m.sg} of handle.\textsc{m.sg} yellow\\
\glt `And he takes a brand-new knife with a yellow handle out of the backpack [...].'
\z

\ea
\label{ex:formazjun}
\langinfo{Tuatschín}{Zarcúns}{m2, l. 1539ff.}\\
\gll  [...] quaj òn vèvan nuṣ da fá sé … \textbf{ina} … \textbf{fòrmazjun} \textbf{nòva} ad ad \textbf{instrumajnts} \textbf{nùfs}.\\
{} \textsc{dem.m.sg} year have.\textsc{impf.1pl} \textsc{1pl} \textsc{comp} make.\textsc{inf} up {} \textsc{def.art.f.sg} {} lineup  new.\textsc{f.sg} and and instrument.\textsc{m.pl} new.\textsc{pl}\\
\glt `[...] that year we had to do ... a ... new lineup and and [buy] new instruments.'
\z

Some adjectives always follow the noun, as adjectives referring to colours or to demonyms: \textit{tgautschas \textbf{najras}} `black trousers' (line 1818) or \textit{ina sòra \textbf{tudestga}} `a Swiss-German nun' (line 799).

\subsection{Absence of agreement}
If the predicative adjective or past participle is left-dislocated in order to topicalize it (\ref{ex:noagr:1}), or if the adjective or the participle forms a semantic unit with the verb as in e.g. \textit{schá líbar} `let free' or \textit{vay mal} `have pain' (\ref{ex:noagr:2} - \ref{ex:noagr:7}), it does not agree with the noun it modifies and the unmarked form of the adjective is used.

\ea\label{ex:noagr:1}
\langinfo{Tuatschín}{Sadrún}{m5}\\
\gll   [...] parquáj ṣè \textbf{impurtònt} lura \textbf{la} \textbf{gramática} tga té fas […]. \\
     {} therefore \textsc{cop.prs.3sg} important.\textsc{adj.unm} then \textsc{def.art.f.sg} grammar \textsc{rel} \textsc{2sg} make.\textsc{prs.2sg}\\
\glt `[…] therefore the grammar you write is important […].'
\z

\ea\label{ex:noagr:2}
\langinfo{Tuatschín}{Sadrún}{m4}\\
\gll Ju a schau \textbf{aviart} agl ésch.   \\
 \textsc{1sg} have.\textsc{prs.1sg} leave.\textsc{ptcp.unm} open.\textsc{unm} \textsc{def.art.m.sg} door\\
\glt `I left the door open.'
\z

\ea\label{ex:noagr:3}
\langinfo{Tuatschín}{Ruèras}{\citealt[64]{Büchli1966}}\\
\gll    [...] al pur ò […] schau \textbf{líbar} l’ uélp.\\
     {} \textsc{def.art.m.sg} peasant have.\textsc{prs.3sg} {} let.\textsc{ptcp.unm} free.\textsc{adj.unm} \textsc{def.art.f.sg} fox\\
\glt `The peasant let […] the fox free.'
\z

\ea
\label{}
\langinfo{Tuatschín}{}{\DRG{10}{462}}\\
\gll schè \textbf{lartg} las vacas\\
let.\textsc{inf} free.\textsc{adj.unm} \textsc{def.art.f.pl} cow.\textsc{pl}\\
\glt `let the cows free'
\z

\ea
\label{ex:noagr:4}
\langinfo{Tuatschín}{Sadrún}{\citealt[105]{Büchli1966}}\\
\gll   […] quèla èra gjalada […] tg' èra nunpussajval li èl da \textbf{fá} \textbf{líbar} èla […].\\
      {} \textsc{dem.f.sg} \textsc{cop.impf.3sg} freeze.\textsc{ptcp.3sg.f} […] \textsc{comp} \textsc{cop.impf.3sg} impossible.\textsc{adj.unm} \textsc{dat} \textsc{3sg.m} to make.\textsc{inf} free.\textsc{adj.unm} \textsc{3sg.f}\\
\glt `[…] that [block] was frozen […] so it was impossible for him to get it free […].'
\z

\ea\label{ex:noagr:5}
\langinfo{Tuatschín}{}{\citealt[31]{Berther2007}}\\
\gll  Mia mèlna tschò \textbf{ò} \textbf{unflau} in téc \textbf{ina} \textbf{tgòmba} \textbf{dav\underline{ù}s}.  \\
     \textsc{poss.f.1sg} yellow.\textsc{f} here have.\textsc{prs.3sg} swell.\textsc{ptcp.unm}  \textsc{indef.art.m.sg} bit \textsc{indef.art.f.sg} leg back\\
\glt `My yellow [cow] has a back leg that is a bit swollen.'
\z

\ea
\label{ex:noagr:6}
\langinfo{Tuatschín}{Ruèras}{m10, l. 1084}\\
\gll   Al quinau vèva \textbf{mal} \textbf{ina} \textbf{tgòmba}. \\
\textsc{def.art.m.sg} brother-in-law have.\textsc{impf.3sg} bad.\textsc{adj.unm} \textsc{indef.art.f.sg} leg \\
\glt `My brother-in-law had leg pain.'
\z

\ea
\label{ex:noagr:7}
\langinfo{Tuatschín}{Cavòrgja}{f1}\\
\gll Nus stgavan ir \textbf{líbar} tùt \textbf{als} \textbf{pòrs}.   \\
 \textsc{1pl} be\_allowed.\textsc{impf.1pl}  go.\textsc{inf} free.\textsc{adj.unm} all  \textsc{def.art.m.pl} pig.\textsc{pl} \\
\glt `We were allowed to let all the pigs go around freely.'
\z

However, if an adjective is left-dislocated for focusing, it agrees with its noun (\ref{ex:agrfoc}).

\ea
\label{ex:agrfoc}
\langinfo{Tuatschín}{Ruèras}{m10}\\
\gll Quaj taur è bjals, ábar \textbf{gròns} ṣè `l bétg.\\
\textsc{dem.m.sg} bull \textsc{cop.prs.3sg} beautiful.\textsc{m.sg} but big.\textsc{m.sg} \textsc{cop.prs.3sg} \textsc{3sg.m} \textsc{neg}\\
\glt `This bull is beautiful, but big he is not.'
\z

\subsection{Conjoining of adjectives}
Adjectives are conjoined by the conjunction \textit{a} `and' (\ref{ex:fraidabletsch}).

\ea
\label{ex:fraidabletsch}
\langinfo{Tuatschín}{Tschamùt}{\citealt[18]{Büchli1966}}\\
\gll I èr' in dé \textbf{frajd} \textbf{a} \textbf{blètsch}.\\
\textsc{expl} \textsc{cop.impf.3sg} \textsc{indef.art.m.sg} day cold and wet\\
\glt `It was a cold and wet day.'
\z

In (\ref{ex:pingrias}) the adjective \textit{pin} `short' is followed by \textit{grias} `fat' without conjunction.

\ea
\label{ex:pingrias}
\langinfo{Tuatschín}{Ruèras}{m4, l. 2148ff.}\\
	\gll [...] quaj èr’ in súpar musicant … in ùm \textbf{pin} \textbf{griaṣ} usché [...].\\
{} \textsc{dem.unm} \textsc{cop.impf.3sg} \textsc{indef.art.m.sg} super musician {} \textsc{indef.art.m.sg} man short fat so \\
\glt `[...] he was a super musician ... a short and fat man [...].'
\z

This is due to the fact that (\ref{ex:pingrias}) is not a case of two conjoined adjectives, but here \textit{grias} modifies \textit{ùm pin}.

\section{Noun phrases and prepositional phrases modifying a noun}
Modifying nouns or noun phrases as well as prepositional phrases follow the modified noun. Prepositional phrases may head noun phrases or infinitive clauses.

\ea
\label{ex:npnoun1}
\langinfo{Tuatschín}{}{\DRG{3}{253}}\\
\gll  Scù `ls \textbf{dis}-\textbf{tgaun} vòn ajn, vòni òra.  \\
     as \textsc{def.art.m.pl} day.\textsc{pl}-dog.\textsc{m.sg} go.\textsc{prs.3pl} in go.\textsc{prs.3pl} out\\
\glt `As the dog days come, they also go.'
\z

\ea\label{}
\langinfo{Tuatschín}{Ruèras}{m10, l. 1054f.}\\
\gll Quaj è ina, asch’ ina \textbf{stazjun} \textbf{amiaz} \textbf{al} \textbf{pas} circa né … strusch séssum al pʰas.\\
\textsc{dem.unm} \textsc{cop.prs.3sg} \textsc{indef.art.f.sg} such \textsc{indef.art.f.sg} station amid \textsc{def.art.m.sg} pass around or {} almost up\_most \textsc{def.art.m.sg} pass\\
\glt `This is a, such a station in the middle [of the road to] the pass, approximately, or … almost on top of the pass.'
\z

\ea\label{}
\langinfo{Tuatschín}{Sadrún}{m5, l. 1180f.}\\
\gll Al sòntgèt dals gjadjuṣ è òdém al \textbf{vitg} \textbf{da} \textbf{Sadrún}.\\
\textsc{def.art.m.sg} little\_chapel of.\textsc{def.art.m.pl} Jew.\textsc{pl} \textsc{cop.prs.3sg}  out\_most \textsc{def.art.m.sg} village of \textsc{pln} \\
\glt `The little chapel of the Jews is located at the lowest part of the village of Sedrun.'
\z


\ea\label{}
\langinfo{Tuatschín}{Sadrún}{m6, l. 1428f.}\\
	\gll    A nuṣ duṣ èra cun la \textbf{tgapjala} \textbf{cun} \textbf{sé} \textbf{ina} \textbf{bjala} \textbf{flur} [...].\\
and \textsc{1pl} two.\textsc{m.pl} also with \textsc{def.art.f.sg} hat with on \textsc{indef.art.f.sg} beautiful flower [...].\\
\glt `And also the two of us with the hats with a beautiful flower on them [...].'
\z

\ea\label{}
\langinfo{Tuatschín}{Sadrún}{f3, l. 14}\\
	\gll «Ah, quaj fùṣ è ina \textbf{lavur} \textbf{pr} \textbf{mè}.»\\
	ah \textsc{dem.unm} \textsc{cop.cond.3sg} also \textsc{indef.art.f.sg} job for \textsc{1sg} \\
\glt `Ah, this could also be a job for me.'
\z

\ea\label{}
\langinfo{Tuatschín}{}{\DRG{3}{204}}\\
\gll  \textbf{calira} \textbf{da} \textbf{barsá} \textbf{vifs} \\
heat.\textsc{f.sg} \textsc{comp} burn.\textsc{inf} alive.\textsc{m.sg}\\
\glt `terrible heat'
\z

\ea\label{}
\langinfo{Tuatschín}{}{\DRG{2}{723-724}}\\
\gll   \textbf{butéglja} \textbf{da} \textbf{scadá} \textbf{pajs} \\
bottle.\textsc{f.sg} \textsc{comp} warm.\textsc{inf} foot.\textsc{pl} \\
\glt `hot-water bottle'
\z

\ea
\label{}
\langinfo{Tuatschín}{Sadrún}{m4, l. 475f.}\\
\gll  Lèdṣ vèva lu dau zatgéj \textbf{étg} \textbf{dad} \textbf{úndṣchar} \textbf{ajn} [...]. \\
\textsc{dem.m.sg} have.\textsc{impf.3sg} then  give.\textsc{ptcp.unm} some ointment.\textsc{m.sg} \textsc{comp} oil.\textsc{inf} in\\
\glt `He had given [him] some ointment to rub in [...].'
\z

\section{Personal pronouns}
\tabref{tab:perspron} shows the paradigm of the personal pronouns with the most widespread forms.

\begin{table}
\caption{Personal pronouns}
\label{tab:perspron}
 \begin{tabular}{lllll}
  \lsptoprule
 &  subject & direct object & indirect object & after preposition\\
  \midrule
\textsc{1sg} & \textit{ju} & \textit{mè} & \textit{da mé} & \textit{cun mè} \\
\textsc{2sg} & \textit{té} & \textit{té} & \textit{da té} & \textit{cun té}\\
\textsc{3sg.m} & \textit{èl} & \textit{èl} & \textit{dad èl} & \textit{cun èl} \\
\textsc{3sg.f} & \textit{èla} & \textit{èla} & \textit{dad èla} & \textit{cun èla}\\
\textsc{1pl} & \textit{nuṣ} & \textit{nuṣ} & \textit{da nuṣ} & \textit{cun nuṣ}\\
\textsc{2pl} & \textit{vuṣ} & \textit{vuṣ} & \textit{da vuṣ} & \textit{cun vuṣ}\\
\textsc{3pl.m} & \textit{èlṣ} & \textit{èlṣ} & \textit{dad èlṣ} & \textit{cun èlṣ}\\
\textsc{3pl.f} & \textit{èlaṣ} & \textit{èlaṣ} & \textit{dad èlaṣ} & \textit{cun èlaṣ}\\
\textsc{generic}\footnote{The generic pronoun will be treated in § 3.7 below} & \textit{ins} & \textit{ins} & \textit{dad ins} & \textit{cun ins}\\
 \lspbottomrule
 \end{tabular}
\end{table}

Subject and object pronouns are only differentiated in the first person singular. There is a certain variation between the object pronouns of the first person singular. In the corpus, there is only one case where \textit{mé} is used instead of \textit{mè}: \textit{cun mé} `with me' (line 1369).

With subject inversion, some contracted forms are optionally used: \textit{craj ju} → \textit{crau}, \textit{sa ju} `know I / can I' → \textit{sau / sòu}; \textit{va ju} `have I' → \textit{vau / vòu}, \textit{savajn nus} `know we / can we' → \textit{savajns / savajnṣ}; \textit{ò ins} `has one' → \textit{òns / ònṣ}.

There is more variation in the domain of passive voice. In a passive construction the agent is introduced by \textit{da}, and some speakers prefer using the pronoun \textit{mé} (dative), whereas others use \textit{mè} (accusative).

\ea
\label{}
\langinfo{Tuatschín}{Cavòrgja}{f1}\\
\gll Quaj cùdisch è vagnús scréts da \textbf{mé}.\\
\textsc{dem.m.sg} book be.\textsc{prs.3sg} \textsc{pass.aux.ptcp.m.sg} write.\textsc{ptcp.m.sg} \textsc{dat} \textsc{1sg}\\
\glt `This book has been written by me.'
\z

\ea
\label{}
\langinfo{Tuatschín}{Sadrún}{m9}\\
\gll Quaj cùdisch è vagnús scréts da \textbf{mè}.\\
\textsc{dem.m.sg} book be.\textsc{prs.3sg} \textsc{pass.aux.ptcp.m.sg} write.\textsc{ptcp.m.sg} by \textsc{1sg}\\
\glt `This book has been written by me.'
\z

This variation is triggered by the fact that \textit{da} is ambiguous: on the one hand it corresponds to the dative marker, hence \textit{mé}, and on the other it functions as a preposition, hence \textit{mè}.

In the second person singular there is no difference between direct and indirect object. As a consultant puts it:

\ea
\label{}
\langinfo{Tuatschín}{Surajn}{f5}\\
\gll  Nuṣ ṣchajn adina \textbf{té}. \textbf{Tè} è da bajbar.\\
\textsc{1pl} say.\textsc{prs.1pl} always \textsc{2sg} tea.\textsc{m.sg} \textsc{cop.prs.3sg} \textsc{cop} drink.\textsc{inf}\\
\glt `We always say \textit{té} (`you'). \textit{Tè} (`tea') is for drinking.'
\z

The subject pronouns \textit{èla} `she' and \textit{èlas} `they (f.)' are sometimes reduced to \textit{la} or \textit{las} when there is subject inversion.

\ea
\label{}
\langinfo{Tuatschín}{Sadrún}{f3}\\
\gll Cu ju sùn vagnús, èra \textbf{la} schòn lò.\\
when \textsc{1sg} be.\textsc{prs.1sg} to \textsc{cop.impf.3sg} \textsc{3sg.f} already there\\
\glt `Now I am going home.'
\z

\ea
\label{}
\langinfo{Tuatschín}{Camischùlas}{\citealt[82]{Büchli1966}}\\
\gll Tùtajnina ò \textbf{las} udju òd gl uaut ṣùt Cavòrgja ina vusch [...].\\
suddenly have.\textsc{prs.3sg} \textsc{3pl.f} hear.\textsc{ptcp.unm} out\_of \textsc{def.art.m.sg} forest under \textsc{pln} \textsc{indef.art.f.s}g voice\\
\glt `Suddenly they heard a voice [coming] out of the forest underneath Cavorgia [...].'
\z

The polite pronoun is \textit{Vus}, which triggers the second person plural form in the verb. There is, however, one exception: past participles. Whereas for masculine singular (and plural) referents, the plural form is used, for feminine singular referents, it is the the singular form that is used for singular reference.

\ea
\label{}
\langinfo{Tuatschín}{Sadrún}{m5}\\
\gll  Nua èssaṣ Vus \textbf{staj}?\\
 where be.\textsc{prs.2pl.pol/prs.2sg.pol} \textsc{2pl/2sg.pol} \textsc{cop.ptcp.m.pl}\\
\glt `Where have you (m. sg. and pl.) been?'
\z

\ea
\label{}
\langinfo{Tuatschín}{Sadrún}{m5}\\
\gll Nua èssaṣ Vus \textbf{stada}?\\
where be.\textsc{prs.2sg.pol} \textsc{2sg.pol} \textsc{cop.ptcp.f.sg}\\
\glt `Where have you (f.sg.) been?'
\z

\ea
\label{}
\langinfo{Tuatschín}{Sadrún}{m5}\\
\gll Nua èssaṣ Vus \textbf{stadas}?\\
where be.\textsc{prs.2pl} \textsc{2pl.pol} \textsc{cop.ptcp.f.sg}\\
\glt `Where have you (f.pl.) been?'
\z

In the first part of the 20th century, a polite pronoun \textit{Èls} was used, which corresponds to the third person plural masculine form. According to some consultants, this form was exclusively used with priests, and according to some others also with teachers and doctors. The verb agrees in number with \textit{Èls}.

\ea
\label{}
\langinfo{Tuatschín}{Camischùlas}{f6}\\
\gll La mùma raquénta tg' i ṣchèvan dá- dal aucs\underline{é}gnar a dál scòlast- ṣchèvani aun \textbf{Èls}. A da tschèls vagnévi détg \textbf{Vus}.\\
\textsc{def.art.f.sg} mother tell.\textsc{prs.1sg} \textsc{comp} \textsc{3pl} say.\textsc{impf.3pl} \textsc{dat} \textsc{dat.def.art.m.sg} priest and \textsc{dat.def.art.m.sg} teacher say.\textsc{impf.3pl.3pl} still \textsc{2sg.m.pol} and \textsc{dat} \textsc{dem.m.pl} \textsc{pass.aux.impf.3sg.expl} say.\textsc{ptcp.unm} \textsc{2sg.m.pol}\\
\glt `My mother says that they would say - to the priest and the teacher - would say \textit{Èls}. And to the others they would say \textit{Vus}.'
\z

The pronoun \textit{èl} `he' refers in very rare cases to entities that have no gender. The only example in the corpus is (\ref{ex:el.quaj}), a function that is normally fulfilled by \textit{quaj}, the demonstrative pronoun which is unmarked for gender.

\ea
\label{ex:el.quaj}
\langinfo{Tuatschín}{Sadrún}{m4, l. 486}\\
\gll «Vajs fatg \textbf{èl}?»\\
have.\textsc{prs.2sg.pol} do.\textsc{ptcp.unm} \textsc{3sg.m}\\
\glt `Did you do it?'
\z

In this context, \textit{èl} refers to the fact that the narrator's grandfather should take care of his wound.

The subject personal pronouns may be modified by the locative adverb \textit{cò} `here' (\ref{ex:perspronco}).

\ea
\label{ex:perspronco}
\langinfo{Tuatschín}{Camischùlas}{f6, l. 810f.}\\
\gll A lu, \textbf{nus} \textbf{cò} sursilvanas, matévan adina da pausa, mavanṣ ajn ṣala da magljè [...].\\
and then \textsc{1pl} here Sursilvan.\textsc{f.pl} put.\textsc{impf.1pl} always of pause go.\textsc{impf.1pl.1pl} in hall.\textsc{f.sg} \textsc{comp} eat.\textsc{inf}\\
\glt `And then we, the Sursilvan students, would always place [it] during the break, we would go into the dining hall [...].'
\z

With subject inversion some contracted forms are optionally used: \textit{craj ju} → \textit{crau}, \textit{sa ju} `know I / can I' → \textit{sau / sòu}; \textit{va ju} `have I' → \textit{vau / vòu}, \textit{savajn nus} `know we / can we' → \textit{savajns / savajnṣ}; \textit{ò ins} `has one' → \textit{òns / ònṣ}.

\ea
\label{}
\langinfo{Tuatschín}{Sadrún}{m8}\\
\gll Gljèz \textbf{crau} bé da té.\\
\textsc{dem.unm} believe.\textsc{1sg.1sg} \textsc{neg} of \textsc{2sg}\\
\glt `This I don't believe you could do it.'
\z

\ea
\label{}
\langinfo{Tuatschín}{Sadrún}{m4, l. 378ff.}\\
\gll   [...] tgé pudévan èlṣ vaj, déjsch quindiṣch armaulṣ gronṣ api lu aun tgauras \textbf{sòu} tga vèvan a … pòrṣ da gaglinaṣ [...]. \\
{} what can.\textsc{impf.3pl} \textsc{3pl.m} have.\textsc{inf} ten fifteen animal.\textsc{m.pl} big.\textsc{pl} and then besides goat.\textsc{f.pl} know.\textsc{prs.1sg.1sg} \textsc{comp} have.\textsc{impf.3pl} and pig.\textsc{m.pl} and hen.\textsc{f.pl}\\
\glt `[...] what could they have, maybe ten, fifteen big animals and then also goats I know they had, and … pigs and hens [...].'
\z

\ea
\label{}
\langinfo{Tuatschín}{Sadrún}{f3, l. 29}\\
\gll Api \textbf{vau} détg: «Ah súpar.»   \\
and have.\textsc{prs.1sg.1sg} say.\textsc{ptcp.unm} oh great\\
\glt `And then I said: «Oh, great!»'
\z

\ea
\label{}
\langinfo{Tuatschín}{Ruèras}{m1, l. 235ff.}\\
\gll    A quaj \textbf{stèvnṣ} èssar … pulits-pulits l’ jamna … tg’ al bap dètschi in frang a miaz.\\
and \textsc{dem.unm} must.\textsc{impf.1pl.1pl} \textsc{cop.inf} {} \textsc{red}\textasciitilde{well\_behaved}.\textsc{m.pl} \textsc{def.art.f.sg} week {} \textsc{comp} \textsc{def.art.m.sg} father  give.\textsc{prs.sbjv.3sg} one.\textsc{m.sg} franc and half.\textsc{m.sg}\\
\glt `And we had to be … very well-behaved during the week … so that my father would give [us] one and a half francs.'
\z

\ea
\label{}
\langinfo{Tuatschín}{Sadrún}{m6, l. 868ff.}\\
\gll    [...] da quèla \textbf{òns} naturálmajn … quasi stavjú … vagní cun ina détga.\\
{} of \textsc{dem.f.sg} have.\textsc{prs.3sg.gnr} of\_course almost must.\textsc{ptcp.unm} {} come.\textsc{inf} with \textsc{indef.art.f.sg} legend\\
\glt `[...] of this slab, one of course had … to ... come up with a legend.'
\z

Besides the pronouns presented in \tabref{tab:perspron}, there are two further personal pronouns: \textit{i} and \textit{aj}, which do not distinguish gender and which are synonyms of \textit{èls} or \textit{èlas}. These two pronouns also function as expletive pronouns (see § 3.5.2 below). Two cases must be distinguished:

\begin{itemize}
\item If \textit{i} and \textit{aj} function as a subject pronoun, \textit{i} is used before (\ref{ex:aj:1} and \ref{ex:aj:2}), and \textit{aj} after the verb (\ref{ex:aj:3} - \ref{ex:aj:5}). In the corpus there is only one case of \textit{aj} functioning as subject that is located before the verb (\ref{ex:aj:7}).
\item If the pronoun functions as an object pronoun, only \textit{aj} is used. It may refer to persons (\ref{ex:aj:6}) or to entities that have no gender (\ref{ex:aj:8}).
\end{itemize}

In (\ref{ex:aj:1}) and (\ref{ex:aj:2}), \textit{i } refers to masculine referents (animals and employees of the municipality), and in (\ref{ex:aj:2b}), it refers to feminine referents (nuns).

\ea
\label{ex:aj:1}
\langinfo{Tuatschín}{Ruèras}{\citealt[67]{Büchli1966}}\\
\gll  […] nòs tiars. \textbf{I} mavan adina gjù ancùnter las  plauncas  da l’ Òndadusa […].\\
{} \textsc{poss.1pl} animal.\textsc{pl} \textsc{3pl.sbj} go.\textsc{impf.3pl} always down towards \textsc{def.art.f.pl} slope.\textsc{pl} of \textsc{def.art.f.sg} \textsc{pln}\\
\glt `[…] our animals. They would always go down towards the slopes of the Ondadusa.'
\z

\ea
\label{ex:aj:2}
\langinfo{Tuatschín}{Sadrún}{f3, l. 25f.}\\ 
\gll  [...] mi' ùm ségi èba mòrts scù \textbf{i} sápjan [..].\\
{} \textsc{poss.1sg.m.sg} man be.\textsc{prs.sbjv.3sg} precisely die.\textsc{ptcp.m.sg} as \textsc{3pl.sbj} know.\textsc{prs.sbjv.3pl}\\
\glt `[...] my husband had died as they knew [...].'
\z

\ea
\label{ex:aj:2b}
\langinfo{Tuatschín}{Camischùlas}{f6, l.780f.}\\ 
\gll    A \textbf{i} miravan schòn da métar ansjaman è … als lungatgs [...].\\
and \textsc{3pl} look.\textsc{impf.3pl} in\_fact \textsc{comp} put.\textsc{inf} together also {} \textsc{def.art.m.pl} language.\textsc{pl}\\
\glt `And in fact, they would make sure to put … the languages together [...].'
\z

\ea
\label{ex:aj:3}
\langinfo{Tuatschín}{}{\DRG{3}{639}}\\
\gll  Schi òn bétg als còrns, schi mòrdan \textbf{aj}.  \\
     if have.\textsc{prs.3sg} \textsc{neg} \textsc{def.art.m.pl} horn.\textsc{pl} \textsc{corr} bite.\textsc{prs.3pl} \textsc{3pl.sbj}\\
\glt `If they [the goats] don’t have horns, they bite.'
\z

\ea
\label{ex:aj:4}
\langinfo{Tuatschín}{Bugnaj}{\citealt[136]{Büchli1966}}\\
\gll Lu ṣèn \textbf{aj} i sén claustra […].\\
then \textsc{cop.prs.3pl} \textsc{3pl}  go.\textsc{ptcp.m.3pl} on monastery\\
\glt `Then they went up to the monastery […].'
\z

\ea
\label{ex:aj:5}
\langinfo{Tuatschín}{Sadrún}{f3, l. 75f.}\\
\gll  Api quèls da la vischnaunca èran grad vid' al, vi da zaná al bògn, api ṣchèvan \textbf{aj}: [...]. \\
and \textsc{dem.m.pl} of \textsc{def.art.f.sg} municipality \textsc{cop.impf.3pl} just \textsc{prog} \textsc{comp.def.art.m.sg} \textsc{prog} \textsc{comp} renovate.\textsc{inf} \textsc{def.art.m.sg} bath and\_then say.\textsc{impf.3pl} \textsc{3pl} \\
\glt `And the municipal employees were just renovating the swimming pool and then they said: [...].'
\z

\ea
\label{ex:aj:7}
\langinfo{Tuatschín}{Ruèras}{m3, l. 2142ff.}\\
\gll  Anqual jèda vagnéva lu al pás[tar] … né usché cu \textbf{aj} vasévan a gidavan tòca tg’ ins èr’ ajn … ajn «ṣchwung»\footnotemark{} [...].\\
some time.\textsc{f.sg} come.\textsc{impf.3sg} then \textsc{def.art.m.sg} herdsman {} or so when \textsc{3pl} see.\textsc{impf.3pl} and help.\textsc{impf.3pl} until \textsc{comp} \textsc{gnr} \textsc{cop.impf.3sg} in {} in momentum.\textsc{m.sg}\\
\glt `Sometimes the herdsman would come ... or so, when they saw and they would help until one was again in momentum [...].'\footnotetext{\textit{Schwung} is German for Romansh \textit{slontsch}.}
\z

\ea
\label{ex:aj:6}
\langinfo{Tuatschín}{Ruèras}{m1, l.274ff.}\\
\gll    Las nòtízjas sa ju bétg danùndar als gjaniturs, als duṣ baps prandèvan \textbf{aj}, i dèva ajnta Ruèras, dèv’ aj in tga vèva r\underline{á}djò.\\
\textsc{def.art.f.pl} news.\textsc{pl} know.\textsc{prs.1sg} \textsc{1sg} \textsc{neg} from\_where \textsc{def.art.m.pl} parents.\textsc{pl} \textsc{def.art.m.pl} two.\textsc{m.pl} father.\textsc{pl} take.\textsc{impf.3pl} \textsc{3pl} \textsc{expl} \textsc{exist.impf.3sg} in \textsc{pln} \textsc{exist.impf.3sg} \textsc{expl}  one.\textsc{m.sg} \textsc{rel} have.\textsc{impf.3sg} radio.\textsc{m.sg} \ \\
\glt `I don’t know where my parents had the news from, the two fathers took them, there was in Rueras, there was [only] one who had a radio.'
\z

\ea
\label{ex:aj:8}
\langinfo{Tuatschín}{Sadrún}{m8, l. 1508}\\
\gll  Álṣò ju a bigja fatg \textbf{aj} agrèssíf [...].\\
well \textsc{1sg} have.\textsc{prs.1sg} \textsc{neg} make.\textsc{ptcm.unm} \textsc{3.unm} aggressive.\textsc{adj.unm}\\
\glt `Well, I didn’t do it in an aggressive way [...].'
\z

\subsection{Dative pronouns}
Nowadays, the dative marker for all persons is \textit{da/dad}, but until approximatively the 1960, \textit{da/dad} was used for first and second persons and \textit{di} or \textit{li} for third persons. Note that in contrast to the correspondent dative article (§ 3.2.1.2 above) the pronoun does not differentiate number. The following examples illustrate the earlier usage. 

\ea\label{}
\langinfo{Tuatschín}{Surajn}{\citealt[128f.]{Büchli1966}}\\
\gll    Mù quèlas òn scumandau \textbf{da} \textbf{mé} da dí òra tgi èlas ségjen […].\\
     but \textsc{dem.f.pl} have.\textsc{prs.3pl} forbid.\textsc{ptcp.unm} \textsc{dat} \textsc{1sg} \textsc{comp} say.\textsc{inf} out who \textsc{3pl.f} \textsc{cop.prs.sbjv.3pl} \\
\glt `But these [girls] forbade me to say who they were […].'
\z

\ea\label{}
\langinfo{Tuatschín}{}{\DRG{3}{499}}\\
\gll   \textbf{Da} \textbf{té} dèṣ ins rúmpar ajn la cavaza.\\
\textsc{dat} \textsc{2sg} must.\textsc{cond.3sg} \textsc{gnr} break.\textsc{inf} in \textsc{def.art.f.sg} skull\\
\glt `One should break your head.'
\z

\ea\label{}
\langinfo{Tuatschín}{Tschamùt}{\citealt[14]{Büchli1966}}\\
\gll  Ad èl ò ancùnuschju sju cuntí a détg quaj \textbf{li} \textbf{èla}. \\
 and 3\textsc{sg.m} have.\textsc{prs.3sg} know.\textsc{ptcp.unm}  \textsc{poss.3sg.m.sg} knife and say.\textsc{ptcp.unm} \textsc{dem.unm} \textsc{dat} \textsc{3sg.f}\\
\glt `And he recognized his knife and told her that.'
\z
 
\ea\label{}
\langinfo{Tuatschín}{Tschamùt}{\citealt[14]{Büchli1966}}\\
\gll  Ad èla ò piau \textbf{li} \textbf{èl} tùts tiars par angrazjá \textbf{li} \textbf{èl}.\\
     and 3\textsc{sg}.\textsc{f} have.\textsc{prs}.3\textsc{sg} pay.\textsc{ptcp.unm} \textsc{dat} 3\textsc{sg}.\textsc{m} all.\textsc{m}.\textsc{pl} animal.\textsc{pl} \textsc{purp} thank.\textsc{inf} \textsc{dat} 3\textsc{sg}.\textsc{m}\\
\glt `And she paid him all the animals in order to thank him.'
\z

\ea\label{}
\langinfo{Tuatschín}{Ruèras}{\citealt[65]{Büchli1966}}\\
\gll  Ju angrazja parsjantar a dùn \textbf{da} \textbf{Vuṣ} al cuntí par in survètsch bégn fatg.\\
    \textsc{1sg} thank.\textsc{prs.1sg} therefore and give.\textsc{prs.1sg} \textsc{dat} \textsc{2sg.pol} \textsc{def.art.m.sg} knife for \textsc{indef.art.m.sg} service well do.\textsc{ptcp.unm}\\
\glt `I thank [you] for it and give you the knife for a well done favour.'
\z

 \ea\label{}
\langinfo{Tuatschín}{Sèlva}{\citealt[53]{Büchli1966}}\\
\gll  La sèra ò in pur dau suttètg \textbf{li} \textbf{èlas} […].\\
     \textsc{def}.\textsc{art}.\textsc{f}.\textsc{sg} evening have.\textsc{prs}.3\textsc{sg} \textsc{indef}.\textsc{art}.\textsc{m}.\textsc{sg} farmer give.\textsc{ptcp.unm} accommodation.\textsc{m.sg} \textsc{dat} \textsc{3pl.f}\\
\glt `In the evening a farmer offered them accommodation […].'
\z

In contrast to full noun phrases, the forms \textit{di} and \textit{li} with pronouns occur neither in the \textit{Dicziunari Rumantsch Grischun} nor in my own corpus.

With third persons there are some examples of the standard Sursilvan dative marker \textit{a/ad} in \citet{Büchli1966}, and in \citet{Gartner1910}, only \textit{a/ad} occurs.

\ea\label{}
\langinfo{Tuatschín}{Bugnaj}{\citealt[145]{Büchli1966}}\\
\gll  […] a lu ṣaj vagnú andamajn \textbf{ad} \textbf{èl} quèla Nòssadùna sél’ alp […].\\
{} and then \textsc{cop.prs.3sg.expl} come.\textsc{ptcp.unm} in\_mind \textsc{dat} \textsc{3sg.m} \textsc{dem.f.sg} Virgin on.\textsc{def.art.f.sg} alp \\
\glt `[...] and then that holy Virgin on the alp came into his mind.'
\z

\ea\label{}
\langinfo{Tuatschín}{}{\citealt[33]{Gartner1910}}\\
\gll leˈʥi kwɛl a daj glajti anavɔz ɛl \textbf{ad} \textbf{ɛl}\\
     read.\textsc{imp.2pl} \textsc{dem.m.sg} and give.\textsc{imp.2pl} soon back \textsc{3sg.m} \textsc{dat} \textsc{3sg.m}\\
\glt `Read it and give it soon back to him.'
\z

 \ea\label{}
\langinfo{Tuatschín}{}{\citealt[86]{Gartner1910}}\\
\gll ad ɛl ɔ partiw \textbf{ad} \textbf{ɛlts} la rawba\\
     and \textsc{3sg.m} have.\textsc{prs.3sg} distribute.\textsc{ptcp.unm} \textsc{dat} \textsc{3pl.m} \textsc{def.art.f.sg} fortune\\
\glt `And he distributed his fortune among them.'
\z

The only occurrence of the form \textit{ada} of the dative article occurs in the DRG; however, there it is only given in parenthesis as an alternative to \textit{da}.

\ea\label{}
\langinfo{Tuatschín}{}{\DRG{3}{462}}\\
\gll  Dá ina castrada \textbf{(a)da} quèla rùsna.\\
     give.\textsc{imp.2sg} \textsc{def.art.f.sg} act\_of\_joining \textsc{dat} \textsc{dem.f.sg} hole\\
\glt `Tie together this hole with a cord. '
\z

As mentioned above, today only \textit{da/dad} is used in all cases. The following examples illustrate the use of \textit{da/dad} with third persons.

\ea
\label{}
\langinfo{Tuatschín}{Sadrún}{f3, l. 18f.}\\ 
\gll  [...], api lura va ju in’ jèda talafònau \textbf{dad} \textbf{èl} [...].\\
{} and then have.\textsc{1sg}  \textsc{1sg} one.\textsc{f.sg} time call.\textsc{ptcp.unm} \textsc{dat} \textsc{3sg.m}\\
\glt `[...] and then I phoned him once [...].'
\z

\ea
\label{}
\langinfo{Tuatschín}{Sadrún}{f6, l. 746f.}\\
\gll  «Cool, ju mòn grad a raquénta \textbf{dad} \textbf{èlas}.»\\
cool \textsc{1sg}  go.\textsc{prs.1sg} right\_away and tell.\textsc{prs.1sg} \textsc{dat} \textsc{3pl.f} \\
\glt `«Cool, I’ll just go and tell them.»'
\z

Other definite pronouns like the demonstratives were treated as the personal pronouns. In (\ref{ex:lisg}), \textit{tschèl} `the other' is pluralized but \textit{li} is not.

\ea\label{ex:lisg}
\langinfo{Tuatschín}{Sadrún}{\citealt[104]{Büchli1966}}\\
\gll  […] lu ò ‘l détg \textbf{li} \textbf{tschèls}: […].\\
{} then have.\textsc{prs.3sg} \textsc{3sg.m} say.\textsc{ptcp.unm} \textsc{dat} \textsc{dem.m.pl}\\
\glt `[…] then he said to the others: […].'
\z


\subsection{Expletive pronoun} 
 The expletive pronoun is usually \textit{i}, in preverbal position (\ref{ex:expl:1} - \ref{ex:expl:4}) as well as in the case of subject inversion (\ref{ex:expl:5}). It is used in existential constructions (\ref{ex:expl:1}), with the verb \textit{vagní} as an auxiliary in impersonal passive constructions (\ref{ex:expl:2}) or as an inchoative copula (\ref{ex:expl:3}). It is also used in active impersonal constructions where the subject is located after the verb (\ref{ex:expl:4}). 
 
 \ea\label{ex:expl:1}
 \langinfo{Tuatschín}{Sadrún}{f3, l. 45f.}\\ 
 \gll [...] tùt quèls … lògans nù \textbf{i} \textbf{èran} muossavias.   \\
 {} all \textsc{dem.m.pl} {} place.\textsc{pl} where \textsc{expl} \textsc{exist.impf.3pl} signpost.\textsc{f.pl}\\
\glt `[...] all these … places where there were signposts.'
 \z
 
 \ea\label{ex:expl:2}
 \langinfo{Tuatschín}{Sadrún}{f3, l. 56ff.}\\ 
 \gll  Quaj è pròpi in ljuc ... nù tg’ \textbf{i} \textbf{vagnéva} \textbf{schau} tùt la munizjun tg’ i vèva, sigir.\\
 \textsc{dem.unm} \textsc{cop. prs.3sg} exactly \textsc{indef.art.m.sg} place {} where \textsc{rel} \textsc{expl} \textsc{pass.aux.impf.3sg} leave.\textsc{ptcp.unm} all \textsc{def.art.f.sg} munition \textsc{rel} \textsc{expl} \textsc{exist.impf.3sg} sure.\textsc{adj.unm}\\
\glt `This is exactly a place ... where they stored all the munition, for sure.'
\z
 
\ea\label{ex:expl:3}
\langinfo{Tuatschín}{Bugnaj}{\citealt[145]{Büchli1966}}\\
\gll […] \textbf{i} végn unviarn a végn frajt. \\
     {} \textsc{expl} come.\textsc{prs.3sg} winter and come.\textsc{prs.3sg} cold \\
\glt `Winter is coming and it is getting cold.'
\z

\ea\label{ex:expl:4}
\langinfo{Tuatschín}{}{\DRG{2}{215}}\\
\gll \textbf{I} briṣcha la cazèta, \textbf{i} végn tga sufla. \\
   \textsc{expl} burn.\textsc{prs.3sg} \textsc{def.art.f.sg} pot \textsc{expl} come.\textsc{prs.3sg} \textsc{rel} blow.\textsc{prs.3sg} \\
\glt `[The soot in] the pot is burning, it is getting stormy.'
\z

\ea\label{ex:expl:5}
\langinfo{Tuatschín}{Sadrún}{m5, l. 1217ff.}\\ 
\gll  [...] sén quaj pas \textbf{duèssi} èssar ina samagljònta caplùta [...].\\
{} on \textsc{dem.m.sg} pass should.\textsc{cond.3sg.expl} \textsc{cop.inf} \textsc{indef.art.f.sg} similar chapel\\
\glt `[...] on this pass there should be a similar chapel [...].'
\z

With subject inversion, the expletive pronoun is sometimes realized \textit{aj} (\ref{ex:expl:6}). In combination with the third person singular present of the verb \textit{èssar} `be' the combination is realized \textit{ṣaj} (\ref{ex:expl:7}), \textit{ṣè} (\ref{ex:expl:8}), or less frequently \textit{ásaj} (\ref{ex:expl:9}).

\ea\label{ex:expl:6}
\langinfo{Tuatschín}{Ruèras}{m1, l. 173f.}\\ 
\gll  [...] plé baut èr’ \textbf{aj} al fagljét tgi fijèv’ al pur.  \\
{} more early \textsc{cop.impf.3sg} \textsc{expl} \textsc{def.art.m.sg}  son.\textsc{m.sg.dim} \textsc{rel} do.\textsc{impf.3sg} \textsc{def.art.m.sg} farmer \\
\glt `[...] in earlier days it was the youngest son who worked as a farmer.'
\z

\ea
\label{ex:expl:7}
\langinfo{Tuatschín}{Ruèras}{m1, l. 178}\\ 
\gll  Api \textbf{ṣaj} stau finju par mè.\\
and be.\textsc{prs.3sg.expl} \textsc{cop.ptcp.unm} finish.\textsc{ptcp.unm} for \textsc{1sg}\\
\glt `And that was it.'
\z

\ea
\label{ex:expl:8}
\langinfo{Tuatschín}{Sadrún}{m4, l. 420ff.}\\
\gll Ah… Nalps è vagnú fraquantau ò scù majṣès ad alps adina, lu \textbf{ṣè} aun dus trajs intarassants lògans ajn cò [...].\\
ah  \textsc{pln}  be.\textsc{prs.3sg} \textsc{pass.aux.ptcp.unm} visit.\textsc{ptcp.unm} out as assembly\_of\_houses and alp.\textsc{m.pl} always then \textsc{exist.prs.3sg} still two three interesting place.\textsc{m.pl} in here \\
\glt `Ah … Nalps has always been visited as an assembly of houses and as pastures, there were furthermore two or three interesting places up there [...].'
\z

\ea
\label{ex:expl:9}
\langinfo{Tuatschín}{Zarcúns}{m2, l. 1678}\\
\gll    Ad òz \textbf{ásaj} bitga trajs, gè, tga vòn á scùla.\\
and today \textsc{exist.prs.3sg.expl} \textsc{neg} three yes \textsc{rel} go.\textsc{prs.3sg} to school.\textsc{f.sg}\\
\glt `And today there aren’t [even] three that attend school.'
\z


\subsection {Intensive personal pronouns}
The personal pronouns may be modified by the intensive pronouns \textit{mèz} `myself' etc. The paradigm is as follows: \textit{mèz, -a}, \textit{tèz, -a}, \textit{sèz, -a}, \textit{nus sèzs, -as}, \textit{vus sèzs, -as}, \textit{sèzs, -as}; however, there is a general tendency in Sursilvan varieties to use \textit{sèz} for all persons, a development which parallels the case of reflexive \textit{sa-} which is used for all persons in all tenses and moods. In the corpus, there are only examples for \textit{mèz} `myself' and \textit{sèz} `herself, himself'. These may be used attributively (\ref{ex:jumeza}) or predicatively (the other examples).

\ea
\label{ex:jumeza}
\langinfo{Tuatschín}{Cavòrgja}{\citealt[126]{Büchli1966}}\\
\gll \textbf{Ju} \textbf{mèza} vaj santju quaj.\\
\textsc{1sg} \textsc{1sg}.self have.\textsc{prs.1sg} feel.\textsc{ptcp.unm} \textsc{dem.unm}\\
\glt`I felt this myself.'
\z

In (\ref{ex:meza}) the narrator uses first \textit{sèza} instead of \textit{tèza} `yourself' and then \textit{mèza} for `myself'.

\ea
\label{ex:meza}
\langinfo{Tuatschín}{Sadrún}{f3, l. 79f.}\\
\gll  Api vau tartgau: «Jò nu, lu fas halt \textbf{sèza}.» ad ábar turnau ajn da quaj da quindisch pétgas \textbf{mèza}.\\
and have.\textsc{prs.1sg.1sg} think.\textsc{ptcp.unm} yes now then do.\textsc{prs.2sg} simply self.\textsc{f.sg} and but turn.\textsc{ptcp.unm} in of \textsc{dem.unm} of fifteen post.\textsc{f.pl} self.\textsc{1sg.f} \\
\glt `And then I thought: «In this case, you simply do it yourself.» and I put in something like fifteen posts myself.'
\z

The following examples show the use of \textit{sèzs} with first and second person plural. There are no examples of first person plural in the corpus.

\ea
\label{ex:sezs1pl}
\langinfo{Tuatschín}{Sadrún}{m6, l. 1353f.}\\
\gll    [...] \textbf{nus} stèvan \textbf{sèzs} … catschá als pòrs ò da nuégl [...].\\
{} \textsc{1pl} must.\textsc{impf.1pl} self.\textsc{m.pl} {}  drove.\textsc{inf} \textsc{def.art.m.pl} pig.\textsc{pl} out of barn.\textsc{m.sg}\\
\glt `[...] we had … to drove the pigs out of the barn ourselves [...].'
\z

\ea
\label{ex:sezs3pl}
\langinfo{Tuatschín}{Sadrún}{f3, l. 36f.}\\
\gll [...] a \textbf{lèzs} òn \textbf{sèzs} stavju lura … métar sén pajs quaj [...]. \\
{} and \textsc{dem.m.pl} have.\textsc{prs.3pl} self.\textsc{m.pl} must.\textsc{ptcp.unm} then {} put.\textsc{inf} on foot.\textsc{m.pl} \textsc{dem.unm}\\
\glt `[...] and these had to get it off the ground themselves [...].'
\z


\section{Relative clauses}
In spontaneous speech only two relativizers occur: \textit{tga} for most cases, and \textit{nùca}, \textit{nùca tga}, or \textit{nù tga} which introduce locative relative clauses. \textit{Tga} is reduced to \textit{tg}' before a word that starts with a vowel, and when followed by \textit{ju} `I', the combination is realized \textit{tgu} `that I' in most cases.\footnote{This form is also used as a contracted form of \textit{cu} `when' with \textit{ju} `I', see § 6.2.3 below.}

The following examples illustrate the function of \textit{tga} heading subjects, direct objects, and temporal arguments.

\ea\label{}
\langinfo{Tuatschín}{Sadrún}{m4, l. 356f.}\\
\gll   Quaj èra pádars \textbf{tga} \textbf{vagnévan}, parvi da mè mintg’ autar òn fagévan misjun. \\
\textsc{dem.unm} \textsc{cop.impf.3sg} Father.\textsc{m.pl} \textsc{rel} come.\textsc{impf.3pl} because of \textsc{1sg} every other.\textsc{m.sg} year make.\textsc{impf.3pl} mission.\textsc{f.sg} \\
\glt `These were Fathers who came, let’s say every two years they undertook a mission.' (subject)
\z

\ea\label{}
\langinfo{Tuatschín}{Sadrún}{f3, l. 1f.}\\
\gll   Ju raquénta da \textbf{mia} \textbf{lavur} \textbf{tga} ju \textbf{\textbf{a}} \textbf{fatg} als dav\underline{ù}ṣ òns. \\
\textsc{1sg} tell.\textsc{prs.1sg} of \textsc{poss.1sg.f.sg} job \textsc{rel} \textsc{1sg}  have.\textsc{prs.1sg} do.\textsc{ptcp.unm} \textsc{def.art.m.pl} last.\textsc{pl} year.\textsc{pl}\\
\glt `I’ll tell [you] about the job I have done during the last years.' (direct object)
\z

\ea\label{}
\langinfo{Tuatschín}{Sadrún}{m4, l. 344f.}\\
\gll A zuar ah da quaj \textbf{mùm\underline{é}n} [...] \textbf{tga} `l tat vèva da fá cun gnarvaṣ. \\
and namely eh from \textsc{dem.unm} moment {} \textsc{rel} \textsc{def.art.m.sg} grandfather have.\textsc{impf.3sg} \textsc{comp} do.\textsc{inf} with nerve.\textsc{f.pl} \\ 
\glt `Namely from the moment [...] when my grandfather had problems with his nerves.' (temporal argument)
\z

Prepositional phrases are not relativized with a preposition; instead, only the relative pronoun \textit{tga} is used and no trace of the preposition phrase occurs in the relative clause. In (\ref{relinstr}), it is the preposition \textit{cun} `with', in (\ref{relprepda}) \textit{da} `of', in (\ref{relpreppar1}) \textit{par} `because of, why', in (\ref{relpreppar2}) `for', and in (\ref{relpreptras}) \textit{atrás} `through' that are intended.

\ea\label{relinstr}
\langinfo{Tuatschín}{Ruèras}{\citealt[8f.]{Valär2013b}}\\
\gll    […] cʊ‘ʎɛːdɐ \textbf{ʨ'} ins lɐ‘vajɐ nʊǝt ɐlz djants dɐ rʊj.\\
{} clotted.milk \textsc{rel} \textsc{gnr} damage.\textsc{prs.3sg} \textsc{neg} \textsc{def.art.m.pl} tooth.\textsc{pl} \textsc{comp} gnaw.\textsc{inf} \\
\glt `[…] clotted milk with which you do not damage your teeth when you gnaw at it.'
\z

\ea\label{relprepda}
\langinfo{Tuatschín}{Sadrún} {m4, l. 505}\\
\gll [...] grat cò nùca \textbf{quèla} \textbf{rùsna} \textbf{tgu} a raquintau.\\
{} just there by \textsc{dem.f.sg} hole \textsc{rel.1sg} have.\textsc{prs.1sg} tell.\textsc{ptcp.unm}\\  
\glt `[...] just by that cave I have mentioned.'
\z

\ea\label{relpreppar1}
\langinfo{Tuatschín}{Sadrún}{m4, l. 569ff.}\\
\gll  [...] quaj è fòrza schòn stau in téc \textbf{al} … \textbf{mòtif} \textbf{tgu} a antschiat dad í par crapa.\\
{} \textsc{dem.unm} be.\textsc{prs.3sg} maybe really  \textsc{cop.ptcp.unm} \textsc{indef.art.m.sg} bit \textsc{def.art.m.sg} {} reason \textsc{rel.1sg} have.\textsc{prs.1sg} begin.\textsc{ptcp.unm} \textsc{comp} go.\textsc{inf} for stone.\textsc{coll}\\
\glt `[...] maybe this has been a bit the reason why I began to go for stones. '
\z

\ea
\label{relpreppar2}
\langinfo{Tuatschín}{Ruèras} {m3, l. 2027ff.}\\
	\gll [...] i èra da quèls tg’ èran angrazjajvels, èra da \textbf{quèls} \textbf{tg}’ ins savèva maj fá avùnda.\\
 {} \textsc{expl} \textsc{\textbf{}exist.impf.3sg} of \textsc{dem.m.pl} \textsc{rel} \textsc{cop.impf.3pl} grateful.\textsc{m.pl} also of \textsc{dem.m.pl} \textsc{rel} \textsc{gnr} can.\textsc{impf.3sg} never do.\textsc{inf} enough\\
\glt `[...] there were some who were grateful, also some for whom you never could do enough.'
\z

\ea
\label{relpreptras}
\langinfo{Tuatschín}{Ruèras}{m10, l. 1042ff.}\\
\gll  A lu vèvan, òni fatg ò dal mir, álṣò òr dal grép òni fatg \textbf{ina} \textbf{pintga} …  \textbf{sènda} \textbf{tg}’ ins sò ira ah á paj flòt.\\
and then have.\textsc{impf.3pl} have.\textsc{prs.3pl.3pl} make.\textsc{ptcp.unm} out of.\textsc{def.art.m.sg} rock\_face this\_is\_to\_say out of.\textsc{def.art.m.sg} rock have.\textsc{prs.3pl.3pl} make.\textsc{ptcp.unm} \textsc{indef.art.f.sg} small {} path \textsc{rel} \textsc{gnr} can.\textsc{prs.3sg} go.\textsc{inf} ah on foot.\textsc{m.sg} easy.\textsc{adj.unm} \\
\glt `And then they made, out of the rock face, this is to say out of the rock they made a small … path through which one could easily go eh on foot.'
\z

If the antecedent is \textit{tgi} `who' or \textit{tgé} `what', the relativizer is realized \textit{ca}.

\ea\label{}
\langinfo{Tuatschín}{Camischùlas}{f6, l. 717ff.}\\
\gll    [...] \textbf{tgi} \textbf{ca} vagnéva traplaus stuèva al vèndardis sèra … stá lò [...].\\
{}  who \textsc{rel} \textsc{pass.aux.impf.3sg} catch.\textsc{ptcp.m.sg} must.\textsc{impf.3sg} \textsc{def.art.m.sg} Friday evening.\textsc{f.sg} {} stay.\textsc{inf} there\\
\glt `[...] the person who got caught had to … remain there on Friday evening [...].'
\z

\ea\label{}
\langinfo{Tuatschín}{}{\DRG{6}{449}}\\
\gll   \textbf{Tgi} {c}’ è da stròm daj sapartgirá dal fjuc.\\
     who \textsc{rel} \textsc{cop.prs.3sg} of straw must.\textsc{prs.3sg} \textsc{refl}.be\_on\_one’s\_guard.\textsc{inf} of.\textsc{def.art.m.sg} fire\\
\glt `A person who is made of straw should be on their guard against fire.'
\z

\ea
\label{}
\langinfo{Tuatschín}{Zarcúns}{m2, l. 1564f.}\\
\gll [...] la sèra vajn nus halt vulju savaj \textbf{tgé} \textbf{ca} cùri aparti cò [..].\\
{} \textsc{def.art.f.sg} evening have.\textsc{prs.1pl} \textsc{1pl} just want.\textsc{ptcp.unm} know.\textsc{inf} what \textsc{comp} run.\textsc{prs.sbjv.3sg} special here\\
\glt `[...] in the evening we wanted to know whether there was something special going on [...].'
\z

The relative clause does not have to be adjacent to its antecedent, as example (\ref{relnonadj}) shows.

\ea
\label{relnonadj}
\langinfo{Tuatschín}{Sadrún}{f3, l. 69ff.}\\
\gll  [...] ábar stòpi prèndar \textbf{malitèr} cun mè, \textbf{tga} vajan… fùnc a sapjan [...] dí cu nus vajan da … ir davùṣ in cuélm.\\
{} but must.\textsc{prs.sbjv.1sg} take.\textsc{inf} military.\textsc{m.sg} with \textsc{1sg} \textsc{rel} have.\textsc{prs.sbjv.3sg} radio.\textsc{m.sg} and can.\textsc{prs.sbjv.3pl} {} say.\textsc{inf} when \textsc{1pl} have.\textsc{prs.sbjv.1pl} \textsc{comp} {} go.\textsc{inf} behind \textsc{indef.art.m.sg} mountain\\
\glt `[...] but I needed to take with me some soldiers that had radio and would say [...] when we should … go behind a mountain [to protect ourselves].'
\z

In rare cases the relativizer \textit{tga} is realised \textit{tgé} or \textit{tgi}. Note, however, that \textit{tgi} as a relativizer is rejected by most of my consultants.

\ea
\label{}
\langinfo{Tuatschín}{Ruèras}{\citealt[62]{Büchli1966}}\\
\gll [...] in pástar gròn, quèl vasèva adina la damaun da bjalaura ina nibla najra \textbf{tgé} satschèntava séla Sjara da Curnèra [...].\\
{} \textsc{indef.art.m.sg} herdsman big \textsc{dem.m.sg} see.\textsc{impf.3sg} always \textsc{def.art.f.sg} morning of nice\_weather.\textsc{f.sg} \textsc{indef.art.f.sg} cloud black \textsc{rel} \textsc{refl}.sit.\textsc{impf.3sg} on.\textsc{def.art.f.sg} \textsc{pln}\\
\glt `[...] a main herdsman, in the morning in nice weather, he always saw a black cloud that lied on the Siara da Curnera [...].'
\z

\ea
\label{}
\langinfo{Tuatschín}{}{\DRG{5}{57}}\\
\gll L' ò ina véglja davùs pégna \textbf{tgi} tgaja danès.\\
\textsc{3sg.m} have.\textsc{prs.3sg} \textsc{indef.art.f.sg} old behind oven.\textsc{f.sg} \textsc{rel} shit.\textsc{prs.3sg} money.\textsc{m.pl}\\
\glt `He has an old woman behind the oven who shits money.'
\z

\ea
\label{}
\langinfo{Tuatschín}{Ruèras}{m1, l. 173f.}\\
\gll  [...] plé baut èr’ aj al fagljét \textbf{tgi} fijèv’ al pur. \\
{} more early \textsc{cop.impf.3sg} \textsc{expl} \textsc{def.art.m.sg}  son.\textsc{m.sg.dim} \textsc{rel} do.\textsc{impf.3sg} \textsc{def.art.m.sg} farmer \\
\glt `[...] in earlier days it was the youngest son who worked as a farmer.'
\z

Locative relative clauses are formed with \textit{nù tga} (\ref{ex:relloc1}), \textit{nùca} (\ref{ex:relloc4} and \ref{ex:relloc5}), \textit{nùca tga} (\ref{ex:relloc2} and \ref{ex:relloc3}), or only with the relativizer \textit{tga} (\ref{ex:relloc6}). \textit{Nù} and \textit{nùca} function as antecedents of the relative clause. \textit{Nùca} already contains the complementizer \textit{ca} –  see standard Survilvan \textit{nua che} ‘where that’, also realized \textit{nu che} – but in Tuatschín, \textit{ca} forms an unanalysable unit with \textit{nù}. If this were not so, the genuine Tuatschin form *\textit{nù tga} should be used, which does not exist.

\ea
\label{ex:relloc1}
\langinfo{Tuatschín}{Sadrún}{f3, 1. 56ff.}\\
\gll  Quaj è pròpi in ljuc ... \textbf{nù} \textbf{tg}’ i vagnéva schau tùt la munizjun tg’ i vèva, sigir.\\
\textsc{dem.unm} \textsc{cop.prs.3sg} exactly \textsc{indef.art.m.sg} place {} where \textsc{rel} \textsc{expl} \textsc{pass.aux.impf.3sg} leave.\textsc{ptcp.unm} all \textsc{def.art.f.sg} munition \textsc{rel} \textsc{expl} \textsc{exist.impf.3sg} sure.\textsc{adj.unm}\\
\glt `This is exactly a place ... where they stored all the munition, for sure.'
\z

\ea
\label{ex:relloc2}
\langinfo{Tuatschín}{}{\citealt[145]{Büchli1966}}\\
\gll   […] sé l’ alp \textbf{nùca} \textbf{tg}’ èl èra staus da buéb […]. \\
     {}  on \textsc{def.art.f.sg} alp where \textsc{rel} \textsc{3sg} be.\textsc{impf.3sg} \textsc{cop.ptcp.3sg.m} as boy.\textsc{m.sg} \\
\glt `[…] on the alp where he had been as a boy [...]. '
\z

\ea
\label{ex:relloc3}
\langinfo{Tuatschín}{Sadrún}{m5}\\
\gll    Al sulèt intarèssant è l’ ampréma sacùnda classa \textbf{nùca} \textbf{tga} fòn la midada tial sursilván […].\\
\textsc{def.art.m.sg} only interesting.\textsc{adj.unm} \textsc{cop.prs.3sg} \textsc{def.art.f.sg} first second form where \textsc{rel} make.\textsc{prs.3pl} \textsc{def.art.f.sg} change towards.\textsc{def.art.m.sg} Sursilvan\\
\glt `The only interesting thing is the first [and] second form where they start switching to Sursilvan.'
\z

\ea
\label{ex:relloc4}
\langinfo{Tuatschín}{Sadrún}{m4, l. 536f.}\\
\gll Ṣùtajn èri la \textbf{tégja} \textbf{nùca} `l caṣchav' èra [...].\\
under\_in \textsc{cop.impf.3sg.expl} \textsc{def.art.f.sg} alpine\_hut \textsc{rel} \textsc{3sg.m} make\_cheese.\textsc{impf.3sg} also\\
\glt `Below was the alpine hut where he would also make cheese [...].'
\z

The following examples illustrate the cases where \textit{nùca} is used without \textit{tga} and \textit{tga} without \textit{nùca}.

\ea
\label{ex:relloc5}
\langinfo{Tuatschín}{Bugnaj}{\citealt[134]{Büchli1966}}\\
\gll    […] la damaun èra la crapa gjù \textbf{nùca} la basèlgja stat òz sén quaj prau.\\
{} \textsc{def.art.f.sg} morning \textsc{cop.impf.3sg} \textsc{def.art.m.sg} stone.\textsc{coll} down \textsc{rel.loc} \textsc{def.art.f.sg} church stay.\textsc{prs.3sg} today on \textsc{dem.m.sg} field\\
\glt `[…] in the morning the stones where down there on the field where the church is located nowadays.'
\z

\ea
\label{ex:relloc6}
\langinfo{Tuatschín}{Sadrún}{f6, l. 838ff.}\\
\gll [...] a lu vèvan nus cò pròpi \textbf{in} \textbf{cantún} \textbf{tga} vagnéva mù raṣdau ròmòntsch.\\
{} and then have.\textsc{impf.1pl} \textsc{1pl} here really \textsc{indef.art.m.sg} corner \textsc{rel} \textsc{pass.aux.impf.3sg} only speak.\textsc{ptcp.unm} Romansh.\textsc{m.sg} \\
\glt `[...] and then we had here a real Romansh corner where only Romansh was spoken.'
\z

Some authors write \textit{nùca} in two words.

\ea\label{}
\langinfo{Tuatschín}{Bugnaj}{\citealt[134]{Büchli1966}}\\
\gll    [...] a lu èni sacussagljai sén vaschnaunca da bagagè èla gjù lò, \textbf{nù}' \textbf{c}' èla stat oz. \\
     {} and then \textsc{cop.prs.3pl.3pl} \textsc{refl}.discuss.\textsc{ptcp.m.pl} on municipality.\textsc{f.sg} \textsc{comp} build.\textsc{inf} \textsc{3sg.f} down there where \textsc{rel} \textsc{3sg.f} stay.\textsc{prs.3sg} today\\
\glt `And then they discussed [the problem] in the municipality [and decided] to build it [the church] where it is located nowadays.'
\z

\textit{Tga} is also used for locative relative clauses if the locative antecedent is not \textit{nùca} or \textit{nù}, as (\ref{relloclo}) and (\ref{relloclogans}) show.

\ea
\label{relloclo}
\langinfo{Tuatschín}{Sadrún}{m4, 1. 437f.}\\
\gll [...] zatgé rastònza ṣè aun \textbf{lò} \textbf{tg}’ ins sa í ajn á mará [...].\\
 {} something remnant.\textsc{f.sg} \textsc{cop.prs.3sg} still there \textsc{rel} \textsc{gnr} can.\textsc{prs.3sg} go.\textsc{inf} into \textsc{comp} see.\textsc{inf}\\
\glt `[...] there still are some remnants there where one can go and see [...].'
\z

\ea
\label{relloclogans}
\langinfo{Tuatschín}{Sadrún}{m4, 1. 566ff.}\\
\gll  Ò lò vòu fòrza schòn è survagnú in téc quajda d' í par crapa \textbf{tgu} a vju difarènts \textbf{lògans} \textbf{tg}’ i vèvan sitau gjù ad èra … vagnú ò cristalaṣ ètc\underline{è}tara [...].   \\
out there  have.\textsc{prs.1sg.1sg} maybe really also get.\textsc{ptcp.unm} \textsc{indef.art.m.sg} bit desire.\textsc{f.sg} \textsc{comp} go.\textsc{inf} for stone.\textsc{coll} \textsc{rel.1sg} have.\textsc{prs.1sg} see.\textsc{ptcp.unm} different.\textsc{m.pl} place.\textsc{pl} \textsc{rel} \textsc{3pl} have.\textsc{impf.3pl} blast.\textsc{ptcp.unm} down and  be.\textsc{impf.3sg} {} come.\textsc{ptcp.unm} out crystal.\textsc{f.pl} and\_so\_on \\
\glt `Out there I might have started enjoying looking for stones a bit, when I saw different places where they had blasted [the rocks], and crystals and so forth … had come out [...].'
\z

\ea
\label{}
\langinfo{Tuatschín}{Sadrún}{f3, l. 44ff.}\\
\gll Ad ju a stuvju ir’ ál’ antiara val á préndar sé gl amprém tùt quèls … tùt quèls … \textbf{lògans} \textbf{nù} i èran muossavías.\\
and \textsc{1sg} have.\textsc{prs.1sg} must.\textsc{ptcp.unm} go.\textsc{inf} in.\textsc{def.art.f.sg} whole valley \textsc{purp} take.\textsc{inf} up \textsc{def.art.m.sg} first all \textsc{dem.m.pl} {} all \textsc{dem.m.pl} {} place.\textsc{pl} where \textsc{expl} \textsc{exist.impf.3pl} signpost.\textsc{f.pl}\\
\glt `And I had to go to the entire valley in order to first take down all these … all these … places where there were signposts.'
\z

In rare cases, \textit{nùca tga} functions as the antecedent of a temporal relative clause (\ref{relnucatemp}). This is also the case with \textit{tga} alone, as is the case with the first occurrence of \textit{tga} in (\ref{relloclogans}).

\ea\label{relnucatemp}
\langinfo{Tuatschín}{Sadrún}{m5, 1. 1274ff.}\\
\gll Álṣò i dèva òns \textbf{nùca} \textbf{tga} gudignavan ... nùndétg, ad i ṣèra òns \textbf{nùca} \textbf{tg}’ èra aua, ad èra òns \textbf{nùca} \textbf{tga} spardévan.\\
well \textsc{expl} \textsc{exist.impf.3sg} year.\textsc{m.pl} where \textsc{rel} earn.\textsc{impf.3pl} {} incredibly and \textsc{expl} \textsc{exist.impf.3sg} year.\textsc{m.pl} where \textsc{rel} \textsc{exist.impf.3sg} water and \textsc{exist.impf.3sg} year.\textsc{m.pl} where \textsc{rel} lose.\textsc{impf.3pl} \\
\glt `Well, there were years when they earned ... a lot of money, and there were years with rain, and precisely years when they would lose money.'
\z

\section{Generic noun phrases}
Generic subject noun phrases are formed with the definite article singular (\ref{ex:gen:1}) or plural (\ref{ex:gen:2} and \ref{ex:gen:3}) as well as with the indefinite article singular or plural (\ref{ex:gen:4} and \ref{ex:gen:5}). Generic object noun phrases are usually bare (\ref{ex:gen:6}).

\ea
\label{ex:gen:1}
\langinfo{Tuatschín}{}{\DRG{10}{644}}\\
\gll \textbf{La} \textbf{lavina} \textbf{da} \textbf{tgaut} scava.\\
\textsc{def.art.f.sg} avalanche of warm.\textsc{adj.unm} scarify.\textsc{prs.3sg}\\
\glt `The warm weather avalanche scarifies the soil.'
\z


\ea
\label{ex:gen:2}
\langinfo{Tuatschín}{}{\DRG{3}{519}}\\
\gll  \textbf{Las} \textbf{tgauraṣ} èn las sulètas fèmnas tga portan barba.\\
     \textsc{def.art.f.pl} goat.\textsc{pl} \textsc{cop.prs.3pl} \textsc{def.art.f.pl} only.\textsc{pl} woman.\textsc{pl} \textsc{rel} carry.\textsc{prs.3pl} beard\\
\glt `Goats are the only women that have a beard.'
\z

\ea\label{ex:gen:3}
\langinfo{Tuatschín}{}{\DRG{3}{630}}\\
\gll \textbf{Las} \textbf{tgautschas} fòn bétg agl ùm. \\
  \textsc{def.art.f.pl} trousers.\textsc{pl} make.\textsc{prs.3pl} \textsc{neg} \textsc{def.art.m.sg} man \\
\glt `Trousers do not make a man.'
\z


\ea\label{ex:gen:4}
\langinfo{Tuatschín}{}{\DRG{6}{437}}\\
\gll  \textbf{In} \textbf{bògn} da flucs til’ ò als tissis.\\
\textsc{indef.art.m.sg} bath of chopped\_straw.\textsc{m.pl} pull.\textsc{prs.3sg} out \textsc{def.art.m.pl} poison.\textsc{pl} \\
\glt `A bath of chopped straw pulls out the poisons.'
\z

\ea\label{ex:gen:5}
\langinfo{Tuatschín}{}{\DRG{6}{432}}\\
\gll  \textbf{Ufauns} \textbf{pins} magljan la flur ad antardan la lavur.\\
child.\textsc{m.pl} small.\textsc{pl} eat.\textsc{prs.3pl} \textsc{def.art.f.sg} flower and delay.\textsc{prs.3pl} \textsc{def.art.f.sg} work\\
\glt `Small children consume power and delay the work.'
\z

\ea\label{ex:gen:6}
\langinfo{Tuatschín}{Sadrún}{m5}\\
\gll Èla ò mù ugèn \textbf{salata}.\\
\textsc{3sg.f} have.\textsc{prs.3sg} only with\_pleasure salad.\textsc{f.sg}\\
\glt `She only likes salad.'
\z

The generic pronoun \textit{ins} (rarely \textit{in}) is not restricted to subjects, but can be found in different syntactic functions.

\ea\label{ex:ins:1}
\langinfo{Tuatschín}{Sadrún}{m4, l. 397f.}\\ 
\gll Ad i èr’ è bitga úṣit tg’ \textbf{ins} mava á scùlèta.\\
and  \textsc{expl} \textsc{cop.impf.3sg} also \textsc{neg} usage.\textsc{m.sg}  \textsc{comp} \textsc{gnr} go.\textsc{impf.3sg} to nursery\_school.\textsc{f.sg}\\
\glt `And it was not usual that one attended nursery school.' (subject)
\z

\ea
\label{ex:ins:2}
\langinfo{Tuatschín}{Surajn}{f5, l. 1291f.}\\ 
\gll   [...] api stavèv’ \textbf{ins} í culas tgauras tòca sé Nalps [...]. \\
{} and  must.\textsc{impf} \textsc{gnr} go.\textsc{inf} with.\textsc{def.art.f.pl} goat.\textsc{pl} until up  \textsc{pln}\\
\glt `[...] and one had to go with the goats till Nalps [...].' (subject)
\z

\ea
\label{ex:ins:3}
\langinfo{Tuatschín}{Selva}{\citealt[46]{Büchli1966}}\\
\gll  […] tg’ al gjával ségi adina spèras lagagjaus par cudizá \textbf{ins} […].\\
{}  \textsc{comp} \textsc{def.art.m.sg} devil \textsc{cop.prs.sbj.3sg} always next lie\_in\_wait.\textsc{ptpc.m.sg} \textsc{purp} tease.\textsc{inf} \textsc{gnr}\\
\glt `[…] that the devil is always next [to us], lying in wait in order to tease us.' (direct object)
\z

In (\ref{ex:ins:4}), \textit{ins} could have a generic reading, but could also be interpreted as a first person plural.

\ea
\label{ex:ins:4}
\langinfo{Tuatschín}{Sadrún}{m4, l. 562ff.}\\ 
\gll  Prquaj tga quaj c’ \textbf{inṣ} vèva bigja grad da partgirá tiars sch’ èr’ \textbf{inṣ} antiar dé cun quèls [...].  \\
because \textsc{comp} \textsc{dem.unm} when \textsc{gnr} have.\textsc{impf.3sg}  \textsc{neg} just \textsc{comp} mind.\textsc{inf} animal.\textsc{m.pl} \textsc{corr} \textsc{cop.impf.3sg} \textsc{gnr} whole.\textsc{m.sg} day with \textsc{dem.m.pl}\\
\glt `Because when we didn’t just have to mind the animals, we were with them [the Italian workers] the whole day [...].'
\z

\section{Structure of the noun phrase}
Determiners precede adjectives and nouns. The modifying noun phrases, prepositional phrases, and relative clauses follow the noun. Adjectives precede or follow the noun; intensifiers precede the adjective. If combinations of determiners occur, the demonstrative precedes the possessive which in turn precedes the numeral: \textit{quèls nòs ufauns} `these our children', but these combinations are rare. The only combination of determiners occurring in the corpus is a possessive with a numeral (\ref{ex:possnum}).

\ea
\label{ex:possnum}
\langinfo{Tuatschín}{Ruèras}{f7, l. 1737f.}\\ 
	\gll  Nuṣ èssan ṣgulaj l’ amprém’ jèd’ uòn, cun \textbf{nòssas} \textbf{duaṣ} buébas.\\
\textsc{1pl} be.\textsc{prs.1pl} fly.\textsc{ptcp.m.pl} \textsc{def.art.f.sg} first time this\_year with \textsc{poss.1pl.f.pl} two.\textsc{f.pl} girl.\textsc{pl}\\
\glt `We flew for the first time this year, with our two daughters.'
\z


