\chapter{Phonology, phonetics, and spelling}

\section{Phonetics and phonology}

\subsection{Vowels}
Tuatschin possesses 9 vowels, which are presented in \tabref{vow}.

\begin{table}
\caption{Vowels}
\label{vow}
 \begin{tabular}{lllll}
  \lsptoprule
            &  front& central  & near back & back \\
  \midrule
 close   &  i <i>  &      &   &    u <u>\\
 near close    &   &      & ʊ <ù> &  \\
close-mid    &  e <é> &   &  &       \\
mid    & ɛ <è>  & ə <a> &        & ɔ <ò>\\
near open    &    &  ɐ <a>  &        \\
open    &   a <a>, <á> & \\
  \lspbottomrule
 \end{tabular}
\end{table}

The reduced vowels [ə] and [ɐ] only occur in unstressed syllables. There are no minimal pairs contrasting [ə] and [ɐ], but the distribution of these two reduced vowels is not clear to me. It seems as if in certain cases a speaker may use [ə] or [ɐ] in the same environment; there is, however, a tendency for [ɐ] to occur in the neighbourhood of stressed [a], as in [ju ˈmavɐ] `I used to go', and for [ə] to occur in the environment of [e] or [ɛ], as in [ˈrwɛrəs] 'Ruèras'. Because of this uncertainty, these two vowel will not be differentiated orthographically, and both will be represented by <a>.

In contrast, [e] and [ɛ] generally occur in stressed syllables in non-compound words, but in some loanwords they may occur in unstressed syllables, as is the case of the second [ɛ] in /\textit{ˈgɛnɛral}/ `general (adj.)'.

There are long and short vowels in Tuatschin, but they do not seem to be relevant phonologically. In unstressed syllables, only short vowels occur, but in stressed syllables, there are both short and long vowels. They will not be represented orthographically, but in the Tuatschin word list (chapter 11), all the lexical entries will be followed by a phonetic transcription indicating lengthening of the vowels.


Regarding [ʊ] and [u], \citet[130]{Liver2010} notes for standard Sursilvan that [ʊ] mostly occurs in short syllables, whereas [u] mostly occurs in long syllables, with some exceptions. This cannot be maintained for Tuatschin, since there is at least one minimal pair which opposes the two vowels in a short syllable: /ʥu/ `had (participle of \textit{vay} 'have')' vs /ʥʊ/ `down'. The realization of /ʊ/ varies between a nearly closed [u] and a very closed [o].

In my corpus, the only rounded vowel is [y], noted \textit{ü}, which only occurs in recent loans from German or Swiss German. It will not be retained in \tabref{vow}. Examples are \textit{bürò} `office', \textit{mütologia} `mythology', and \textit{tüp} `person'.


\begin{table}
	\caption{Vowel minimal pairs}
	\label{vmp}
	\begin{tabular}{llllllll}
	 \lsptoprule
		/i/&vs&/ɛ/& \textit{fil} &`thread' &vs& \textit{fɛl}&`fiel'\\
		/e/&vs&/ɛ/&\textit{leʨ} &`bed'&vs&\textit{lɛʨ} &`marriage'\\
		&&&\textit{me}&`me' (dative) & vs &\textit{mɛ} & `me' (accusative) \\
		/ɛ/&vs&/ɔ/&\textit{fɛl} &`gall’ &vs & \textit{fɔl} & `bellows’\\
		&&&\textit{sɛɲ} &`sign’ & vs & \textit{sɔɲ} & `holy’\\
		/u/&vs&/ʊ/&\textit{kʊ} & `how’ &vs& \textit{kɔ} &`here’\\
	/ʊ/ & vs & /ɔ/ & \textit{ˈrʊma} & `Rome' &vs & \textit{ˈrɔma} & `branches'\\
		\lspbottomrule
	\end{tabular}
\end{table}


\subsection{Diphthongs}
In Tuatschin, diphthongs are built with the semiconsonants /j/ and /w/ as well as with /i/ and /u/ in combination with /ə/. Tuatschin possesses 5 falling and 11 rising diphthongs. \tabref{diph} shows the diphthongs built with /j/ and /w/. Note that /ɔw/ is very rare and only occurs in the inverted forms of the first person singular present indicative forms of /saˈvaj/ `know' and /vaj/ `have': /sɔw/ `know I' and /vɔw/ `have I'. The following diphthongs do not exist: /ɔj/, /ʊj/, /uj/, /jʊ/, /ɛw/, /ew/, /iw/, /ʊw/, /uw/, /wʊ/, and /wu/.

The falling diphthongs /iə/ and /uə/ are not very frequent. Some examples are /ˈmiə/ `mine' (\textsc{f}) or /sɐˈniəʃtər/ `left', and /ˈʃkuə/ `broom' or /ˈuəʃt/ `August'.

Minimal pairs opposing diphthongs are scarce in the corpus; there are only oppositions between /aj/ and /ej/, as in /najf/ `snow' vs /nejf/ `new', or /majl/ `apple' vs /mejl/ `honey'.

\begin{table}
\caption{Diphthongs}
	\label{diph}
	\begin{tabular}{llllll}
		\lsptoprule
		falling & & & raising & & \\
		\midrule
		/aj/ & /kwaj/ & `this' & /ja/ & /uˈjarɐ/ & `war'\\
		/ɛj/ & /sɛj/ & `is (\textsc{sbjv})' & /jɛ/ & /ˈjɛdɐ/ & `time'\\
		/ej/ & /sejs/ & `their' & /je/ & /ˈjeli/ & `oil'\\
		/ɔj/ & -- & & /jɔ/ & /kurˈjɔs/ & `strange'\\
		/uj/ & -- & & /ju/ & /ju/ & `I'\\
		/aw/ & /awn/ & `still' & /wa/ & /ˈawa/ & `water'\\
		/ɛw/ & -- & & /wɛ/ & /kwɛl/ & `this'\\
		/ew/ &  -- & & /we/ & /kwelm/ & `mountain'\\
		/iw/ & -- & & /wi/ & /kwɛlˈwizɐ/ & `in this way'\\
		/ɔw/ & /sɔw/ & `know I' & /wɔ/ & -- \\
		\lspbottomrule
	\end{tabular}
\end{table}

The difference between diphthongs and vowels in hiatus is not always straightforward. A spontaneous production of /piun/ `lard' with /i/ and /u/ in hiatus is found in (\ref{ex:lard}).

\ea\label{ex:lard}
\langinfo{Tuatschín}{Sadrún}{m5}\\
\gll La mùm' ò mèz ajn \textbf{piun}.\\
\textsc{def.art.f.sg} mother have.\textsc{prs.3sg} put.\textsc{ptcp.unm} into lard.\textsc{m.sg}\\
\glt `Mother added some lard.'
\z

But when asked whether \textit{piun} has one or two syllables, the consultant answered that it has only one syllable and pronounced it /pjun/. However, there are uncontroversial cases of hiatus, as e.g. /ˈuɔn/ `this year', which is never pronounced /wɔn/.

--> wie stédiamajn behandeln? Ist ja hiatus.


\ea
\label{}
\langinfo{Tuatschín}{Sèlva}{\citealt[34]{Büchli1966}}\\
\gll    \textbf{Vasèvan} \textbf{ins} ina signura […] cun schuba cuérta, còtschna, […] \textbf{lura} spitgavan als purs ina gronda malaura […].\\
see.\textsc{impf.3sg.euph} \textsc{gnr} \textsc{indef.art.f.sg} woman [...] with shirt.\textsc{f.sg} short red [...] \textsc{corr} expect.\textsc{impf.3pl} \textsc{def.art.m.pl} farmer.\textsc{pl} \textsc{indef.art.f.sg} big storm\\
\glt `If one saw a woman with a short shirt, a red one, the farmers would expect a heavy storm.'
\z

n = Bindekonsonant bei vaseevan ins.

 
\subsection{Consonants}
Consonants are presented in \tabref{cons} and consonant minimal pairs in \tabref{cmp}.

\begin{table}
\caption{Consonants}
\label{cons}
 \begin{tabular}{llllllll}
  \lsptoprule
      &  & bilabial & labio-  & alveolar  &  palatal & palato- &velar\\
     &&& dental &&& alveolar \\
  \midrule
nasal    &    &  m   & &  n       &  	ɲ & \\

stop &voiced   &  b  &   &  d     &  &  & g\\
  & unvoiced   &  p   &      & t  &  & & k\\
fricative  &  voiced  &      & v        & z &  	ʒ\\
  &  unvoiced  &      &   f      & s & ʃ\\
  affricate & voiced & & & &&ʥ \\
  & unvoiced &&&ʦ & ʧ &ʨ\\
trill  &    &      &         & r \\
lateral appr.  &    &      &         & l & ʎ \\
  \lspbottomrule
 \end{tabular}
\end{table}

A major problem in analysing consonants is the question whether the unvoiced word final consonants are in many cases to be considered as such or as underlyingly voiced. For instance, in standard Sursilvan the 1st person singular conditional is written \textit{cantass} `I would sing', but if this form is followed by a vowel or a voiced consonant, it is pronounced /z/, as in \textit{stèṣ} /ʃtɛz/ in the Tuatschin example (\ref{ex:stes}).

\ea\label{ex:stes}
\langinfo{Tuatschín}{Sadrún}{m10, l. 1061f.}\\
\gll [...] alṣò sch' ju \textit{\textbf{stèṣ}} aun fá in' jèda quaj, \textit{\textbf{figès}} ju bétga.\\
{} well if \textsc{1sg} must.\textsc{cond.1sg} still do.\textsc{inf} one.\textsc{f.sg} time \textsc{dem.unm} do.\textsc{cond.1sg} \textsc{1sg} \textsc{neg}\\
\glt `[...] well, if I had to do it once more, I wouldn't do it.'
\z

Another example is `not even', which is spelled \textit{gnanc} in standard Sursilvan, but which is pronounced with /g/ if followed by a vowel or a voiced consonant in Tuatschin, as in /ɲaŋg in/ `not even one'.

If in all cases voiceless consonants were pronounced voiced if followed by a voiced element, it would be very easy to establish the rule that every voiceless consonant in word final position is pronounced voiced if followed by a vowel or a voiced consonant. However, this is not the case, as in /in brˈiək əd in kʊp/ `a wooden bucket and a bowl', which is never pronounced /in *brˈiəg əd in kʊp/ or as in /in ljuk amparˈnajval/ `a cosy place', which is not pronounced /in *ljug amparˈnajval/.

Therefore I will depart from the standard Sursilvan spelling and write voiced consonants in case they are pronounced voiced if followed by a voiced element, and the rule will be formulated as follows:

Every word final voiced consonant is pronounced voicelessly in isolation or preceding a word starting with a voiceless consonant.

The consequences for the spelling system used in this book will be discussed in § 2.2 below.

\begin{table}
\caption{Consonant minimal pairs}
\label{cmp}
 \begin{tabular}{llllllll}
 \lsptoprule
p&vs&t&\textit{pawn}&`bread'&vs&\textit{tawn}&`so much'\\
p&vs&n&\textit{pawk}&`little (adv.')&vs&\textit{pawn}&`bread' \\
p&vs&ɲ&\textit{kʊp}&`bowl'&vs&\textit{kʊɲ}&`wedge' \\
p&vs&s&\textit{pawn}&`bread'&vs&\textit{sawn}&`blood' \\
p&vs&ʨ&\textit{pawn}&`bread'&vs&\textit{ʨawn}&`dog' \\
t&vs&k& \textit{bʊt}&`barrel&vs&\textit{bʊk}&`billy goat' \\
t&vs&n&\textit{sɐˈlit}&`greeting'&vs&\textit{sɐˈlin}&`wheat' \\
t&vs&ʨ&\textit{vit}&empty&vs&\textit{viʨ}&village \\
k&vs&f&\textit{ʥuk}&play&vs&\textit{ʥuf}&yoke\\
b&vs&n&\textit{ˈrawbɐ}&merchandise&vs&\textit{ˈrawnɐ}&frog\\
d&vs&n&\textit{ˈfri:dɐ}&wound&vs&\textit{ˈfri:nɐ}&flour\\
d&vs&ʦ&\textit{ˈsɛndɐ}&path&vs&\textit{ˈsɛnʦɐ}&without\\
m&vs&l&\textit{fɔm} & hunger & vs & \textit{fɔl} & bellows\\
&&&\textit{fiˈma:}& smoke & vs& \textit{fiˈla:}& spin\\
n&vs&ɲ&\textit{ɔn}&year&vs&\textit{ɔɲ}&alder\\
n&vs&ʦ&\textit{pʊn}&bridge&vs&\textit{pʊʦ}&pond\\
n&vs&ʨ&\textit{lɛn}&firewood&vs&\textit{lɛʨ}&marriage\\
n&vs&ʥ& \textit{nuf}&knot&vs&\textit{ʥuf}&yoke\\
ɲ&vs&ʨ&\textit{peɲ}&fir tree&vs&\textit{peʨ} & pick\\
ɲ&vs&ʦ&\textit{pɛɲ}&pledge&vs&\textit{pɛʦ}&chest\\
f&vs&r& \textit{najf} &snow&vs&\textit{najr}&black\\
s&vs&ʦ&\textit{ˈfɔrsɐ}&maybe&vs&\textit{ˈfɔrʦɐ}&power\\
s&vs&ʨ&\textit{glas}&glass&vs&\textit{glaʧ}&ice\\
s&vs&l& \textit{pas}&step&vs&\textit{pal}&post\\
ʃ&vs&ʨ&\textit{eʃ}&door&vs &\textit{eʨ}&ointment\\
ʨ&vs&ʦ&\textit{ɔʨ}&eight&vs&\textit{ɔʦ}&today\\
ʨ&vs&ʧ&\textit{deʨ}&`said'&vs&\textit{deʧ}&`(I) say'\\
ʥ&vs&l&\textit{ʥuf}&yoke&vs&\textit{luf}&wolf\\
  \lspbottomrule
 \end{tabular}
\end{table}

/h/ only  exists in Swiss German loans, as in /halt/ `simply'.

/n/ has two allomorphs: /ŋ/ before g and k, and m before b 

(check!).
 
/l/ and /r/ have different realizations according to the speaker. Some speakers pronounce [ł] instead of /l/ in  cases that still must be determined, and the uvular /ʁ/ is not uncommon among younger and young speakers, certainly under influence of some Sursilvan varieties, partly spoken by the teachers of the school in Sedrun. Note that all my consultants produce an alveolar [r].

Furthermore, /r/ and /l/ may have a syllabic realization due to the dropping of [ə] as in [pr̩-ma-ˈvɛ-ra] (/pər-ma-ˈvɛ-ra/) 'Lent' (line 1209) or [pl̩s] (/pər əls/ `for the' (line 1006).

The voiced stops /b/, /d/, and /g/ are sometimes realized voiceless as in [ˈab̥ər] `but'; however, this is never the case with the palatal fricatives /z/ and /ʒ/.

In rare cases, the voiceless stops show an aspirated realization, as in [əmˈpʰaw] `a bit' (line 776) or [pʰas] `pass' (line 1023).

There are some cases of assimilation of consonants across word boundaries. In (\ref{ex:ass1}), \textit{détg di} `said to' is realized [det:i], and in (\ref{ex:ass2}), \textit{détg: té} `said: you' is pronounced [det:e].


\ea\label{ex:ass1}
\langinfo{Tuatschín}{Sadrún}{m4, l. 397}\\
\gll [...] api vau \textbf{détg} \textbf{di} mùma [...].\\
{} and have.\textsc{prs.1sg.1sg} say.\textsc{ptcp.unm} \textsc{dat} mother\\
\glt `[...] I said to my mother [...].'
\z

\ea\label{ex:ass2}
\langinfo{Tuatschín}{Sadrún}{m4, l. 401f.}\\
\gll Pi ò èla \textbf{détg}: «\textbf{Té} savèssaṣ í cul tat [...].\\
then have.\textsc{prs.3sg} \textsc{3sg} say.\textsc{ptcp.unm} \textsc{2sg} can.\textsc{cond.2sg} go.\textsc{inf}  with.\textsc{def.art.m.sg} grandfather\\
\glt `Then she said: «You could go up [...] with your grandfather.'
\z

There are also some cases where complex consonant clusters are avoided, as for instance \textit{tgs} [ʨs], which is sometimes pronounced \textit{ts}, as in \textit{mats} instead of \textit{matgs} `bunches' (chapter 9, lines 1384f.).

If two vowels are adjacent across word boundaries, the last vowel of the first word is often elided if it is a weak vowel (\textit{ə} and \textit{ɐ}, both spelled <a>). Example (\ref{ex:sandhi1}) contains two examples of the wek vowel [-a] which is elided (\textit{bigj' idéa} for \textit{bigja idéa} and \textit{stad' ajn} for \textit{stada ajn}), a well as one example of strong vowels that do not trigger elision (\textit{ju èra}).



\ea
\label{ex:sandhi1}
\langinfo{Tuatschín}{Ruèras}{f7, l. 1664f.}\\
\gll Nua ṣè quaj hotel? \textbf{Bigj'} \textbf{idéa}, \textbf{ju} \textbf{èra} schòn òns bigja \textbf{stad’} \textbf{ajn} quaj martgau.   \\
where \textsc{cop.prs.3sg} \textsc{dem.m.sg} hotel \textsc{neg} idea \textsc{1sg} be.\textsc{impf.1sg} already year.\textsc{m.pl} \textsc{neg} \textsc{cop.ptcp.f.sg} in \textsc{dem.m.sg} hotel\\
\glt `Where is this hotel? No idea, I hadn't been in that city for years.'
\z

\section{Spelling system}

The spelling system used in this grammar is a compromise between standard Sursilvan spelling and the aim of making pronunciation and stress transparent to the reader, which means that one grapheme has to correspond to one sound (or phoneme in most cases) (see \tabref{graphIpaI} and \tabref{graphIpaII}). 

The most outstanding problems with Sursilvan standard spelling are

\begin{itemize}
\item whether <e> and <o> are close-mid or mid, 
\item whether two adjacent vowels form a rising or falling diphthong or whether they represent two vowels in hiatus,
\item whether <s> and <sch> are voiced or not, 
\item and, in some cases, where stress falls.
\end{itemize}
 
To disambiguate these problems, I will indicate with an acute accent <é, ó> that the vowel is close-mid, and with a grave accent that the vowel is mid (<è, ò>) or near close (<ù>). Voiced palatal fricatives get a point under the \textit{s} (<ṣ> for /z/ and <ṣch> for /ʒ/), as is the usage in Romansh bilingual dictionaries. The other cases will be explained below.

There are several problems with dz: only for instance in lèdz véva (lèz véva, l. 142)


\begin{table}
\caption{Correspondences between spelling and IPA I}
\label{graphIpaI} 
\begin{tabular}{lll}
    \lsptoprule
        grapheme      & IPA\\
    \midrule  
  a & ə, ɐ\\
  á & a\\
  b & b\\
  c & k before a, ó, ò, ù, u\\
  & ʦ before é, è, i\\
  ch & k before é, è, i\\
  d & d\\
  é & e\\
  è & ɛ\\
  f & f\\
  g & g before a, ó, ò, ù, u\\
  & ʥ before é, è, i\\
  gh & g before é, è, i\\
  gj & ʥ before a, ó, ò, ù, u\\
  gl & ʎ before i and word finally\\
  & gl before a, é, è, ó, ò, ù, u\\
  glj & ʎ before a, è, é, ò, ó, ù, u\\
  gn& ɲ\\
  h & h\\
  i & i\\
  j & j\\
  l & l\\
  m & m\\
  n & n\\
  ó & o\\
  ò & ɔ\\
  p & p\\
  qu & kw\\
  r & r, R\\
  s & s word initially, word finally, and preceding <l, n, z>\\
  & z between two vowels\\
  & ʃ before <c, f, m, n, p, qu, r, t>\\
  ss & s between two vowels\\
  ṣ & z\\
  & ʒ preceding <b, d, g, v>\\
  \lspbottomrule
\end{tabular} 
\end{table}


\begin{table}
\caption{Correspondences between spelling and IPA II}
\label{graphIpaII} 
\begin{tabular}{lll}
    \lsptoprule
        grapheme      & IPA\\
    \midrule  
  sch & ʃ\\
  ṣch & ʒ\\
  t & t\\
  tg& ʨ\\
  tsch & ʧ\\
  u & u\\
  ù & ʊ\\
  v & v\\
  x & ks\\
  z & ʦ before a, ó, ò, ù, u\\
  \lspbottomrule
\end{tabular} 
\end{table}




Stress rules are as follows:

\begin{itemize}
\item Diphthongs are always stressed.
\item Words without a diphthong which end in a vowel or <-n> or <-s> are stressed on the penultimate syllable.
\item Words without a diphthong ending in a consonant, except for words ending in <-n> or <-s>, are stressed on the last syllable.
\item Close and mid-open vowels (<é, è, ó, ò>) are stressed except if they occur in a word containing a diphthong.
\item Words ending in a vowel which are stressed on a syllable other than the penultimate get an acute accent. This concerns the vowels a, i, and u. In other words, á, í, and ú are always stressed.
\item Words with two diphthongs, with two vowels with diacritics, or words with a diphthong and a vowel with a diacritic get an underscore, as e.g. \underline{è} in \textit{ajn g\underline{è}nèral} /ˈʥɛnɛral/ `in general'.
\end{itemize}
      
The reason for giving <-n >and <-s> a special treatment is the fact that <-n> is used for verbal plural and <-s> for nominal and verbal plural. If <-n >and <-s> were treated as the other final consonants, much more diacritics should be used.

A further problem is the treatment of <s> followed by a consonant. Here, I follow the standard Sursilvan spelling:

\begin{itemize}

\item <s> followed by <c, f, m, n, p, qu, r, t> is pronounced [ʃ].
\item <s> followed by <b, d, g, v> is written <ṣ> and pronounced [ʒ].
\item <s> followed by <l, n, z> is pronounced [s].
\item If <s> should not be pronounced [ʃ], a hyphen separates the two consonants as in <ris-plí> (/risˈpli/) `pencil'.
\end{itemize}

In the texts (chapter 9) and hence also in the examples taken from these texts, the final consonants are transcribed as they were pronounced. An example is the word for `ten', which can be transcribed [déjʃ] or [déjʒ] according to the context in which it occurs.

al bab èra lu ... vèva ina mùma da l'Austria (Ricardo biografia)
