\chapter{Phonology, phonetics, and spelling}

\section{Phonetics and phonology}

\subsection{Vowels}
Tuatschin possesses 9 vowels, which are presented in \tabref{vow}.

\begin{table}
\caption{Vowels}
\label{vow}
 \begin{tabular}{lllll}
  \lsptoprule
            &  front& central  & near back & back \\
  \midrule
 close   &  i  &      &   &    u    \\
 near close    &   &      & ʊ &  \\
close-mid    &  e  &   &  &       \\
mid    & ɛ   & ə  &        & ɔ\\
near open    &    &  ɐ   &        \\
open    &   a  &        \\
  \lspbottomrule
 \end{tabular}
\end{table}

The reduced vowels [ə] and [ɐ] only occur in unstressed syllables; in contrast, [e] and [ɛ] generally occur in stressed syllables in non-compound words. These two vowels may occur in unstressed syllables of some loan words like \textit{gènèral} 'general' or .
There is no minimal pair contrasting [ə] and [ɐ], but the distribution of these two reduced vowels is not clear to me. It seems as if in certain cases a speaker may use [ə] or [ɐ] in the same environment; there is, however, a tendency for [ɐ] to occur in the neighbourhood of stressed [a], as in , and for [ə] to occur in the environment of [e] or [ɛ], as in ['rwɛrəs] 'Ruèras'. For this reason, these two vowel will not be differentiated orthographically, and both will be represented by <a>.

There are long and short vowels in Tuatschin. In unstressed syllables, only short vowels occur, but in stressed syllables, there are both short and long vowels. 


Regarding [ʊ] and [u], \citet[130]{Liver2010} notes for Standard Sursilvan that [ʊ] mostly occurs in short syllables, whereas [u] mostly occurs in long syllables, with some exceptions. This cannot be maintained for Tuatschin, since there is at least one minimal pair which opposes the two vowels in a short syllable: /ʥu/ 'had (participle of \textit{vay} 'have')' vs /ʥʊ/ 'down'.

rounded vowels ü, ö

Problems:

existence of /o/ or [o]

long and short vowels

--> diphthongs



\begin{table}
\caption{Vowel minimal pairs}
\label{vmp}
 \begin{tabular}{llllllll}
 \midrule
/i/&vs&/ɛ/&fil&thread&vs&fɛl&fiel\\
/e/&vs&/ɛ/&leʨ&bed&vs&lɛʨ, &marriage\\
&&&te&you (nominative)& vs &tɛ & you (accusative)\\
/ɛ/&vs&/ɔ/&fɛl &gall &vs & fɔl &bellows\\
&&&sɛɲ&sign & vs & sɔɲ & holy\\
/u/&vs&/ʊ/&pun &bridge&vs&pʊn &cloth\\
&&& duʨ & sweet & vs & dʊʨ & irrigation canal\\
/o/&vs&/ɔ/&ko&how&vs&kɔ&here\\
 \lspbottomrule
 \end{tabular}
\end{table}

ròma vs. Rùma

Long vowels :
feːʎ ‘leaf’ vs feʎ ‘son’
fil 'thread': long or short?

èrani vs. èran i (short vs long)




Diphthongs
nejf ‘new.m.sg’ vs näjv ‘snow’
mejl ‘honey’ vs mäjl ‘apple’

culs pézs sil pèz (Büchli 1966: 124) (mit den Spitzen auf der Brust)


\subsection{Consonants}
Consonants are presented in \tabref{cons} and consonantic minimal pairs in \tabref{cmp}.


\begin{table}
\caption{Consonants}
\label{cons}
 \begin{tabular}{llllllll}
  \lsptoprule
      &  & bilabial & labio-  & alveolar  &  palatal & palato- &velar\\
     &&& dental &&& alveolar \\
  \midrule
nasal    &    &  m   & &  n       &  	ɲ & \\

stop &voiced   &  b  &   &  d     &  &  & g\\
  & unvoiced   &  p   &      & t  &  & & k\\
fricative  &  voiced  &      & v        & z &  	ʒ\\
  &  unvoiced  &      &   f      & s & ʃ\\
  affricate & voiced & & & &&ʥ \\
  & unvoiced &&&ʦ & ʧ &ʨ\\
trill  &    &      &         & r \\
lateral appr.  &    &      &         & l & ʎ \\
  \lspbottomrule
 \end{tabular}
\end{table}

dz: only for instance in lèdz véva (lèz véva, l. 142)


\begin{table}
\caption{Consonant minimal pairs}
\label{cmp}
 \begin{tabular}{llllllll}
  \midrule
p&vs&t&pawn&bread&vs&tawn&so much\\
p&vs&n&pawk&little (adv.)&vs&pawn&bread \\
p&vs&ɲ&kʊp&bowl&vs&kʊɲ&wedge \\
p&vs&x&pawn&bread&vs&sawn&blood \\
p&vs&ʨ&pawn&bread&vs&ʨawn&dog \\
t&vs&k& bʊt&barrel&vs&bʊk&billy goat \\
t&vs&n&sɐ'lit&greeting&vs&sɐ'lin&wheat \\
t&vs&ʨ&vit&empty&vs&viʨ&village \\
k&vs&m&sɛk&dry&vs&sɛm&seed \\
k&vs&ɲ&pɛk&baker&vs&pɛɲ&deposit \\
k&vs&f&ʥuk&play&vs&ʥuf&yoke\\
b&vs&n&rawbɐ&merchandise&vs&rawnɐ&frog\\
d&vs&n&fri:dɐ&wound&vs&fri:nɐ&flour\\
d&vs&ʦ&ˈsɛndɐ&path&vs&ˈsɛnʦɐ&without\\
m&vs&l&fɔm & hunger & vs & fɔl & bellows\\
&&&fi'ma:& smoke & vs& fila:& spin\\
n&vs&ɲ&ɔn&year&vs&ɔɲ&alder\\
n&vs&ʦ&pʊn&bridge&vs&pʊʦ&pond\\
n&vs&ʨ&lɛn&firewood&vs&lɛʨ&marriage\\
n&vs&ʥ& nuf&knot&vs&ʥuf&yoke\\
ɲ&vs&ʨ&peɲ&fir tree&vs&peʨ & pick\\
ɲ&vs&ʦ&pɛɲ&deposit&vs&pɛʦ&chest\\
f&vs&r& näjf &snow&vs&näjr&black\\
s&vs&ʦ&fɔrsɐ&maybe&vs&fɔrʦɐ&power\\
s&vs&ʨ&glas&glass&vs&glaʧ&ice\\
s&vs&l& pas&step&vs&pal&poteau\\
ʃ&vs&ʨ&eʃ&door&vs &eʨ&ointment\\
ʨ&vs&ʦ&ɔʨ&eight&vs&ɔʦ&today\\
ʥ&vs&l&ʥuf&yoke&vs&luf&wolf\\
  \lspbottomrule
 \end{tabular}
\end{table}


unvoiced fricatives: ábar

geminated consonants

consonants in final position: voiced or voiceless?

Tarcisi: in gries um (bo: in griess um)

al Tarcisi ò è fagj lò in pèr placats (ca. 108)

fatg ni fagj?

A quaj èra schòn strètg, alṣò sch'ju \textit{stèṣ} aun fá in'jèda quaj, \textit{figjès} ju bétga.

mètar a sut/mètar sut, vs. ṣ

velar l

uvular r


Sandhi

aspiration of voiceless stops

syllabic l, r

nu i dèva pával pls méls (ca. 545)


Ju sun juṣ in tjamṣ a plaṣchéva da mé ṣchùbar nuét api vau détg di (dét:di) mùma in dé:

Pi ò èla dét[g]:«Té savèssaṣ í cul tat ajn Pardatsch.» --> détté savessesthe 

Assimilation: blut dj → bludj:avel (miu tat)

\section{Spelling system}

The spelling system used in this grammar is a compromise between standard Sursilvan spelling and the aim of making pronunciation and stress transparent for the reader, which means that one grapheme has to correspond to one sound (or phoneme in most cases) (see \tabref{graphIpaI} and \tabref{graphIpaII}). The most outstanding problems with Sursilvan standard spelling are that it does not indicate systematically

\begin{itemize}
\item whether <e> and <o> are close-mid or mid, 
\item whether two adjacent vowels form a rising or falling diphthong or whether they represent two vowels in hiatus,
\item whether <s> and <sch> are voiced or not, 
\item and, in some cases, where stress falls.
\end{itemize}
 
To disambiguate these problems, I will indicate with an accute accent <é, ó> that the vowel is close-mid, and with a grave accent that the vowel is mid (<è, ò>) or near close (<ù>). Voiced palatal fricatives get a point under the \textit{s} (<ṣ> for /z/ and <ṣch> for /ʒ/), as is the usage in Romansh bilingual dictionaries. The other cases will be explained below.

\begin{table}
\caption{Correspondences between spelling and IPA I}
\label{graphIpaI} 
\begin{tabular}{lll}
    \lsptoprule
        grapheme      & IPA & examples\\
    \midrule  
  a & ə, ɐ\\
  á & a\\
  b & b\\
  c & k before a, ó, ò, ù, u\\
  & ʦ before é, è, i\\
  ch & k before é, è, i\\
  d & d\\
  é & e\\
  è & ɛ\\
  f & f\\
  g & g before a, ó, ò, ù, u\\
  & ʥ before é, è, i\\
  gh & g before é, è, i\\
  gj & ʥ before a, ó, ò, ù, u\\
  gl & ʎ before i and word finally\\
  & gl before a, é, è, ó, ò, ù, u\\
  glj & ʎ before a, è, é, ò, ó, ù, u\\
  gn& ɲ\\
  h & h\\
  i & i\\
  j & j\\
  l & l\\
  m & m\\
  n & n\\
  ó & o\\
  ò & ɔ\\
  p & p\\
  qu & kw\\
  r & r, R\\
  s & s word initially and finally\\
  & z between two vowels\\
  & ʃ before p\\
  & ʒ before \\
  ss & s between two vowels\\
  ṣ & z\\
  \lspbottomrule
\end{tabular} 
\end{table}


\begin{table}
\caption{Correspondences between spelling and IPA II}
\label{graphIpaII} 
\begin{tabular}{lll}
    \lsptoprule
        grapheme      & IPA & examples\\
    \midrule  
  sch & ʃ\\
  ṣch & ʒ\\
  t & t\\
  tg& ʨ\\
  tsch & ʧ\\
  u & u\\
  ù & ʊ\\
  v & v\\
  x & ks\\
  z & ʦ before a, ó, ò, ù, u\\
  \lspbottomrule
\end{tabular} 
\end{table}




Stress rules are as follows:

\begin{itemize}
\item Words ending in a vowel or in -n or -s are stressed on the penultimate syllable.
\item Words ending in a consonant, except for words ending in -n or -s, are stressed on the last syllable.
\item Close and mid-open vowels, which all have a diacritic, as well as diphthongs, are stressed. 
\item Exceptions get an acute accent (<ʹ>).
\end{itemize}
      
Some observations.

\begin{itemize}
\item The reason for giving -n and -s a special treatment is the fact that -\textit{n} is used for verbal plural and -\textit{s} for nominal and verbal plural.
\item If there are two diphthongs in a word, which sometimes happens in compound words, the stressed diphthong gets an acute accent.
\item If there are two or more close or mid-open vowels in a word, the stressed one is treated in the following way: <ê> and <ô>: closed and stressed; <ě> and <ǒ>: mid-open and stressed. 
\end{itemize}

Problems:

cáput or cápút?

éxact or éxáct?

accidèn or aczidèn?

quátar or cuátar?

saravagnús or saravagnjús or saravagnjus?

s-ch: mes-chin adj, gemein, schäbig
zues-cher Zwetschgenbaum, s-chivi 'Widerwille'

s + c, f, m, n, p, qu, r, t -> voiceless sch

s + b, d, g, v -> voiced sch

s + l, n, z = ts -> s




