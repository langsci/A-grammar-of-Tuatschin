\chapter{Complex sentences}

\section{Coordination}

\section{Subordination}

\subsection{Subject clauses}

\ea\label{}
\langinfo{Tuatschin}{}{\DRG{}{}}\\
\gll \textbf{Fugí} è bian, a \textbf{mitschè} è aun mégliar. \\
   flee.\textsc{inf} \textsc{cop.prs.3sg} good and escape.\textsc{inf} \textsc{cop.prs.3sg} still better  \\
\glt `To flee is good, and to escape is even better.'
\z


\subsection{Object clauses}

\subsection{Temporal clauses}

The usual temporal subordinator is \textit{cura} or its short form \textit{cu}. Sometimes, a correlative construction with \textit{sche} `if; then' is used.

\ea\label{ex:1:}
\langinfo{Tuatschin}{}{\citealt[23]{Berther2007}}\\
\gll  […] cu i dat la benedicziun, \textbf{sche} fo la sontga crusch.\\
    […] when \textsc{expl} give.\textsc{prs.3sg} \textsc{def.art.f.sg} blessing then make.\textsc{imp.2sg} \textsc{def.art.f.sg} holy cross\\
\glt `[…] when it comes to the blessing, make the sign of the cross.'
\z



\ea\label{ex:1:}
\langinfo{Tuatschin}{Selva} {\citealt[30]{Büchli1966}}\\
\gll    Fagèi vinavon; avon che tgǝ Gion ǝ Paul greschan!\\
     do.\textsc{imp.2pl} forward before \textsc{comp} \textsc{comp} \textsc{pn} and \textsc{pn} shout.\textsc{prs.3pl}\\
\glt `Hurry up before Gion and Paul start to shout.'
\z



\ea\label{ex:1:}
\langinfo{Tuatschin}{Tschamut} {\citealt[17]{Büchli1966}}\\
\gll    sund ju ida a tgèesa, avon che tgǝ vegni stgir.\\
     be.\textsc{prs.1sg} \textsc{1sg} go.\textsc{ptcp.f.sg} to house before \textsc{comp} \textsc{comp} get.\textsc{prs.sbjv.3sg} dark \\
\glt `[…] I went home before it got dark.'
\z

\ea\label{ex:1:}
\langinfo{Tuatschin}{Tschamut} {\citealt[19]{Büchli1966}}\\
\gll    Ju a giu tèma, tg’ ǝls tiers réisdien duront tge la sontga messa de Nadal vegni legida.\\
      \textsc{1sg} have.\textsc{prs.1sg} have.\textsc{ptcp} fear \textsc{comp} \textsc{def.art.m.pl} animal.\textsc{pl} talk.\textsc{prs.sbjv.3pl} during \textsc{comp} \textsc{def.art.f.sg} holy mass of Christmas come.\textsc{prs.sbjv.3sg} read.\textsc{ptcp.f.sg}\\
\glt `I feared […] that the animals could talk when the Holy Mass would be read.'
\z

\ea\label{}
\langinfo{Tuatschin}{Cavorgia}{\citealt[120]{Büchli1966}}\\
\gll    Dǝ vegnî anavůůs vond ju plaunsiu !\\
    \textsc{comp} come.\textsc{inf} back go.\textsc{prs.1sg} \textsc{1sg} slowly \\
\glt `When I come back, I’ll walk slowly!'
\z

\ea\label{}
\langinfo{Tuatschin}{Ruèras}{\citealt[8]{Valär2013b}}\\
\gll    […] faː ain […] ʦa’kɔnʣ diants aʎ rǝʃ’tiː dɐ lɐ dʊːnɐ ʨ ɐl ɔ rut ɔːr eir dɐ mɐ’ʦaː in ruʃp\\
    […] make.\textsc{inf} in […] some tooth.\textsc{m.pl} {dat} rake of \textsc{def.art.f.sg} woman \textsc{rel} \textsc{3sg} have.\textsc{prs.3sg} break.\textsc{ptcp} out also \textsc{comp} kill.\textsc{inf} \textsc{indef.art.m.sg} toad\\
\glt `[…] put in some teeth to the woman’s rake he had broken when he killed a toad.'
\z

\ea\label{}
\langinfo{Medelin}{Curaglia}{\DRG{3}{592}}\\
\gll \textbf{Anqual} \textbf{ja} \textbf{tga} nus prendevan giu, vagnevi giu pli bia miurs tga monas.\\
   some time \textsc{comp} \textsc{1pl} take.\textsc{impf.1pl} down come.\textsc{impf.3sg.expl} down more many mouse.\textsc{pl} than sheave.\textsc{pl}.'\\
\glt `Sometimes, when we would take [the sheaves] down, there would be more mice coming down that sheaves.'
\z



\subsection{Modal clauses}

\subsection{Purposive clauses}

 \ea\label{ex:1:}
\langinfo{Tuatschin}{Bugnei} {\citealt[143]{Büchli1966}}\\
\gll  mez in pétg sc’ ins drova per tůt las lavurs dǝ pǝrmavèera å gl’ atun \textbf{dǝ} \textbf{cavâ} \textbf{truffels}.\\
     put.\textsc{ptcp} \textsc{indef.art.m.sg} hoe like \textsc{impers} need.\textsc{prs.3sg} for all \textsc{def.art.f.pl} work.\textsc{pl} of spring and \textsc{def.art.m.sg} autumn \textsc{purp} dig potato.\textsc{pl}\\
\glt `[…] put a hoe like [the one] one needs for all the work that must be done in spring, and in autumn in order to dig out potatoes.'
\z

\ea\label{}
\langinfo{Tuatschin}{Ruèras}{\citealt[9]{Valär2013b}}\\
\gll ‘ʥeːvjǝ zɛ l ‘tɔni dɐ lɐ mɐt’lajnɐ sǝʃfǝrdɐntaws ʃi feʨ \textbf{dɐ} ‘\textbf{bajbǝr} \textbf{trajs} \textbf{mjɔːlas} \textbf{pɛːn} \textbf{frajt}\\
     Thursday \textsc{cop.prs.3sg} \textsc{def.art.m.sg} \textsc{pn} of \textsc{def.art.f.sg} \textsc{pn} \textsc{refl}.catch.cold.\textsc{ptcp.m.sg} so much \textsc{comp} drink.\textsc{inf} three cup.\textsc{pl} buttermilk cold\\
\glt `Thursday Matlaina’s Toni caught a very strong cold because he drank three cups of cold buttermilk.'
\z

In the DRG materials, there is one occurrence of \textit{bétg} `negator' located between \textit{par} `for' and \textit{tga} `complementizer', i.e. outside of the subordinate clause.

This construction has not been accepted by my informants; however, \textit{par bétg tga} exists or has existed also in other Romansh dialects, as e.g. in the Sutsilvan dialect of Dalin.

\ea\label{}
\langinfo{Sutsilvan}{Dalin}{\DRG{4}{607}}\\
\gll  \textbf{Par} \textbf{bétg} \textbf{tg}'in schleschi dat il calger eign in pêr guspas els calzers.\\
     \textsc{purp} \textsc{neg}\textsc{comp=gdl} slip.\textsc{prs.sbjv.3sg} give.\textsc{prs.3sg} \textsc{def.art.m.sg}shoemaker in \textsc{indef.art.m.sg} some nail.\textsc{pl} in.\textsc{def.art.m.pl} shoe.\textsc{pl}\\
\glt `In order not to slip, the shoemaker beat some nails into the shoes.'
\z

\ea\label{}
\langinfo{Sutsilvan}{Dagliegn}{\DRG{6}{297}}\\
\gll   Fil stiert stogn strihar cun tschera \textbf{par} \textbf{betg} \textbf{tg}’el sasfili.\\
     thread doubly.twisted must.\textsc{pres.3sg=gnr} smear.\textsc{inf} with wax \textsc{purp} \textsc{neg} \textsc{comp=3sg} fray.\textsc{pres.sbjv.3sg}\\
\glt `Doubly twisted thread must get smeared with wax so it doesn’t get frayed.'
\z






\subsection{Causal clauses}
 
 \ea\label{ex:1:}
\langinfo{Tuatschin}{Rueras} {\citealt[9]{Valär2013b}}\\
\gll ʥeːvjǝ zɛ l 'tɔni dɐ lɐ mɐt'lajnɐ sǝʃfǝrdɐntaws ʃi feʨ \textbf{dɐ} \textbf{'bajbǝr} \textbf{trajs} \textbf{'mjɔːlas} \textbf{pɛːn} \textbf{frajt} \\
 Thursday  be.\textsc{prs.3sg}  \textsc{def.art.m.sg} \textsc{pn} of \textsc{def.art.f.sg} \textsc{pn} \textsc{refl}.catch.cold.\textsc{ptcp.m.sg} so much \textsc{comp} drink.\textsc{inf} three cup.\textsc{pl}  buttermilk cold  \\
\glt `Thursday Matlaina’s Toni caught a very strong cold because he drank three cups of cold buttermilk.'
\z

\ea\label{}
\langinfo{Tuatschin}{}{\DRG{3}{719}}\\
\gll Ùssa leiva \textbf{tgi} è clar dé.\\
  now get.up.\textsc{imp.2sg} \textsc{comp.expl} \textsc{cop.prs.3sg} clear day\\
\glt `Get up now since day has already broken.'
\z




\subsection{Conditional clauses}
Conditional clauses are formed in four different ways: (1) the protasis is introduced by the subordinator \textit{sche} 'if', (2) a correlative construction with \textit{sche} both in the protasis and the apodosis, (3) with a correlative construction that has \textit{sche} in the protasis and \textit{lura} in the apodosis, or (4) without subordinator in the protasis, but with subject inversion, probably under German influence.

\ea\label{}
\langinfo{Tuatschin}{}{\citealt[60]{Berther1998}}\\
\gll \textbf{Sche} ju entupass quella gliut sebetess ju giu en ganuglias e bitschass ils cazes.\\
     if \textsc{1sg} meet.\textsc{cond.1sg} \textsc{dem.f.sg} people \textsc{refl}.throw.\textsc{cond.1sg} \textsc{1sg} down in knee.\textsc{f.pl} and kiss.\textsc{cond.1sg} \textsc{def.art.m.pl} shoe.\textsc{pl}\\
\glt `If I met these people, I’d kneel down and kiss their shoes.'
\z

\ea\label{}
\langinfo{Tuatschin}{}{\citealt[120]{Berther1998}}\\
\gll \textbf{Sche} te as lu memia biè da reclamà a grì \textbf{sche} mattein nus te ainagiu ‘l Run.\\
if \textsc{2sg} have.\textsc{prs.2sg} then too much to complain and shout then put.\textsc{prs.1pl} \textsc{1pl} \textsc{2sg} in.down \textsc{def.art.m.sg} \textsc{pln}\\
     \glt ` If you really have so much to complain and to shout, we will throw you down into the Run [river].'
\z

\ea\label{}
\langinfo{Tuatschin}{}{\citealt[60]{Berther1998}}\\
\gll Sche ju entupass quella gliut sebetess ju giu en ganuglias e bitschass ils cazes.\\
     if \textsc{1sg} meet.\textsc{cond.1sg} \textsc{dem.f.sg} people \textsc{refl}.throw.\textsc{cond.1sg} \textsc{1sg} down in knee.\textsc{f.pl} and kiss.\textsc{cond.1sg} \textsc{def.art.m.pl} shoe.\textsc{pl}\\
\glt `If I met these people, I’d kneel down and kiss their shoes.'
\z


\ea\label{}
\langinfo{Tuatschin}{Selva}{\citealt[34]{Büchli1966}}\\
\gll    \textbf{Vǝsèevan} \textbf{ins} ina signura […] cun schuba cuerta, cotschna, […] \textbf{lura} spitgavan ǝls purs ina gronda malaura […].\\
     see.\textsc{impf.3sg} \textsc{gener} \textsc{indef.art.f.sg} woman [...] with shirt short red [...] then expect.\textsc{impf.3pl} \textsc{def.art.m.pl} peasant.\textsc{pl} \textsc{indef.art.f.sg} big storm\\
\glt `If one saw a woman with a short shirt, a red one, the peasants would expect a heavy storm.'
\z

n = Bindevokal bei vaseevan ins.


\subsection{Consecutive clauses}

\ea\label{ex:1:}
\langinfo{Tuatschin}{Tschamut}{\citealt[20]{Büchli1966}}\\
\gll  Ella detgi ingnèeda ina curnada li el, \textbf{tg’} \textbf{el} \textbf{stetschi} \textbf{sel} \textbf{plaz}.\\
     3SG give.\textsc{prs.sbjv.3sg} once \textsc{def.art.f.sg} push.with.horn  \textsc{dat} \textsc{3sg.m} \textsc{comp} \textsc{3sg} stay.\textsc{prs.sbjv.3sg} on.\textsc{def.art.m.sg} place \\
\glt `She [the cow] would give him a push with her horns so that he would remain on the spot.'
\z

 \ea\label{}
\langinfo{Tuatschin}{Camischolas}{\DRG{3}{583}}\\
\gll   La bucca stuev'esser \textbf{tga} la pudev'ain la latta.\\
     \textsc{def.art.f.sg} mouth should.\textsc{impf.3sg}{=cop.inf} \textsc{comp} \textsc{3sg} can.\textsc{impf.3sg}=in \textsc{def.art.f.sg} slat\\
\glt `The cutting would have to be such that the slat could fit into it.'
\z


\ea\label{}
\langinfo{Medelin}{Curaglia}{\DRG{6}{337}}\\
\gll  Quèl èra \textbf{schi} fins \textbf{tga}’l saveva saglí sigl uíarcal da la pipa dals signurs.\\
    \textsc{dem} \textsc{cop.impf.3sg} so fine.\textsc{m.sg.pred} \textsc{comp}=\textsc{3sg} can.\textsc{impf.3sg} jump.\textsc{inf}  on.\textsc{def.art.m.sg} lid of \textsc{def.art.f.sg} pipe  \textsc{def.art.m.pl} gentleman.\textsc{pl}\\
\glt `This one was so smart that he was able to play up the gentlemen.'
\z


\subsection{Comparative clauses}

\ea\label{}
\langinfo{Tuatschin}{}{\DRG{6}{300}}\\
\gll  I vò fil a fil \textbf{scò} \textbf{da} \textbf{caná} in anṣéil.\\
    \textsc{expl} go.\textsc{prs.3sg} jet to jet \textsc{cpr} \textsc{comp} stab.\textsc{inf} \textsc{indef.art.m.sg} kid \\
\glt `[Blood] flows like when one stabs a kid.'
\z



\section{Topic and focus}




\ea\label{ex:1:büchli}
\langinfo{Tuatschin}{Sedrun}{\citealt[106]{Büchli1966}}\\
\gll  Ju a cò in bagliet tochen giů Turitg, aber \textbf{ira} \textbf{vond} \textbf{ju} mů giů Sumvitg.\\
     1\textsc{SG} have.3\textsc{sg} here \textsc{indef.art.m.sg} ticket until down \textsc{pln} but go.\textsc{inf} go.\textsc{prs}.1\textsc{sg} 1\textsc{sg} only down \textsc{pln}\\
\glt `I have here a ticket to Zurich, but I only go till Sumvitg.'
\z

With modal verbs, the main verb is left-dislocated and the modal verb is left in the background clause.

\ea\label{}
\langinfo{Tuatschin}{Cavorgia}{\citealt[125]{Büchli1966}}\\
\gll ǝls tiers vèzan ǝl bǝrlot ǝ tèman mů \textbf{dî} \textbf{sòn} i nuet.\\
     \textsc{def.art.m.pl} animal.\textsc{pl} see.\textsc{prs.3pl} \textsc{def.art.m.sg} sorcery and be.afraid.\textsc{prs.3pl} but say.\textsc{inf} can.\textsc{prs.3pl} \textsc{3pl} nothing\\
\glt `The animals see the sorcery and are afrait, but they cannot say anymething.'
\z


