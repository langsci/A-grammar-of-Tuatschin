\chapter{Complex sentences}

\section{Coordination}

\ea
\label{}
\langinfo{Tuatschín}{Ruèras}{\DRG{9}{452}}\\
\gll Basta, da fá fùssi ùssa gl atún la pjal plajna da tùtas uisas. \textbf{Dantaun} sche l' aura tégn ansjaman vòi schòn.\\
\\
\glt `Nun gut, jetzt im Herbst, hätte man die Hände voll zu tun mit allerlei Arbeiten.Aber wenn das Wetter anhält, geht es schon.'
\z


\section{Subordination}

\subsection{Subject clauses}


\ea
\label{}
\langinfo{Tuatschín}{Zarcúns}{m2, l. 1634ff.}\\
	\gll    Òh \textbf{gl} \textbf{ampréndar} \textbf{tudèstg} è stau, l’ antschata ṣè quaj schòn stau in téc curiùs.\\
oh \textsc{def.art.m.sg} learn.\textsc{inf} German.\textsc{m.sg} be.\textsc{prs.3sg}  \textsc{cop.ptcp.unm} \textsc{def.art.f.sg} beginning be.\textsc{prs.3sg} \textsc{dem.unm} indeed \textsc{cop.ptcp.unm} \textsc{indef.art.m.sg} bit strange.\textsc{adj.unm}\\
\glt `Oh, to learn German was, at the beginning this was indeed a little bit strange.'
\z

\ea
\label{}
\langinfo{Tuatschín}{}{\DRG{}{}}\\
\gll \textbf{Fugí} è bian, a \textbf{mitschè} è aun mégljar. \\
   flee.\textsc{inf} \textsc{cop.prs.3sg} good and escape.\textsc{inf} \textsc{cop.prs.3sg} still better  \\
\glt `To flee is good, and to escape is even better.'
\z

\ea
\label{}
\langinfo{Tuatschín}{Sadrún}{m5}\\
\gll I è stau mégljar \textbf{da} \textbf{còschar}.\\
\textsc{expl} be.\textsc{prs.3sg} \textsc{cop.ptcp.unm} better \textsc{comp} keep\_silent.\textsc{inf}\\
\glt `It was better to keep silent.'
\z

\ea
\label{}
\langinfo{Tuatschín}{Sadrún}{m6, l. 1379ff.}\\
\gll    A \textbf{dumagnè} als amprèmṣ dis quèls pòrs gjù da da quaj trùtg ajnagjù a sjantar atrás l’ aua dal Drun tga vagnéva mù pauc vi da tschèla vart, \textbf{qu}’ è stau álṣò in martéri.\\
and cope.\textsc{inf} \textsc{def.art.m.sg} first.\textsc{pl}  day.\textsc{pl} \textsc{dem.m.pl}  pig.\textsc{pl} down from from \textsc{dem.m.sg} path into\_and\_down and after through  \textsc{def.art.f.sg} water of.\textsc{def.art.m.sg}  \textsc{rn} \textsc{rel} come.\textsc{impf.3sg} only little over of \textsc{dem.f.sg} side \textsc{dem.unm.} be.\textsc{prs.3sg} \textsc{cop.ptcp.unm} well \textsc{indef.art.m.sg} ordeal \\
\glt `And the first days, to cope with these pigs [going] down this path and then trough the water of the Drun, of which only few would go over to the other side, well, this was an ordeal [...].'
\z

\ea
\label{}
\langinfo{Tuatschín}{Ruèras}{m3, l. 2140f.}\\
	\gll [...] ábar \textbf{al} \textbf{trá} \textbf{ajn} \textbf{èla} èra … da maz.   \\
{} but \textsc{def.art.m.sg} pull.\textsc{inf} in \textsc{3sg.f} \textsc{cop.impf.3sg} {} of killing.\textsc{m.sg}\\
\glt `[...] but bringing it in was terrible.'
\z

Following ex: subject clause in predicative position triggers use of expl pron

\ea
\label{}
\langinfo{Tuatschín}{Sadrún}{m9, l. 1831}\\
	\gll [...] \textbf{i} è gréjv \textbf{dí} [...].\\
 {} \textsc{expl} \textsc{cop.prs.3sg} difficult.\textsc{adj.unm} say.\textsc{inf}\\
\glt `[...] it is difficult to say [...].'
\z

\subsection{Object clauses}

\ea
\label{}
\langinfo{Tuatschín}{Surajn}{\citealt[128]{Büchli1966}}\\
\gll Al pástar gròn ò \textbf{admonju} \textbf{da} bétga \textbf{fá} tupadats.\\
\textsc{def.art.m.sg} herdsman big have.\textsc{prs.3sg} admonish.\textsc{ptcp.unm} \textsc{comp} \textsc{neg} do.\textsc{inf} stupidity.\textsc{f.pl}\\
\glt `The main herdsman admonished not to commit stupidities.'
\z

\ea
\label{}
\langinfo{Tuatschín}{Ruèras}{m10, l. 1171ff.}\\
\gll  [...] api vagi èl \textbf{tartgau} ... \textbf{dad} \textbf{í} \textbf{vi} ajn via ... a tanaj sé èls.  \\
{} and have.\textsc{prs.sbjv.3sg} \textsc{3sg.m} think.\textsc{ptcp.unm} {} \textsc{comp} go.\textsc{inf} over on road.\textsc{f.sg} {} and hold.\textsc{inf} up \textsc{3pl.m}\\
\glt `[...] and he thought... that he would go on the road ... and stop them.'
\z

\ea
\label{}
\langinfo{Tuatschín}{Cavòrgja}{m7, l.2184f.}\\
\gll [...] \textbf{cù} \textbf{barsá} stù ju bigja dí [...].»\\
{} how roast.\textsc{inf} must.\textsc{prs.1sg} \textsc{1sg} \textsc{neg} say.\textsc{inf}\\
\glt `[...] how to roast [it] I don't have to tell [you] [...].'
\z


The object clause may be separated from the main verb by a temporal clause, as in (\ref{sacutga}).

\ea
\label{sacutga}
\langinfo{Tuatschín}{Zarcúns}{m2, l. 1630fff.}\\
\gll    Ju \textbf{sa} cu cu … cu ju sùn maridaus \textbf{tga} … èri ajn Camischùlas circa quindiṣch ufauns … tga mavan á scùla da Camischùlas.\\
\textsc{1sg} know.\textsc{prs.1sg} when when {} when \textsc{1g} be.\textsc{prs.1sg}  marry.\textsc{ptcp.m.sg} \textsc{comp} {} \textsc{exist.impf.3sg.expl} in \textsc{pln} about fifteen child.\textsc{m.pl} {} \textsc{rel} go.\textsc{impf.3pl} to school.\textsc{f.sg} of  \textsc{pln}\\
\glt `I know that when when … when I got married … there were in Camischolas about fifteen … children who attended the school of Camischolas.'
\z

\ea
\label{}
\langinfo{Tuatschín}{Sadrún}{m9, l. 1755ff.}\\
	\gll [...] in téc da la gjuvantétgna sa ju schòn \textbf{raquintá} … \textbf{usché} \textbf{scù} i mava da nòs tjams a \textbf{tgé} \textbf{ca} va ùsa [...].\\
{} \textsc{indef.art.m.sg} bit of \textsc{def.art.f.sg} youth can.\textsc{prs.1sg} \textsc{1sg} indeed tell.\textsc{inf} {} so as \textsc{expl} go.\textsc{impf.3sg} of \textsc{poss.1pl.m.sg} time.\textsc{pl} and what \textsc{rel} go.\textsc{prs.3sg} now \\
\glt `[...] a bit of my youth I can indeed tell [you about] ... the way it was when we were young and the way it is nowadays [...].'
\z

\ea
\label{}
\langinfo{Tuatschín}{Sadrún}{m9, l. 1810}\\
	\gll [...] a lu \textbf{sa} ju schòn in téc \textbf{sc}’ \textbf{i} \textbf{funczjanava}.   \\
{} and then know.\textsc{prs.1sg} \textsc{1sg} indeed \textsc{indef.art.m.sg} bit how \textsc{expl} function.\textsc{impf.3sg}\\
\glt `[...] and therefore I know a bit how it used to function.'
\z


\subsection{Temporal clauses}
The usual temporal subordinator is \textit{cu} (derived from the interrogative pronoun \textit{cura} 'when'), which in combination with \textit{ju} `I' is realized \textit{tgu}, as in (\ref{tgu:1}). Sometimes, a correlative construction with \textit{scha} `if; then' is used.

\ea
\label{tgu:1}
\langinfo{Tuatschín}{Sadrún}{m4, l. 471f.}\\
\gll  [...] \textbf{avaun} \textbf{c}’ \textbf{í} \textbf{vidajn}, ah, staus lu tial miadi[...]. \\
{} before \textsc{comp} go.\textsc{inf} uphill eh \textsc{cop.ptcp.m.sg} then  at.\textsc{def.art.m.sg} doctor\\
\glt `[...] before going uphill, he went to the doctor [...].'
\z

With the subordinator \textit{cu}, the main clause is introduced by the correlative element \textit{scha}.
\ea
\label{}
\langinfo{Tuatschín}{}{\citealt[23]{Berther2007}}\\
\gll  […] \textbf{cu} i dat la bènèdiczjun, \textbf{scha} fò la sòntga crusch.\\
    […] when \textsc{expl} give.\textsc{prs.3sg} \textsc{def.art.f.sg} blessing \textsc{corr} make.\textsc{imp.2sg} \textsc{def.art.f.sg} holy cross\\
\glt `[…] when it comes to the blessing, make the sign of the cross.'
\z

\ea
\label{}
\langinfo{Tuatschín}{Ruèras}{m3, l. 2206f.}\\
	\gll  A pér \textbf{cu} quaj èra fatg \textbf{scha} èra la scòtga mèmi tgauda [...].  \\
and only when \textsc{dem.unm} \textsc{pass.aux.impf.3sg} do.\textsc{ptcp.unm} \textsc{corr} \textsc{cop.impf.3sg} \textsc{def.art.f.sg} whey too hot\\
\glt `And only when this was done, the whey was too hot [...].'
\z


\ea
\label{}
\langinfo{Tuatschín}{Ruèras}{m1, l. 280f.}\\
\gll    A sjantar [...] ṣè 'l bap lu mòrts, \textbf{tgu} vèva…. mù vèntgadúṣ òns.\\
and after {} be.\textsc{prs.3sg} \textsc{def.art.m.sg} father then die.\textsc{ptcp.m.sg} \textsc{rel.1sg} have.\textsc{impf.3sg} only twenty-two year.\textsc{m.pl}\\
\glt `And after that, yes, I had … my father then died when I was … only 22 years old.' (temporal argument)
\z

\ea
\label{}
\langinfo{Tuatschín}{Tschamùt} {\citealt[17]{Büchli1966}}\\
\gll    sùnd ju ida a tgèsa, \textbf{avòn} \textbf{ca} \textbf{tga} vagní stgir.\\
     be.\textsc{prs.1sg} \textsc{1sg} go.\textsc{ptcp.f.sg} to house before \textsc{comp} \textsc{comp} get.\textsc{prs.sbjv.3sg} dark \\
\glt `[…] I went home before it got dark.'
\z

\ea
\label{}
\langinfo{Tuatschín}{Tschamùt} {\citealt[19]{Büchli1966}}\\
\gll    Ju a gju tèma tg’ als tiars réjsdjan \textbf{durònt} \textbf{tga} la sòntga mèssa da Nadal végni lagida.\\
      \textsc{1sg} have.\textsc{prs.1sg} have.\textsc{ptcp.unm} fear \textsc{comp} \textsc{def.art.m.pl} animal.\textsc{pl} talk.\textsc{prs.sbjv.3pl} during \textsc{comp} \textsc{def.art.f.sg} holy mass of Christmas come.\textsc{prs.sbjv.3sg} read.\textsc{ptcp.f.sg}\\
\glt `I feared […] that the animals could talk when the Holy Mass would be read.'
\z

\ea
\label{}
\langinfo{Tuatschín}{Cavòrgja}{\citealt[120]{Büchli1966}}\\
\gll    \textbf{Da} \textbf{vagní} \textbf{anavùs} vònd ju plaunsju!\\
    \textsc{comp} come.\textsc{inf} back go.\textsc{prs.1sg} \textsc{1sg} slowly \\
\glt `When I come back, I’ll walk slowly!'
\z

\ea\label{}
\langinfo{Tuatschín}{Sadrún}{m4, l. 562f.}\\
\gll  Ò lò vòu fòrza schòn è survagnú in téc quajda d' í par crapa, \textbf{tgu} a vju difarènts lògans tg’ i vèvan sitau gjù ad èra… vagnú ò cristalaṣ [...]. \\
down there  have.\textsc{prs.1sg.1sg} maybe really also get.\textsc{ptcp.m.unm} \textsc{indef.art.m.sg} bit desire.\textsc{f.sg} \textsc{comp} go.\textsc{inf} for stone.\textsc{coll} \textsc{rel.1sg} have.\textsc{prs.1sg} see.\textsc{ptcp.m.unm} different.\textsc{m.pl} place.\textsc{pl} \textsc{rel} \textsc{3pl} have.\textsc{impf.3pl} blast.\textsc{ptcp.m.unm} down and  be.\textsc{impf.3sg} come.\textsc{ptcp.m.unm} out crystal.\textsc{f.pl}\\
\glt `Out there I might have started enjoying a bit going for stones, when I saw different places where they had blasted [the rocks], and crystals [...] had come out [...].'
\z


\ea\label{}
\langinfo{Tuatschín}{Ruèras}{\citealt[8]{Valär2013b}}\\
\gll    […] faː ajn […] ʦa’kɔnʣ diants aʎ rǝʃ’tiː dɐ lɐ dʊːnɐ \textbf{ʨ} ɐl ɔ rut ɔːr eir dɐ mɐ’ʦaː in ruʃp\\
    […] make.\textsc{inf} in […] some tooth.\textsc{m.pl} {dat} rake of \textsc{def.art.f.sg} woman \textsc{rel} \textsc{3sg} have.\textsc{prs.3sg} break.\textsc{ptcp.unm} out also \textsc{comp} kill.\textsc{inf} \textsc{indef.art.m.sg} toad\\
\glt `[…] put in some teeth to the woman’s rake he had broken when he killed a toad.'
\z


\ea
\label{}
\langinfo{Tuatschín}{Sadrún}{m5}\\
\gll \textbf{Dafartáuntiar} \textbf{tga} la mùma fò quaj, miras té dal pòp.\\
whilst \textsc{rel} \textsc{def.art.f.sg} mother do.\textsc{prs.3sg} \textsc{dem.unm} look.\textsc{prs.2sg} \textsc{2sg} of.\textsc{def.art.m.sg} baby\\
\glt `Whilst mother is doing this, you look after the baby.'
\z

\ea
\label{}
\langinfo{Tuatschín}{Camischùlas}{f6, l. 741f.}\\
\gll    Api èra la sòra òra uschéja … avaun niaṣ ésch ad ò spatgau a spatgau \textbf{tòca} \textbf{la} \textbf{audi} \textbf{anzatgéj} [...].\\
and \textsc{cop.impf.3sg} \textsc{def.art.f.sg} nun out so {} in\_front\_of \textsc{poss.1pl.m.sg} door and have.\textsc{prs.3sg} wait.\textsc{ptcp.unm} and wait.\textsc{ptcp.unm} until \textsc{3sg.f} hear.\textsc{prs.sbjv.3sg} something\\
\glt `And then the nun was out [on the corridor] like this ... in front of our door, waiting and waiting until she would hear something [...]'
\z

\ea
\label{}
\langinfo{Tuatschín}{Ruèras}{m10, l.  1134f.}\\
\gll  A schi … \textbf{schi} \textbf{glajti} \textbf{scù} nuṣ èssan staj sén la via cantunala òni antschiat á galòpá.  \\
and so {} so soon as \textsc{1pl} be.\textsc{prs.1pl} \textsc{cop.ptcp.m.pl} on \textsc{def.art.f.sg} way cantonal have.\textsc{prs.3pl.3pl} begin.\textsc{ptcp.unm} \textsc{comp} gallop.\textsc{inf}\\
\glt `And as … as soon as we were on the cantonal way they started to gallop.'
\z



\subsection{Manner clauses}

\subsection{Purposive clauses}

There are two purposive prepositions heading an infinitive: \textit{a} after a verb of movement, and \textit{par/pr} in all other cases. Example (\ref{ex:purp:inf:1}) contains both prepositions.

\ea\label{ex:purp:inf:3}
\langinfo{Tuatschín}{Ruèras}{m11, l.960ff.}\\
\gll Als méls èran, qu’ èra… fèrm… fèrms tiarṣ ad èl duvrava quaj mél pr trá, \textbf{pr} \textbf{trá} lèna sé da Cavòrgja.   \\
\textsc{def.art.m.pl} mule \textsc{cop.impf.3sg} \textsc{dem.unm} \textsc{cop.impf.3sg} strong.\textsc{adj.unm} strong.\textsc{m.pl} animal.\textsc{pl} and \textsc{3sg.m} use.\textsc{impf.3sg} \textsc{dem.m.sg} mule \textsc{purp} pull.\textsc{inf} \textsc{purp} pull.\textsc{inf} wood.\textsc{coll} up from \textsc{pln}  \\
\glt `The mules were, these were in fact …. strong … strong animals and he used that mule for transporting wood up from Cavorgia.'
\z


\ea\label{ex:purp:inf:2}
\langinfo{Tuatschín}{Sèlva}{f2, l.939f.}\\
\gll  Api ṣchèvan aj èba in tg’ è mòrts tga \textbf{végn} \textbf{á} \textbf{mètar} als, als tjarms... la nòtg [...].\\
and say.\textsc{impf.3pl} \textsc{3pl} precisely one \textsc{rel} \textsc{cop.prs.3sg} dead.\textsc{m.sg} \textsc{rel} come.\textsc{prs.3sg} \textsc{purp} put.\textsc{inf} \textsc{def.art.m.pl} \textsc{def.art.m.pl} boundary\_stone.\textsc{pl} \textsc{def.art.f.sg} night \\
\glt `And, precisely, they also used to say [that] somebody who was dead came and put the, the boundary stones ... at night [...].'
\z

\ea
\label{ex:purp:inf:1}
\langinfo{Tuatschín}{Sadrún}{m4, l.416ff.}\\
\gll [...] ju vèva… siṣ òns tga èran vida prapará la via \textbf{par} \textbf{í} sé Nalps \textbf{á} \textbf{bagagè} al mir da farmada.\\
{} \textsc{1sg} have.\textsc{impf.1sg} six year.\textsc{m.pl} \textsc{comp} \textsc{cop.impf.3pl} \textsc{prog} prepare.\textsc{inf} \textsc{def.art.f.sg} road \textsc{purp} go.\textsc{inf} up \textsc{pln} and build.\textsc{inf} \textsc{def.art.m.sg} wall.\textsc{m.sg} of reservoir.\textsc{f.sg} \\ 
\glt `[...] I was six years old when they were preparing the road in order to go to Nalps to build the wall of the reservoir.'
\z

There is one example where the complementizer á is absent.

\ea
\label{}
\langinfo{Tuatschín}{Ruèras}{f4, l. 1948}\\
	\gll [...] api èri dad \textbf{í} \textbf{métar} trúfals [...] .\\
{} and be.\textsc{impf.3sg.expl} \textsc{comp} go.\textsc{inf} put potato.\textsc{m.pl}\\
\glt `[...] and then one had to sow potatoes [...].'
\z


 \ea\label{}
\langinfo{Tuatschín}{Bugnaj} {\citealt[143]{Büchli1966}}\\
\gll  [...] méz in pétg sc’ ins drùva par tùt las lavurs da parmavèra a gl’ atún \textbf{da} \textbf{cavá} \textbf{trúfals}.\\
{} put.\textsc{ptcp.unm} \textsc{indef.art.m.sg} hoe like \textsc{impers} need.\textsc{prs.3sg} for all \textsc{def.art.f.pl} work.\textsc{pl} of spring and \textsc{def.art.m.sg} autumn \textsc{purp} dig potato.\textsc{pl}\\
\glt `[…] put a hoe like [the one] one needs for all the work that must be done in spring, and in autumn in order to dig out potatoes.'
\z

\ea\label{}
\langinfo{Tuatschín}{Ruèras}{\citealt[9]{Valär2013b}}\\
\gll ‘ʥeːvjǝ zɛ l ‘tɔni dɐ lɐ mɐt’lajnɐ sǝʃfǝrdɐntaws ʃi feʨ \textbf{dɐ} ‘\textbf{bajbǝr} \textbf{trajs} \textbf{mjɔːlas} \textbf{pɛːn} \textbf{frajt}\\
     Thursday \textsc{cop.prs.3sg} \textsc{def.art.m.sg} \textsc{pn} of \textsc{def.art.f.sg} \textsc{pn} \textsc{refl}.catch.cold.\textsc{ptcp.m.sg} so much \textsc{comp} drink.\textsc{inf} three cup.\textsc{pl} buttermilk cold\\
\glt `Thursday Matlaina’s Toni caught a very strong cold because he drank three cups of cold buttermilk.'
\z

In the DRG materials, there is one occurrence of \textit{bétg} `negator' located between \textit{par} `for' and \textit{tga} `complementizer', i.e. outside of the subordinate clause.

This construction has not been accepted by my informants; however, \textit{par bétg tga} exists or has existed also in other Romansh dialects, as e.g. in the Sutsilvan dialect of Dalin.

\ea\label{}
\langinfo{Sutsilvan}{Dalin}{\DRG{4}{607}}\\
\gll  \textbf{Par} \textbf{bétg} \textbf{tg}'in schleschi dat il calger eign in pêr guspas els calzers.\\
     \textsc{purp} \textsc{neg}\textsc{comp=gdl} slip.\textsc{prs.sbjv.3sg} give.\textsc{prs.3sg} \textsc{def.art.m.sg}shoemaker in \textsc{indef.art.m.sg} some nail.\textsc{pl} in.\textsc{def.art.m.pl} shoe.\textsc{pl}\\
\glt `In order not to slip, the shoemaker beat some nails into the shoes.'
\z

\ea\label{}
\langinfo{Sutsilvan}{Dagliegn}{\DRG{6}{297}}\\
\gll   Fil stiert stogn strihar cun tschera \textbf{par} \textbf{betg} \textbf{tg}’el sasfili.\\
     thread doubly.twisted must.\textsc{pres.3sg=gnr} smear.\textsc{inf} with wax \textsc{purp} \textsc{neg} \textsc{comp=3sg} fray.\textsc{pres.sbjv.3sg}\\
\glt `Doubly twisted thread must get smeared with wax so it doesn’t get frayed.'
\z

DA + INF

\ea\label{}
\langinfo{Tuatschín}{Zarcúns}{m2, l. 1579f.}\\
	\gll    Api savèv’ ins bigja cù, cù fá \textbf{da} \textbf{purtá} \textbf{las} \textbf{nèglas} [...].\\
and know.\textsc{impf.3sg} \textsc{gnr} \textsc{neg} how how do.\textsc{inf} \textsc{comp} carry.\textsc{inf} \textsc{def.art.f.pl} carnation.\textsc{pl} \\
\glt `And one would not know how to put the carnations [...].'
\z




\subsection{Causal clauses}

 
 
\ea
\label{}
\langinfo{Tuatschín}{Sadrún}{f3, l. 24ff.}\\ 
\gll  [...] api lura va ju in’ jèda talafònau dad èl \textbf{prquaj} \textbf{tg}' èl vèva tarmèz in’ anunzja da mòrt [...].\\
then have.\textsc{1sg}  \textsc{1sg} one.\textsc{f.sg} time call.\textsc{ptcp.unm} \textsc{dat} \textsc{3sg.m} because \textsc{comp} \textsc{3sg.m} have.\textsc{impf.3sg} send.\textsc{ptcp.unm} \textsc{indef.art.f.sg} announcement of death.\textsc{f.sg}\\ 
\glt `[...] then I phoned him once, because I should send a death notice [...].'
\z
 
 
 \ea\label{ex:1:}
\langinfo{Tuatschín}{Rueras} {\citealt[9]{Valär2013b}}\\
\gll ʥeːvjǝ zɛ l 'tɔni dɐ lɐ mɐt'lajnɐ sǝʃfǝrdɐntaws ʃi feʨ \textbf{dɐ} \textbf{'bajbǝr} \textbf{trajs} \textbf{'mjɔːlas} \textbf{pɛːn} \textbf{frajt} \\
 Thursday  be.\textsc{prs.3sg}  \textsc{def.art.m.sg} \textsc{pn} of \textsc{def.art.f.sg} \textsc{pn} \textsc{refl}.catch.cold.\textsc{ptcp.m.sg} so much \textsc{comp} drink.\textsc{inf} three cup.\textsc{pl}  buttermilk cold  \\
\glt `Thursday Matlaina’s Toni caught a very strong cold because he drank three cups of cold buttermilk.'
\z

\ea\label{}
\langinfo{Tuatschín}{}{\DRG{3}{719}}\\
\gll Ùssa léjva \textbf{tgi} è clar dé.\\
  now get.up.\textsc{imp.2sg} \textsc{comp.expl} \textsc{cop.prs.3sg} clear day\\
\glt `Get up now since day has already broken.'
\z




\subsection{Conditional clauses}
Conditional clauses are formed in four different ways: (1) the protasis is introduced by the subordinator \textit{sche} 'if', (2) a correlative construction with \textit{sche} both in the protasis and the apodosis, (3) with a correlative construction that has \textit{sche} in the protasis and \textit{lura} in the apodosis, or (4) without subordinator in the protasis, but with subject inversion, probably under German influence.

\ea\label{}
\langinfo{Tuatschín}{}{\citealt[60]{Berther1998}}\\
\gll \textbf{Sche} ju entupass quella gliut sebetess ju giu en ganuglias e bitschass ils cazes.\\
     if \textsc{1sg} meet.\textsc{cond.1sg} \textsc{dem.f.sg} people \textsc{refl}.throw.\textsc{cond.1sg} \textsc{1sg} down in knee.\textsc{f.pl} and kiss.\textsc{cond.1sg} \textsc{def.art.m.pl} shoe.\textsc{pl}\\
\glt `If I met these people, I’d kneel down and kiss their shoes.'
\z

\ea\label{}
\langinfo{Tuatschín}{}{\citealt[120]{Berther1998}}\\
\gll \textbf{Sche} te as lu memia biè da reclamà a grì \textbf{sche} mattein nus te ainagiu ‘l Run.\\
if \textsc{2sg} have.\textsc{prs.2sg} then too much to complain and shout then put.\textsc{prs.1pl} \textsc{1pl} \textsc{2sg} in.down \textsc{def.art.m.sg} \textsc{pln}\\
     \glt ` If you really have so much to complain and to shout, we will throw you down into the Run [river].'
\z

\ea\label{}
\langinfo{Tuatschín}{}{\citealt[60]{Berther1998}}\\
\gll Sche ju entupass quella gliut sebetess ju giu en ganuglias e bitschass ils cazes.\\
     if \textsc{1sg} meet.\textsc{cond.1sg} \textsc{dem.f.sg} people \textsc{refl}.throw.\textsc{cond.1sg} \textsc{1sg} down in knee.\textsc{f.pl} and kiss.\textsc{cond.1sg} \textsc{def.art.m.pl} shoe.\textsc{pl}\\
\glt `If I met these people, I’d kneel down and kiss their shoes.'
\z


\ea\label{}
\langinfo{Tuatschín}{Sèlva}{\citealt[34]{Büchli1966}}\\
\gll    \textbf{Vǝsèevan} \textbf{ins} ina signura […] cun schuba cuerta, cotschna, […] \textbf{lura} spitgavan ǝls purs ina gronda malaura […].\\
     see.\textsc{impf.3sg} \textsc{gener} \textsc{indef.art.f.sg} woman [...] with shirt short red [...] then expect.\textsc{impf.3pl} \textsc{def.art.m.pl} peasant.\textsc{pl} \textsc{indef.art.f.sg} big storm\\
\glt `If one saw a woman with a short shirt, a red one, the peasants would expect a heavy storm.'
\z

n = Bindevokal bei vaseevan ins.


\subsection{Consecutive clauses}

\ea\label{}
\langinfo{Tuatschín}{Sadrún}{m6, l. 1397ff.}\\
	\gll    [...] quaj piartg èra juṣ atráṣ a vèva rùt gjù al matg \textbf{tga} ’l vèva mù la còrda plé antùrn.\\
{} \textsc{dem.m.sg} pig be.\textsc{impf.3sg} go.\textsc{ptcp.m.sg} through and have.\textsc{impf.3sg} break.\textsc{ptcp.unm} down \textsc{def.art.m.sg} bunch  \textsc{comp} \textsc{3sg.m} have.\textsc{impf.3sg} only \textsc{def.art.f.sg} rope more around\\
\glt `[...] this pig had gone through and had broken the bunch of flowers so that he only had the rope around him.'
\z


\ea
\label{}
\langinfo{Tuatschín}{Tschamùt}{\citealt[20]{Büchli1966}}\\
\gll  Ella detgi ingnèeda ina curnada li el, \textbf{tg’} \textbf{el} \textbf{stetschi} \textbf{sel} \textbf{plaz}.\\
     3SG give.\textsc{prs.sbjv.3sg} once \textsc{def.art.f.sg} push.with.horn  \textsc{dat} \textsc{3sg.m} \textsc{comp} \textsc{3sg} stay.\textsc{prs.sbjv.3sg} on.\textsc{def.art.m.sg} place \\
\glt `She [the cow] would give him a push with her horns so that he would remain on the spot.'
\z

\ea\label{}
\langinfo{Tuatschín}{Ruèras}{m10, l. 1085f.}\\
\gll  Bjè jèdaṣ ṣèni scapaj \textbf{tga} nuṣ vajn ah pròpi gju ah gròndas mis\underline{é}rjas [...].\\
many time.\textsc{pl} be.\textsc{prs.3pl.3pl} escape.\textsc{ptcp.m.pl} \textsc{comp} \textsc{1pl}   have.\textsc{prs.1pl} ah really have.\textsc{ptcp.unm} ah big.\textsc{f.pl} trouble.\textsc{pl}\\
\glt `They escaped many times so that we had big troubles [...].'
\z


\ea
\label{}
\langinfo{Tuatschín}{Camischùlas}{\DRG{3}{583}}\\
\gll   La bucca stuev'esser \textbf{tga} la pudev'ain la latta.\\
     \textsc{def.art.f.sg} mouth should.\textsc{impf.3sg}{=cop.inf} \textsc{comp} \textsc{3sg} can.\textsc{impf.3sg}=in \textsc{def.art.f.sg} slat\\
\glt `The cutting would have to be such that the slat could fit into it.'
\z




\subsection{Comparative clauses}

\ea\label{}
\langinfo{Tuatschín}{}{\DRG{6}{300}}\\
\gll  I vò fil á fil \textbf{scù} \textbf{da} \textbf{caná} in anṣéjl.\\
    \textsc{expl} go.\textsc{prs.3sg} jet to jet \textsc{cpr} \textsc{comp} stab.\textsc{inf} \textsc{indef.art.m.sg} kid \\
\glt `[Blood] flows like when one stabs a kid.'
\z

\ea
\label{}
\langinfo{Tuatschín}{Sadrún}{f3, l. 24ff.}\\
\gll  [...] lu va ju … tlafònau dad èl a détg, éba, mi' ùm ségi èba mòrts \textbf{scù} \textbf{i} \textbf{sápjan} [...].\\
{} then have.\textsc{prs.1sg} \textsc{1sg} {} call.\textsc{ptcp.unm} \textsc{dat} \textsc{3sg.m} and say.\textsc{ptcp.unm} exactly \textsc{poss.1sg.m.sg} man be.\textsc{prs.sbjv.3sg} precisely die.\textsc{ptcp.m.sg} as \textsc{3pl} know.\textsc{prs.sbjv.3pl}\\ 
\glt `[...] then I … phoned him and said that my husband had died as he knew [...].'
\z



\subsection{Concessive clauses}

\ea
\label{}
\langinfo{Tuatschín}{Sadrún}{m4, l. 566ff.}\\
\gll \textbf{Schabi} \textbf{tga} lu, cun siṣ òns capév’ ins lu halt aun mèmja pauc a vèva bigja la... fòrsa\footnotemark{} da fá zatgéj.   \\
although \textsc{comp} then with six year.\textsc{m.pl} understand.\textsc{impf.3sg} \textsc{gnr} then just still too little and have.\textsc{impf.3sg} \textsc{neg} \textsc{def.art.f.sg} strength \textsc{comp} do.\textsc{inf} something\\
\glt `Although then, at the age of six, one would understand too little and wouldn’t have the ... strength to do something.'
\z


\subsection{Indirect interrogative clauses}

\ea
\label{}
\langinfo{Tuatschín}{Ruèras}{m10, l. 1079ff.}\\
\gll   Ad èr' è zatgé bi da mirá \textbf{cù} quèls tiars luvravan, cù quèls… mavan ad èran ruassajvalṣ a... pazjènts.\\
and \textsc{cop.impf.3sg} also something beautiful.\textsc{adj.unm} \textsc{comp} look.\textsc{inf} how \textsc{dem.m.pl} animal.\textsc{pl} work.\textsc{impf.3pl} how \textsc{dem.m.pl} go.\textsc{impf.3sg} and \textsc{cop.impf.3pl} calm.\textsc{m.pl} and patient.\textsc{m.pl}  \\
\glt `Also something nice to look at, how these animals worked, how they … used to go and keep calm and patient.'
\z

\ea\label{}
\langinfo{Tuatschín}{Sadrún} {\citealt[103]{Büchli1966}}\\
\gll    El ò dumandau èlas, \textbf{partgéj} èlas ségien bétg idas á mèssa.\\
\textsc{3sg.m} have.\textsc{prs.3sg} ask.\textsc{ptcp.unm} \textsc{3pl.f} why  \textsc{3pl.f} be.\textsc{prs.sbjv.3pl} \textsc{neg} go.\textsc{ptcp.3pl.f} to mass\\
\glt `He asked them why they didn’t go to mass.'
\z

\ea\label{}
\langinfo{Tuatschín}{Sadrún} {\citealt[105]{Büchli1966}}\\
\gll    [...] a damònda \textbf{cù} i vòndi.\\
{} and ask.\textsc{prs.3sg} how \textsc{expl} go.\textsc{prs.sbjv.3sg}\\
\glt `[…] and he asked how he was.'
\z

\ea
\label{}
\langinfo{Tuatschín}{Sadrún}{m4, l. 432ff.}\\
\gll [...] al \textsc{pn} ò è fagj lò in pèr placats tga mùssan ajn via \textbf{nòc}’ ins sa è mirá quaj.\\
{} \textsc{def.art.m.sg} \textsc{pn} have.\textsc{prs.3sg} also make.\textsc{ptcp.unm} there \textsc{indef.art.m.sg} pair poster.\textsc{m.pl} \textsc{rel} show.\textsc{prs.3pl} in way where \textsc{gnr} can.\textsc{prs.3sg} also see.\textsc{inf} \textsc{dem.unm} \\
\glt `[...] \textsc{pn} has also put there some posters which show on the way where one can have a look at this.'
\z

	
\ea
\label{}
\langinfo{Tuatschín}{Ruèras}{m10, l. 1034ff.}\\
\gll A lu vajn nus, quaj èra tùt fatg a racògnòszau avaun tg’ ins savèva \textbf{nua} inṣ vèva da durmí , \textbf{nu} i èra … da mètar ah ṣur nòtg als als méls, \textbf{nu} i dèva pával pls méls [...]. \\
and then have.\textsc{prs.1pl} \textsc{1pl} \textsc{dem.unm} \textsc{pass.aux.impf.3sg} all do.\textsc{ptcp.unm} and  reconnoitre.\textsc{ptcp.unm} before \textsc{comp} \textsc{gnr}  know.\textsc{impf.3sg} where \textsc{gnr} have.\textsc{impf.3sg} \textsc{comp} sleep.\textsc{inf} where \textsc{expl} be.\textsc{impf.3sg} {} \textsc{comp} put.\textsc{inf} ah over night.\textsc{f.sg} \textsc{def.art.m.pl} \textsc{def.art.m.pl} mule.\textsc{pl} where \textsc{expl} \textsc{exist.impf.3sg} food.\textsc{m.sg} for.\textsc{def.art.m.pl} mule.\textsc{pl}\\
\glt `And then we have, this had all been done and reconnoitred before, so that one knew where to sleep, where to put the mules over night, where there was food for the mules[...].'
\z

\ea
\label{}
\langinfo{Tuatschín}{Bugnaj} {\citealt[134]{Büchli1966}}\\
\gll [...] a lu ò `l grju li gljut [...] tga ségi trajs rùsnas; ajn \textbf{tgénina} èl dégi mètar ajn la crusch.\\
{} and then have.\textsc{prs.3sg} \textsc{3sg.m} shout.\textsc{ptcp.unm} \textsc{dat.sg} people.\textsc{f.sg} {} \textsc{comp} \textsc{exist.prs.sbjv.3sg.expl} three hole.\textsc{f.pl} into which.\textsc{f.sg} \textsc{3sg.m} must.\textsc{prs.sbjv.3sg} put into \textsc{def.art.f.sg} cross\\
\glt `[...] and then he shouted to the people [...] [saying that] there were three holes; [asking] into which he should put the cross.'
\z



\section{Other clauses}

\ea
\label{}
\langinfo{Tuatschín}{Ruèras}{m3, l. 2206f.}\\
	\gll [...] \textbf{anstagl} \textbf{da} \textbf{mùngjar} òtgònta vacas èri fòrsa mù tschuncònta [...].\\
 {} instead of milk.\textsc{inf} eighty cow.\textsc{f.pl} \textsc{exist.impf.3sg.expl} maybe only fifty\\
\glt `[...]  instead of milking eighty cows there were maybe fifty [...].'
\z


\section{Topic and focus}

\ea
\label{}
\langinfo{Tuatschín}{Cavòrgja}{\citealt[106]{Büchli1966}}\\
\gll Ju sùn dada gjù séla fatscha, mù \textbf{fatg} \textbf{òi} nuét.\\
\textsc{1sg} be.\textsc{prs.1sg} give.\textsc{ptcp.f.sg} down on.\textsc{def.art.f.sg} face but do.\textsc{ptcp.unm} have.\textsc{prs.3sg.expl} nothing\\
\glt `I fell on my face but it didn't do anything.'
\z

\ea
\label{ex:1:büchli}
\langinfo{Tuatschín}{Sadrún}{\citealt[106]{Büchli1966}}\\
\gll  Ju a cò in bagljèt tòcan gjù Turitg, ábar \textbf{ira} \textbf{vònd} \textbf{ju} mù gjù Sumvitg.\\
     \textsc{1sg} have.\textsc{3sg} here \textsc{indef.art.m.sg} ticket until down \textsc{pln} but go.\textsc{inf} go.\textsc{prs}.\textsc{1sg} \textsc{1sg} only down \textsc{pln}\\
\glt `I have here a ticket to Zurich, but I only go till Sumvitg.'
\z

\ea
\label{}
\langinfo{Tuatschín}{Surajn}{f5, l. 1319}\\
\gll Na na, a \textbf{durmí} \textbf{durmévan} nus cò.\\
no no and sleep.\textsc{inf} sleep.\textsc{impf.1pl} \textsc{1pl} here \\
\glt `No, no, and as for sleeping, we would sleep here.'
\z

With modal verbs, the main verb is left-dislocated and the modal verb is left in the background clause.

\ea\label{}
\langinfo{Tuatschín}{Cavorgia}{\citealt[125]{Büchli1966}}\\
\gll Als tiars vèzan al barlòt a tèman, mù \textbf{dí} \textbf{sòn} i nuét.\\
     \textsc{def.art.m.pl} animal.\textsc{pl} see.\textsc{prs.3pl} \textsc{def.art.m.sg} sorcery and be.afraid.\textsc{prs.3pl} but say.\textsc{inf} can.\textsc{prs.3pl} \textsc{3pl} nothing\\
\glt `The animals see the sorcery and are afrait, but they cannot say anything.'
\z

The following example shows a left-dislocation of a focalised element without subject inversion. Contrastive focus?

\ea
\label{}
\langinfo{Tuatschín}{Sadrún}{m11}\\
\gll Question: Dati da quaj cò? Answer: \textbf{Bjè} i dat.\\
{} \textsc{exist.prs.3sg.expl} of \textsc{dem.gl} here {} a\_lot \textsc{expl} \textsc{exist.prs.3sg} \\
\glt `Question: Is there some of it here? Answer: There is a lot of it.'
\z

\ea\label{}
\langinfo{Tuatschín}{Sadrún}{m5}\\
\gll Alṣò \textbf{cùntar} prandès 'l tùt ùsa?\\
well towards take.\textsc{cond.3sg} \textsc{3sg} all	now\\
\glt ´Well, he would receive everything now.'
\z

\ea
\label{}
\langinfo{Tuatschín}{Ruèras}{f4, l. 1942ff.}\\
	\gll [...] ábar \textbf{stju} \textbf{luvrá} \textbf{còrpòrálmajn} vajn nus schi fétg scù quèls.\\
{}	but must.\textsc{ptcp.unm} work.\textsc{inf} physical.\textsc{adj.m.adv} have.\textsc{prs.1pl} \textsc{1pl} so much as \textsc{dem.m.pl}\\
\glt `[...] but physically we had to work as hard as those [children].'
\z

\section{Extraction}
Extraction out of a subordinate clause for focussing occurs and leaves a trace in the clause if the extracted element is a noun or noun phrase. In the case of (\ref{ex:extr1}), the pronoun \textit{aj} `it' does not agree in number nor in gender with its antecedent \textit{las nòtízjas}.

In the case of an extracted interrogative pronoun, no trace is left in the indirect interrogative clause, as in example (\ref{ex:extr2}).  

\ea\label{ex:extr1}
\langinfo{Tuatschín}{Sadrún}{m1, l. 270f.}\\
\gll  \textbf{Las} \textbf{nòtízjas} sa ju bétg danùndar als gjaniturs, als dus baps prandèvan \textbf{aj} [...]. \\
\textsc{def.art.f.pl} news.\textsc{pl} know.\textsc{prs.1sg} \textsc{1sg} \textsc{neg} from\_where \textsc{def.art.m.pl} parents.\textsc{pl} \textsc{def.art.m.pl} two.\textsc{m.pl} father.\textsc{pl} take.\textsc{impf.3pl} \textsc{3sg}\\
\glt `I don’t know where my parents had the news from, the two fathers took them, there was in Rueras, there was [only] one who had a radio.'
\z

\ea
\label{ex:extr2}
\langinfo{Tuatschín}{Sadrún}{m4, l.453}\\
\gll   Aah, \textbf{tgé}.. prandévan pròpi òra sa ins bégj éxáct [...]. \\
ah what take.\textsc{impf.3pl} exactly out know.\textsc{prs.3sg} \textsc{gnr} \textsc{neg} exactly\\
\glt `Ah, what … they really mined one does not know exactly [...].'
\z




