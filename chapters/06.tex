\chapter{Complex sentences}

\section{Coordination}
Coordinating conjunctions are \textit{a/ad} `and' (\ref{ex:coord1}), \textit{api} `and, and then' (\ref{ex:coord1}), \textit{né} `or', \textit{pi} `then' (\ref{ex:coord2}), \textit{ábar} `but' (\ref{ex:coord3}), and \textit{dantaun} `however, but' (\ref{ex:coord4}).

\ea
\label{ex:coord1}
\langinfo{Tuatschín}{Zarcúns}{m2, l. }\\
\gll    In’ jèda tschò, in’ jèda lò, \textbf{ad} in’ jèda saragòrd’ ju aun … tga nuṣ èran í ajn ustria òl Mírar, \textbf{api} sa ju bégja sén tgé nuṣ èran vagní [...].\\
one.\textsc{f.sg} time here one.\textsc{f.sg} time there and one.\textsc{f.sg} time \textsc{refl}.remember.\textsc{prs.1sg} \textsc{prs.1sg} still {}  \textsc{comp} \textsc{1pl} be.\textsc{impf.1pl} go.\textsc{ptcp.m.pl} in restaurant.\textsc{f.sg} out.\textsc{def.art.m.sg} \textsc{pn} and  know.\textsc{prs.1sg} \textsc{prs.1sg} \textsc{neg} upon what \textsc{1pl} be.\textsc{impf.1pl} come.\textsc{ptcp.m.pl}\\
\glt `Once here, once there, and once I still remember ... that we had gone to Mirer's restaurant, and I don't know what we conceived [...].'
\z

\ea
\label{ex:coord2}
\langinfo{Tuatschín}{Surajn}{f5, l. }\\
\gll    Mintgataun mavan nuṣ èra … plas pitgògnas á cavá cristalas anstagl mirá dlas tgauras, \textbf{pi} vignévan nus halt in téc tart. \\
sometimes go.\textsc{impf.1pl} \textsc{1pl} also {} in.\textsc{def.art.f.pl} steep\_slope.\textsc{pl} \textsc{purp} dig.\textsc{inf} crystal.\textsc{f.pl} instead look\_for.\textsc{inf} of.\textsc{def.art.f.pl} goat.\textsc{pl} and come.\textsc{impf.1pl} \textsc{1pl} simply \textsc{indef.art.m.sg} bit late \\
\glt `From time to time we would also … go farther up to extract crystals instead of looking after the goats, and then we would come back a bit late.'
\z


\ea
\label{ex:coord3}
\langinfo{Tuatschín}{Ruèras}{m3, l. }\\
\gll Ju èr’ in gjuvanòtar, sùn maj staus fumégl \textbf{ábar} ins mav’ á gidá òra [...].\\
textsc{1sg} \textsc{cop.impf.1sg} \textsc{indef.art.m.sg} youngster be.\textsc{prs.1sg} never \textsc{cop.ptcp.m.sg} farmhand but \textsc{gnr} go.\textsc{impf.3sg} \textsc{comp} help.\textsc{inf} out \\
\glt `I was a youngster, I never was a farmhand but we would go and help out [...]'
\z

\ea
\label{ex:coord4}
\langinfo{Tuatschín}{Ruèras}{\DRG{9}{452}}\\
\gll Basta, da fá fùssi ùssa gl atún la pjal plajna da tùtas uisas. \textbf{Dantaun} scha l' aura tégn ansjaman vòi schòn.\\
enough \textsc{comp} do.\textsc{inf} be\textsc{.cond.3sg.expl} now \textsc{def.art.m.sg} autumn \textsc{def.art.f.sg} skin full of all.\textsc{f.pl} way.\textsc{pl} however if \textsc{def.art.f.sg} weather hold.\textsc{prs.3sg} together go.\textsc{prs.3sg.expl} \\
\glt `Well, now during autumn there would be a lot of things to do. However, if the weather holds, it will do indeed.'
\z

\medskip
Alys: das ist die dt. Übersetzung: Nun gut, jetzt im Herbst, hätte man die Hände voll zu tun mit allerlei Arbeiten.Aber wenn das Wetter anhält, geht es schon.'

\medskip

In the case of coordination with \textit{né} `or', the complementiser \textit{tga} must be used as in (\ref{ex:coord5}).

\ea
\label{ex:coord5}
\langinfo{Tuatschín}{Sadrún}{m9, l. }\\
\gll A né grad è gl unviarn cunzún mavanṣ bjè a bagagjavan sprungṣ a dèvan cun skis né cun ajssa \textbf{né} \textbf{tg}’ ins mava cun bòb da vias gjù [...].\\
and right precisely also \textsc{def.art.m.sg} winter especially go.\textsc{impf.1pl.1pl} often and build.\textsc{impf.1pl} jump.\textsc{m.pl} and give.\textsc{impf.1pl} with ski.\textsc{m.pl} or with board.\textsc{f.sg} right \textsc{comp} \textsc{gnr} go.\textsc{impf.3sg} with bob.\textsc{m.sg} from street.\textsc{f.pl} down\\
\glt `And, right?, especially during winter we often went and built ski jumps and would go skiing or snowboarding, or we would go down the streets on bobsleigh [...].'
\z


\section{Subordination}

\subsection{Subject clauses}
Subject clauses are either finite or non-finite. If the non-finite subject clause is located at the beginning of the sentence, the infinitive can be modified by the definite masculine article (\ref{ex:subjclause1} and \ref{ex:subjclause2}) or not (\ref{ex:subjclause3} or \ref{ex:subjclause4}).

\ea
\label{ex:subjclause1}
\langinfo{Tuatschín}{Zarcúns}{m2, l. 1634ff.}\\
	\gll    Òh \textbf{gl} \textbf{ampréndar} \textbf{tudèstg} è stau, l’ antschata ṣè quaj schòn stau in téc curiùs.\\
oh \textsc{def.art.m.sg} learn.\textsc{inf} German.\textsc{m.sg} be.\textsc{prs.3sg}  \textsc{cop.ptcp.unm} \textsc{def.art.f.sg} beginning be.\textsc{prs.3sg} \textsc{dem.unm} indeed \textsc{cop.ptcp.unm} \textsc{indef.art.m.sg} bit strange.\textsc{adj.unm}\\
\glt `Oh, to learn German was, at the beginning this was indeed a little bit strange.'
\z

\ea
\label{ex:subjclause2}
\langinfo{Tuatschín}{Ruèras}{m3, l. 2140f.}\\
\gll [...] ábar \textbf{al} \textbf{trá} \textbf{ajn} \textbf{èla} èra … da maz.   \\
{} but \textsc{def.art.m.sg} pull.\textsc{inf} in \textsc{3sg.f} \textsc{cop.impf.3sg} {} of killing.\textsc{m.sg}\\
\glt `[...] but bringing it in was ... terrible.'
\z

\ea
\label{ex:subjclause3}
\langinfo{Tuatschín}{}{\DRG{}{}}\\
\gll \textbf{Fugí} è bian, a \textbf{mitschè} è aun mégljar. \\
   flee.\textsc{inf} \textsc{cop.prs.3sg} good and escape.\textsc{inf} \textsc{cop.prs.3sg} still better  \\
\glt `To flee is good, and to escape is even better.'
\z

\ea
\label{ex:subjclause4}
\langinfo{Tuatschín}{Sadrún}{m6, l. 1379ff.}\\
\gll    A \textbf{dumagnè} als amprèmṣ dis quèls pòrs gjù da da quaj trùtg ajnagjù a sjantar atrás l’ aua dal Drun tga vagnéva mù pauc vi da tschèla vart, \textbf{qu}’ è stau álṣò in martéri.\\
and cope.\textsc{inf} \textsc{def.art.m.sg} first.\textsc{pl}  day.\textsc{pl} \textsc{dem.m.pl}  pig.\textsc{pl} down from from \textsc{dem.m.sg} path into\_and\_down and after through  \textsc{def.art.f.sg} water of.\textsc{def.art.m.sg}  \textsc{rn} \textsc{rel} come.\textsc{impf.3sg} only little over of \textsc{dem.f.sg} side \textsc{dem.unm.} be.\textsc{prs.3sg} \textsc{cop.ptcp.unm} well \textsc{indef.art.m.sg} ordeal \\
\glt `And the first days, to cope with these pigs [going] down this path and then trough the water of the Drun, of which only few would go over to the other side, well, this was an ordeal [...].'
\z

It is also possible to use the complementiser \textit{da} (\ref{ex:subjclause7}), but in this case the referent of the infinitive is highlighted.

\ea
\label{ex:subjclause7}
\langinfo{Tuatschín}{Ruèras}{m10}\\
\gll \textbf{Da} \textbf{còschar} fùs stau mégljar .\\
\textsc{comp} keep\_silent.\textsc{inf} be.\textsc{cond.3sg} \textsc{cop.ptcp.unm} better \\
\glt `To keep silent would have been better.'
\z

If the subject clause is located after the verb, the expletive pronoun \textit{i} is required in subject position and the infinitive may be headed by the preposition \textit{da} (\ref{ex:subjclause5}) or not (\ref{ex:subjclause6}).

\ea
\label{ex:subjclause5}
\langinfo{Tuatschín}{Sadrún}{m5}\\
\gll I è stau mégljar \textbf{da} \textbf{còschar}.\\
\textsc{expl} be.\textsc{prs.3sg} \textsc{cop.ptcp.unm} better \textsc{comp} keep\_silent.\textsc{inf}\\
\glt `It was better to keep silent.'
\z

\ea
\label{ex:subjclause6}
\langinfo{Tuatschín}{Sadrún}{m9, l. 1831}\\
	\gll [...] \textbf{i} è gréjv \textbf{dí} [...].\\
 {} \textsc{expl} \textsc{cop.prs.3sg} difficult.\textsc{adj.unm} say.\textsc{inf}\\
\glt `[...] it is difficult to say [...].'
\z

Finite object clauses are headed by the complementiser \textit{tga}; the expletive pronoun \textit{i} must occur in subject position (\ref{ex:subjclause8}).

\ea
\label{ex:subjclause8}
\langinfo{Tuatschín}{Zarcúns}{m2, l. }\\
\gll    Ad i è aun schabagjau plé … \textbf{tga} nuṣ èssan i … tiar duas, álṣò ajn duaṣ jèdas … tiar ina api sjántar aun í ad í vi tiar tschèla.\\
and \textsc{expl} be.\textsc{prs.3sg} in\_addition happen.\textsc{ptcp.unm} more {} \textsc{comp}  \textsc{1pl} be.\textsc{prs.1pl} go.\textsc{ptcp.m.pl} {} to two.\textsc{f.pl} well in two.\textsc{f.pl} time.\textsc{pl} {} to \textsc{one.f.sg} and afterwars in\_addition go.\textsc{ptcp.m.pl} and  go.\textsc{ptcp.m.pl} over to \textsc{dem.f.sg}\\
\glt `And it also happened more … that we went … to two, well at two different moments … to the one and went also, and went over to the other.'
\z


\subsection{Object clauses}
Object clauses can be finite or non-finite. Infinitive clauses are headed by the complementiser \textit{da} (\ref{ex:objcl1} and \ref{ex:objcl2}).

\ea
\label{ex:objcl1}
\langinfo{Tuatschín}{Surajn}{\citealt[128]{Büchli1966}}\\
\gll Al pástar gròn ò \textbf{admonju} \textbf{da} bétga \textbf{fá} tupadats.\\
\textsc{def.art.m.sg} herdsman big have.\textsc{prs.3sg} admonish.\textsc{ptcp.unm} \textsc{comp} \textsc{neg} do.\textsc{inf} stupidity.\textsc{f.pl}\\
\glt `The main herdsman admonished not to commit stupidities.'
\z

\ea
\label{ex:objcl2}
\langinfo{Tuatschín}{Ruèras}{m10, l. 1171ff.}\\
\gll  [...] api vagi èl \textbf{tartgau} ... \textbf{dad} \textbf{í} \textbf{vi} ajn via ... a tanaj sé èls.  \\
{} and have.\textsc{prs.sbjv.3sg} \textsc{3sg.m} think.\textsc{ptcp.unm} {} \textsc{comp} go.\textsc{inf} over on road.\textsc{f.sg} {} and hold.\textsc{inf} up \textsc{3pl.m}\\
\glt `[...] and he thought... that he would go on the road ... and stop them.'
\z

Finite object clauses are headed by the complementiser \textit{tga} `that' (\ref{ex:objcl13}) or \textit{scha} `whether' (\ref{ex:objcscha1}). The clause may be separated from the main verb by a temporal clause as in (\ref{ex:sacutga}).

\ea
\label{ex:objcl13}
\langinfo{Tuatschín}{Zarcúns}{m2, l. 1630fff.}\\
\gll    In’ jèda tschò, in’ jèda lò, ad in’ jèda saragòrd’ ju aun … \textbf{tga} nuṣ èran í ajn ustria òl Mírar [...].\\
one.\textsc{f.sg} time here one.\textsc{f.sg} time there and one.\textsc{f.sg} time \textsc{refl}.remember.\textsc{prs.1sg} \textsc{prs.1sg} still {} \textsc{comp} \textsc{1pl} be.\textsc{impf.1pl} go.\textsc{ptcp.m.pl} in restaurant.\textsc{f.sg} out.\textsc{def.art.m.sg} \textsc{pn}\\
\glt `Once here, once there, and once I still remember ... that we had gone to Mirer's restaurant [...].'
\z

\ea
\label{ex:sacutga}
\langinfo{Tuatschín}{Zarcúns}{m2, l. 1630fff.}\\
\gll    Ju \textbf{sa} cu cu … cu ju sùn maridaus \textbf{tga} … èri ajn Camischùlas circa quindiṣch ufauns … tga mavan á scùla da Camischùlas.\\
\textsc{1sg} know.\textsc{prs.1sg} when when {} when \textsc{1g} be.\textsc{prs.1sg}  marry.\textsc{ptcp.m.sg} \textsc{comp} {} \textsc{exist.impf.3sg.expl} in \textsc{pln} about fifteen child.\textsc{m.pl} {} \textsc{rel} go.\textsc{impf.3pl} to school.\textsc{f.sg} of  \textsc{pln}\\
\glt `I know that when when … when I got married … there were in Camischolas about fifteen … children who attended the school of Camischolas.'
\z

\ea
\label{ex:objcscha1}
\langinfo{Tuatschín}{Ruèras}{m10, l. }\\
\gll   [...] a … lu vajn nuṣ, va ju dumandau \textbf{sch}’ èl prèndèssi mè tòcan … á Ruèras. \\
{} and {} then have.\textsc{prs.1pl} \textsc{1pl} have.\textsc{prs.1sg}  \textsc{1sg} ask.\textsc{ptcp.unm} if \textsc{3sg.m} take.\textsc{cond.indir.3sg} \textsc{1sg} until {} to  \textsc{pln}\\
\glt `[...] and … then we, I asked whether he could take me down to Rueras.'
\z


\subsection{Temporal clauses}
Temporal clauses are either finite or non-finite. Non-finite temporal clauses are headed by \textit{da} (\ref{ex:nonfintempcl1}) or \textit{avaun ca} (\ref{ex:nonfintempcl2} and \ref{ex:nonfintempcl3}).

\ea
\label{ex:nonfintempcl1}
\langinfo{Tuatschín}{Cavòrgja}{\citealt[120]{Büchli1966}}\\
\gll    \textbf{Da} \textbf{vagní} \textbf{anavùs} vònd ju plaunsju!\\
\textsc{comp} come.\textsc{inf} back go.\textsc{prs.1sg} \textsc{1sg} slowly \\
\glt `When I come back, I’ll walk slowly!'
\z

\ea
\label{ex:nonfintempcl2}
\langinfo{Tuatschín}{Tschamùt} {\citealt[17]{Büchli1966}}\\
\gll  [...]  sùnd ju ida a tgèsa, \textbf{avòn}\footnotemark {} \textbf{ca} \textbf{tga} vagní stgir.\\
{} be.\textsc{prs.1sg} \textsc{1sg} go.\textsc{ptcp.f.sg} to house before \textsc{comp} \textsc{comp} get.\textsc{prs.sbjv.3sg} dark \\
\glt `[…] I went home before it got dark.'\footnotetext{The form \textit{avòn} is not a standard Sursilvan form, but belongs to the dialect of Tschamut and Selva.}
\z

\ea
\label{ex:nonfintempcl3}
\langinfo{Tuatschín}{Sadrún}{m4, l. 471f.}\\
\gll  [...] \textbf{avaun} \textbf{c}’ \textbf{í} \textbf{vidajn}, ah, staus lu tial miadi [...]. \\
{} before \textsc{comp} go.\textsc{inf} uphill eh \textsc{cop.ptcp.m.sg} then  at.\textsc{def.art.m.sg} doctor\\
\glt `[...] before going uphill, he went to the doctor [...].'
\z

Finite temporal clauses are headed by \textit{avaun ca tga} `before', \textit{cu} `when', \textit{dafartáuntiar tga} `whilst', \textit{durònt tga} `when', \textit{schi glajti scù} `as soon as', and \textit{tòca} `until'.

With the complementiser \textit{cu} (derived from the interrogative pronoun \textit{cura} 'when') correlative textit{scha} `if, then' is usually used (\ref{ex:cuscha1} and \ref{ex:cuscha2}).

\ea
\label{ex:cuscha1}
\langinfo{Tuatschín}{}{\citealt[23]{Berther2007}}\\
\gll  […] \textbf{cu} i dat la bènèdiczjun, \textbf{scha} fò la sòntga crusch.\\
    […] when \textsc{expl} give.\textsc{prs.3sg} \textsc{def.art.f.sg} blessing \textsc{corr} make.\textsc{imp.2sg} \textsc{def.art.f.sg} holy cross\\
\glt `[…] when it comes to the blessing, make the sign of the cross.'
\z

\ea
\label{ex:cuscha2}
\langinfo{Tuatschín}{Ruèras}{m3, l. 2206f.}\\
	\gll  A pér \textbf{cu} quaj èra fatg, \textbf{scha} èra la scòtga mèmi tgauda [...].  \\
and only when \textsc{dem.unm} \textsc{pass.aux.impf.3sg} do.\textsc{ptcp.unm} \textsc{corr} \textsc{cop.impf.3sg} \textsc{def.art.f.sg} whey too hot\\
\glt `And only when this was done, the whey was too hot [...].'
\z

Instead of \textit{cu} `when' one also finds \textit{tga}, which in combination with \textit{ju} `I' is realised as \textit{tgu} (\ref{ex:tempcltgu1} - \ref{ex:tempcltgu3}).

\ea
\label{ex:tempcltgu1}
\langinfo{Tuatschín}{Ruèras}{\citealt[8]{Valär2013b}}\\
\gll    […] faː ajn […] ʦa’kɔnʣ diants aʎ rǝʃ’tiː dɐ lɐ dʊːnɐ \textbf{ʨ} ɐl ɔ rut ɔːr eir dɐ mɐ’ʦaː in ruʃp\\
[…] make.\textsc{inf} in […] some tooth.\textsc{m.pl} {dat} rake of \textsc{def.art.f.sg} woman \textsc{rel} \textsc{3sg} have.\textsc{prs.3sg} break.\textsc{ptcp.unm} out also \textsc{comp} kill.\textsc{inf} \textsc{indef.art.m.sg} toad\\
\glt `[…] put in some teeth to the woman’s rake he had broken when he killed a toad.'
\z

\ea
\label{ex:tempcltgu2}
\langinfo{Tuatschín}{Ruèras}{m1, l. 280f.}\\
\gll    A sjantar [...] ṣè `l bap lu mòrts \textbf{tgu} vèva…. mù vèntgadúṣ òns.\\
and after {} be.\textsc{prs.3sg} \textsc{def.art.m.sg} father then die.\textsc{ptcp.m.sg} \textsc{rel.1sg} have.\textsc{impf.3sg} only twenty-two year.\textsc{m.pl}\\
\glt `And after that, yes, I had … my father then died when I was … only 22 years old.' (temporal argument)
\z

\ea
\label{ex:tempcltgu3}
\langinfo{Tuatschín}{Sadrún}{m4, l. 562f.}\\
\gll  Ò lò vòu fòrza schòn è survagnú in téc quajda d' í par crapa, \textbf{tgu} a vju difarènts lògans tg’ i vèvan sitau gjù ad èra… vagnú ò cristalaṣ [...]. \\
down there  have.\textsc{prs.1sg.1sg} maybe really also get.\textsc{ptcp.m.unm} \textsc{indef.art.m.sg} bit desire.\textsc{f.sg} \textsc{comp} go.\textsc{inf} for stone.\textsc{coll} \textsc{rel.1sg} have.\textsc{prs.1sg} see.\textsc{ptcp.m.unm} different.\textsc{m.pl} place.\textsc{pl} \textsc{rel} \textsc{3pl} have.\textsc{impf.3pl} blast.\textsc{ptcp.m.unm} down and  be.\textsc{impf.3sg} come.\textsc{ptcp.m.unm} out crystal.\textsc{f.pl}\\
\glt `Out there I might have started enjoying a bit going for stones, when I saw different places where they had blasted [the rocks], and crystals [...] had come out [...].'
\z

Examples (\ref{ex:tempcl1} - \ref{ex:tempcl4}) illustrate the other complementisers that head a temporal clause.

\ea
\label{ex:tempcl1}
\langinfo{Tuatschín}{Sadrún}{m5}\\
\gll \textbf{Dafartáuntiar} \textbf{tga} la mùma fò quaj, miras té dal pòp.\\
whilst \textsc{rel} \textsc{def.art.f.sg} mother do.\textsc{prs.3sg} \textsc{dem.unm} look.\textsc{prs.2sg} \textsc{2sg} of.\textsc{def.art.m.sg} baby\\
\glt `Whilst mother is doing this, you look after the baby.'
\z


\ea
\label{ex:tempcl2}
\langinfo{Tuatschín}{Tschamùt} {\citealt[19]{Büchli1966}}\\
\gll    Ju a gju tèma tg’ als tiars réjsdjan \textbf{durònt} \textbf{tga} la sòntga mèssa da Nadal végni lagida.\\
      \textsc{1sg} have.\textsc{prs.1sg} have.\textsc{ptcp.unm} fear \textsc{comp} \textsc{def.art.m.pl} animal.\textsc{pl} talk.\textsc{prs.sbjv.3pl} during \textsc{comp} \textsc{def.art.f.sg} holy mass of Christmas come.\textsc{prs.sbjv.3sg} read.\textsc{ptcp.f.sg}\\
\glt `I feared […] that the animals could talk when the Holy Mass would be read.'
\z

\ea
\label{ex:tempcl3}
\langinfo{Tuatschín}{Ruèras}{m10, l.  1134f.}\\
\gll  A schi … \textbf{schi} \textbf{glajti} \textbf{scù} nuṣ èssan staj sén la via cantunala òni antschiat á galòpá.  \\
and so {} so soon as \textsc{1pl} be.\textsc{prs.1pl} \textsc{cop.ptcp.m.pl} on \textsc{def.art.f.sg} way cantonal have.\textsc{prs.3pl.3pl} begin.\textsc{ptcp.unm} \textsc{comp} gallop.\textsc{inf}\\
\glt `And as … as soon as we were on the cantonal way they started to gallop.'
\z

\ea
\label{ex:tempcl4}
\langinfo{Tuatschín}{Camischùlas}{f6, l. 741f.}\\
\gll    Api èra la sòra òra uschéja … avaun niaṣ ésch ad ò spatgau a spatgau \textbf{tòca} \textbf{la} \textbf{audi} \textbf{anzatgéj} [...].\\
and \textsc{cop.impf.3sg} \textsc{def.art.f.sg} nun out so {} in\_front\_of \textsc{poss.1pl.m.sg} door and have.\textsc{prs.3sg} wait.\textsc{ptcp.unm} and wait.\textsc{ptcp.unm} until \textsc{3sg.f} hear.\textsc{prs.sbjv.3sg} something\\
\glt `And then the nun was out [on the corridor] like this ... in front of our door, waiting and waiting until she would hear something [...]'
\z


\subsection{Manner clauses}
Non-finite manner clauses are headed by \textit{cun} `with' (\ref{ex:manncl1}) and finite manner clauses either by \textit{scù} (\ref{ex:manncl2}), \textit{scù tga} (\ref{ex:manncl3}), or only \textit{tga} (\ref{ex:manncl4}).

\ea
\label{ex:manncl1}
\langinfo{Tuatschín}{Ruèras}{m1, l. 176f.}\\
\gll   «Gè, i fùs schòn flòt, ábar ju sùn è vagnú\footnotemark{} atrás \textbf{cun} \textbf{fá} `\textbf{l} \textbf{pur}.» \\
yes \textsc{expl} \textsc{cop.cond.3sg} really great.\textsc{adj.unm} but \textsc{1sg}  be.\textsc{prs.1sg} also  come.\textsc{ptcp.m.sg} through with  do.\textsc{inf} \textsc{def.art.m.sg} farmer\\
\glt `«Yes, this would be really great, but being a farmer I also could earn a living.»'\footnotetext{\textit{Vagnú} is a performance error for \textit{vagnúṣ}.}
\z

\ea
\label{ex:manncl2}
\langinfo{Tuatschín}{Ruèras}{m1, l. 755f.}\\
\gll    Òz ṣaj al al Furka né al … la ban\footnotemark{} né \textbf{scù} i végn raṣdau òz sén tudèstg.\\
today \textsc{cop.prs.3sg.expl} \textsc{def.art.m.sg} \textsc{def.art.m.sg} \textsc{pln} or \textsc{def.art.m.sg} {} \textsc{def.art.f.sg} train right as \textsc{expl} \textsc{pass.aux.prs.3sg} speak.\textsc{ptcp.unm} today on German.\textsc{m.sg}\\
\glt `Today it is the Furka or the … the \textit{ban} `train', as nowadays it is called in German.' (i.e the Matterhorn-Gotthard-Bahn)\footnotetext{\textit{Ban} is another German word (\textit{Bahn}) for `train'.}
\z

\ea
\label{ex:manncl3}
\langinfo{Tuatschín}{Sadrún}{m9, l. }\\
\gll Nus mavan bjè, gè, á fá gjugs, \textbf{scù} \textbf{tg}’ ins fagèva plé baut [...].   \\
\textsc{1pl} go.\textsc{impf.1pl} often yes \textsc{purp} do.\textsc{inf} game.\textsc{m.pl} as \textsc{comp} \textsc{gnr} do.i\textsc{mpf.3sg} more early\\
\glt `We would often go and play, as one would do it earlier [...].'
\z

\ea
\label{ex:manncl4}
\langinfo{Tuatschín}{Camischùlas}{\DRG{3}{583}}\\
\gll   La bùca stuèv' èssar \textbf{tga} la pudèv' ajn la lata.\\
\textsc{def.art.f.sg} mouth should.\textsc{impf.3sg} \textsc{cop.inf} \textsc{comp} \textsc{3sg.f} can.\textsc{impf.3sg} into \textsc{def.art.f.sg} slat\\
\glt `The cutting should be such that the slat could fit into it.'
\z

\subsection{Purposive clauses}
Non-finite purposive clauses are headed by \textit{a} after a verb of movement (\ref{ex:purp:inf:1} and \ref{ex:purp:inf:2}) and by \textit{par/pr} (\ref{ex:purp:inf:3}) or \textit{da} (\ref{ex:purp:inf:4} and \ref{ex:purp:inf:5}) in all other cases, whereby \textit{da} is very rare. Example (\ref{ex:purp:inf:2}) contains both \textit{par} and \textit{á}.

\ea\label{ex:purp:inf:1}
\langinfo{Tuatschín}{Sèlva}{f2, l.939f.}\\
\gll  Api ṣchèvan aj èba in tg’ è mòrts tga \textbf{végn} \textbf{á} \textbf{mètar} als, als tjarms... la nòtg [...].\\
and say.\textsc{impf.3pl} \textsc{3pl} precisely one \textsc{rel} \textsc{cop.prs.3sg} dead.\textsc{m.sg} \textsc{rel} come.\textsc{prs.3sg} \textsc{purp} put.\textsc{inf} \textsc{def.art.m.pl} \textsc{def.art.m.pl} boundary\_stone.\textsc{pl} \textsc{def.art.f.sg} night \\
\glt `And, precisely, they also used to say [that] somebody who was dead came and put the, the boundary stones ... at night [...].'
\z

\ea
\label{ex:purp:inf:2}
\langinfo{Tuatschín}{Sadrún}{m4, l.416ff.}\\
\gll [...] ju vèva… siṣ òns tga èran vida prapará la via \textbf{par} \textbf{í} sé Nalps \textbf{á} \textbf{bagagè} al mir da farmada.\\
{} \textsc{1sg} have.\textsc{impf.1sg} six year.\textsc{m.pl} \textsc{comp} \textsc{cop.impf.3pl} \textsc{prog} prepare.\textsc{inf} \textsc{def.art.f.sg} road \textsc{purp} go.\textsc{inf} up \textsc{pln} and build.\textsc{inf} \textsc{def.art.m.sg} wall.\textsc{m.sg} of reservoir.\textsc{f.sg} \\ 
\glt `[...] I was six years old when they were preparing the road in order to go to Nalps to build the wall of the reservoir.'
\z

\ea
\label{ex:purp:inf:3}
\langinfo{Tuatschín}{Ruèras}{m11, l.960ff.}\\
\gll Als méls èran, qu’ èra… fèrm… fèrms tiarṣ ad èl duvrava quaj mél pr trá, \textbf{pr} \textbf{trá} lèna sé da Cavòrgja.   \\
\textsc{def.art.m.pl} mule \textsc{cop.impf.3sg} \textsc{dem.unm} \textsc{cop.impf.3sg} strong.\textsc{adj.unm} strong.\textsc{m.pl} animal.\textsc{pl} and \textsc{3sg.m} use.\textsc{impf.3sg} \textsc{dem.m.sg} mule \textsc{purp} pull.\textsc{inf} \textsc{purp} pull.\textsc{inf} wood.\textsc{coll} up from \textsc{pln}  \\
\glt `The mules were, these were in fact …. strong … strong animals and he used that mule for transporting wood up from Cavorgia.'
\z

\ea
\label{ex:purp:inf:4}
\langinfo{Tuatschín}{Bugnaj} {\citealt[143]{Büchli1966}}\\
\gll  [...] méz in pétg sc’ ins drùva par tùt las lavurs da parmavèra a gl’ atún \textbf{da} \textbf{cavá} \textbf{trúfals}.\\
{} put.\textsc{ptcp.unm} \textsc{indef.art.m.sg} hoe like \textsc{impers} need.\textsc{prs.3sg} for all \textsc{def.art.f.pl} work.\textsc{pl} of spring and \textsc{def.art.m.sg} autumn \textsc{purp} dig potato.\textsc{pl}\\
\glt `[…] put a hoe like [the one] one needs for all the work that must be done in spring, and in autumn in order to dig out potatoes.'
\z

\ea
\label{ex:purp:inf:5}
\langinfo{Tuatschín}{Zarcúns}{m2, l. 1579f.}\\
\gll    Api savèv’ ins bigja cù, cù fá \textbf{da} \textbf{purtá} \textbf{las} \textbf{nèglas} [...].\\
and know.\textsc{impf.3sg} \textsc{gnr} \textsc{neg} how how do.\textsc{inf} \textsc{comp} carry.\textsc{inf} \textsc{def.art.f.pl} carnation.\textsc{pl} \\
\glt `And one would not know how to put the carnations [...] (literally `how to do in order to put the carnations').'
\z

There is one example where the complementiser \textit{á} is absent.

\ea
\label{ex:purp:inf:6}
\langinfo{Tuatschín}{Ruèras}{f4, l. 1948}\\
	\gll [...] api èri dad \textbf{í} \longrule {} \textbf{métar} trúfals [...] .\\
{} and be.\textsc{impf.3sg.expl} \textsc{comp} go.\textsc{inf} {}  put potato.\textsc{m.pl}\\
\glt `[...] and then one had to sow potatoes [...].'
\z

In the DRG materials, there is one occurrence of \textit{bétg} `negator' located between \textit{par} `for' and \textit{tga} `complementiser', i.e. outside of the subordinate clause.

\ea
\label{ex:betgtga1}
\langinfo{Tuatschín}{}{\DRG{3}{385}}\\
\gll Vèndardé sògn stù in muantá la tgarn a las ljòngjas \textbf{bétga} \textbf{tga} vignan ajn als baus.\\
Friday.\textsc{m.sg} holy must.\textsc{prs.3sg} \textsc{gnr} move.\textsc{inf} \textsc{def.art.f.sg} meat and \textsc{def.art.f.pl} sausage.\textsc{pl} \textsc{neg} \textsc{comp} come.\textsc{prs.3pl} in \textsc{def.art.m.pl} beetle.\textsc{pl}\\
\glt `On Good Friday one must move the meat and the sausages so that the maggots cannot go into the meat.'
\z

This construction has not been accepted by my informants; however, a similar construction which, in contrast to the Tuatschin example (\ref{ex:betgtga1}) includes the complementiser \textit{par}, can be found in other Romansh varieties like e.g. in the Sutsilvan dialect of Dalin.

\ea\label{ex:betgtga2}
\langinfo{Ṣutsilván}{Dagljégn}{\DRG{4}{607}}\\
\gll  \textbf{Par} \textbf{bétg} \textbf{tg}' in schleschi dat il calger eign in pêr guspas els calzers.\\
     \textsc{purp} \textsc{neg} \textsc{comp} \textsc{gnr} slip.\textsc{prs.sbjv.3sg} give.\textsc{prs.3sg} \textsc{def.art.m.sg} shoemaker in \textsc{indef.art.m.sg} some nail.\textsc{pl} in.\textsc{def.art.m.pl} shoe.\textsc{pl}\\
\glt `In order not to slip, the shoemaker beats some nails into the shoes.'
\z

Therefore it is possible that the Tuatschin construction in (\ref{ex:betgtga1}) belonged to an older variety of the language.


\subsection{Causal clauses}
Non-finite causal clauses are headed by \textit{da} (\ref{ex:caus1}), and finite causal clauses are headed by \textit{parquaj/prquaj tga} (\ref{ex:caus3}), \textit{partgé} (\ref{ex:caus4}), \textit{ma tga} (\ref{ex:caus5}), \textit{tga} (\ref{ex:caus6}), or \textit{cunquaj tga} `since' (\ref{ex:caus7}).
 
 \ea
 \label{ex:caus1}
 \langinfo{Tuatschín}{Ruèras}{\citealt[9]{Valär2013b}}\\
 \gll ‘ʥeːvjǝ zɛ l ‘tɔni dɐ lɐ mɐt’lajnɐ sǝʃfǝrdɐntaws ʃi feʨ \textbf{dɐ} ‘\textbf{bajbǝr} \textbf{trajs} \textbf{mjɔːlas} \textbf{pɛːn} \textbf{frajt}\\
 Thursday \textsc{cop.prs.3sg} \textsc{def.art.m.sg} \textsc{pn} of \textsc{def.art.f.sg} \textsc{pn} \textsc{refl}.catch.cold.\textsc{ptcp.m.sg} so much \textsc{comp} drink.\textsc{inf} three cup.\textsc{pl} buttermilk cold\\
 \glt `Thursday Matlaina’s Toni caught a very strong cold because he drank three cups of cold buttermilk.'
 \z

\ea
\label{ex:caus3}
\langinfo{Tuatschín}{Sadrún}{f3, l. 24ff.}\\ 
\gll  [...] api lura va ju in’ jèda talafònau dad èl \textbf{prquaj} \textbf{tg}' èl vèva tarmèz in’ anunzja da mòrt [...].\\
then have.\textsc{1sg}  \textsc{1sg} one.\textsc{f.sg} time call.\textsc{ptcp.unm} \textsc{dat} \textsc{3sg.m} because \textsc{comp} \textsc{3sg.m} have.\textsc{impf.3sg} send.\textsc{ptcp.unm} \textsc{indef.art.f.sg} announcement of death.\textsc{f.sg}\\ 
\glt `[...] then I phoned him once, because I should send a death notice [...].'
\z

\ea
\label{ex:caus4}
\langinfo{Tuatschín}{Zarcúns}{m2, l. }\\ 
\gll    Avaun ina fjasta mavan aj … tialas … gjufnas … par nègla, \textbf{partgé} matévan sé sé la tgapjala … ina nègla.\\
before \textsc{indef.art.f.sg} celebration go.\textsc{impf.3pl} \textsc{3pl} {} to.\textsc{def.art.f.pl} {}  young\_woman.\textsc{pl} {} for carnation.\textsc{f.pl} beacusee put.\textsc{impf.3pl} up up  \textsc{def.art.f.sg} hat {} \textsc{indef.art.f.sg} carnation \\
\glt `Before a celebration they would go … to the … girls for carnations, because they would put … a carnation on their hat.'
\z

\ea
\label{ex:caus5}
\langinfo{Tuatschín}{Cavòrgja}{m7, l. }\\ 
\gll «Cò ò la mùma dau ina sèrvla, \textbf{ma} \textbf{tg}’ i è damaun duméngja».\\
here have.\textsc{prs.3sg} \textsc{def.art.f.sg} mother give.\textsc{ptcp.unm} \textsc{indef.art.f.sg} cervelat because \textsc{comp} \textsc{expl} \textsc{cop.prs.3sg} tomorrow Sunday\\
\glt `«Here, mother provided a cervelat, because tomorrow is Sunday.»'
\z

\ea
\label{ex:caus6}
\langinfo{Tuatschín}{}{\DRG{3}{719}}\\
\gll Ùssa léjva \textbf{tg'} i è clar dé.\\
  now get\_up.\textsc{imp.2sg} \textsc{comp} \textsc{expl} \textsc{cop.prs.3sg} clear day\\
\glt `Get up now since day has already broken.'
\z

\ea
\label{ex:caus7}
\langinfo{Tuatschín}{Ruèras}{m10, l. }\\ 
\gll [...] \textbf{cunquaj} \textbf{tg}' èl èra … fòrsa staus ah da malitèr tials tgavals, né gju da fá cun tgavals, scha… vèv’ èl … cumprau in asan, álṣò in mél, bitg in asan, in mél.\\
{} since \textsc{comp} \textsc{3sg.m} be.\textsc{impf.3sg} {} maybe \textsc{cop.ptcp.m.sg} ah  of army.\textsc{m.sg} at.\textsc{def.art.m.pl} horse.\textsc{pl} or have.\textsc{ptcp.unm}  \textsc{comp} do.\textsc{inf} with horse.\textsc{m.pl} then have.\textsc{impf.3sg} \textsc{3sg.m} {}  buy.\textsc{ptcp.unm}  \textsc{indef.art.m.sg} donkey this\_is \textsc{indef.art.m.sg} mule \textsc{neg} \textsc{indef.art.m.sg} donkey \textsc{indef.art.m.sg} mule \\
\glt `[...] since in the army he had maybe been with the horses, or had to do with horses, then he had bought a donkey, that is to say a mule, not a donkey, a mule.'
\z

\subsection{Conditional clauses}
Conditional clauses are formed in three different ways:

\begin{itemize}
	\item (1) a correlative construction with \textit{scha} in both the protasis and the apodosis (\ref{ex:cond1} and \ref{ex:cond2}), 
	\item (2) only the protasis is headed by \textit{scha} 'if' (\ref{ex:cond3} - \ref{ex:cond5}),
	\item (3) without subordinator in the protasis but with subject inversion and \textit{scha} or \textit{lura} in the apodosis (\ref{ex:cond6}).
\end{itemize}


In all these cases there is subject inversion in the apodosis, with some rare exceptions. Furthermore, correlative \textit{lura} is very rare in the corpus; it only occurs in \citet{Büchli1966}.

\ea
\label{ex:cond1}
\langinfo{Tuatschín}{}{\citealt[120]{Berther1998}}\\
\gll \textbf{Scha} té as lu mèmja bjè da raclamá a grí, \textbf{scha} matajn nus té ajnagjù ‘l Run.\\
if \textsc{2sg} have.\textsc{prs.2sg} then too much to complain and shout then put.\textsc{prs.1pl} \textsc{1pl} \textsc{2sg} in.down \textsc{def.art.m.sg} \textsc{rn}\\
\glt ` If you really have so much to complain and to shout, we will throw you down into the Run [river].'
\z

\ea
\label{ex:cond2}
\langinfo{Tuatschín}{Camischùlas}{f6, l. }\\ 
\gll    A da gjantá … \textbf{sch}’ ina sòra … tudèstga èra … vida majṣa, \textbf{scha} stuèvan tùt quèla- nuṣ ròmòntschas raṣdá tudèstg.\\
and of lunch.\textsc{inf} {} if \textsc{indef.art.f.sg} nun {} German \textsc{cop.impf.3sg} {} at\_of table.\textsc{f.sg} then must.\textsc{impf.3pl} all \textsc{dem.f.sg} \textsc{1pl} Romansh.\textsc{f.pl} speak.\textsc{inf} German.\textsc{m.sg}\\
\glt `And during lunch … if a German … nun was … at table, all these – we, the Romansh speaking people, had to speak German.'
\z

\ea
\label{ex:cond3}
\langinfo{Tuatschín}{}{\citealt[60]{Berther1998}}\\
\gll \textbf{Scha} ju antupás quèla gljut, \textbf{sabatès} \textbf{ju} gjù ajn ganugljas a bitscháṣ als cazès.\\
if \textsc{1sg} meet.\textsc{cond.1sg} \textsc{dem.f.sg} people \textsc{refl}.throw.\textsc{cond.1sg} \textsc{1sg} down in knee.\textsc{f.pl} and kiss.\textsc{cond.1sg} \textsc{def.art.m.pl} shoe.\textsc{pl}\\
\glt `If I met these people, I’d kneel down and kiss their shoes.'
\z

\ea
\label{ex:cond4}
\langinfo{Tuatschín}{Sadrún}{m6, l. }\\ 
\gll Da mintga pur stèvanṣ í a \textbf{scha} `l vèva gju dus, \textbf{stèva} \textbf{`l} pijè …  dus pòrs né in né uschéa [...].\\
of every farmer.\textsc{m.sg} must.\textsc{impf.1pl.1pl} go.\textsc{inf} and if \textsc{3sg.m} have.\textsc{impf.3sg} have.\textsc{ptcp.unm} two.\textsc{m.pl} must.\textsc{impf.3sg}  \textsc{3sg.m} pay.\textsc{inf} {} two.\textsc{m.pl} pig.\textsc{pl} or one.\textsc{m} or so\\
\glt `We had to go to every farmer and if he had given two [pigs], he should [pay] more … two pigs or one or so [...].'
\z


\ea
\label{ex:cond5}
\langinfo{Tuatschín}{Sadrún}{m4, l. }\\ 
\gll  [...] \textbf{scha} `l vèva bian, \textbf{vagnév}’ \textbf{ins} schòn séssúra inqual tgaussas.\\
{} if \textsc{3sg.m}  have.\textsc{impf.3sg} good.\textsc{adj.unm} come.\textsc{impf.3sg} \textsc{gnr} indeed upon some thing.\textsc{f.pl} \\
\glt `[...] if he was in a good mood, one could get to know some things.'
\z

\ea
\label{ex:cond6}
\langinfo{Tuatschín}{Sèlva}{\citealt[34]{Büchli1966}}\\
\gll    \textbf{Vasèvan} \textbf{ins} ina signura […] cun schuba cuérta, còtschna, […] \textbf{lura} spitgavan als purs ina gronda malaura […].\\
     see.\textsc{impf.3sg.euph} \textsc{gnr} \textsc{indef.art.f.sg} woman [...] with shirt.\textsc{f.sg} short red [...] \textsc{corr} expect.\textsc{impf.3pl} \textsc{def.art.m.pl} farmer.\textsc{pl} \textsc{indef.art.f.sg} big storm\\
\glt `If one saw a woman with a short shirt, a red one, the farmers would expect a heavy storm.'
\z

\subsection{Consecutive clauses}
Concessive clauses are headed by tga.

\ea
\label{}
\langinfo{Tuatschín}{Tschamùt}{\citealt[20]{Büchli1966}}\\
\gll  Èla dètgi in' jèda ina curnada li èl \textbf{tg’} \textbf{èl} \textbf{stètschi} \textbf{sél} \textbf{plaz}.\\
\textsc{3sg.f} give.\textsc{prs.sbjv.3sg} one.\textsc{f.sg} time \textsc{def.art.f.sg} push\_with\_horn.\textsc{ptcp.f.sg} \textsc{dat} \textsc{3sg.m} \textsc{comp} \textsc{3sg.m} stay.\textsc{prs.sbjv.3sg} on.\textsc{def.art.m.sg} place \\
\glt `She [the cow] would give him a push with her horns so that he would remain on the spot.'
\z

\ea\label{}
\langinfo{Tuatschín}{Sadrún}{m6, l. 1397ff.}\\
	\gll    [...] quaj piartg èra juṣ atráṣ a vèva rùt gjù al matg \textbf{tga}  `l vèva mù la còrda plé antùrn.\\
{} \textsc{dem.m.sg} pig be.\textsc{impf.3sg} go.\textsc{ptcp.m.sg} through and have.\textsc{impf.3sg} break.\textsc{ptcp.unm} down \textsc{def.art.m.sg} bunch  \textsc{comp} \textsc{3sg.m} have.\textsc{impf.3sg} only \textsc{def.art.f.sg} rope more around\\
\glt `[...] this pig had gone through and had broken the bunch of flowers so that he only had the rope around him.'
\z

\ea
\label{}
\langinfo{Tuatschín}{Ruèras}{m10, l. 1085f.}\\
\gll  Bjè jèdaṣ ṣèni scapaj \textbf{tga} nuṣ vajn ah pròpi gju ah gròndas mis\underline{é}rjas [...].\\
many time.\textsc{pl} be.\textsc{prs.3pl.3pl} escape.\textsc{ptcp.m.pl} \textsc{comp} \textsc{1pl}   have.\textsc{prs.1pl} ah really have.\textsc{ptcp.unm} ah big.\textsc{f.pl} trouble.\textsc{pl}\\
\glt `They escaped many times so that we had big troubles [...].'
\z

\subsection{Comparative clauses}
Non-finite comparative clauses are headed by \textit{scù da}, literally `how of' or `like of', (\ref{ex:comparative1}), and finite comparative clauses are headed by \textit{scù} (\ref{ex:comparative2}) or \textit{scù tga} (\ref{ex:comparative3}).


\ea
\label{ex:comparative1}
\langinfo{Tuatschín}{}{\DRG{6}{300}}\\
\gll  I vò fil á fil \textbf{scù} \textbf{da} \textbf{caná} in anṣéjl.\\
    \textsc{expl} go.\textsc{prs.3sg} jet to jet \textsc{cpr} \textsc{comp} stab.\textsc{inf} \textsc{indef.art.m.sg} kid \\
\glt `[Blood] flows like when one stabs a kid.'
\z

\ea
\label{ex:comparative2}
\langinfo{Tuatschín}{Sadrún}{f3, l. 24ff.}\\
\gll  [...] lu va ju … tlafònau dad èl a détg, éba, mi' ùm ségi èba mòrts \textbf{scù} \textbf{i} \textbf{sápjan} [...].\\
{} then have.\textsc{prs.1sg} \textsc{1sg} {} call.\textsc{ptcp.unm} \textsc{dat} \textsc{3sg.m} and say.\textsc{ptcp.unm} exactly \textsc{poss.1sg.m.sg} man be.\textsc{prs.sbjv.3sg} precisely die.\textsc{ptcp.m.sg} as \textsc{3pl} know.\textsc{prs.sbjv.3pl}\\ 
\glt `[...] then I … phoned him and said that my husband had died as he knew [...].'
\z

\ea
\label{ex:comparative3}
\langinfo{Tuatschín}{Sadrún}{m9, l. 1784f.}\\	
\gll Nus mavan bjè, gè, á fá gjugs, \textbf{scù} \textbf{tg}’ ins fagèva plé baut … fòrsa da quaj da polizi’ a ládar …  da quaj.   \\
\textsc{1pl} go.\textsc{impf.1pl} often yes \textsc{purp} do.\textsc{inf} game.\textsc{m.pl} as \textsc{comp} \textsc{gnr} do.i\textsc{mpf.3sg} more early {} maybe of \textsc{dem.unm} of police.\textsc{f.sg} and thief.\textsc{m.sg} {} of \textsc{dem.unm}\\
\glt `We would often go and play, as one would do it earlier ... maybe games of police and thief ... things like that.'
\z

\subsection{Concessive clauses}
Consecutive clauses are headed by \textit{schabi tga}, literally `if nice that'.

\ea
\label{}
\langinfo{Tuatschín}{Sadrún}{m4, l. 566ff.}\\
\gll \textbf{Schabi} \textbf{tga} lu, cun siṣ òns capév’ ins lu halt aun mèmja pauc a vèva bigja la... fòrsa\footnotemark{} da fá zatgéj.   \\
although \textsc{comp} then with six year.\textsc{m.pl} understand.\textsc{impf.3sg} \textsc{gnr} then just still too little and have.\textsc{impf.3sg} \textsc{neg} \textsc{def.art.f.sg} strength \textsc{comp} do.\textsc{inf} something\\
\glt `Although then, at the age of six, one would understand too little and wouldn’t have the ... strength to do something.'
\z

\subsection{Indirect interrogative clauses}
Indirect interrogative clauses are characterised by the fact that, in contrast to direct interrogative sentences (see § 5.2), they do not trigger subject inversion (\ref{ex:indinterr1} - \ref{ex:indinterr8}). An exception is a topicalised non-finite manner clause which is left-dislocated and therefore precedes the main verb (\ref{ex:indinterr9}).

\ea
\label{ex:indinterr1}
\langinfo{Tuatschín}{Sadrún} {\citealt[105]{Büchli1966}}\\
\gll    [...] a \textbf{damònda} \textbf{cù} i vòndi.\\
{} and ask.\textsc{prs.3sg} how \textsc{expl} go.\textsc{prs.sbjv.3sg}\\
\glt `[…] and he asked how he was.'
\z

\ea
\label{ex:indinterr2}
\langinfo{Tuatschín}{Ruèras}{m10, l. 1079ff.}\\
\gll   Ad èr' è zatgé bi da \textbf{mirá} \textbf{cù} quèls tiars luvravan, cù quèls… mavan ad èran ruassajvalṣ a... pazjènts.\\
and \textsc{cop.impf.3sg} also something beautiful.\textsc{adj.unm} \textsc{comp} look.\textsc{inf} how \textsc{dem.m.pl} animal.\textsc{pl} work.\textsc{impf.3pl} how \textsc{dem.m.pl} go.\textsc{impf.3sg} and \textsc{cop.impf.3pl} calm.\textsc{m.pl} and patient.\textsc{m.pl}  \\
\glt `Also something nice to look at, how these animals worked, how they … used to go and keep calm and patient.'
\z

\ea
\label{ex:indinterr3}
\langinfo{Tuatschín}{Ruèras}{m1, l. 274ff.}\\
\gll    Las nòtízjas \textbf{sa} ju bétg \textbf{danùndar} als gjaniturs, als dus baps prandèvan aj, i dèva ajnta Ruèras, dèv’ aj in ca vèva r\underline{á}djò. \\
\textsc{def.art.f.pl} news.\textsc{pl} know.\textsc{prs.1sg} \textsc{1sg} \textsc{neg} from\_where \textsc{def.art.m.pl} parents.\textsc{pl} \textsc{def.art.m.pl} two.\textsc{m.pl} father.\textsc{pl} take.\textsc{impf.3pl} \textsc{3sg} \textsc{expl} \textsc{exist.impf.3sg} in \textsc{pln} \textsc{exist.impf.3sg} \textsc{expl}  one.\textsc{m.sg} \textsc{rel} have.\textsc{impf.3sg} radio.\textsc{m.sg}\\
\glt `I don’t know where my parents had the news from, the two fathers took them, there was in Rueras, there was [only] one who had a radio.'
\z

\ea
\label{ex:indinterr4}
\langinfo{Tuatschín}{Sadrún}{m4, l. 432ff.}\\
\gll [...] al \textsc{pn} ò è fagj lò in pèr placats tga \textbf{mùssan} ajn via \textbf{nùc}’ ins sa è mirá quaj.\\
{} \textsc{def.art.m.sg} \textsc{pn} have.\textsc{prs.3sg} also make.\textsc{ptcp.unm} there \textsc{indef.art.m.sg} pair poster.\textsc{m.pl} \textsc{rel} show.\textsc{prs.3pl} in way where \textsc{gnr} can.\textsc{prs.3sg} also see.\textsc{inf} \textsc{dem.unm} \\
\glt `[...] \textsc{pn} has also put there some posters which show on the way where one can have a look at this.'
\z

\ea
\label{ex:indinterr5}
\langinfo{Tuatschín}{Ruèras}{m10, l. 1034ff.}\\
\gll A lu vajn nus, quaj èra tùt fatg a racògnòszau avaun tg’ ins \textbf{savèva} \textbf{núa} inṣ vèva da durmí, \textbf{nu} i èra … da mètar ah ṣur nòtg als als méls, \textbf{nu} i dèva pával pls méls [...]. \\
and then have.\textsc{prs.1pl} \textsc{1pl} \textsc{dem.unm} \textsc{pass.aux.impf.3sg} all do.\textsc{ptcp.unm} and  reconnoitre.\textsc{ptcp.unm} before \textsc{comp} \textsc{gnr}  know.\textsc{impf.3sg} where \textsc{gnr} have.\textsc{impf.3sg} \textsc{comp} sleep.\textsc{inf} where \textsc{expl} be.\textsc{impf.3sg} {} \textsc{comp} put.\textsc{inf} ah over night.\textsc{f.sg} \textsc{def.art.m.pl} \textsc{def.art.m.pl} mule.\textsc{pl} where \textsc{expl} \textsc{exist.impf.3sg} food.\textsc{m.sg} for.\textsc{def.art.m.pl} mule.\textsc{pl}\\
\glt `And then we have, this had all been done and reconnoitred before, so that one knew where to sleep, where to put the mules over night, where there was food for the mules[...].'
\z

\ea\label{ex:indinterr6}
\langinfo{Tuatschín}{Sadrún} {\citealt[103]{Büchli1966}}\\
\gll    El ò \textbf{dumandau} èlas \textbf{partgéj} èlas ségian bétg idas á mèssa.\\
\textsc{3sg.m} have.\textsc{prs.3sg} ask.\textsc{ptcp.unm} \textsc{3pl.f} why  \textsc{3pl.f} be.\textsc{prs.sbjv.3pl} \textsc{neg} go.\textsc{ptcp.3pl.f} to mass\\
\glt `He asked them why they didn’t go to mass.'
\z

\ea
\label{ex:indinterr7}
\langinfo{Tuatschín}{Sadrún}{m9, l. 1810}\\
\gll [...] a lu \textbf{sa} ju schòn in téc \textbf{sc}’ \textbf{i} \textbf{funczjanava}.   \\
{} and then know.\textsc{prs.1sg} \textsc{1sg} indeed \textsc{indef.art.m.sg} bit how \textsc{expl} function.\textsc{impf.3sg}\\
\glt `[...] and therefore I know a bit how it used to function.'
\z

\ea
\label{ex:indinterr8}
\langinfo{Tuatschín}{Bugnaj} {\citealt[134]{Büchli1966}}\\
\gll [...] a lu ò `l \textbf{grju} li gljut [...] tga ségi trajs rùsnas; ajn \textbf{tgénina} èl dégi mètar ajn la crusch.\\
{} and then have.\textsc{prs.3sg} \textsc{3sg.m} shout.\textsc{ptcp.unm} \textsc{dat.sg} people.\textsc{f.sg} {} \textsc{comp} \textsc{exist.prs.sbjv.3sg.expl} three hole.\textsc{f.pl} into which.\textsc{f.sg} \textsc{3sg.m} must.\textsc{prs.sbjv.3sg} put into \textsc{def.art.f.sg} cross\\
\glt `[...] and then he shouted to the people [...] [saying that] there were three holes; [asking] into which he should put the cross.'
\z

\ea
\label{ex:indinterr9}
\langinfo{Tuatschín}{Cavòrgja}{m7, l.2184f.}\\
\gll [...] \textbf{cù} \textbf{barsá} stù ju bigja \textbf{dí} [...].»\\
{} how roast.\textsc{inf} must.\textsc{prs.1sg} \textsc{1sg} \textsc{neg} say.\textsc{inf}\\
\glt `[...] how to roast [it] I don't have to tell [you] [...].'
\z

A special case is (\ref{ex:indinterr10}) in which the manner clause headed by \textit{scù} modifies the manner adverb \textit{usché} `like, as'.

\ea
\label{ex:indinterr10}
\langinfo{Tuatschín}{Sadrún}{m9, l. 1755ff.}\\
\gll [...] in téc da la gjuvantétgna sa ju schòn \textbf{raquintá} … \textbf{usché} \textbf{scù} i mava da nòs tjams a \textbf{tgé} \textbf{ca} va ùsa [...].\\
{} \textsc{indef.art.m.sg} bit of \textsc{def.art.f.sg} youth can.\textsc{prs.1sg} \textsc{1sg} indeed tell.\textsc{inf} {} so as \textsc{expl} go.\textsc{impf.3sg} of \textsc{poss.1pl.m.sg} time.\textsc{pl} and what \textsc{rel} go.\textsc{prs.3sg} now \\
\glt `[...] a bit of my youth I can indeed tell [you about] ... the way it was when we were young and the way it is nowadays [...].'
\z


\subsection{Instead of}
`Instead of' only occurs as non-finite clauses in the corpus. They are headed either by \textit{anstagl} (\ref{ex:anstagl1}) or \textit{anstagl da} (\ref{ex:anstagl2}).

\ea
\label{ex:anstagl1}
\langinfo{Tuatschín}{Zarcúns}{m2, l. 1579f.}\\
\gll    Api \textbf{anstagl} \textbf{bájbar} \textbf{al} \textbf{vin} … èran nus lu i sé, vèvani fatg ina bòla.\\
and instead drink.\textsc{inf} \textsc{def.art.m.sg} wine {} be.\textsc{impf.1pl} \textsc{1sg} then go.\textsc{ptcp.m.pl} up have.\textsc{impf.3pl.3pl} do.\textsc{ptcp.unm} \textsc{indef.art.m.sg} punch \\
\glt `And instead of drinking the wine … we went up, they had prepared a punch.'
\z


\ea
\label{ex:anstagl2}
\langinfo{Tuatschín}{Ruèras}{m3, l. 2206f.}\\
	\gll [...] \textbf{anstagl} \textbf{da} \textbf{mùngjar} òtgònta vacas èri fòrsa mù tschuncònta [...].\\
 {} instead of milk.\textsc{inf} eighty cow.\textsc{f.pl} \textsc{exist.impf.3sg.expl} maybe only fifty\\
\glt `[...]  instead of milking eighty cows there were maybe fifty [...].'
\z



\section{Focus}
If verbs, participles, particles belonging to verbs, certain adverbs, or clauses which give new information are focussed on, they are moved to the beginning of the sentence or clause.

In case a verb is focussed on, it is moved in its infinitive form and the finite form is left behind in the background clause (\ref{ex:focverb1} and \ref{ex:focverb2}).

\ea
\label{ex:focverb1}
\langinfo{Tuatschín}{Sadrún}{\citealt[106]{Büchli1966}}\\
\gll  Ju a cò in bagljèt tòcan gjù Turitg, ábar \textbf{ira} \textbf{vònd} \textbf{ju} mù gjù Sumvitg.\\
\textsc{1sg} have.\textsc{3sg} here \textsc{indef.art.m.sg} ticket until down \textsc{pln} but go.\textsc{inf} go.\textsc{prs}.\textsc{1sg} \textsc{1sg} only down \textsc{pln}\\
\glt `I have here a ticket to Zurich, but I only go till Sumvitg.'
\z

\ea
\label{ex:focverb2}
\langinfo{Tuatschín}{Surajn}{f5, l. 1319}\\
\gll Na na, a \textbf{durmí} \textbf{durmévan} nus cò.\\
no no and sleep.\textsc{inf} sleep.\textsc{impf.1pl} \textsc{1pl} here \\
\glt `No, no, and as for sleeping, we would sleep here.'
\z

In the perfect tense, participles are moved without their auxiliary verb, which remains in the background clause (\ref{ex:focptcp1} and \ref{ex:focptcp2}). Example (\ref{ex:focptcp3}) shows that the participle can be moved with its complements.

\ea
\label{ex:focptcp1}
\langinfo{Tuatschín}{Cavòrgja}{\citealt[106]{Büchli1966}}\\
\gll Ju sùn dada gjù séla fatscha, mù \textbf{fatg} \textbf{òi} nuét.\\
\textsc{1sg} be.\textsc{prs.1sg} give.\textsc{ptcp.f.sg} down on.\textsc{def.art.f.sg} face but do.\textsc{ptcp.unm} have.\textsc{prs.3sg.expl} nothing\\
\glt `I fell down on my face but it didn't do anything.'
\z

\ea
\label{ex:focptcp2}
\langinfo{Tuatschín}{Sadrún}{f3, l. 82f.}\\
\gll  [...] quaj è vagnú da bètòn’ ajn, a \textbf{tanju} ò laṣ aun adina. \\
{} \textsc{dem.unm} be.\textsc{prs.3sg} come.\textsc{ptcp.unm} \textsc{comp} concrete.\textsc{inf} in and hold.\textsc{ptcp.unm} have.\textsc{prs.3sg} \textsc{3pl.f} still always \\
\glt `I would simply take ... bad ... sand, that is to say cement and water with me, and this has been concreted, and they still hold.'
\z

\ea
\label{ex:focptcp3}
\langinfo{Tuatschín}{Ruèras}{f4, l. 1942ff.}\\
\gll [...] ábar \textbf{stju} \textbf{luvrá} \textbf{còrpòrálmajn} vajn nus schi fétg scù quèls.\\
{}	but must.\textsc{ptcp.unm} work.\textsc{inf} physical.\textsc{adj.m.adv} have.\textsc{prs.1pl} \textsc{1pl} so much as \textsc{dem.m.pl}\\
\glt `[...] but physically we had to work as hard as those [children].'
\z

Infinitives modified by a modal verb are left-dislocated without the modal verb, which remains in the background clause (\ref{ex:focmod1}).

\ea
\label{ex:focmod1}
\langinfo{Tuatschín}{Cavorgia}{\citealt[125]{Büchli1966}}\\
\gll Als tiars vèzan al barlòt a tèman, mù \textbf{dí} \textbf{sòn} i nuét.\\
     \textsc{def.art.m.pl} animal.\textsc{pl} see.\textsc{prs.3pl} \textsc{def.art.m.sg} sorcery and be\_afraid.\textsc{prs.3pl} but say.\textsc{inf} can.\textsc{prs.3pl} \textsc{3pl} nothing\\
\glt `The animals see the sorcery and are afraid, but they cannot say anything.'
\z

Example (\ref{ex:focpcl1}) shows the movement of the particle \textit{cùntar} `towards' out of the particle verb \textit{prèndar ancùntar} `receive' (calqued on German \textit{entgegennehmen}).
\ea
\label{ex:focpcl1}
\langinfo{Tuatschín}{Sadrún}{m5}\\
\gll Alṣò \textbf{ancùntar} prandès `l tùt ùsa?\\
well towards take.\textsc{cond.3sg} \textsc{3sg} all	now\\
\glt ´Well, would he receive everything now?'
\z

If a clause is focussed on, it may (\ref{ex:focclause1}) or may not be resumed by a demonstrative pronoun (\ref{ex:focclause2}). In the case of (\ref{ex:focclause1}), the demonstrative pronoun used is \textit{gljèz}.

\ea
\label{ex:focclause1}
\langinfo{Tuatschín}{Sadrún}{m4. l.}\\
\gll Ins vèz’ aun tg’ èra dau vidajn pùntgas né trádals; \textbf{sch}’ \textbf{i} \textbf{sitavan} gljèz sau bétg.\\
\textsc{gnr} see.\textsc{prs.3sg} still \textsc{comp} \textsc{pass.aux.impf.3sg} give.\textsc{ptcp.unm} into chisel.\textsc{f.pl} or power\_drill.\textsc{m.pl} whether \textsc{3pl} blow\_up.\textsc{impf.3pl} \textsc{dem.unm} know.\textsc{prs.1sg.1sg}  \textsc{neg}  \\
`One still can see that chisels or power drills had been used; whether they would blow up I don’t know.'
\z

\ea
\label{ex:focclause2}
\langinfo{Tuatschín}{Sadrún}{m4, l.453}\\
\gll   Aah, \textbf{tgé}.. \textbf{prandévan} \textbf{pròpi} \textbf{òra} sa ins bégj éxáct [...]. \\
ah what take.\textsc{impf.3pl} exactly out know.\textsc{prs.3sg} \textsc{gnr} \textsc{neg} exactly\\
\glt `Ah, what … they really mined one does not know exactly [...].'
\z

If a noun is focussed on, a pronoun referring to it must be left in the background clause (\ref{ex:focnoun1}).

\ea
\label{ex:focnoun1}
\langinfo{Tuatschín}{Sadrún}{m1, l. 270f.}\\
\gll  \textbf{Las} \textbf{nòtízjas} sa ju bétg danùndar als gjaniturs, als dus baps prandèvan \textbf{aj} [...]. \\
\textsc{def.art.f.pl} news.\textsc{pl} know.\textsc{prs.1sg} \textsc{1sg} \textsc{neg} from\_where \textsc{def.art.m.pl} parents.\textsc{pl} \textsc{def.art.m.pl} two.\textsc{m.pl} father.\textsc{pl} take.\textsc{impf.3pl} \textsc{3sg}\\
\glt `I don’t know where my parents had the news from, the two fathers took them, there was in Rueras, there was [only] one who had a radio.'
\z

Contrastive focus is done by intonation; they remain in their place according to their syntactic function (\ref{ex:foccontrnoun1} and \ref{ex:foccontrnoun2}).

\ea
\label{ex:foccontrnoun1}
\langinfo{Tuatschín}{Sadrún}{m5}\\
\gll \textbf{Gjòn} ò angulau la gaglina, bigja Maria.\\
\textsc{pn} have.\textsc{prs.3sg} steal.\textsc{ptcp.unm} \textsc{def.art.f.sg} hen \textsc{neg} \textsc{pn}\\
\glt `It is Gjon who stole the hen, not Maria.'
\z

\ea
\label{ex:foccontrnoun2}
\langinfo{Tuatschín}{Sadrún}{m4}\\
\gll Èl vut dá in cùdisch \textbf{da} \textbf{Gjòn}, bigja da Maria.\\
\textsc{3sg.m} want.\textsc{prs.3sg} give.\textsc{inf} \textsc{indef.art.m.sg} book \textsc{dat} \textsc{pn} \textsc{neg} \textsc{dat} \textsc{pn}\\
\glt `It is to Gion that he wants to give a book, not to Maria.'
\z







